
\par
En este apartado trataremos sobre c�mo configurar e inicializar el tracker, as� como los complementos necesarios para el funcionamiento del sistema\footnote{Tenemos que tener en cuenta que, como ya hemos comentado en el apartado \ref{decisiones}, utilizaremos el usuario ''\textbf{pfc}''.}.

\subsection{Configuraci�n de los elementos complementarios}
\subsubsection{Configuraci�n de la base de datos}

\subsection{Configuraci�n del tracker}
Para configurar el tracker, se deber�n seguir los siguientes pasos (todos ellos en modo superusuario):
\begin{enumerate}
	\item Crear directorio para incluir el tracker \footnote{NOTA: en el caso hipot�tico de que futuras verisiones consideraran que deber� configurarse m�s de un tracker, se utilizar� un �nico directorio para todos.}. 
\begin{center}
\verb|mkdir /opt/roundup/trackers|
\end{center}
	\item A�adir al path la direcci�n donde se encuentran los scripts (en nuestro caso en /opt/roundup/bin
	\item Ejecutar el siguiente comando: \verb|roundup-admin install|. Al ejecutarse esta opci�n, la aplicaci�n nos pedir� que seleccionemos la plantilla (template) y el sistema gestor de bases de datos subyacente (backend) a utilizar:
\begin {itemize}
	\item Inserte directorio base (Enter tracker home): en nuestro caso introduciremos:
\begin{center}
\verb|/opt/roundup/trackers/pfc/|.
\end{center}
	\item Seleccione plantilla (Select template):por defecto introduciremos 'classic'.
	\item Seleccione base de datos (Select backend): por defecto introduciremos 'anydbm'.
\end {itemize}
	\item Realizaremos las siguientes modificaciones en el archivo 'config.ini'\footnote{NOTA: para facilitar la configuraci�n, se adjunta en el cd de instalaci�n una versi�n del fichero config.ini que podr� tomarse tal cual para sustituir la que genera por defecto.}, cuyo sentido se explica en el apartado \ref{config}, sobre el dise�o de la aplicaci�n. (para lo cual, comprobaremos los correspondientes permisos y ajustaremos seg�n sea preciso):
\begin {itemize}
	\item \verb|[main]|
\begin {itemize}
	\item \verb|admin_email = pfc|
	\item \verb|dispatcher_email = pfc|
\end {itemize}
	\item \verb|[tracker]|
\begin {itemize}
	\item \verb|web = http://localhost:8080/pfc|\footnote{NOTA: Esta direcci�n deber� ser modificada de acuerdo a la ubicaci�n que tenga el equipo que sirva realmente la aplicaci�n. Indicar Localhost �nicamente valdr� para un escenario de pruebas en que cliente y servidor est�n en el mismo equipo}
	\item \verb|email = pfc|
\end {itemize}
	\item \verb|[mail]|
\begin {itemize}
	\item \verb|domain = localhost|
	\item \verb|host = localhost|
\end {itemize}
	
\end {itemize}
\end {enumerate}
\par
Para cualquier duda que pueda surgir sobre sobre la configuraci�n de roundup, se recomienda encarecidamente ver \cite{roundupweb}.
\par
\subsection{Inicializaci�n del tracker}\label{inicializacion}
\par
Antes que nada, hay que tener en cuenta que se tienen que verificar los permisos antes y despu�s de realizar este paso, pues al crearse nuevas carpetas puede ser que los permisos de �stas nos impidan realizar alg�n paso.
\par
La inicializaci�n del tracker debe hacerse tambi�n como superusuario. Los pasos a seguir ser�n los siguientes
\begin{enumerate}
	\item Se ejecutar� el comando:
\begin{center}
\verb|roundup-admin initialize|
\end{center}
	\item Se nos pedir� introducir el directorio base del tracker:
\begin{center}
\verb|/opt/roundup/trackers/pfc/|
\end{center}
	\item Se nos pide introducir una contrase�a de administraci�n (en este caso hemos utilizado 'pfc')
\par
\end{enumerate}


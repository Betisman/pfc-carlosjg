\chapter{Licencia P�blica General (GPL)}\label{gpl}
\lhead{Anexo \ref{gpl}}
\rhead{Licencia P�blica General (GPL)}

\par
Es una licencia creada por la FSF (Free Software Foundation) orientada principalmente a proteger la libre distribuci�n, modificaci�n y uso de software. Esta licencia declara que el cualquier software cubierto por ella es software libre, con el objetivo de protegerlo de intentos de apropiaci�n que restrinjan esas libertades a los usuarios.
\par
La definici�n de esta licencia viene marcada por el respeto a una serie de derechos que la FSF considera fundamentales para el usuario de software libre. 
\par

\section{Derechos del usuario de software libre}
\par
A continuaci�n se muestran, los cuatro derechos fundamentales que definen esta licencia.
\par
\subsubsection{El derecho a utilizar}
\par
El primer derecho o libertad, el que trata sobre el derecho a utilizar software, aunque pueda sorprender, tiene mucho sentido. . Cuando una persona ''compra'' un programa de ordenador que no es software libre por lo general no dispone del derecho de utilizaci�n ilimitada: el usuario est� limitado a utilizar el programa para determinados objetivos (prohibido usar este programa de forma comercial) o en determinados sitios (prohibido usar este programa en el pa�s X y el pa�s Y) o en un n�mero determinado de m�quinas (prohibido usar este programa en m�s de una m�quina al mismo tiempo). Estas restricciones son muy habituales cuando hablamos de software privativo, y el software libre debe respetar el derecho a que �stas no existan en la utilizaci�n.
\par
\subsubsection{El derecho a entender}
\par
La segunda libertad para el usuario: el derecho a entender c�mo funcionan los programas que nos distribuyen, y a adaptarlo a nuestras necesidades. Este derecho no existe cuando hablamos de software privativo: por lo general, el software privativo se distribuye en forma de ejecutables (equivalentes a los ficheros ''.exe'' en entornos windows) sin que le acompa�e el c�digo fuente correspondiente. El c�digo fuente de un programa es su forma entendible y modificable por un programador. 
\par
\subsubsection{El derecho a distribuir}
\par
El derecho a distribuir programas de ordenador de forma gratuita o, alternativamente, cobrando algo a cambio de hacerlo. Es natural, ya que la industria del software privativo hace cont�nuos esfuerzos para intentar convencer a la sociedad de que copiar programas de ordenador es algo que no debe hacerse. La FSF entiende que el poder copiar sin necesidad de grandes recursos (con una unidad de grabaci�n basta) y la caracter�stica peculiar de que la copia no pierde calidad respecto al original no es algo malo: por el contrario, es casi lo mejor que tiene el software. Copiar programas de ordenador y distribuirlas es algo que beneficia a la sociedad. Es de sentido com�n. Realizar copias de programas privativos es algo ilegal en la mayor�a de los pa�ses. Por eso proporcionamos software libre: es perfectamente legal copiarlo. De esta forma tanto el usuario como la sociedad se benefician, y nadie sale perdiendo (la copia original no funciona peor por haber hecho una o millones de copias).
\par
Es importante un detalle: el software libre no tiene por qu� ser gratis. Es perfectamente posible distribuir software libre a cambio de dinero. As� es como pueden ganarse la vida los programadores y distribuidores. Ahora bien, eso no justifica el hecho de vulnerar los derechos de la gente que paga por obtener una copia del programa: el usuario puede distribuir sus propias copias, cobrando por ello si lo desea.
\par
\subsubsection{El derecho a mejorar}
\par
El derecho a mejorar el software y distribuir las mejoras, es tal vez el que m�s controversia genera. El usuario de software privativo no puede mejorar los programas que utiliza: aunque quisiera y supiera hacerlo, por lo general no tiene acceso al c�digo fuente. Y aunque lo tuviera (puede distribuirse el c�digo fuente y no obstante no ser software libre) ser�a ilegal modificar ese c�digo fuente.
\par
Sin embargo, el software libre siempre se distribuye con su c�digo fuente, y adem�s es totalmente legal modificarlo. Y otra cosa importante: el usuario tambi�n tiene derecho a no distribuir sus mejoras si no quiere. Una persona puede descargar o comprar software libre, introducirle mejoras, y no redistribuir ni hacer p�blicas dichas mejoras. 
\par
\section{La licencia}
\par
Puede obtenerse m�s informaci�n, y la versi�n completa de la licencia en \cite{fsfweb}, y una versi�n de la misma en castellano en \cite{gnuweb}.

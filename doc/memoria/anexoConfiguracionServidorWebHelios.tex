\section{Configuraci�n del Servidor Web del Sistema de Votaci�n} \label{App:ConfiguracionServidorWeb}

El framework web Django provee su propio Servidor Web, pero para poder utilizar un canal seguro \gls{HTTPS} y configurar el acceso con el DNIe es preciso utilizar un Servidor Web como Apache en el que apoyar el framework. Aqu� una lista de pasos que se han llevado a cabo para su configuraci�n. 

\begin{lstlisting}[language=bash]
	# Instalamos Apache2 y OpenSSL
	sudo apt-get install apache2
	sudo apt-get install openSSL
	
	cd /etc/apache2/mods-available
	
	# Activamos SSL en Apache
	sudo a2enmod ssl
	
	sudo apt-get install libapache2-mod-wsgi
	sudo a2enmod wsgi
	
	cd /etc/apache2/sites-available
	sudo cp 000-default.conf sitednie.conf
	cd /etc/apache2/sites-available
	
	# Preparamos los certificados para el DNIe
	mkdir ~/Descargas/certificados
	cd certificados/
	
	# Creamos una clave
	openssl genrsa -des3 -out server.key 2048
	
	# Petici�n del certificado, asoci�ndolo con la clave que acabamos de crear
	openssl req -new -key server.key -out server.csr
	
	# Pese a que no somos una entidad certificadora, firmamos nuestro propio certificado, as�
	# obtenemos un certificado autofirmado.
	openssl x509 -req -days 365 -in server.csr -signkey server.key -out server.crt
	
	
	#Bajamos el certificado de la AC Ra�z del DNIE desde:
	#http://www.dnielectronico.es/PortalDNIe/PRF1_Cons02.action?pag=REF_077
	wget http://www.dnielectronico.es/ZIP/ACRAIZ-SHA2.CAB
	
	# Lo extraemos
	cabextract ACRAIZ-SHA2.CAB
	
	# Creamos el certificado x509
	openssl x509 -in ACRAIZ-SHA2.crt -inform DER -out ACRAIZ-SHA2.crt -outform PEM
	sudo cp ACRAIZ-SHA2.crt acraiz-dnie.cer
	
	cd /etc/apache2/sites-available
	sudo cat default-ssl.conf >> sitednie.conf
	
	
	# Configuramos el servidor web para que escuche por un puerto seguro, con SSL y 
	# requiriendo los certificados del DNIe
	sudo gedit sitednie.conf &	
\end{lstlisting}

Puede ser con el comando anterior para usar el editor gedit o con cualquier otro editor de nuestra preferencia, pero se deben modificar los siguientes par�metros del fichero:

En <VirtualHost *:80>:
\begin{lstlisting}[language=XML]
DocumentRoot /home/carlos/Descargas/helios-server
	<Directory /home/carlos/Descargas/helios-server>
		<Files wsgi.py>
			Require all granted
		</Files>
	</Directory>

	WSGIDaemonProcess / python-path=/home/carlos/Descargas/helios-server:/home/carlos/Descargas/helios-server/venv/lib/python2.7/site-packages
    WSGIProcessGroup /
    WSGIScriptAlias / /home/carlos/Descargas/helios-server/wsgi.py
\end{lstlisting}


En <VirtualHost \_default\_:443>:
\begin{lstlisting}[language=XML]
	DocumentRoot /home/carlos/Descargas/helios-server
	### Antes de cerrar </VirtualHost>:

	<Directory /home/carlos/Descargas/helios-server/>
	     Options Indexes FollowSymLinks MultiViews
	     AllowOverride None
	     Order allow,deny
	     allow from all
	   </Directory>
		
	### Modificar:
	SSLCertificateFile	/home/carlos/Descargas/certificados/server.crt
	SSLCertificateKeyFile /home/carlos/Descargas/certificados/server.key
	SSLCACertificateFile /home/carlos/Descargas/certificados/acraiz-dnie.cer 
	SSLVerifyClient require
	SSLVerifyDepth  2
	
	### Modificar donde se deba:
	SSLOptions +StdEnvVars +ExportCertData
	
	
	<Directory /home/carlos/Descargas/helios-server>
		<Files wsgi.py>
			Require all granted
		</Files>
	</Directory>

	#WSGIDaemonProcess / python-path=/home/carlos/Descargas/helios-server:/home/carlos/Descargas/helios-server/venv/lib/python2.7/site-packages
    WSGIProcessGroup /
    WSGIScriptAlias / /home/carlos/Descargas/helios-server/wsgi.py
\end{lstlisting}


A continuaci�n hay que editar el fichero ports.conf si es necesario:
\begin{lstlisting}[language=bash]
	sudo gedit /etc/apache2/ports.conf
\end{lstlisting}
\begin{lstlisting}[language=XML]
	### Modificar si es necesario:
	<IfModule ssl_module>
		Listen 443
		Listen 1443
	</IfModule>
\end{lstlisting}

Cargamos la configuraci�n modificada en Apache y reseteamos el servicio:
\begin{lstlisting}[language=bash]
	sudo a2ensite sitednie.conf
	sudo service apache2 restart
\end{lstlisting}


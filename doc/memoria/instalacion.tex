\subsection{Requisitos para la instalacion}\label{requisitosinst}
\subsection{Pasos previos a la instalaci�n}
\par
Para poder realizar la instalaci�n de los programas necesitaremos previamente instalar la distribuci�n correspondiente de Linux, en este caso la distribuci�n SUSE 9.3, incluyendo los siguientes paquetes:
\begin{itemize}
\item
	python: int�rprete de Python (versi�n superior a TODO:qu� versi�n?)
\item
 	python-devel: paquete necesario, pues roundup necesita "distutils". Este paquete requiere de la instalaci�n de python-tk y blt.
\end{itemize}

\subsection{Instalaci�n de Roundup Issue Tracker}
\par
A continuaci�n se muestran las distintas acciones a seguir para una correcta instalaci�n y configuraci�n del programa.
	{Operaciones a realizar en modo superusuario}
\par
A continuaci�n se muestran los pasos a seguir, que deber�n ser realizados desde la consola, habi�ndose identificado como "root":

\begin{itemize}
\item
	Elecci�n de la direcci�n para la instalaci�n (en este caso, elegiremos: 
\begin{center}
\verb|/opt/roundup/bin| 
\end{center}
\item
	Ejecutar, situ�ndonos en el directorio donde hayamos descomprimido el programa:
\begin{center}
\verb|python setup.py install --install-scripts=/opt/roundup/bin| .
\end{center}
\end{itemize}
\par
Para cualquier duda que pueda surgir sobre la instalaci�n de roundup, se recomienda encarecidamente ver \cite{roundupweb}.
\par

\subsection{Instalaci�n PostgreSQL}\label{instpost}
\par
\textbf{DECIR LAS INDICACIONES PROPIAS QUE REQUIERE POSTGRES}
\par
Para cualquier duda que pueda surgir sobre la instalaci�n de roundup, se recomienda encarecidamente ver \cite{postweb} y \cite{postgres}.
\par
\subsection{Instalaci�n Xapian}\label{xapinst}
\textbf{DECIR LAS INDICACIONES PROPIAS QUE REQUIERE ROUNDUP PARA ESTO}
\par
Cualquier aclaraci�n adicional que se necesite sobre c�mo debe hacerse la instalaci�n de Xapian, se puede obtener en \cite{xapianweb}.
\par


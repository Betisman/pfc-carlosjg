\section{Verificar un voto individualmente} \label{anexo:verificarVoto}

	Esta documentaci�n se puede consultar en la documentaci�n de Helios Voting\footnote{\url{http://documentation.heliosvoting.org/verification-specs/helios-v4}}.
	
	Recall the Chaum-Pedersen proof that a ciphertext ($\alpha$,$\beta$) under public key $(y, (g,p,q))$ is proven to encode the value $m$ by proving knowledge of $r$, the randomness used to create the ciphertext, specifically that $g$, $y$, $\alpha$, $\beta/g^m$ is a DDH tuple, noting that $\alpha = g^r and \beta/g^m = y^r$.

	\begin{itemize}
		\item Prover sends $A = g^w \bmod p$ and $B = y^w \bmod p$ for a random $w$.
		\item Verifier sends challenge, a random challenge $\bmod q$.
		\item Prover sends $response = w + challenge * r$.
		\item Verifier checks that:
		\begin{itemize}
			\item $g^{response} = A * \alpha^{challenge}$
			\item $y^{response} = B * (\beta/g^m)^{challenge}$
		\end{itemize}
	\end{itemize}
	
\begin{lstlisting}[language=Python]
	def verify_proof(ciphertext, plaintext, proof, public_key):
		if pow(public_key.g, proof.response, public_key.p) !=
		    ((proof.commitment.A * pow(ciphertext.alpha, proof.challenge, public_key.p)) % public_key.p):
		       return False
		
		beta_over_m = modinverse(pow(public_key.g, plaintext, public_key.p), public_key.p) * ciphertext.beta
		beta_over_m_mod_p = beta_over_m % public_key.p
		
		if pow(public_key.y, proof.response, public_key.p) !=
		   ((proof.commitment.B * pow(beta_over_m_mod_p, proof.challenge, public_key.p)) % public_key.p):
		      return False
		
		return True
\end{lstlisting}

	In a disjunctive proof that the ciphertext is the encryption of one value between 0 and max, all max+1 proof transcripts are checked, and the sum of the challenges is checked against the expected challenge value. Since we use this proof in non-interactive Fiat-Shamir form, we generate the expected challenge value as $SHA1(A_0 + B_0 + A_1 + B_1 + ... + A_{max} + B_{max})$ with $A_0, B_0, A_1, B_1, ... , A_{max}, B_{max}$ in decimal form. ($A_i$ and $B_i$ are the components of the commitment for the i'th proof.)

	Thus, to verify a $<ZK\_PROOF\_0..max>$ on a $<ELGAMAL\_CIPHERTEXT>$, the following steps are taken.

\begin{lstlisting}[language=Python]
	def verify_disjunctive_0..max_proof(ciphertext, max, disjunctive_proof, public_key):
		for i in range(max+1):
		  # the proof for plaintext "i"
		  if not verify_proof(ciphertext, i, disjunctive_proof[i], public_key):
		    return False
		
		# the overall challenge
		computed_challenge = sum([proof.challenge for proof in disjunctive_proof]) % public_key.q
		
		# concatenate the arrays of A,B values
		list_of_values_to_hash = sum([[p.commitment.A, p.commitment.B] for p in disjunctive_proof], [])
		
		# concatenate as strings
		str_to_hash = ",".join(list_of_values_to_hash)
		
		# hash
		expected_challenge = int_sha(str_to_hash)
		
		# last check
		return computed_challenge == expected_challenge
\end{lstlisting}

Thus, given <ELECTION> and a <VOTE>, the verification steps are as follows:

\begin{lstlisting}[language=Python]
	def verify_vote(election, vote):
		# check hash (remove the last character which is a useless '=')
		computed_hash = base64.b64encode(hash.new(election.toJSON()).digest())[:-1]
		if computed_hash != vote.election_hash:
		    return False
		
		# go through each encrypted answer by index, because we need the index
		# into the question array, too for figuring out election information
		for question_num in range(len(vote.answers)):
		   encrypted_answer = vote.answers[question_num]
		   question = election.questions[question_num]
		
		   # initialize homomorphic sum (assume operator overload on __add__ with 0 special case.)
		   homomorphic_sum = 0
		
		   # go through each choice for the question (loop by integer because two arrays)
		   for choice_num in range(len(encrypted_answer.choices)):
		     ciphertext = encrypted_answer.choices[choice_num]
		     disjunctive_proof = encrypted_answer.individual_proofs[choice_num]
		     
		     # check the individual proof (disjunctive max is 1)
		     if not verify_disjunctive_0..max_proof(ciphertext, 1, disjunctive_proof, election.public_key):
		        return False
		        
		     # keep track of homomorphic sum
		     homomorphic_sum = ciphertext + homomorphic_sum
		   
		   # check the overall proof
		   if not verify_disjunctive_0..max_proof(homomorphic_sum, question.max,
		                                          encrypted_answer.overall_proof,
		                                          election.public_key):
		       return False
		       
		# done, we succeeded
		return True
\end{lstlisting}


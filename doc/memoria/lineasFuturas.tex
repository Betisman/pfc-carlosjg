\chapter{L�neas futuras}\label{lineasFuturas}
\lhead{Cap�tulo \ref{lineasFuturas}}
\rhead{L�neas futuras}

\todo[inline]{Uso de Enigma \url{http://enigma.media.mit.edu/enigma_full.pdf}(8.9, p�gina 13) para el voto por Internet.}
\todo[inline]{{tro: \url{http://www.pabloyglesias.com/cifrado-homomorfico-blockchain/}}}

\todo[inline]{Realizar auditor�a de la criptograf�a y seguridad de los procesos}
\todo[inline]{Mejorar el protocolo oAuth y montar un sistema REST para poder usado como servicio web y llamado desde diferentes plataformas: web estilo Angular, app m�vil con llamadas Http directas, app de escritorio...}

\todo[inline]{Investigar m�todos de identificaci�n - biometr�a, SIM con IDs ... }

\todo[inline]{Carlos, aseg�rate de que comentamos en alg�n sitio que es el primer sistema con el que se puede votar con el nuevo DNIe 3.0. Importante por la novedad.}


	Tras la elaboraci�n de la memoria de este proyecto y la implementaci�n del mismo, se advierten varios �mbitos donde hay alguna carencia tecnol�gica o se observa una oportunidad de desarrollo futuro.
	
	Una necesidad de este sistema es una migraci�n hacia TDD. Un sistema tan importante como una votaci�n, que debe minimizar los errores al m�ximo por la criticidad de los mismos, deber�a apoyarse en un sistema automatizado de tests. Deber�an implementarse tests unitarios y punto a punto para todos los subsistemas y funcionalidades del sistema. Con esto, se puede asegurar la integridad del mismo y facilitar los cambios en el software, provocando un software m�s estable.
	
	 
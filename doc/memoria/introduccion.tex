\chapter*{Introducci�n}
\rhead{Introducci�n}
\par
Este proyecto trata de entrar en la problem�tica del voto electr�nico remoto y presencial, de las reticencias sociales y tecnol�gicas que influyen en su reducida implantaci�n en procesos electorales de gran importancia y alto n�mero de electores. Para ello, vamos a reproducir la situaci�n a escala reducida. Plantearemos una posible soluci�n al proceso necesario para llevar a cabo las Elecciones a la Junta de Escuela de la Escuela Polit�cnica Superior de la Universidad San Pablo - CEU.
\par
Con este planteamiento es obvio que no vamos a solucionar las trabas t�cnicas y sociales del voto por internet a nivel de unas elecciones legislativas en, por ejemplo, Espa�a. Es un tema que se escapa del objetivo de este PFC, pero s� que vamos a tratar de identificar algunos de los agentes influyentes y buscar una posible soluci�n aplicable a la elecci�n a la Junta de Escuela.
\par
As�, conseguiremos dos objetivos. Por un lado, estudiar la dificultad existente para la implantaci�n del voto electr�nico en las elecciones nacionales. Por otro, un soporte electr�nico al proceso completo de las Elecciones a la Junta de Escuela, con el cual obtendremos una mejora significativa en el mismo respecto a procesos anteriores.
\par
La forma de llegar a la soluci�n buscada debe comenzar identificando los factores que afectan a un proceso electoral general y, a continuaci�n, personalizar los que se encuentran en el que vamos a estudiar.
Una vez identificados estos agentes, definiremos las fases que comportan unas elecciones y estudiaremos c�mo podr�an ser apoyadas tecnol�gicamente, evaluando c�mo llegar al punto �ptimo de integraci�n con el sistema tradicional para mejorar el proceso.
\par
La primera fase se concentrar� en desarrollar los sistemas asociados a la fase preelectoral. En ella, se recoge el censo electoral y se identifican tanto los candidatos como los diferentes cargos que se votan.
%%OJO: �hay que hablar de la l�gica de la elecci�n? De c�mo se vota y qui�n para elegir el qu� y c�mo??
\par
La segunda fase, la electoral, la identificamos con los procesos que se requieren durante el periodo que dura la elecci�n (ya sea un d�a o varios). Esta consistir� en desarrollar los sistemas de identificaci�n y validaci�n de votantes, el sistema de votaci�n, ss
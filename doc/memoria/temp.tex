\chapter*{temp}
\par
Seg�n el documento \textbf{NORMAS DE ORGANIZACI�N Y FUNCIONAMIENTO DE LA UNIVERSIDAD SAN PABLO-CEU}, en su Art�culo 9, ''Las Facultades, Escuelas y Centros integrados o adscritos son las instancias responsables de la organizaci�n de la ense�anza e investigaci�n, de acuerdo con las directrices emanadas de los �rganos superiores de la Universidad, y de los procesos acad�micos, administrativos y de gesti�n conducentes a la obtenci�n de t�tulos de car�cter oficial y validez en todo el territorio nacional, as� como de aquellas otras funciones que determinen las presentes Normas de Organizaci�n y Funcionamiento y los restantes reglamentos universitarios.''
\par
A partir de esta definici�n, en el Cap�tulo II De los �rganos acad�micos, encontramos el Art�culo 22 Tipos de �rganos, donde se establece que (1c) que las Juntas de Facultad, Escuela o Centro son �rganos colegiados. Y encontramos su definici�n en el Art�culo 31 Las Juntas de Centros, donde podemos leer que ''La Junta de Facultad, Escuela o Centro es el �rgano colegiado de gobierno del mismo, que ejerce sus funciones con vinculaci�n a los acuerdos del Patronato, Consejo de Gobierno y resoluciones del Rector.''
\par
Tambi�n podemos destacar los art�culos 32 y 33, donde se establece la composici�n y funciones de las Juntas de Facultad, Centro o Escuela.
\par
Art�culo 32
Composici�n de las Juntas
La Junta de Facultad, Escuela o Centro estar� compuesta por miembros natos y electos.
Son miembros natos: El Decano o Director, que presidir� sus reuniones; los Vicedecanos o Subdirectores, el Secretario acad�mico, que levantar� acta de sus sesiones y los Directores de los Departamentos integrados en la Facultad o Escuela.
Son miembros electos: Quienes resulten elegidos en representaci�n del profesorado y de los alumnos de acuerdo con la normativa que reglamentariamente se establezca.
\par
Art�culo 33
Funciones de las Juntas
Las competencias de la Junta de Facultad, Escuela o Centro son:
a) Colaborar con el Decano o Director en la gesti�n de la Facultad, Escuela o Centro.
b) Promover el perfeccionamiento de los planes de estudio y de la metodolog�a docente, as� como el establecimiento de nuevos t�tulos tanto propios como oficiales.
c) Participar en la programaci�n de las actividades de extensi�n universitaria.
d) Velar por la adecuada dotaci�n de los servicios necesarios para su correcto funcionamiento.
e) Cualquier otra competencia que le pueda ser atribuida en el desarrollo de estas Normas de Organizaci�n y Funcionamiento.


\chapter{Manuales}\label{manuales}
\lhead{Cap�tulo \ref{manuales}}
\rhead{Manuales}
%*******************************************************************************
\section{Notas sobre los manuales}
\par
El objetivo de este cap�tulo es facilitar unos manuales m�nimos pero suficiente para poder instalar, configurar, utilizar y administrar el sistema, de una forma correcta pero sencilla. 
\par
Aunque podr�amos entender que los contenidos de este cap�tulo podr�an estar dentro de los apartados relativos a la construcci�n del sistema y a la implantaci�n y despliegue, se ha optado por presentarlo a parte, debido a la importancia que tiene dentro de los requisitos descritos. Dicha importancia viene descrita en el apartado \ref{alcance_final}, sobre el alcance final del proyecto, donde se indicaba la creaci�n de los manuales como el tercero de los cuatro objetivos a resolver (los dos primeros quedaron resueltos en el cap�tulo \ref{solucion}, y el cuarto se resolver� en el cap�tulo \ref{lineas_futuras}).
\par
Si se precisa ayuda, o ampliar informaci�n sobre el sistem Linux, en quien reside la aplicaci�n, ver \cite{autounix}. Puede ser muy util tanto a la hora de la instalaci�n, configuraci�n, uso, administraci�n y mantenimiento del sistema.
\section{Manual de Instalaci�n}
\subsection{Requisitos para la instalacion}\label{requisitosinst}
\subsection{Pasos previos a la instalaci�n}
\par
Para poder realizar la instalaci�n de los programas necesitaremos previamente instalar la distribuci�n correspondiente de Linux, en este caso la distribuci�n SUSE 9.3, incluyendo los siguientes paquetes:
\begin{itemize}
\item
	python: int�rprete de Python (versi�n superior a TODO:qu� versi�n?)
\item
 	python-devel: paquete necesario, pues roundup necesita "distutils". Este paquete requiere de la instalaci�n de python-tk y blt.
\end{itemize}

\subsection{Instalaci�n de Roundup Issue Tracker}
\par
A continuaci�n se muestran las distintas acciones a seguir para una correcta instalaci�n y configuraci�n del programa.
	{Operaciones a realizar en modo superusuario}
\par
A continuaci�n se muestran los pasos a seguir, que deber�n ser realizados desde la consola, habi�ndose identificado como "root":

\begin{itemize}
\item
	Elecci�n de la direcci�n para la instalaci�n (en este caso, elegiremos: 
\begin{center}
\verb|/opt/roundup/bin| 
\end{center}
\item
	Ejecutar, situ�ndonos en el directorio donde hayamos descomprimido el programa:
\begin{center}
\verb|python setup.py install --install-scripts=/opt/roundup/bin| .
\end{center}
\end{itemize}
\par
Para cualquier duda que pueda surgir sobre la instalaci�n de roundup, se recomienda encarecidamente ver \cite{roundupweb}.
\par

\subsection{Instalaci�n PostgreSQL}\label{instpost}
\par
\textbf{DECIR LAS INDICACIONES PROPIAS QUE REQUIERE POSTGRES}
\par
Para cualquier duda que pueda surgir sobre la instalaci�n de roundup, se recomienda encarecidamente ver \cite{postweb} y \cite{postgres}.
\par
\subsection{Instalaci�n Xapian}\label{xapinst}
\textbf{DECIR LAS INDICACIONES PROPIAS QUE REQUIERE ROUNDUP PARA ESTO}
\par
Cualquier aclaraci�n adicional que se necesite sobre c�mo debe hacerse la instalaci�n de Xapian, se puede obtener en \cite{xapianweb}.
\par


\section{Manual de Configuraci�n}

\par
En este apartado trataremos sobre c�mo configurar e inicializar el tracker, as� como los complementos necesarios para el funcionamiento del sistema\footnote{Tenemos que tener en cuenta que, como ya hemos comentado en el apartado \ref{decisiones}, utilizaremos el usuario ''\textbf{pfc}''.}.

\subsection{Configuraci�n de los elementos complementarios}
\subsubsection{Configuraci�n de la base de datos}

\subsection{Configuraci�n del tracker}
Para configurar el tracker, se deber�n seguir los siguientes pasos (todos ellos en modo superusuario):
\begin{enumerate}
	\item Crear directorio para incluir el tracker \footnote{NOTA: en el caso hipot�tico de que futuras verisiones consideraran que deber� configurarse m�s de un tracker, se utilizar� un �nico directorio para todos.}. 
\begin{center}
\verb|mkdir /opt/roundup/trackers|
\end{center}
	\item A�adir al path la direcci�n donde se encuentran los scripts (en nuestro caso en /opt/roundup/bin
	\item Ejecutar el siguiente comando: \verb|roundup-admin install|. Al ejecutarse esta opci�n, la aplicaci�n nos pedir� que seleccionemos la plantilla (template) y el sistema gestor de bases de datos subyacente (backend) a utilizar:
\begin {itemize}
	\item Inserte directorio base (Enter tracker home): en nuestro caso introduciremos:
\begin{center}
\verb|/opt/roundup/trackers/pfc/|.
\end{center}
	\item Seleccione plantilla (Select template):por defecto introduciremos 'classic'.
	\item Seleccione base de datos (Select backend): por defecto introduciremos 'anydbm'.
\end {itemize}
	\item Realizaremos las siguientes modificaciones en el archivo 'config.ini'\footnote{NOTA: para facilitar la configuraci�n, se adjunta en el cd de instalaci�n una versi�n del fichero config.ini que podr� tomarse tal cual para sustituir la que genera por defecto.}, cuyo sentido se explica en el apartado \ref{config}, sobre el dise�o de la aplicaci�n. (para lo cual, comprobaremos los correspondientes permisos y ajustaremos seg�n sea preciso):
\begin {itemize}
	\item \verb|[main]|
\begin {itemize}
	\item \verb|admin_email = pfc|
	\item \verb|dispatcher_email = pfc|
\end {itemize}
	\item \verb|[tracker]|
\begin {itemize}
	\item \verb|web = http://localhost:8080/pfc|\footnote{NOTA: Esta direcci�n deber� ser modificada de acuerdo a la ubicaci�n que tenga el equipo que sirva realmente la aplicaci�n. Indicar Localhost �nicamente valdr� para un escenario de pruebas en que cliente y servidor est�n en el mismo equipo}
	\item \verb|email = pfc|
\end {itemize}
	\item \verb|[mail]|
\begin {itemize}
	\item \verb|domain = localhost|
	\item \verb|host = localhost|
\end {itemize}
	
\end {itemize}
\end {enumerate}
\par
Para cualquier duda que pueda surgir sobre sobre la configuraci�n de roundup, se recomienda encarecidamente ver \cite{roundupweb}.
\par
\subsection{Inicializaci�n del tracker}\label{inicializacion}
\par
Antes que nada, hay que tener en cuenta que se tienen que verificar los permisos antes y despu�s de realizar este paso, pues al crearse nuevas carpetas puede ser que los permisos de �stas nos impidan realizar alg�n paso.
\par
La inicializaci�n del tracker debe hacerse tambi�n como superusuario. Los pasos a seguir ser�n los siguientes
\begin{enumerate}
	\item Se ejecutar� el comando:
\begin{center}
\verb|roundup-admin initialize|
\end{center}
	\item Se nos pedir� introducir el directorio base del tracker:
\begin{center}
\verb|/opt/roundup/trackers/pfc/|
\end{center}
	\item Se nos pide introducir una contrase�a de administraci�n (en este caso hemos utilizado 'pfc')
\par
\end{enumerate}


\section{Manual de Uso}
\par
Una vez configurado el programa, e inicializado el tracker, en este cap�tulo se indicar� c�mo se debe proceder a levantar el servicio. Tambi�n se realiza alguna peque�a aclaraci�n sobre los distintos manuales que se deben utilizar.
\subsection{Levantar el servicio}\label{levantar}
\par
Para arrancar el tracker y que el sistema entre en funcionamiento, se debe levantar el servicio. 
\par
Esta operaci�n se deber� realizar tambi�n en caso de que la m�quina caiga, o tras cualquier operaci�n de mantenimiento del sistema. Es decir, cada vez que ocurra un evento por el cual debamos volver a poner en funcionamiento el sistema.
\par
Para levantar el servicio, deberemos trabajar como usuario normal, y no como superusuario (como hemos hecho en la instalaci�n y en la configuraci�n). Para ello, se deben revisar los permisos en las carpetas involucradas, y comprobar que sean los pertinentes.
\par
Se ejecutar�n en la consola las siguientes instrucciones, de acuerdo a los casos expuestos en la instalaci�n y la configuraci�n:
\begin{center}
\verb|export PATH=$PATH:/opt/roundup/bin|
\end{center}
\begin{center}
\verb|roundup-server pfc=/opt/roundup/trackers/pfc|
\end{center}
\par
Por comodidad en el uso, se recomienda encarecidamente el uso del script ''levantar.sh'' que se adjunta en el CD, y contiene s�mplemente esas dos instrucciones. Para utilizar este script, habr�a que alojarlo dentro del directorio ''home/pfc''. Para levantar la aplicaci�n bastar� unicamente con situarnos en ese directorio desde la consola y ejecutar:
\begin{center}
\verb|./levantar.sh|
\end{center}
\par
\subsection{Manual de usuario}
\par
Para un correcto uso del sistema, se ven involucrados los siguientes documentos:
\begin{itemize}
	\item Documentaci�n de uso de Round-up Issue Tracking (incluida en el CD adjunto, y tambi�n disponible y actualizada en \cite{roundupweb}).
	\item Cualquier manual, normativa o documentaci�n propuesta para la utilizaci�n de este sistema, comunicaci�n en los roles m�s bajos, etc, as� como cualquier otro documento que sea considerado pertinente por las Fuerzas Armadas.
\end{itemize}
\par
Aunque el funcionamiento del sistema es muy intuitivo y sencillo, cualquier duda podr� ser aclarada en la documentaci�n del programa, por lo que esa documentaci�n debe estar disponible. \par
Igualmente, el programa no deber� entrar en producci�n hasta que los usuarios finales tengan pleno conocimiento de las normativas de uso impuestas por las Fuerzas Armadas. 
 \label{uso}
\section{Notas sobre la administraci�n y el mantenimiento}
Hablar de que existe m�s documentaci�n para los backend,selenium etc.

Hacer una aclaraci�n sobre las cosas que debe hacer el administrador. (crear usuarios... ver guia de administraci�n de roundup
 \label{notasadmin}

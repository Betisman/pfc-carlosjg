\chapter{Conclusiones}\label{conclusiones}
\lhead{Cap�tulo \ref{conclusiones}}
\rhead{Conclusiones}

	El problema del voto por Internet es m�s complejo de lo que a priori puede parecer.
	
	El hecho de tener que lidiar con el reto de la dualidad verificsbilidad - secreto (\ref{estadocuestion.votoelectronico.verifvssecreto}) complica el dise�o de cualquier sistema de voto electr�nico por Internet. Es muy complicado llegar al punto medio en el cual la privacidad del voto de un votante se mantiene lo justo para poder demostrar que el voto es correcto y de la misma forma ha sido inclu�do en el escrutinio.
	
	Esta lucha entre verificabilidad y secreto provoca desconfianza. Tanto en los propios votantes, por la privacidad de su elecci�n. Como en los afectados por el resultado, que pueden dudar de la transparencia del sistema, de la honestidad del mismo a la hora de contar los votos sufragados.
	
	La mayor�a de los retos tecnol�gicos est�n superados. La seguridad de las comunicaciones, la identificaci�n digital de los votantes, las herramientas criptogr�ficas. Incluso en este proyecto avanzamos una prueba de concepto para utilizar el nuevo \gls{DNIe} 3.0 usando sensores \gls{NFC} para poder votar desde un dispositivo m�vil. A diario utilizamos sistemas que requieren una seguridad muy importante, como las transacciones bancarias, consultas m�dicas, \gls{VPN}s corporativas, etc. 
	
	Sin embargo, en la mayor�a de estos casos de uso, un fallo en estos sistemas es visible cuando ocurre, incluso demostrable. Si falla el sistema del banco y nos cobra irregularmente m�s dinero del que debe, lo podemos ver en el extracto de nuestra cuenta y reclamarlo, por ejemplo.
	
	En un sistema de voto por Internet, esto no es posible. Al menos si queremos asegurar la privacidad del votante. Si el sistema no totaliza correctamente nuestro voto, no tenemos forma de saberlo o demostrarlo si no procedemos a desencriptar e identificar, de alguna forma, nuestro voto en la "urna digital".
	
	La confianza es b�sica, por tanto, para estos sistemas.
	
	Otro problema importante es la coacci�n. La capacidad de que un sistema proteja al votante de presiones externas a la hora de emitir su voto.
	
	El desarrollo de este proyecto, basado en Helios Voting, es factible para llevar a cabo elecciones con bajo riesgo de coacci�n, como las de nivel universitario, organizacional o corporativas. En estos casos, aunque la importancia de la elecci�n puede variar, normalmente el riesgo de coacci�n al votante es bajo. No es lo mismo cuando hablamos de elecciones de cargos p�blicos, donde s� que se pueden producir presiones importantes al censo.
	
	En este caso, queda un largo recorrido para que el software pueda 












\iffalse
Desde que comenc� el proyecto hasta el momento de entregarlo he ido adquiriendo nuevos conocimientos y t�cnicas de desarrollo.
En los �ltimos tiempos he comenzado a desarrollar en base a integraci�n continua. Es una pena que haya sido cuando ya estaba implementado el proyecto y no haber podido realizarlo desde el comienzo. Creo que es una buena t�cnica, muy �til y que va perfecta para desarrollos en los que un proyecto est� en producci�n 

\fi

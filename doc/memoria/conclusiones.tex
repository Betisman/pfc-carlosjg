\chapter{Conclusiones}\label{conclusiones}
\lhead{Cap�tulo \ref{conclusiones}}
\rhead{Conclusiones}
%*******************************************************************************
\par
En este trabajo nos hemos centrado en encontrar una soluci�n para los roles m�s bajos de la jerarqu�a empleada en el sistema de telemedicina de las Fuerzas Armadas Espa�olas. Las soluciones aqu� propuestas se han centrado en solucionar toda la problem�tica existente en estos niveles de la jerarqu�a, y en crear un sistema que satisfaga todas las necesidades requeridas.
\par
Dentro de estos niveles se han descrito dos tipos de abstracci�n, representados por dos sistemas distintos: un consistir�a en un sistema que contenga la informaci�n de las fichas m�dicas que prev�n los est�ndares OTAN, y otro, que es el que en este proyecto estamos estudiando, que gestionar�a los eventos sucedidos en el tr�mite de las urgencias m�dicas.
\par
Puesto que los temas derivados de niveles superiores de la jerarqu�a ya se encuentran resueltos, este proyecto no ha tocado en ning�n momento ning�n aspecto relativo a los sistemas que resuelven esos problemas.
\par
Para la creaci�n del sistema planteado en este proyecto, y el buen cumplimiento de los requisitos planteados se han cumplido los siguientes objetivos: se ha dise�ado un modelo que representaba de un modo apropiado la situaci�n real que se quer�a resolver; se ha encontrado una soluci�n software capaz de resolver el problema; se han creado los manuales precisos para su correcta instalaci�, configuraci�n y uso; se ha facilitado el desarrollo para futuras lineas de desarrollo.
\par
Para obtener esos resultados no solo se han elegido muy cuidadosamente todas las opciones posibles (desde la aplicaci�n a utilizar, el sistema operativo o la base de datos, a los lenguajes de programaci�n, los modelos para el despliegue, etc), sino que adem�s se ha puesto una importancia muy especial en las pruebas, que han sido automatizadas, y se han utilizado algunas pr�cticas de metodolog�as �giles que han enriquecido mucho este trabajo. Todos estos factores han hecho que esos resultados sean muy satisfactorios.
\par
Los unicos inconvenientes que se han encontrado son los derivados a la insuficiencia de este sistema por si solo (en este caso carece de sentido, si no es correctamente integrado en las otras fases de desarrollo descritas), y a la posible dificultad que podr�a tener este sistema llegado el momento de ingresarse con los sistemas ya existentes, en futuras fases de desarrollo.
\par
Pero independientemente de estos inconvenientes, se ha encontrado una soluci�n que satisface todos los requisitos y objetivos propuestos. Adem�s, en todo momento se ha facilitado el trabajo para las futuras fases de desarrollo, como se puede ver a lo largo del documento en todas las decisiones tomadas, y de forma m�s concreta en el cap�tulo dedicado a las futuras lineas de desarrollo.
\section{Verificar la Totalizaci�n de una Elecci�n} \label{anexo:verificarEleccion}
	Esta documentaci�n se puede consultar en la documentaci�n de Helios Voting\footnote{\url{http://documentation.heliosvoting.org/verification-specs/helios-v4}}.


	To verify a complete election tally, one should:
	
	\begin{itemize}
		\item display the computed election fingerprint.
		\item ensure that the list of voters matches the election voter-list hash.
		\item display the fingerprint of each cast ballot.
		\item check that each cast ballot is correctly formed by verifying the proofs.
		\item homomorphically compute the encrypted tallies
		\item verify each trustee's partial decryption
		\item combine the partial decryptions and verify that those decryptions, the homomorphic encrypted tallies, and the claimed plaintext results are consistent.
	\end{itemize}

	In other words, the complete results of a verified election includes: the election fingerprint, the list of ballot fingerprints, the trustee decryption factors and proofs, and the final plaintext counts. Any party who verifies the election should re-publish all of these items, as they are meaningless without one another. This is effectively a "re-tally".

	Part of this re-tally requires checking a partial decryption proof, which is almost the same, but not quite the same, as checking an encryption proof with given randomness.

	Given a ciphertext denoted ($alpha$,$beta$), and a trustee's private key $x$ corresponding to his public key $y$, a partial decryption is:
	
\begin{equation}
  dec\_factor = \alpha^x \bmod p
\end{equation}

    

	The trustee then provides a proof that ($g$, $y$, $\alpha$, $dec\_factor$) is a proper DDH tuple, which yields a Chaum Pedersen proof of discrete log equality. Verification proceeds as follows:
	
\begin{lstlisting}[language=Python]
  def verify_partial_decryption_proof(ciphertext, decryption_factor, proof, public_key):
	# Here, we prove that (g, y, ciphertext.alpha, decryption_factor) is a DDH tuple, proving knowledge of secret key x.
	# Before we were working with (g, alpha, y, beta/g^m), proving knowledge of the random factor r.
	if pow(public_key.g, proof.response, public_key.p) !=
	   ((proof.commitment.A * pow(public_key.y, proof.challenge, public_key.p)) % public_key.p):
	      return False
	
	if pow(ciphertext.alpha, proof.response, public_key.p) !=
	   ((proof.commitment.B * pow(decryption_factor, proof.challenge, public_key.p)) % public_key.p):
	      return False
	
	# compute the challenge generation, Fiat-Shamir style
	str_to_hash = str(proof.commitment.A) + "," + str(proof.commitment.B)
	computed_challenge = int_sha(str_to_hash)
	
	# check that the challenge matches
	return computed_challenge == proof.challenge
\end{lstlisting}

	Then, the decryption factors must be combined, and we check that:
    
\begin{equation}
  dec\_factor_1 * dec\_factor_2 * \dots * dec\_factor_k * m = \beta \pmod{p}
\end{equation}

	Then, the re-tally proceeds as follows.

\begin{lstlisting}[language=Python]
  def retally_election(election, voters, result, result_proof): #compute the election fingerprint
	  election_fingerprint = b64_sha(election.toJSON())
	
	  # keep track of voter fingerprints
	  vote_fingerprints = []
	
	  # keep track of running tallies, initialize at 0# again, assuming operator overloading
	  for homomorphic addition
	  tallies = [
	    [0
	      for a in question.answers
	    ]
	    for question in election.questions
	  ]
	
	  # go through each voter, check it
	  for voter in voters:
	    if not verify_vote(election, voter.vote):
	    return False
	
	  # compute fingerprint
	  vote_fingerprints.append(b64_sha(voter.vote.toJSON()))
	
	  # update tallies, looping through questions and answers within them
	  for question_num in range(len(election.questions)):
	    for choice_num in range(len(election.questions[question_num].answers)):
	    tallies[question_num][choice_num] = voter.vote.answers[question_num].choices[choice_num] +
	    tallies[question_num][choice_num]
	
	
	  # now we have tallied everything in ciphertexts, we must verify proofs
	  for question_num in range(len(election.questions)):
	    for choice_num in range(len(election.questions[question_num].answers)):
	    decryption_factor_combination = 1
	  for trustee_num in range(len(election.trustees)):
	    trustee = election.trustees[trustee_num]# verify the tally
	  for that choice within that question# check that it decrypts to the claimed result with the claimed proof
	  if not verify_partial_decryption_proof(tallies[question_num][choice_num],
	      trustee.decryption_factors[question_num][choice_num],
	      trustee.decryption_proof[question_num][choice_num],
	      trustee.public_key):
	    return False# combine the decryption factors progressively
	  decryption_factor_combination *= trustee.decryption_factors[question_num][choice_num]
	  if (decryption_factor_combination * election.result[question_num][choice_num]) % election.public_key.p != tallies[question_num][choice_num].beta % election.public_key.p:
	    return False
	
	  # return the complete tally, now that it is confirmed
	  return {
	    'election_fingerprint': election_fingerprint,
	    'vote_fingerprints': vote_fingerprints,
	    'verified_tally': result
	  }
\end{lstlisting}


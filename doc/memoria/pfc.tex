%% Para arreglar el error que da el ifpdf
%% Sacado de internet:
%% Yes, \ifpdf is defined both in the class file and in the package ifpdf that is loaded by graphics.cfg. Workaround:
%%  \RequirePackage{ifpdf}
%%  \documentclass{JHEP3}
%% Then \ifpdf will be overwritten by the class, but with a similar meaning (the class is wrong for negative numbers of \pdfoutput). But the warning of ifpdf is not triggered and the package will not be loaded again later, because the package is already loaded.
\RequirePackage{ifpdf}

%%%%%%%%%%%%%%%%%%%%%%%%%%%%%%%%%%%%%%%%%%%%%%%%%%%%%%%%%%%%%
%% HEADER
%%%%%%%%%%%%%%%%%%%%%%%%%%%%%%%%%%%%%%%%%%%%%%%%%%%%%%%%%%%%%
\documentclass[a4paper,oneside,12pt]{report}
% Alternative Options:
%	Paper Size: a4paper / a5paper / b5paper / letterpaper / legalpaper / executivepaper
% Duplex: oneside / twoside
% Base Font Size: 10pt / 11pt / 12pt
\usepackage{helvet}
\renewcommand{\familydefault}{\sfdefault}

%% Normal LaTeX or pdfLaTeX? %%%%%%%%%%%%%%%%%%%%%%%%%%%%%%%%
%% ==> The new if-Command "\ifpdf" will be used at some
%% ==> places to ensure the compatibility between
%% ==> LaTeX and pdfLaTeX.
\newif\ifpdf
\ifx\pdfoutput\undefined
	\pdffalse              %%normal LaTeX is executed
\else
	\pdfoutput=1           
	\pdftrue               %%pdfLaTeX is executed
\fi


%% Fonts for pdfLaTeX %%%%%%%%%%%%%%%%%%%%%%%%%%%%%%%%%%%%%%%
%% ==> Only needed, if cm-super-fonts are not installed
%\ifpdf
	%\usepackage{ae}       %%Use only just one of these packages:
	%\usepackage{zefonts}  %%depends on your installation.
%\else
	%%Normal LaTeX - no special packages for fonts required
%\fi


%% Language %%%%%%%%%%%%%%%%%%%%%%%%%%%%%%%%%%%%%%%%%%%%%%%%%
\usepackage[spanish]{babel}
\usepackage[T1]{fontenc}
\usepackage[latin1]{inputenc}


%% Packages for Graphics & Figures %%%%%%%%%%%%%%%%%%%%%%%%%%
\ifpdf %%Inclusion of graphics via \includegraphics{file}
	\usepackage[pdftex]{graphicx} %%graphics in pdfLaTeX
\else
	\usepackage[dvips]{graphicx} %%graphics and normal LaTeX
\fi
%\usepackage[hang,tight,raggedright]{subfigure} %%Subfigures inside a figure
%\usepackage{pst-all} %%PSTricks - not useable with pdfLaTeX


%% Math Packages %%%%%%%%%%%%%%%%%%%%%%%%%%%%%%%%%%%%%%%%%%%%
\usepackage{amsmath}
\usepackage{amsthm}
\usepackage{amsfonts}

%% Lists packages %%%%%%%%%%%%%%%%%%%%%%%%%%%%%%%%%%%%%%%%%%%
\usepackage{enumerate}
%\usepackage[shortlabels]{enumitem}


%% Line Spacing %%%%%%%%%%%%%%%%%%%%%%%%%%%%%%%%%%%%%%%%%%%%%
\usepackage{setspace}
%\singlespacing        %% 1-spacing (default)
\onehalfspacing       %% 1,5-spacing
%\doublespacing        %% 2-spacing


%% Other Packages %%%%%%%%%%%%%%%%%%%%%%%%%%%%%%%%%%%%%%%%%%%
%\usepackage{a4wide} %%Smaller margins = more text per page.
\usepackage{fancyhdr} %%Fancy headings
\usepackage{longtable} %%For tables, that exceed one page
\usepackage{rotating} %%permite rotar tablas
\usepackage{multirow} %% permite tener multilineas en una tabla

\usepackage{wrapfig}

%%\usepackage{tex4ht}
%%%%%%%%%%%%%%%%%%%%%%%%%%%%%%%%%%%%%%%%%%%%%%%%%%%%%%%%%%%%%
%% Remarks
%%%%%%%%%%%%%%%%%%%%%%%%%%%%%%%%%%%%%%%%%%%%%%%%%%%%%%%%%%%%%
%
% TODO:
% 1. Edit the used packages and their options (see above).
% 2. If you want, add a BibTeX-File to the project
%    (e.g., 'literature.bib').
% 3. Happy TeXing!
%
%%%%%%%%%%%%%%%%%%%%%%%%%%%%%%%%%%%%%%%%%%%%%%%%%%%%%%%%%%%%%

%%%%%%%%%%%%%%%%%%%%%%%%%%%%%%%%%%%%%%%%%%%%%%%%%%%%%%%%%%%%%
%% Options / Modifications
%%%%%%%%%%%%%%%%%%%%%%%%%%%%%%%%%%%%%%%%%%%%%%%%%%%%%%%%%%%%%

%\input{options} %You need a file 'options.tex' for this
%% ==> TeXnicCenter supplies some possible option files
%% ==> with its templates (File | New from Template...).

%% MARCADEAGUA
%%%%%%%%%%%%%%%%%%%%%%%%%%%%%%%%%%%%%%%%%%%%%%%%%%%%%%%%%%%%%
%% Marca de agua (mientras se escribe el documento)
%%%%%%%%%%%%%%%%%%%%%%%%%%%%%%%%%%%%%%%%%%%%%%%%%%%%%%%%%%%%%

%% C�digo para poner una marca de agua mientras escribimos el documento.
%% Recomiendo buscar el tag MARCADEAGUA por el documento para ver d�nde se usa.

%\usepackage{eso-pic}
%\newcommand\BackgroundPic{
	%\put(0,0){
		%\parbox[b][\paperheight]{\paperwidth}{
			%\vfill
			%\centering
			%\includegraphics[width=\paperwidth, height=\paperheight, keepaspectratio]{imgs/marcadeagua_endesarrollo.png}
			%\vfill
		%}
	%}
%}
%% FIN MARCADEAGUA
\usepackage{wallpaper}

%% SVN Version
\usepackage{svn-multi}
\svnidlong
{$HeadURL$}
{$LastChangedDate$}
{$LastChangedRevision$}
{$LastChangedBy$}
\svnid{$Id$}
%% Based on a TeXnicCenter-Template by Tino Weinkauf.
%%%%%%%%%%%%%%%%%%%%%%%%%%%%%%%%%%%%%%%%%%%%%%%%%%%%%%%%%%%%%


%% Links en la tabla de contenidos
%\usepackage[linktoc=all]{hyperref}
%\setcounter{secnumdepth}{0}
%\usepackage{titlesec}
%\hypersetup{
    %colorlinks,
    %citecolor=black,
    %filecolor=black,
    %linkcolor=black,
    %urlcolor=black
%}

%%%%%%%%%%%%%%%%%%%%%%%%%%%%%%%%%%%%%%%%%%%%%%%%%%%%%%%%%%%%%
%% DOCUMENT
%%%%%%%%%%%%%%%%%%%%%%%%%%%%%%%%%%%%%%%%%%%%%%%%%%%%%%%%%%%%%
\begin{document}

%% MARCADEAGUA
%\AddToShipoutPicture{\BackgroundPic}
\ULCornerWallPaper{1}{imgs/marcadeagua_endesarrollo.png}
%% FIN MARCADEAGUA

%% File Extensions of Graphics %%%%%%%%%%%%%%%%%%%%%%%%%%%%%%
%% ==> This enables you to omit the file extension of a graphic.
%% ==> "\includegraphics{title.eps}" becomes "\includegraphics{title}".
%% ==> If you create 2 graphics with same content (but different file types)
%% ==> "title.eps" and "title.pdf", only the file processable by
%% ==> your compiler will be used.
%% ==> pdfLaTeX uses "title.pdf". LaTeX uses "title.eps".
\ifpdf
	\DeclareGraphicsExtensions{.pdf,.jpg,.png}
\else
	\DeclareGraphicsExtensions{.eps}
\fi

\pagestyle{fancy} %No headings for the first pages.


%% Title Page %%%%%%%%%%%%%%%%%%%%%%%%%%%%%%%%%%%%%%%%%%%%%%%
%% ==> Write your text here or include other files.

%% The simple version:
\begin{figure}
	\centering
		\includegraphics[width=0.50\textwidth]{imgs/logo_usp_12star.jpg}
\end{figure}


\title{Aqu� va el nombre del proyecto}
\author{Jos� Carlos Jim�nez G�mez}
%\date{7 de febrero de 2012} %%If commented, the current date is used.


\maketitle
 



%% The nice version:
%%% Based on a TeXnicCenter-Template by Tino Weinkauf.
%%%%%%%%%%%%%%%%%%%%%%%%%%%%%%%%%%%%%%%%%%%%%%%%%%%%%%%%%%%%%

%%%%%%%%%%%%%%%%%%%%%%%%%%%%%%%%%%%%%%%%%%%%%%%%%%%%%%%%%%%%%
%% Deckblatt
%%%%%%%%%%%%%%%%%%%%%%%%%%%%%%%%%%%%%%%%%%%%%%%%%%%%%%%%%%%%%
%%
%% ATTENTION: You need a main file to use this one here.
%%            Use the command "\input{filename}" in your
%%            main file to include this file.
%%

%% Definici�n de autor, t�tulo y fecha para reutilizar
\author{Jos� Carlos Jim�nez G�mez}
\title{<T�TULO DEL\\PROYECTO FINAL DE CARRERA>}

\newcommand{\director}{Ra�l Garc�a Garc�a}
\newcommand{\miUniversidad}{Universidad San Pablo - CEU}
\newcommand{\miFacultad}{Escuela Polit�cnica Superior}
\newcommand{\miCarrera}{Ingenier�a en Inform�tica}

\date{\today}
\newdateformat{mydate}{\monthname[\THEMONTH] de \THEYEAR} 
%http://www.howtotex.com/packages/customize-the-date-format-in-your-latex-documents/#sthash.rmRWrHms.dpuf


\makeatletter	%% Para poder usar las definiciones anteriores. Hay que cerrar con \makeatother

\begin{titlepage}
	\begin{center}

		\Large
		\vspace{1cm}
		%\textsf{UNIVERSIDAD SAN PABLO - CEU\\}
		\textsf{\textsc{\miUniversidad}}
		\\
		%\vspace{1cm}
		\large
		%\textsf{ESCUELA POLIT�CNICA SUPERIOR\\}
		\textsf{\textsc{\miFacultad}}
		\\
		\vspace{2cm}
		%\textsf{INGENIER�A INFORM�TICA\\}
		\textsf{\textsc{\miCarrera}}
		\\
		\vspace{2cm}

		\begin{figure}[htbp]
			\centering
				\includegraphics[width=0.30\textwidth]{imgs/logoceu.jpg}
				%\includegraphics[width=0.50\textwidth]{imgs/logo_usp_12star.jpg}
			\label{fig:logoceu}
		\end{figure}
		%\begin{figure}
			%\centering
				%\includegraphics[width=0.50\textwidth]{imgs/logo_usp_12star.jpg}
		%\end{figure}

		\large
		\vspace*{2cm}
		\textsf{\textsc{Proyecto Final de Carrera}}
		\\

		\LARGE
		\vspace*{1cm}
		\textbf{\textsf{\@title}}
		\vspace{1cm}

		\large
		\textsf{Autor: \textbf{\@author}\\
		%Director: Ra�l Garc�a Garc�a}
		Director: \textbf{\director}}

		\vspace{1cm}


		%\textsf{Enero 2008}\\ %%Date - better you write it yourself.

		%\textsf{\date{\today}}\\ %%Date - better you write it yourself.

		\mydate
		\date{\today}

		\textsf{\@date}


	\end{center}

\end{titlepage}

\makeatother	%% Necesario este ''cierre'' para el \makeatletter para poder usar las referencias a title, author y date
 %%You need a file 'titlepage.tex' for this.
%% ==> TeXnicCenter supplies a possible titlepage file
%% ==> with its templates (File | New from Template...).


\svnidlong
{$HeadURL$}
{$LastChangedDate$}
{$LastChangedRevision$}
{$LastChangedBy$}
\svnid{$Id$}
%% Based on a TeXnicCenter-Template by Tino Weinkauf.
%%%%%%%%%%%%%%%%%%%%%%%%%%%%%%%%%%%%%%%%%%%%%%%%%%%%%%%%%%%%%

\chapter*{Versi�n SVN}

%%%%%%%% SVN Version
\par
Version control information :\\
Last changed by : \svnFullAuthor {\svnauthor }\\
Last changed date : \svndate \\
Last changed revision : \svnrev \\
Document URL : \svnmainurl \\
Document filename \svnmainfilename \\

%%%%%%%% Fin SV Version

%% Resumen, abstract y agradecimientos %%%%%%%%%%%%%%%%%%%%%%%%
\chapter*{Resumen}
\addcontentsline{toc}{chapter}{Resumen}

\vspace{-30pt}
El presente proyecto desarrolla una soluci�n tecnol�gica de voto por Internet con el que realizar las Elecciones a la Junta de Escuela de la Escuela Polit�cnica Superior de la Universidad San Pablo CEU, sita en Madrid, Espa�a. \\

Trata de ofrecer una prueba de concepto de un sistema seguro de voto por Internet que hace uso del nuevo \gls{DNIe} 3.0 como herramienta de identificaci�n remota del votante. \\

En el momento de la redacci�n de esta memoria, existen soluciones implementadas para este tipo de problemas, pero ning�n sistema actual permite utilizar el nuevo documento de identidad espa�ol para identificar de forma remota al votante. \\

Por tanto, podemos considerar que con este desarrollo se establece el primer sistema de voto por Internet que utiliza el \gls{DNIe} 3.0 para identificar al votante con tecnolog�a \gls{NFC}. \\

Para desarrollar el sistema se ha decidido adaptar una soluci�n ya existente, Helios Voting. Este proyecto, creado por Ben Adida y nacido en el \gls{MIT}, es considerado un est�ndar de facto en votaci�n electr�nica basada en protocolos de Verificaci�n Punto-a-Punto y en un esquema criptogr�fico homom�rfico. Es el proyecto libre m�s completo para aquellos procesos electorales con riesgo bajo de coacci�n. \\

No obstante, para poder cumplir con los objetivos del \gls{PFC}, que contiene el uso del \gls{DNIe} 3.0 como documento digital de identificaci�n de usuario, ha sido necesario realizar una integraci�n de sistemas. El proyecto Helios Voting, pese a proporcionar un gran n�mero de opciones de login, no soporta por defecto identificaci�n con certificados digitales, lo cual es b�sico para poder utilizar los que contiene el \gls{DNIe}. Por ello, ha sido necesario dise�ar un m�dulo de identificaci�n alternativo, basado en protocolo oAuth 2.0 con un servidor web configurado para aceptar estos certificados. \\

Para facilitar el uso de este documento, se ha integrado tambi�n una app de Android desarrollada por la Polic�a. Esta app requiere una adaptaci�n para las necesidades del proyecto, pero permite a los votantes usar sus dispositivos m�viles para votar utilizando el sensor \gls{NFC} de los mismos y su propio \gls{DNIe} 3.0, sin necesidad de requerir de hardware externo como los lectores de chips con contacto. Esta aproximaci�n posibilita que realmente se pueda votar desde cualquier lugar con conexi�n a Internet. \\

\chapter*{Abstract}
\addcontentsline{toc}{chapter}{Abstract}
\begin{otherlanguage}{british}
	This document shows the development of an Internet voting solution to be used to hold the Elections for the Junta de Escuela in the Escuela Polit�cnica Superior of the Universidad San Pablo CEU, placed in Madrid, Spain. \\
	
	It tries to offer a proof of concept of a secure Internet voting system which uses the new \gls{DNIe} 3.0 as a tool for the remote voter identification. \\
	
	At the moment of writing this document, there are several solutions implemented to solve this kind of problems, but there is none system which allow the use of the new Spanish ID card to identify the voter remotely. \\
	
	The goal of this project, along with the implementation of the solution, is that it will serve as a start point for future developments with voting and  the new \gls{DNIe} with \gls{NFC} chip. \\
	
	So then, we can consider that this project establishes the first Internet voting system that uses the \gls{DNIe} 3.0 to identify the voter using its \gls{NFC} chip. \\
	
	To accomplish the solution, it has been decided to adapt an existing solution, Helios Voting. This project, created by Ben Adida in the \gls{MIT}, is considered as a de-facto standard in electronic voting based in End-to-end Verifiability protocols and a homomorphic cryptography. It is the more complete free software project for low coercion risk elections. \\
	
	Neverthless, to fulfill the goals of this \gls{PFC}, which includes the use of the\gls{DNIe} as digital user identification, it has been necessary a system integration. Helios Voting, though it offers several login options, it does not support digital certificates login by default, which is basic for using the ones included in the \gls{DNIe} card. This causes the need to design an alternative identification module, based in oAuth 2.0 protocol with a web server configured to accept these digital certificates. \\
	
	To ease the use of this document, an Android app developed by Spanish Police Department has been integrated. This app has needed to be adapted in order to fulfill the requirements of the project, but it allows the voters to use their mobile devices to cast a vote using their \gls{NFC} sensor and their own \gls{DNIe} 3.0 cards, with no needs of external hardware as contact chip readers. This approach makes possible to cast votes from anywhere with an Internet connection. \\
\end{otherlanguage}
\chapter*{Agradecimientos}
\par
No quiero dejar pasar esta oportunidad sin agradecer a todas aquellas personas que en mayor o menor medida han ayudado a que este proyecto pudiera realizarse. No puedo nombrar a todos, pero si quiero reconocer espec�ficamente el valor a algunos de ellos:
\begin{itemize}
	\item A Ruth, porque sin su apoyo jam�s habr�a comenzado esta andadura universitaria, y sin su inestimable ayuda, personal y did�ctica, no habr�a podido concluirla.
	\item A mis padres, a mi hermano... Por estar ah�. Por el apoyo moral y econ�mico durante la carrera. Por lo que les ha tocado aguantarme todo este tiempo: ex�menes, trabajos y malos humores. Y al resto de mi familia, los que tambi�n han compartido estos a�os conmigo, y a los que nos han dejado pero siguen presentes.
	\item A mi director de proyecto, Gianluca Cornetta, y a mi profesor Ra�l Garc�a, por todo el tiempo, apoyo e interes que ambos han dedicado en este proyecto, y en general por mi carrera.
	\item A la Universidad San Pablo CEU, y a las Fuerzas Armadas Espa�olas, por darme la oportunidad de participar en este proyecto.
	\item A mis profesores, por lo que he aprendido de ellos, tanto en lo t�cnico y lo profesional como en lo humano. Por las lecciones magistrales que he recibido de ellos, tanto did�cticas como de la propia vida.
	\item A esos amigos que me han sabido dar una palmada en la espalda cuando me ha hecho falta. A los que me han sabido escuchar. A los que se han tomado una ca�a conmigo cuando ha hecho falta. A los ''Jedis'', a la gente del ''Sanagus'', a los asiduos del ''Boomerang'', a los del ''CIB'', a quienes tengo desperdigados por Espa�a, y en general a todos los que me han hecho pasar tan buenos momentos estos a�os.
 	\item A los compa�eros de trabajo. De Telef�nica I+D y AXPE. A los que esos �ltimos meses de proyecto me han acompa�ado y dado �nimos para acabar. Pero por supuesto a Ra�l, ya que sin �l no les habr�a conocido (ni hubiera comenzado con buen pie mis andaduras profesionales).
	\item A mis compa�eros de estudios. A los que s� que nunca perder� el contacto con ellos y a  los que hoy considero mis amigos. Al ''Gang of five''. Por todo ese tiempo que hemos compartido juntos, y la ayuda que siempre he recibido de ellos.
	\item Al Dr. Emilio Gonz�lez. Por ser un modelo de superaci�n en quien me he podido fijar. Por poder tener su tesis en mente cuando me ha costado avanzar con mi proyecto. Por hacerme con ello y otras muchas cosas recordar que, luchando, a todo se llega.
	\item A Carmen, por ese tiempo que ha invertido siempre en hacer que yo tenga m�s tiempo. Por ayudarme a que los momentos dif�ciles no lo fueran tanto.
	\item A mi buen amigo Carlos. Mi querido ''Betism�n''. Que sabe tan bien como yo que, si no hubiera tirado de m� en los malos momentos, me habr�a quedado en la cuneta y nunca habr�a podido finalizar mis estudios.
	\item Pero muy especialmente a quien me ha apoyado desde el principio hasta el final, que me ha levantado cada vez que me he ca�do y ha compartido cada momento bueno y malo a lo largo de toda la carrera. A quien ha sufrido cada d�a que me ha visto quedarme sin dormir, y pacientemente ha aguantado todos los cambios de humor. A quien sin duda debo por encima de todo el poder presentar hoy este proyecto. A mi madre.
\end{itemize}
\par
A todos ellos, y a todos los que no he podido nombrar, gracias.


%% Introducci�n %%%%%%%%%%%%%%%%%%%%%%%%%%%%%%%%%%%%%%%
\chapter*{Introducci�n}
\rhead{Introducci�n}
\par
Este proyecto trata de entrar en la problem�tica del voto electr�nico remoto y presencial, de las reticencias sociales y tecnol�gicas que influyen en su reducida implantaci�n en procesos electorales de gran importancia y alto n�mero de electores. Para ello, vamos a reproducir la situaci�n a escala reducida. Plantearemos una posible soluci�n al proceso necesario para llevar a cabo las Elecciones a la Junta de Escuela de la Escuela Polit�cnica Superior de la Universidad San Pablo - CEU.
\par
Con este planteamiento es obvio que no vamos a solucionar las trabas t�cnicas y sociales del voto por internet a nivel de unas elecciones legislativas en, por ejemplo, Espa�a. Es un tema que se escapa del objetivo de este PFC, pero s� que vamos a tratar de identificar algunos de los agentes influyentes y buscar una posible soluci�n para la elecci�n a la Junta de Escuela.
\par
As�, conseguiremos dos objetivos. Por un lado, estudiar la dificultad existente para la implantaci�n del voto electr�nico en las elecciones nacionales. Por otro, un soporte electr�nico al proceso completo de las Elecciones a la Junta de Escuela, con el cual obtendremos una mejora significativa en el mismo respecto a procesos anteriores.
\par
La forma de llegar a la soluci�n buscada debe comenzar identificando los factores que afectan a un proceso electoral general y, a continuaci�n, personalizar los que se encuentran en el que vamos a estudiar.
Una vez identificados estos agentes, definiremos las fases que comportan unas elecciones y estudiaremos c�mo podr�an ser apoyadas tecnol�gicamente, evaluando c�mo llegar al punto �ptimo de integraci�n con el sistema tradicional para mejorar el proceso.
\par
La primera fase se concentrar� en desarrollar los sistemas asociados a la fase preelectoral. En ella, se recoge el censo electoral y se identifican tanto los candidatos como los diferentes cargos que se votan.
%%OJO: �hay que hablar de la l�gica de la elecci�n? De c�mo se vota y qui�n para elegir el qu� y c�mo??
\par
La segunda fase, la electoral, la identificamos con los procesos que se requieren durante el periodo que dura la elecci�n (ya sea un d�a o varios). Esta consistir� en desarrollar los sistemas de identificaci�n y validaci�n de votantes, el sistema de votaci�n, ss


%% Indice %%%%%%%%%%%%%%%%%%%%%%%%%%%%%%%%%%%%%%%
\lhead{}
\rhead{�ndice General}
\tableofcontents %Table of contents

%% The List of Figures
\clearpage
\addcontentsline{toc}{chapter}{�ndice de Figuras}
\listoffigures

%% The List of Tables
\clearpage
\renewcommand{\tablename}{Tabla}
\renewcommand{\listtablename}{�ndice de tablas}
\addcontentsline{toc}{chapter}{�ndice de Tablas}
\listoftables

\pagestyle{fancy}
%\pagestyle{plain} %Now display headings: headings / fancy / ...



%% Chapters %%%%%%%%%%%%%%%%%%%%%%%%%%%%%%%%%%%%%%%%%%%%%%%%%
%% ==> Write your text here or include other files.

\chapter{Planteamiento}\label{planteamiento}
\lhead{Cap�tulo \ref{planteamiento}}
\rhead{Planteamiento}
%*******************************************************************************
\section{Objetivos finales del proyecto}\label{objetivos}
%*
%*
%*
%*
%*
%*

	\subsection{Descripci�n del sistema real}\label{sistemareal}

\section{Alcance del proyecto}\label{alcance}

\section{Especificaci�n de requisitos}\label{requisitos}
\begin{enumerate}
	\item Votaci�n por internet
	\item Disponibilidad 24/7
	\item Todos los requisitos del voto electr�nico
\end{enumerate}
	
\chapter{Riesgos}\label{riesgos}
\lhead{Cap�tulo \ref{riesgos}}
\rhead{Riesgos}
%*******************************************************************************
\section{Identificaci�n y gesti�n de riesgos}\label{idGestRiesgos}
%*
%*
%*
%*
%*
%*

	\subsection{Identificaci�n de riesgos}\label{identificacionRiesgos}
\chapter{Soluci�n}\label{solucion}
\lhead{Cap�tulo \ref{solucion}}
\rhead{Descripci�n detallada de la soluci�n}
%*******************************************************************************
% *************************************************************************************************************** %
% 			SOLUCI�N
% *************************************************************************************************************** %
%%%%%%%\section{Identificaci�n y gesti�n de riesgos}\label{idGestRiesgos}
%*
%*
%*
%*
%*
%*

	%%%%%%%%\subsection{Identificaci�n de riesgos}\label{identificacionRiesgos}
	\todo[inline]{He eliminado los diagramas que ten�a porque ya no se corresponden con el nuevo sistema.}
	%\begin{figure}[htbp]
		%\centering
			%\includegraphics[width=1\textwidth]{imgs/sistema.png}
		%\caption{Diagrama de flujo del Sistema}
		%\label{fig:sistema}
	%\end{figure}
	%
	%\begin{figure}[htbp]
		%\centering
			%\includegraphics[width=0.60\textwidth]{imgs/flujoUsuario.jpg}
		%\caption{Esquema del flujo que sigue el votante}
		%\label{fig:flujoUsuario}
	%\end{figure}
	%\begin{figure}[htbp]
		%\centering
			%\includegraphics[width=0.60\textwidth]{imgs/flujoSistema.jpg}
		%\caption{Esquema del flujo del Sistema}
		%\label{fig:flujoSistema}
	%\end{figure}
	
		
		\section{Dise�o}\label{dise�o}
		
			Tal como se estudia en el an�lisis (\ref{analisis}), la soluci�n que se propone para este Proyecto consiste en adaptar el sistema Helios Voting a las necesidades de la EPS.
			
			Para ello, habr� que tener en cuenta qui�nes son los actores que han de interactuar, as� como los cambios que hay que realizar para llegar a implementar el sistema en este nuevo escenario.
			
			
		
			\subsection{Dise�o del esquema de votaci�n}\label{disenhoEsquemaVoto}
				\subsubsection{Registro}
					La fase de registro de votantes en el sistema no ser� interactivo en cuanto a que no es el propio votante el que debe inscribirse para poder votar en las elecciones, sino que es la Autoridad Electoral la que lo registra en el censo. En esta fase, pues, se trata de establecer el censo de votantes que tienen autoridad para votar en el proceso electoral.
					
					Como se advierte en el an�lisis se ha tomado en consideraci�n que sean los administradores del sistema quienes tengan responsabilidad sobre el tratamiento del censo, por lo que se ha de cargar en el sistema y �ste es el que lo va a tratar.
					
					El censo ha de cargarse en dos servicios del sistema, tanto en el servidor de autenticaci�n como en el sistema de votaci�n.
					
					El censo del subsistema de autenticaci�n se utilizar� para llevar el control de los votantes que tienen derecho de acceso al sistema, por lo que a los votantes se les pueden a�adir otros usuarios necesarios para llevar a t�rmino la votaci�n, como pueden ser administradores o auditores, aunque estos no tengan derecho de voto. As� se conformar�a la base de usuarios activos del sistema.
					
					En el subsistema de voto tambi�n se vuelca el censo para cada uno de los diferentes procesos de voto que conformen la elecci�n.
					
					El procedimiento a seguir consistir� en que la Autoridad Organizadora del Proceso Electoral, la Universidad, proveer� una lista del censo a los administradores del sistema. El administrador utilizar� la funci�n de carga de votantes con la lista proporcionada para realizar la carga inicial de votantes para cada una de las subelecciones que se configuren.
					
					La lista proporcionada por la Universidad debe contener la siguiente informaci�n de cada uno de los votantes:
						\begin{itemize}
							\item Nombre y apellidos
							\item DNI
							\item Clase / grupo de votantes
							\item E-mail
						\end{itemize}
						
					La carga de los votantes a trav�s de su aplicaci�n se realiza subiendo un fichero csv con la informaci�n requerida. Este fichero se pone a disposici�n de la cola de procesos, la cual, llegado el momento volcar� cada uno de los registros en la base de datos del sistema de votaci�n.
					
					\notasDuda{Falta definir el proceso de carga de votantes para el servicio de autenticaci�n.}
					
				\subsubsection{Identificaci�n / Autenticaci�n}
					El servicio de identificaci�n es un subsistema clave en el proceso electoral. En �l recae parte de la responsabilidad de la robustez del sistema, en cuanto a que debe asegurar varios de los requisitos b�sicos que definen el voto electr�nico en concreto:
					
		\notasInfo[inline]{Esto es as� seg�n los requisitos de Fujioka, si se utilizan los de la UNEX, se puede modificar.}
		\begin{description}
			\item[Solidez:] Debe asegurar que un votante deshonesto no tenga capacidad de acceder al sistema e interrumpir la votaci�n. Es decir, que s�lo debe dar acceso a los votantes que realmente deben ingresar al sistema de votaci�n.
			
			\item[Elegibilidad:] Este requisito implica que el sistema debe controlar que ning�n votante que no tenga permitido el voto pueda votar. Aunque es el proceso de votaci�n el que debe controlar esta circunstancia cuando un usuario trata de emitir un voto, el sistema de votaci�n, de forma an�loga al requisito anterior, tambi�n debe proteger el sistema evitando el acceso a aquellos que, directamente, no tengan permisos para votar.
			
			\item[Sin duplicados:] El sistema debe evitar que un votante duplique o reemplace el voto de otro. Igualmente, aunque es el sistema de votaci�n el que debe tener mecanismos que controlen esta situaci�n, la primera barrera debe ser la servicio de autenticaci�n del votante.
			
		\end{description}

					
			 El sistema de identificaci�n del votante se apoya en el protocolo oAuth2.
			\notasInfo[inline]{Aqu� desarrollamos la implementaci�n oAuth que hemos desarrollado para este sistema}
					
					
					\begin{figure}[!ht]
						\centering
							\includegraphics[width=.95\textwidth]{imgs/flujoOAuth01.jpg}
						\caption{Flujo oAuth para servicio de autenticaci�n}
						\label{fig:flujo.oauth}
					\end{figure}
					
					
					
					
					
					
					
					
				\subsubsection{Elecci�n de candidatura}
				\subsubsection{Votaci�n}
				\subsubsection{Escrutinio}
				\subsubsection{Difusi�n de resultados}
				
			\subsection{Dise�o de la arquitectura}
				\todo[inline]{Aqu� metemos la explicaci�n de los dos servidores, el de auth y el de Helios, con sus conexiones por Internet. Podemos adelantar la implementaci�n pensando en dos raspberrys? Aqu� se habla de los dos Apaches en uno de los servidores (el de DNIe y el otro)??}
				
				\begin{figure}[!ht]
						\centering
							\includegraphics[width=.5\textwidth]{imgs/arquitectura01.jpg}
						\caption{Arquitectura}
						\label{fig:arquitecura01}
					\end{figure}
					
					\begin{figure}[!ht]
						\centering
							\includegraphics[width=.5\textwidth]{imgs/arquitectura02.jpg}
						\caption{Arquitectura nivel 2}
						\label{fig:arquitectura02}
					\end{figure}
				
				
			\subsection{Dise�o de la capa de datos}
				\todo[inline]{Al hablar de Helios ya se puso un diagrama de datos. Repetir destacando los cambios. A�adir el diagrama E-R del servidor de oAuth}
			
			\subsection{Dise�o de la red}
				\todo[inline]{Para esto s� que necesito ayuda, por el tema de Firewalls y esas cosas.}
				
			\subsection{Dise�o de la app de identificaci�n}
							
			\subsection{Dise�o de la interfaz de usuario}\label{dise�o_interfaz_usuario}
				Para la interfaz de usuario hay varias necesidades que se han debido de satisfacer.
				
				Por un lado, la imagen corporativa. Al tratarse de un proceso electoral dise�ado para una entidad, la plataforma en la que se basa deber�a mostrar inequ�vocamente la imagen de la entidad que lo organiza.
				
				Otro aspecto a tener en cuenta ser� la adaptaci�n a dispositivos m�viles, pues lo que se busca es el voto seguro desde este tipo de dispositivos, algo que no cubre con suficiencia la versi�n actual de Helios Voting.
				
				\subsubsection{Estructura de la p�gina web}
					\todo[inline]{Cambios en la interfaz original de Helios. Responsive....}
					
				\subsubsection{Estructura de la aplicaci�n m�vil}
					\todo[inline]{Interfaz de usuario de la app.}
					
				\subsubsection{Accesibilidad}
					\todo[inline]{Destacar accesibilidad}
					
				\subsubsection{Imagen corporativa}
					\todo[inline]{Como imagen de marca, bla bla bla, al no haber un logo del proceso, bla bla bla, se puede introducir uno o utilizar el de la Universidad, bla bla}
					La interfaz de Helios Voting se corresponde con un sistema neutro para realizar elecciones en la plataforma web de ejemplo que tienen publicada junto con el c�digo fuente del proyecto.
					
					Esta interfaz est� bien para el prop�sito que tienen de, por un lado, mostrar un ejemplo de funcionamiento del proyecto y, por otro, dar una herramienta funcional para la realizaci�n de peque�os procesos electorales seguros por Internet para peque�os grupos que no necesiten una implementaci�n propia.
					
					A la hora de utilizar esta herramienta para montar un sistema electoral propio, en infraestructuras de un cliente, esta interfaz no se corresponde con la que deber�a tener un proyecto serio.
					Cualquier proceso electoral suele y debe tener una imagen corporativa propia que pueda identificarse claramente con el propio proceso y con las autoridades que lo organizan, con la finalidad de que el votante identifique claramente el origen del proceso y no d� lugar a equ�vocos, adem�s de que, en cierto modo, aumentan la seguridad del mismo en el proceso.
					
					La Universidad, al no haber estar habituada a procesos electorales por Internet, no suele llevar a cabo el desarrollo de una imagen corporativa para este tipo de eventos. No obstante, s� que lo hace con otros que suele organizar, ya sean acad�micos o sociales, adem�s de que, como la mayor�a de entidades, tiene construida una imagen de marca propia.
					
					Bas�ndonos en esta imagen de marca propia de la Universidad San Pablo CEU, se ha dise�ado una gama de colores y una serie de logos e im�genes para dar una imagen corporativa al proceso electoral. Aunque se ha utilizado la interfaz base del proyecto Helios, ha habido que modificarla para hacer que se relacione la web con la Autoridad que organiza la Elecci�n.
					
					\notasInfo[inline]{Dar una muestra de colores y de logos de la elecci�n}
					
					\begin{figure}[!ht]
						\centering
							\includegraphics[width=.5\textwidth]{imgs/logo_usp_ceu.jpg}
						\caption{Logo de la Universidad San Pablo CEU en su web}
						\label{fig:logo.universidad_san_pablo_ceu}
					\end{figure}
				
				
					\notasInfo[inline]{Colores...accesibilidad!}				
				
				
				
				
				
				
				
				\todo[inline]{Desde aqu� hasta el comienzo de \ref{solucion_protocolo} hay que moverlo a An�lisis.}
				En varios de los sistemas estudiados que se han desarrollado para intentar implantar el voto electr�nico a un nivel medio, como pueden ser los mexicanos SELES \ref{seles} y SEVI \ref{sevi} o los espa�oles de V�ctor Moreno \cite{moreno07} \notasDuda{???????}o Votescript \ref{ivotingVotescript}\notasDuda{???????} se observa que se realiza una divisi�n del proceso electoral en cuatro fases (Registro, Votaci�n, Consolidaci�n de resultados y Auditor�a). En el desarrollo de este sistema vamos a identificar las mismas fases, pero con matices.
				
				As�, en una primera visi�n global del sistema, en este se definen cuatro fases:
				\begin{itemize}
					\item Preelectoral
					\item Votaci�n
					\item Consolidaci�n de resultados
					\item Postelectoral
				\end{itemize}
				
				Realmente, la mayor diferencia con las fases definidas en los esquemas anteriores se corresponden con el alcance de la primera y la �ltima fase.
				La fase Preelectoral, denominada com�nmente en los ejemplos estudiados en la Introducci�n como fase de Registro, en este sistema tiene un alcance mayor. En este proceso electoral no se requiere que el votante se registre para poder votar. El censo lo proporciona la Autoridad Electoral y se carga en el sistema. Igualmente, en los d�as previos a la jornada electoral el sistema permitir� a los votantes comprobar si est�n en el censo y qu� informaci�n contiene �ste, tanto personal - para asegurarse de que podr�n identificarse - como de permisos de cara a realizar la votaci�n.
				
				
				
				Las fase de Votaci�n tambi�n tiene un alcance diferente. En primer lugar, empieza con la identificaci�n del votante en el sistema electoral. Una vez el votante ha sido correctamente identificado por el sistema (tal como lo har�a contra los miembros de la mesa en el voto tradicional), debe recibir una boleta electr�nica que le ofrezca las opciones entre las que, por su circunscripci�n, deba elegir la que desea votar. Una vez seleccionado, es el momento en el que realmente el votante realiza la votaci�n, traspasando el voto de forma digital al sistema, a la \textit{urna digital} donde se anonimizar�n y almacenar�n hasta la fase de consolidaci�n.
				
				En la fase de consolidaci�n de resultados, el sistema se encargar� del conteo de los votos que han sido emitidos
				\todo[inline]{continuar...}
				
				La fase Postelectoral, que denominan Auditor�a, prefiero dejarla con este nombre, ya que considero que la auditor�a del sistema es una operativa que se realiza durante toda la jornada electoral, no s�lo al finalizar �sta. No obstante, es cierto que al final se llevar�n a cabo auditor�as de los resultados y el funcionamiento. Adem�s de las auditor�as llevadas a cabo por los auditores \textit{oficiales}, se va a implementar un mecanismo que permita a los propios votante auditar que su voto ha sido correctamente incluido y contado en el proceso. Esta fase postelectoral tambi�n tiene m�s pasos ... \todo[inline]{continuar con fase postelectoral}
				
			\todo[inline]{Antes de este punto hay que hacer un resumen de los diferentes esquemas de votaci�n, teniendo estos como Firma ciega, mixnets, etc...}


\iffalse
			\subsection{Protocolo}\label{solucion_protocolo}
			\textst{
				Como se ha comentado en cap�tulos anteriores, hay una multitud de soluciones propuestas para el voto telem�tico.
				
				Teniendo en cuenta el objetivo de este Proyecto Fin de Carrera, de los sistemas implementados a gran escala, a nivel nacional o regional, podemos destacar Estonia, Noruega y los cantones suizos como las tres experiencias m�s exitosas y aquellas de las que se pueden estudiar las soluciones, esquemas y protocolos utilizados. No obstante, el alcance de las mismas supera sobremanera el de este proyecto. Igualmente, muchas decisiones las toman en base a satisfacer requisitos que resultan muy importantes en su an�lisis, pero que en este trabajo no se ha considerado que tengan igual trascendencia, y viceversa, por lo que se han de tomar diferentes consideraciones frente a los mismos problemas dependiendo del impacto que suponen en cada proyecto.
				}
				Tambi�n se han presentado casos de proyectos de voto telem�tico pensados a menor escala. Entre ellos, hay muchas soluciones que, en parte, podr�an satisfacer los requisitos de este proyecto. No obstante en ninguno de ellos encontramos un protocolo que se adapte completamente a los requerimientos planteados, ya que, en alg�n momento, se analiza un elemento que los hace diferir. Por ejemplo, un proyecto ya maduro como Votescript (\ref{ivotingVotescript}) realiza un estudio acad�mico y t�cnico muy profundo acerca del voto telem�tico pero, por su propia definici�n, el modelo de identificaci�n y emisi�n del voto lo sit�an f�sicamente en centros de votaci�n. Este elemento es diferencial para este proyecto, pensado en el voto telem�tico remoto, aunque puede integrarse cuando se estudian alternativas para que aquellos votantes que, por alg�n motivo, no pueden o quieren votar por Internet de forma remota tengan la oportunidad de ejercer su derecho de sufragio desde un lugar habilitado para ello por la propia Escuela.
				\textst{
				A partir de los esquemas criptogr�ficos estudiados y con ayuda de algunos protocolos ya publicados en otros proyectos, el siguiente paso es dise�ar el protocolo de votaci�n que se adapte a las necesidades del Proyecto, cumpliendo con los requisitos y asegurando los niveles de seguridad planteados.
				
				En muchas de las soluciones estudiadas se observa que no recibe la importancia necesaria la fase de identificaci�n del votante. Los mecanismos de identificaci�n y autenticaci�n del mismo resultan laxos desde el punto de vista de la seguridad ante el fraude electoral. Por ello han sido descartadas las soluciones basadas en identificaci�n por medio de bases de datos con el t�pico protocolo de usuario/contrase�a o incluso con elementos de seguridad de una generaci�n algo posterior, como pin, patrones, captchas, operaciones aritm�ticas o m�todos similares con mayor o menor complejidad. Igualmente, se han descartado aquellos m�todos de identificaci�n que requieran la presencia f�sica del votante frente a los responsables de la mesa de votaci�n, ya que se busca el dise�o de un sistema remoto. As� descartamos protocolos de identificaci�n como los publicados por Votescript, en el que el votante acude a un centro o local de votaci�n, se identifica ante la mesa electoral y recibe un token criptogr�fico personalizado con el que se le permite ejercer el sufragio.
				
				La mayor�a de las soluciones estudiadas previamente a la realizaci�n de esta memoria centran sus esfuerzos en la fase de votaci�n. Buscan la elaboraci�n de un protocolo robusto, basado en esquemas criptogr�ficos, que permita la mayor seguridad posible al cumplimiento de los requisitos fundamentales del voto electr�nico, dotando al sistema de privacidad del votante, 
			}
				\todo[inline]{Hay que modificarlo. Es anterior a la decisi�n de utilizar Helios}
\fi				
				
				
				
				
				
				
				
				
				
				
				
				
				
				%\subsubsection{Descripci�n del sistema}\label{solucion_descripcion}
				%El sistema contar� de cinco fases, determinadas por el flujo temporal de la votaci�n.
				%Preelectoral, Identificaci�n, Votaci�n, Escrutinio y publicaci�n de resultados.
				%Adicionalmente, se tendr� en cuenta un sistema de auditor�a, de car�cter transversal a este flujo, ya que debe estar disponible durante todo el proceso de votaci�n.
				%
				%\todo[inline]{No he podido conseguir reglamentaci�n oficial de la elecci�n, as� que, b�sicamente, propongo yo las fases y la problem�tica ... esto, con palabras aqu� escrito y bien puesto }
				%
				
\iffalse
				\section{ESTO ES EL PFC}
				\todo[inline]{ESTO NO VA EN EL PFC, ES UNA EXPLICACI�N PARA TENER PRESENTE QU� ES EL PFC YU PODER DESARROLLAR LA MEMORIA EN TORNO A LA IDEA QUE TENEMOS.}
				El sistema que se propone en este PFC es un sistema integral. Busca sostener el proceso electoral desde el comienzo hasta el final del mismo. Por ello empieza en el momento mismo de definici�n del censo y no termina hasta que la publicaci�n de resultados y su auditor�a son oficializadas por el �rgano rector de la Elecci�n.
				
				La primera fase, preelectoral, es aquella previa al d�a electoral, en la cual se definen las bases en las que se rige el proceso electoral.
				
				As�, es imprescindible cumplimentar varias acciones por parte de los desarrolladores, administradores y �rgano electoral.
				
				En primer lugar, es fundamental la elaboraci�n de un censo electoral. En �ste se recogen los potenciales votantes, aquellos con derecho a voto, identificando, adem�s, la circunscripci�n \notasDuda{Cir-cuns-crip-ci�n??? No hay una forma mejor de expresarlo??} a la que pertenece. En unas elecciones legislativas, una circunscripci�n electoral se puede definir como el conjunto de electores a partir del cual se procede la distribuci�n de los esca�os asignados, en funci�n de la distribuci�n del los votos sufragados. En las elecciones legislativas espa�olas, las circunscripciones se corresponden con las provincias espa�olas (excepto en el caso de Aturias, que est� subdividida en 3 distritos electorales, y la Regi�n de Murcia, que lo hace en 5). Esto significa que del total de diputados que se eligen en este proceso para la totalidad de Espa�a, en vez de repartirlos con el recuento total de los votos, se reparten los cargos por cada circunscripci�n, dependiendo del n�mero de electores de cada una, con lo que los votantes censados en una circunscripci�n, digamos por ejemplo la provincia de M�laga, elegir�n a un n�mero determinado de diputados que ser�n quienes les representen en el Congreso junto a los elegidos en el resto de territorios espa�oles. En las Elecciones al Parlamento Europeo, sin embargo, Espa�a act�a como una �nica circunscripci�n, por lo que los diputados que representar�n al pa�s en la c�mara supranacional se obtendr�n a base de repartir los esca�os con respecto al total de votos recogidos en todo el territorio espa�ol.
				
				Algo parecido es lo que se va a definir en el censo electoral. Adem�s de recoger de forma un�voca a los electores con derecho al voto, se tendr�n que sumar las ************\notasCambio{*********} necesarias para su correcta identificaci�n, as� como la ``circunscripci�n'' a la que pertenece, es decir, el grupo sobre el que debe escoger a sus representantes, con el fin de que la opci�n de voto que el sistema le presente y la que introduzca en el sistema sea correcta.
				
				Se vislumbran aqu� dos requisitos del voto electr�nico que necesitan ser satisfechos para la integridad del proceso electoral.
				
				En primer lugar, es b�sico que el censo defina claramente los votantes con derecho al voto y provea de la informaci�n necesaria para que se pueda comprobar la identidad del votante en el momento en el que se disponga a votar. En las elecciones con voto tradicional esto se consegu�a a�adiendo datos personales tales como el n�mero del DNI, del Pasaporte o, en caso de estas elecciones, el n�mero de identificaci�n del alumno. As�, al acudir a la mesa electoral todos los votantes ten�an estos datos con los que se pod�an identificar frente a los miembros de la misma, los cuales tienen la potestad de permitirles votar o no.
				
				Integridad del voto. El hecho de relacionar cada votante con una ``circunscripci�n'' es esencial a la hora de mantener la integridad de la votaci�n, pues hay que tener en cuenta los candidatos a los que cada votante puede votar, ya que no son los mismos para todos. Igual que en unas legislativas espa�olas un votante de M�laga no elige entre los mismos candidatos que lo hace un votante de Lugo, en estas elecciones, un alumno elige sus representantes entre los delegados de curso, mientras que los profesores, por su parte, lo hacen entre otros colegas profesores. Es indispensable, pues, gestionar correctamente estas relaciones ya que no se deben recoger votos de votantes a candidatos a los que no tiene derecho a elegir.
				
				En el caso de esta elecci�n, es la propia Escuela Polit�cnica Superior la que debe proveer el censo oficial a los administradores del sistema, los cuales proceder�n a cargarlo en el mismo a trav�s de los mecanismos implementados para ello.
				\notasInfo[inline]{(Aqu� encontramos un primer punto de auditor�a importante).}
				\notasInfo[inline]{(En algunos pa�ses, en vez de elaborarse un censo oficial, son los propios votantes los que han de registrarse)}
				
				
				Es requisito de la Instituci�n que convoca el proceso electoral el definir las ``reglas del juego''. En este caso, el �rgano de la EPS encargado de la celebraci�n de las elecciones ha de definir los mecanismos de votaci�n para que el sistema se pueda adaptar y mantener .......
				\todo[inline]{continuar...}
				
				Candidatos. Es necesario que los candidatos puedan presentar su candidatura e incorporarse al sistema para que �ste pueda gestionarlos para presentarlos como opciones a los votantes determinados, adem�s de en el momento de consolidaci�n de los votos y posterior publicaci�n de resultados.
				En muchos procesos se realizan desarrollos que permiten a los partidos pol�ticos registrar sus listas electorales y/o candidatos de forma remota durante el plazo determinado que la Ley Electoral les indica. As�, los partidos inscriben a sus representantes en el proceso electoral. En el caso de esta elecci�n, debido a su car�cter tan localizado no vemos necesidad de ello y corresponde a la Escuela Polit�cnica Superior proporcionar el listado de candidatos elegible y las circunscripciones a las que se presentan.
				\todo[inline]{Para futuros desarrollos, pensando en la escalabilidad del sistema, se podr�a desarrollar este punto para que este proceso sea independiente de los �rganos electorales de la EPS}.
				
				En las elecciones tradicionales, es tambi�n necesaria la formaci�n de las mesas electorales, con la definici�n del n�mero de ellas que son necesarias y la designaci�n de los miembros que van a formar parte de ella. En una elecci�n electr�nica y remota, como la que hemos dise�ado, el concepto de mesa se puede mantener, sobre todo para poder gestionar las circunscripciones y para continuar con las estad�sticas de participaci�n tradicionales, basadas en agrupaciones y disgregaciones de mesas. Sin embargo, al transformarse en un concepto l�gico, se pierde el sentido de la designaci�n de los miembros de mesa, por lo que no ser� un punto a tener en cuenta en el proceso.
				
				
				\todo[inline]{Pasamos a la siguiente fase: Identificaci�n}
				Una vez acometidas todas las gestiones de la fase preelectoral, pasamos a la fase correspondiente al llamado D�a Electoral (aunque realmente la elecci�n en vez de en un d�a, se pueda alargar a lo largo de un per�odo de tiempo mayor).
				Tratando de emular a las elecciones tradicionales, esta fase comienza con la apertura de los colegios electorales y las mesas que los componen. En el caso digital, ser�n los miembros designados por la Junta Electoral los que, previa identificaci�n y requerimiento de sus credenciales digitales, pongan en marcha el sistema en su fase electoral. Ser� una apertura de los colegios de forma virtual, permitiendo que los votantes puedan acceder al sistema y proceder a votar.
				La fase de identificaci�n del votante es una fase realmente importante. En las elecciones de voto tradicional, el proceso normal consiste en que el votante acude a la mesa electoral y muestra a los miembros de mesa alguna identificaci�n de curso legal, respaldada por alguna instituci�n estatal reconocida y capacitada. Los miembros de la mesa electoral contrastan la identificaci�n presentada con la informaci�n recogida en el censo electoral de dicha mesa y deciden si es suficiente o no para permitir al votante introducir su voto en la urna. En el caso de las elecciones legislativas espa�olas los documentos que se pueden mostrar son DNI, pasaporte o permiso de conducir. Todos estos documentos son v�lidos para votar incluso estando caducados. Han de mostrar la fotograf�a del votante para permitir la identificaci�n por parte de los miembros de mesa, por lo que, aunque sea v�lido que est�n caducados, no se permite utilizar el resguardo de DNI en tr�mite.
				\todo[inline]{En el caso de las elecciones de la EPS, los documentos v�lidos son .-......}
				Es requisito el sustituir este sistema de identificaci�n del elector por otro en el que no sea necesaria la presencia f�sica de �ste ni de los miembros de mesa para permitir el voto, aunque manteniendo el mismo nivel de seguridad en el proceso. Aqu� se hace indispensable estudiar las opciones de identificaci�n digital que se pueden implementar para .............
				\todo[inline]{continuar}
				
				Lo ideal es disponer de documentos que contengan tokens criptogr�ficos propios que puedan ser utilizados en los diferentes procesos de identificaci�n y voto. Por ello, vamos a utilizar documentos que los disponen.
				
				As�, los documentos v�lidos para ejercer el derecho al voto ser�n el DNIe (tanto la primera versi�n como la denominada 3.0, presentada en enero de 2015) y la TUI \notasMejorar{***InfoTUI***} de la Universidad San Pablo-CEU. En sendos documentos encontramos elementos criptogr�ficos que identifican un�vocamente a su due�o. Adem�s encontramos en ellos certificados para la firma digital, que ser�n necesarios para la fase de votaci�n.
				
				El votante se identifica con su documento digital de forma remota. Es necesario que disponga de un lector de chip electr�nico conectado al dispositivo desde el que va a realizar el voto, aunque utilizando DNIe con lector de chip sin contacto, no har�a falta si se hace uso de un dispositivo con sensor de radiofrecuencia, con capacidad para leer informaci�n a trav�s de NFC.
				
				A trav�s de la app Android (o la app web), el votante accede al servicio de votaci�n por Internet. El primer paso es la identificaci�n del votante. Es la primera vez que har� uso de los certificados del DNIe. En este caso, la app leer� (con NFC o chip con contacto) el certificado de Autenticaci�n del DNIe, por el cual se asegura la identidad del votante. Con la identidad del votante verificada (por la DGP), se contrasta con el censo, para comprobar:
				\begin{itemize}
					\item Si el votante existe en el censo.
					\item Si el votante ha votado previamente.
					\item Los datos censales del votante, para comprobar circunscripci�n, mesa electoral y, por ende, ser capaz de obtener los candidatos entre los que puede escoger.
				\end{itemize}
				
				Una vez verificado el votante y comprobados sus datos censales, se procede a construir la boleta con los candidatos que entre los que le corresponde elegir bas�ndose en su circunscripci�n electoral. El sistema ha de presentar la boleta al votante y permitir que �ste marque la o las opciones que permita el sistema electoral para constituir el voto a emitir.
				
				Una vez constituido el voto (papeleta), hay que proceder a la votaci�n digital. Para ello nos basamos en cifrado y firma ciega. As�, el primer paso es que la app utiliza la clave p�blica de la Entidad Electoral para cifrar el voto. Con el voto cifrado, el votante ha de firmarlo. La firma se realiza con el certificado de Firma que posee el DNIe. As�, el votante firma un conjunto de [voto cifrado + votante], que es el paquete que se pasar� al subsistema de gesti�n del voto.
				
				Una vez el votante ha emitido el voto, el sistema le devuelve un resguardo (c�digo QR como en Estonia, un c�digo alfanum�rico, no s� todav�a) con el cual puede verificar que el voto ha sido correctamente incluido en el sistema. Adem�s, podr� verificar que el voto ha sido correctamente incluido en el escrutinio. \notasCambio[inline]{(No s� si con este resguardo debe poder llegar a la opci�n de voto elegida, todo depende de c�mo tomemos el requisito de coerci�n y qu� es lo que menos le afecta)}
				
				El votante puede votar tantas veces como desee cambiar su voto \notasInfo{As� disminuimos el riesgo de coerci�n}. Para ello, hay un protocolo por el cual cuando un votante emite su voto, todos los anteriores son anulados. \todo[inline]{Hay que definir el protocolo para la anulaci�n de votos por 'revoto'}
				
				El sistema de gesti�n del voto es el encargado de los votos sufragados durante la jornada electoral. El sistema almacena los votos firmados (voto cifrado + votante) en una \textit{urna} digital durante el tiempo que dura la jornada electoral. En caso de recibir un voto de un votante que ya previamente hab�a emitido su voto, debe ser capaz de anular los votos anteriores que �ste hubiese sufragado\notasInfo{Como digo antes, hay que definir c�mo se hace esto de anular votos emitidos}.
				Una vez que el Administrador del Proceso Electoral da por terminada la Jornada Electoral, los Miembros de la Junta Electoral utilizan sus claves para formar la clave maestra que permite dar por terminada la fase de votaci�n y comenzar con el Escrutinio.
				La primera fase del Escrutinio es que los votos firmados deben ser \textit{anonimizados}. Esto lo vamos a realizar en dos pasos. Primero, comprobamos la validez de la firma del voto firmado. Si la firma se corresponde con u voto a descartar, se elimina. Si la firma es v�lida, se extrae (abrimos el sobre donde va la info del votante y el sobre con su voto secreto) del contenido del voto firmado tanto el voto cifrado como la informaci�n asociada del votante. Por un lado, la informaci�n del votante se almacena para sacar un listado de votantes (que podr� compararse con el resultado de votantes del censo). Por otra parte, los votos cifrados pasan a otro almac�n ya sin asociaci�n con su votante. Para terminar de separar los votos de sus votantes, pasamos por un proceso anonimizador que ... \todo[inline]{Aqu� entra en juego ElGamal y sus amigos o las mixnets}.
				Una vez tenemos los votos separados de sus votantes, procedemos a la siguiente fase del escrutinio, que es la de (abrir el sobre del voto secreto) descifrar el voto. El sistema necesita la clave privada de la Entidad Electoral para descifrar los votos que, recordemos, est�n cifrados con la clave p�blica. Con esta clave privada, extraemos el contenido del voto cifrado y obtenemos cada uno de los votos en plano de las urnas digitales.
				Una vez obtenidos el conjunto de los votos en plano de cada urna digital, podemos proceder a la consolidaci�n de los votos. Se realiza el conteo de cada urna y, con los resultados obtenidos, se puede realizar la totalizaci�n para llegar al resultado final de la Elecci�n.
				
				El �ltimo paso del sistema ser� el de la Difusi�n de los Resultados. El sistema de Escrutinio (o Totalizaci�n) informa de los resultados al m�dulo de Difusi�n, el cual les aplicar� el formato necesario para cumplir con las necesidades de publicaci�n de los mismos. En el caso del Proceso Electoral asociado a este proyecto, una web y diversos listados PDFs para poder ser cotejados.

				Ser�a muy interesante que, como en Estonia, los votantes tuvieran una herramienta para poder verificar que su voto ha sido correctamente incluido y escrutado en el Proceso.
				
				Paralelamente a todo el proceso, cada subsistema ha de generar una serie de registros, ficheros logs, que puedan ser visualizados por un conjunto de auditores, observadores u otro grupo de profesionales que tengan que dar cuenta del correcto funcionamiento del Proceso y de la transparencia del mismo, as� como del �xito t�cnico del Sistema.
\fi		
				
%%\chapter{Manuales}\label{manuales}
\lhead{Cap�tulo \ref{manuales}}
\rhead{Manuales}
%*******************************************************************************
\section{Notas sobre los manuales}
\par
El objetivo de este cap�tulo es facilitar unos manuales m�nimos pero suficiente para poder instalar, configurar, utilizar y administrar el sistema, de una forma correcta pero sencilla. 
\par
Aunque podr�amos entender que los contenidos de este cap�tulo podr�an estar dentro de los apartados relativos a la construcci�n del sistema y a la implantaci�n y despliegue, se ha optado por presentarlo a parte, debido a la importancia que tiene dentro de los requisitos descritos. Dicha importancia viene descrita en el apartado \ref{alcance_final}, sobre el alcance final del proyecto, donde se indicaba la creaci�n de los manuales como el tercero de los cuatro objetivos a resolver (los dos primeros quedaron resueltos en el cap�tulo \ref{solucion}, y el cuarto se resolver� en el cap�tulo \ref{lineas_futuras}).
\par
Si se precisa ayuda, o ampliar informaci�n sobre el sistem Linux, en quien reside la aplicaci�n, ver \cite{autounix}. Puede ser muy util tanto a la hora de la instalaci�n, configuraci�n, uso, administraci�n y mantenimiento del sistema.
\section{Manual de Instalaci�n}
\subsection{Requisitos para la instalacion}\label{requisitosinst}
\subsection{Pasos previos a la instalaci�n}
\par
Para poder realizar la instalaci�n de los programas necesitaremos previamente instalar la distribuci�n correspondiente de Linux, en este caso la distribuci�n SUSE 9.3, incluyendo los siguientes paquetes:
\begin{itemize}
\item
	python: int�rprete de Python (versi�n superior a TODO:qu� versi�n?)
\item
 	python-devel: paquete necesario, pues roundup necesita "distutils". Este paquete requiere de la instalaci�n de python-tk y blt.
\end{itemize}

\subsection{Instalaci�n de Roundup Issue Tracker}
\par
A continuaci�n se muestran las distintas acciones a seguir para una correcta instalaci�n y configuraci�n del programa.
	{Operaciones a realizar en modo superusuario}
\par
A continuaci�n se muestran los pasos a seguir, que deber�n ser realizados desde la consola, habi�ndose identificado como "root":

\begin{itemize}
\item
	Elecci�n de la direcci�n para la instalaci�n (en este caso, elegiremos: 
\begin{center}
\verb|/opt/roundup/bin| 
\end{center}
\item
	Ejecutar, situ�ndonos en el directorio donde hayamos descomprimido el programa:
\begin{center}
\verb|python setup.py install --install-scripts=/opt/roundup/bin| .
\end{center}
\end{itemize}
\par
Para cualquier duda que pueda surgir sobre la instalaci�n de roundup, se recomienda encarecidamente ver \cite{roundupweb}.
\par

\subsection{Instalaci�n PostgreSQL}\label{instpost}
\par
\textbf{DECIR LAS INDICACIONES PROPIAS QUE REQUIERE POSTGRES}
\par
Para cualquier duda que pueda surgir sobre la instalaci�n de roundup, se recomienda encarecidamente ver \cite{postweb} y \cite{postgres}.
\par
\subsection{Instalaci�n Xapian}\label{xapinst}
\textbf{DECIR LAS INDICACIONES PROPIAS QUE REQUIERE ROUNDUP PARA ESTO}
\par
Cualquier aclaraci�n adicional que se necesite sobre c�mo debe hacerse la instalaci�n de Xapian, se puede obtener en \cite{xapianweb}.
\par


\section{Manual de Configuraci�n}

\par
En este apartado trataremos sobre c�mo configurar e inicializar el tracker, as� como los complementos necesarios para el funcionamiento del sistema\footnote{Tenemos que tener en cuenta que, como ya hemos comentado en el apartado \ref{decisiones}, utilizaremos el usuario ''\textbf{pfc}''.}.

\subsection{Configuraci�n de los elementos complementarios}
\subsubsection{Configuraci�n de la base de datos}

\subsection{Configuraci�n del tracker}
Para configurar el tracker, se deber�n seguir los siguientes pasos (todos ellos en modo superusuario):
\begin{enumerate}
	\item Crear directorio para incluir el tracker \footnote{NOTA: en el caso hipot�tico de que futuras verisiones consideraran que deber� configurarse m�s de un tracker, se utilizar� un �nico directorio para todos.}. 
\begin{center}
\verb|mkdir /opt/roundup/trackers|
\end{center}
	\item A�adir al path la direcci�n donde se encuentran los scripts (en nuestro caso en /opt/roundup/bin
	\item Ejecutar el siguiente comando: \verb|roundup-admin install|. Al ejecutarse esta opci�n, la aplicaci�n nos pedir� que seleccionemos la plantilla (template) y el sistema gestor de bases de datos subyacente (backend) a utilizar:
\begin {itemize}
	\item Inserte directorio base (Enter tracker home): en nuestro caso introduciremos:
\begin{center}
\verb|/opt/roundup/trackers/pfc/|.
\end{center}
	\item Seleccione plantilla (Select template):por defecto introduciremos 'classic'.
	\item Seleccione base de datos (Select backend): por defecto introduciremos 'anydbm'.
\end {itemize}
	\item Realizaremos las siguientes modificaciones en el archivo 'config.ini'\footnote{NOTA: para facilitar la configuraci�n, se adjunta en el cd de instalaci�n una versi�n del fichero config.ini que podr� tomarse tal cual para sustituir la que genera por defecto.}, cuyo sentido se explica en el apartado \ref{config}, sobre el dise�o de la aplicaci�n. (para lo cual, comprobaremos los correspondientes permisos y ajustaremos seg�n sea preciso):
\begin {itemize}
	\item \verb|[main]|
\begin {itemize}
	\item \verb|admin_email = pfc|
	\item \verb|dispatcher_email = pfc|
\end {itemize}
	\item \verb|[tracker]|
\begin {itemize}
	\item \verb|web = http://localhost:8080/pfc|\footnote{NOTA: Esta direcci�n deber� ser modificada de acuerdo a la ubicaci�n que tenga el equipo que sirva realmente la aplicaci�n. Indicar Localhost �nicamente valdr� para un escenario de pruebas en que cliente y servidor est�n en el mismo equipo}
	\item \verb|email = pfc|
\end {itemize}
	\item \verb|[mail]|
\begin {itemize}
	\item \verb|domain = localhost|
	\item \verb|host = localhost|
\end {itemize}
	
\end {itemize}
\end {enumerate}
\par
Para cualquier duda que pueda surgir sobre sobre la configuraci�n de roundup, se recomienda encarecidamente ver \cite{roundupweb}.
\par
\subsection{Inicializaci�n del tracker}\label{inicializacion}
\par
Antes que nada, hay que tener en cuenta que se tienen que verificar los permisos antes y despu�s de realizar este paso, pues al crearse nuevas carpetas puede ser que los permisos de �stas nos impidan realizar alg�n paso.
\par
La inicializaci�n del tracker debe hacerse tambi�n como superusuario. Los pasos a seguir ser�n los siguientes
\begin{enumerate}
	\item Se ejecutar� el comando:
\begin{center}
\verb|roundup-admin initialize|
\end{center}
	\item Se nos pedir� introducir el directorio base del tracker:
\begin{center}
\verb|/opt/roundup/trackers/pfc/|
\end{center}
	\item Se nos pide introducir una contrase�a de administraci�n (en este caso hemos utilizado 'pfc')
\par
\end{enumerate}


\section{Manual de Uso}
\par
Una vez configurado el programa, e inicializado el tracker, en este cap�tulo se indicar� c�mo se debe proceder a levantar el servicio. Tambi�n se realiza alguna peque�a aclaraci�n sobre los distintos manuales que se deben utilizar.
\subsection{Levantar el servicio}\label{levantar}
\par
Para arrancar el tracker y que el sistema entre en funcionamiento, se debe levantar el servicio. 
\par
Esta operaci�n se deber� realizar tambi�n en caso de que la m�quina caiga, o tras cualquier operaci�n de mantenimiento del sistema. Es decir, cada vez que ocurra un evento por el cual debamos volver a poner en funcionamiento el sistema.
\par
Para levantar el servicio, deberemos trabajar como usuario normal, y no como superusuario (como hemos hecho en la instalaci�n y en la configuraci�n). Para ello, se deben revisar los permisos en las carpetas involucradas, y comprobar que sean los pertinentes.
\par
Se ejecutar�n en la consola las siguientes instrucciones, de acuerdo a los casos expuestos en la instalaci�n y la configuraci�n:
\begin{center}
\verb|export PATH=$PATH:/opt/roundup/bin|
\end{center}
\begin{center}
\verb|roundup-server pfc=/opt/roundup/trackers/pfc|
\end{center}
\par
Por comodidad en el uso, se recomienda encarecidamente el uso del script ''levantar.sh'' que se adjunta en el CD, y contiene s�mplemente esas dos instrucciones. Para utilizar este script, habr�a que alojarlo dentro del directorio ''home/pfc''. Para levantar la aplicaci�n bastar� unicamente con situarnos en ese directorio desde la consola y ejecutar:
\begin{center}
\verb|./levantar.sh|
\end{center}
\par
\subsection{Manual de usuario}
\par
Para un correcto uso del sistema, se ven involucrados los siguientes documentos:
\begin{itemize}
	\item Documentaci�n de uso de Round-up Issue Tracking (incluida en el CD adjunto, y tambi�n disponible y actualizada en \cite{roundupweb}).
	\item Cualquier manual, normativa o documentaci�n propuesta para la utilizaci�n de este sistema, comunicaci�n en los roles m�s bajos, etc, as� como cualquier otro documento que sea considerado pertinente por las Fuerzas Armadas.
\end{itemize}
\par
Aunque el funcionamiento del sistema es muy intuitivo y sencillo, cualquier duda podr� ser aclarada en la documentaci�n del programa, por lo que esa documentaci�n debe estar disponible. \par
Igualmente, el programa no deber� entrar en producci�n hasta que los usuarios finales tengan pleno conocimiento de las normativas de uso impuestas por las Fuerzas Armadas. 
 \label{uso}
\section{Notas sobre la administraci�n y el mantenimiento}
Hablar de que existe m�s documentaci�n para los backend,selenium etc.

Hacer una aclaraci�n sobre las cosas que debe hacer el administrador. (crear usuarios... ver guia de administraci�n de roundup
 \label{notasadmin}

%%\chapter{Lineas Futuras} \label{lineas_futuras}
\lhead{Cap�tulo \ref{lineas_futuras}}
\rhead{Lineas Futuras}
\section{Aclaraciones sobre las lineas futuras}
\par
En el apartado \ref{alcance_final}, sobre el alcance final, se defin�an cuatro objetivos a alcanzar en este proyecto. El primer objetivo y el segundo (encontrar un modelo y estructura v�lidos, y encontrar una soluci�n software v�lida) quedan resueltos en el cap�tulo \ref{solucion}. El tercer objetivo (Creaci�n de manuales) queda resuelto en el cap�tulo \ref{manuales}.
\par
El cuarto objetivo planteaba la necesidad de dar todas las facilidades posibles a futuras l�neas de desarrollo. A lo largo de todo el proyecto, cada vez que se ha tomado una decisi�n, se ha tenido en cuenta este objetivo. A lo largo de este cap�tulo se dar�n indicaciones adicionales y aclaraciones para cumplir finalmente este �ltimo objetivo.
\par
De entre todas las decisiones tomadas a lo largo del proyecto, hay que destacar el empleo del dise�o dirigido por pruebas, que facilitar� cualquier modificaci�n o ampliaci�n al sistema, aprovechando que las pruebas que deber�n superarse ya est�n programadas. Para mayor informaci�n sobre este punto, ver \cite{} y los apartados \ref{metod} y \ref{pruebas}.
\par
\section{Siguientes fases de desarrollo}
\par
En el apartado \ref{fases} se indicaba que este proyecto se correspond�a con la primera fase del desarrollo de un sistema de telemedicina militar mucho m�s amplio. La principal linea futura a seguir, despu�s de este proyecto, ser� la ejuci�n de las siguientes fases. Para satisfacer el �ltimo objetivo de este proyecto, ser� necesario facilitar ese futuro trabajo.
\par
A lo largo de todo el documento, se ha podido apreciar como las decisiones siempre se han tomado teniendo en cuenta estos futuros desarrollos. En todo momento se ha mostrado qu� decisiones se han tomado para la creaci�n de este primer prototipo, y las orientaciones convenientes para las siguientes fases de desarrollo, orient�ndonos ya en el despliegue del sistema real. Por este motivo, ser� fundamental la lectura del apartado \ref{solucion} para una buena consecuci�n.
\par
Adem�s de lo comentado en el apartado \ref{solucion}, a continuaci�n (y como ayuda a futuras mejoras que se quieran a�adir para implantarse en el sistema real) se describen algunas indicaciones importantes que se deber�n tener en consideraci�n para el desarrollo de dichas fases. No obstante se quiere recordar la importancia de la lectura de todo el apartado \ref{solucion}, y de la consulta a la bibliograf�a aportada, en los casos que sea preciso.
\par
\subsection{Modelo de despliegue real}
\par
El principal planteamiento que se deber� tomar en las futuras fases dentro del desarrollo es el modelo de despliegue que se deber� seguir. 
\par
Como se expone en este documento, los futuros desarrolladores no deber�an alejarse del modelo descrito en la figura \ref{fig:sistema_distribuido}, dentro del apartado \ref{modelo}, ya que dicho modelo encaja perfectamente con el sistema real expuesto en \ref{sistemareal}.
\par
En caso de decidirse emplear alg�n otro tipo de modelo de despliegue, tampoco deber�a ser complejo, pues la aplicaci�n a utilizar seguir� siendo la misma, y no se obligar�a a replantear ninguna otra caracter�stica ajena al propio despliegue.  
\par
\subsection{Soporte para el despliegue}\label{soport}
\par
Cuando se realice el despliegue en futuras fases, deber� plantearse cambiar las m�quinas sobre las que se instalar� la aplicaci�n. En este proyecto se propuso la instalaci�n sobre PC, pero cuando en fases posteriores sea desplegado todo el sistema, podr�a resultar interesante utilizar otro tipo de equipos.
\par
La f�cil adaptaci�n del sistema a un sistema operativo Solaris, podr�a dar como una buena opci�n utilizar m�quinas SUN. Pero en cualquier caso, al ser Unix un sistema multiplataforma en general, el principal argumento para cambiar los equipos donde instalarse ser� la propia decisi�n del cliente y de los administradores.
\par
\subsection{Cambio de Backend}
\par
El cambio de esta fase a las posteriores, donde ya nos adentraremos en el sistema real, y por tanto existir� un nivel de carga mucho mayor, requerir� plantearse la aplicaci�n de este cambio.
\par
En el apartado \ref{bd}, justificabamos que, si bien para este prototipo ser�a suficiente usar AnyDBM, en futuras fases la opci�n m�s razonable ser�a la utilizaci�n de PostgreSQL. 
\par 
Para la utilizaci�n de PostgreSQL se deber�n tener en cuenta varias cosas:
\par
\begin{itemize}
	\item En el momento de la instalaci�n del sistema, se deber�n tener en cuenta las indicaciones para instalaci�n de Postgres que se facilitan en el punto \ref{instpost}.
	\item En el momento de configuraci�n, deber� replantearse dentro del fichero config.ini el apartado [rbdms]. En el apartado \ref{config} se aporta una configuraci�n que deber�a ser v�lida, pero deber� reconsiderarse seg�n las opciones que se tomen a la hora de la isntalaci�n de PostgreSQL, as� como nuevas sugerencias o indicaciones de los administradores del sistema.
	\item Como su uso ralentizar� los accesos, se deber�a plantear la inclusi�n de Xapian en el sistema (ver apartado \ref{usoxap}).
	\item Se deber� estudiar la conveniencia de que la base de datos resida o no en la misma m�quina que Roundup, si bien hay que aclarar que en principio en este proyecto no se ha contemplado ning�n motivo que pueda empujar a tomar esta decisi�n.
\end{itemize}
\par
Para ampliar informaci�n sobre PostgreSQL, ver \cite{postgres} y {postweb}.
\par
\subsection{Uso de Xapian}\label{usoxap}
\par
El inconveniente que encontr�bamos en el uso de Roundup era una menor velocidad que si us�bamos otros SGBD, como se mostraba en la tabla \ref{tab:ComparativaSGBD}. En el apartado \ref{bd} se indicaba como algo interesante la utilizaci�n de Xapian para acelerar las b�squedas.
\par
Para la �ncorporaci�n de Xapian en el sistema, se debe consultar el apartado \ref{xapinst}, donde se aportan indicaciones sosbre la instalaci�n, y se puede ampliar informaci�n en \cite{xapianweb}.
\par

\subsection{Cambio de sistema operativo}
\par
En principio, si en fases posteriores se siguen instalando los servidores en PC's, no tiene ning�n sentido el cambio de sistema operativo, ya que Linux es totalmente v�lido. �nicamente podr�a plantearse el cambio de distribuci�n, pero siempre y cuando se trate de una distribuci�n fiable (ver apartado \ref{suse}).
\par
No obstante, en caso de optar por cambiar los servidores a m�quinas Sun, como se suger�a en el apartado \ref{soport}, se recomienda plantearse la utilizaci�n de Solaris.
\par
\subsection{Mejoras Interfaz}
\par
Otro aspecto que se deber�a plantear si se mejora en las siguientes fases es la interfaz. Seg�n los requisitos planteados, en principio ser�a suficiente con la interfaz web, como se describ�a en el apartado \ref{interfaz}. Pero en caso de considerarse cualquier tipo de mejora, podr�n realizarse las siguientes mejoras (para mayor informaci�n ver la documentaci�n actualizada disponible en \cite{roundupweb}.
\par
Otras opciones como aportar una interfaz email, o crear una interfaz a medida a partir de la salida por linea de comando que aporta Rondup, pueden ser planteadas seg�n nuevos requisitos que se puedan exigir en siguientes fases. 
\par
Finalmente, se podr�an requerir en futuras fases tambi�n cambios en la propia interfaz web ya presente, los cuales podr�an hacerse f�cilmente creando nuevas css, sin necesidad de cambiar nada m�s.
\par
\subsection{Sistemas de Backup}
\par
En fases futuras, habr� que prever sistemas de backup, para impedir que la informaci�n en cada servidor pueda perderse en caso de ocurrir alguna eventualidad o catastrofe.
\par
Ser� imprescindible desarrollar un sistema que guarde la informaci�n de los tickets de cada uno de los servidores ubicados en unidades de rol 2, en una unidad de rol 3 (o incluso 4, este aspecto deber� ser determinado, en base a las normativas ya existentes para comunicaci�n entre roles 3 y 4, y a las necesidades que las Fuerzas Armadas puedan requerir llegado el momento del despliegue).
\par
Puesto que la informaci�n es gestionada por un sistema gestor de bases de datos, la informaci�n en �ste contenida ser� lo susceptible a ser resguardado. Por tanto, los sistemas de backup a dise�ar, ser�n los de las propias bases de datos.
\par
Para mayor informaci�n sobre c�mo realizar el backup, as� como cualquier otra labor de administraci�n sobre la base de datos PostgreSQL ver \cite{postgres} y {postweb}.
\par
\subsection{Sistema de log}
\par
En este proyecto no se ha contemplado la necesidad de utilizaci�n de un sistema de logging. En las siguientes fases, si se considera apropiada su utilizaci�n, deber�n cambiarse los par�metros del fichero config.ini del apartado [logging], seg�n se indica en el apartado \ref{config}.
\par
\section{Otros �mbitos de aplicaci�n}
\par
Como futuras lineas de desarrollo de este proyecto, fuera del proyecto global al que pertenece, podr�an plantearse otros ambitos de aplicaci�n donde encajar�a tambi�n este sistema, en condiciones muy similares.
\par
\begin{itemize}
	\item \textbf{Telemedicina Civil.} El sistema es �ptimo para su utilizaci�n en medicina, no solamente militar, sino tambi�n civil. La adaptaci�n de este sistema a un sistema civil, �nicamente requerir�a de un replanteamiento del modelo de despliegue, pues el funcionamiento ser�a el mismo. No solo se podr�a utilizar a nivel de lo planteado este proyecto en concreto, el sistema de eventos, sino utilizarlo integrado al sistema de fichas m�dicas (que tambi�n deber�a adaptarse) tambi�n desarrollado en distintas fases.
	\item \textbf{Sistemas dirigidos por eventos.} En general, cualquier sistema cuyo cambio de estados sea dirigido por eventos podr� ser susceptible de ser solucionado con la implantaci�n de esta misma soluci�n. Podemos destacar, a modo de ejemplo:
	
\begin{itemize}
	\item Sistema de carta electr�nica de un restaurante. La petici�n de un men�, que el plato est� listo, que el plato se sirva, que se pida la cuenta, que se pague: son eventos que hacen cambiar el estado del pedido, que en este caso ser�a el ticket.
	\item Sistemas de atenci�n al cliente y atenci�n t�cnica. La incidencia ser� el ticket en este caso, y se registrar� cada vez que se asigne a un nuevo personaje, se a�adan comentarios, haya cambios en el estado de la incidencia, etc.
	\item Tratamiento de siniestros en compa�ias aseguradoras. El concepto ser�a el mismo que el anterior, siendo cada parte de siniestro el equivalente a un ticket.
	\item Sistemas de asignaci�n de actividades en un proceso empresarial. Una actividad que va pasando de mano en mano, puede ser asignada con facilidad con un sistema como este.
\end{itemize}
\par
Esto es solo un ejemplo de las utilidades que podr�an darse a este sistema, pero hay infinidad de casos donde su utilizaci�n es la soluci�n m�s apropiada.
\end{itemize}



%%\chapter{Conclusiones}\label{conclusiones}
\lhead{Cap�tulo \ref{conclusiones}}
\rhead{Conclusiones}
%*******************************************************************************
\par
En este trabajo nos hemos centrado en encontrar una soluci�n para los roles m�s bajos de la jerarqu�a empleada en el sistema de telemedicina de las Fuerzas Armadas Espa�olas. Las soluciones aqu� propuestas se han centrado en solucionar toda la problem�tica existente en estos niveles de la jerarqu�a, y en crear un sistema que satisfaga todas las necesidades requeridas.
\par
Dentro de estos niveles se han descrito dos tipos de abstracci�n, representados por dos sistemas distintos: un consistir�a en un sistema que contenga la informaci�n de las fichas m�dicas que prev�n los est�ndares OTAN, y otro, que es el que en este proyecto estamos estudiando, que gestionar�a los eventos sucedidos en el tr�mite de las urgencias m�dicas.
\par
Puesto que los temas derivados de niveles superiores de la jerarqu�a ya se encuentran resueltos, este proyecto no ha tocado en ning�n momento ning�n aspecto relativo a los sistemas que resuelven esos problemas.
\par
Para la creaci�n del sistema planteado en este proyecto, y el buen cumplimiento de los requisitos planteados se han cumplido los siguientes objetivos: se ha dise�ado un modelo que representaba de un modo apropiado la situaci�n real que se quer�a resolver; se ha encontrado una soluci�n software capaz de resolver el problema; se han creado los manuales precisos para su correcta instalaci�, configuraci�n y uso; se ha facilitado el desarrollo para futuras lineas de desarrollo.
\par
Para obtener esos resultados no solo se han elegido muy cuidadosamente todas las opciones posibles (desde la aplicaci�n a utilizar, el sistema operativo o la base de datos, a los lenguajes de programaci�n, los modelos para el despliegue, etc), sino que adem�s se ha puesto una importancia muy especial en las pruebas, que han sido automatizadas, y se han utilizado algunas pr�cticas de metodolog�as �giles que han enriquecido mucho este trabajo. Todos estos factores han hecho que esos resultados sean muy satisfactorios.
\par
Los unicos inconvenientes que se han encontrado son los derivados a la insuficiencia de este sistema por si solo (en este caso carece de sentido, si no es correctamente integrado en las otras fases de desarrollo descritas), y a la posible dificultad que podr�a tener este sistema llegado el momento de ingresarse con los sistemas ya existentes, en futuras fases de desarrollo.
\par
Pero independientemente de estos inconvenientes, se ha encontrado una soluci�n que satisface todos los requisitos y objetivos propuestos. Adem�s, en todo momento se ha facilitado el trabajo para las futuras fases de desarrollo, como se puede ver a lo largo del documento en todas las decisiones tomadas, y de forma m�s concreta en el cap�tulo dedicado a las futuras lineas de desarrollo.
%%%\chapter{Glosario}\label{glosario}
%\lhead{Cap�tulo \ref{glosario}}
%\rhead{Glosario}

%from documentation
%\newacronym[?key-val list?]{?label ?}{?abbrv ?}{?long?}
%above is short version of this
% \newglossaryentry{?label ?}{type=\acronymtype,
% name={?abbrv ?},
% description={?long?},
% text={?abbrv ?},
% first={?long? (?abbrv ?)},
% plural={?abbrv ?\glspluralsuffix},
% firstplural={?long?\glspluralsuffix\space (?abbrv ?\glspluralsuffix)},
% ?key-val list?}



%\newglossaryentry{DNIe}
%{name={DNIe}, 
%description={Documento Nacional de Indentidad Electr�nico}
%}

\newacronym{DNIe}{DNIe}{Documento Nacional de Identidad Electr�nico}
\newacronym{CAN}{CAN}{Card Access Number}
\newacronym{PACE}{PACE}{Password Authentication Connection Establishment}
\newacronym{PIN}{PIN}{Personal Identification Number}
\newacronym{NFC}{NFC}{Near Field Communication}
\newacronym{CRL}{CRL}{Certificate Revocation List (Lista de Revocaci�n de Certificados)}
\newacronym{OCSP}{OCSP}{Online Certificate Status Protocol}
\newacronym{E2E}{E2E}{End-to-End (Punto a punto)}
\newacronym{UNED}{UNED}{Universidad Nacional de Educaci�n a Distancia}
\newacronym{UPV/EHU}{UPV/EHU}{Universidad del Pa�s Vasco / Euskal Herriko Unibertsitatea}
\newacronym{CERES-FNMT}{CERES-FNMT}{CERtificaci�n ESpa�ola - F�brica Nacional de Moneda y Timbre}
\newacronym{PFC}{PFC}{Proyecto Final de Carrera}
\newacronym{W3C}{W3C}{World Wide Web Consortium}
\newacronym{PC}{PC}{Personal Computer}





\chapter{Temp}\label{temp}
\lhead{Cap�tulo \ref{temp}}
\rhead{Temp}
%*******************************************************************************
\par
Seg�n el documento \textbf{NORMAS DE ORGANIZACI�N Y FUNCIONAMIENTO DE LA UNIVERSIDAD SAN PABLO-CEU}, en su Art�culo 9, ''Las Facultades, Escuelas y Centros integrados o adscritos son las instancias responsables de la organizaci�n de la ense�anza e investigaci�n, de acuerdo con las directrices emanadas de los �rganos superiores de la Universidad, y de los procesos acad�micos, administrativos y de gesti�n conducentes a la obtenci�n de t�tulos de car�cter oficial y validez en todo el territorio nacional, as� como de aquellas otras funciones que determinen las presentes Normas de Organizaci�n y Funcionamiento y los restantes reglamentos universitarios.''
\par
A partir de esta definici�n, en el Cap�tulo II De los �rganos acad�micos, encontramos el Art�culo 22 Tipos de �rganos, donde se establece que (1c) que las Juntas de Facultad, Escuela o Centro son �rganos colegiados. Y encontramos su definici�n en el Art�culo 31 Las Juntas de Centros, donde podemos leer que ''La Junta de Facultad, Escuela o Centro es el �rgano colegiado de gobierno del mismo, que ejerce sus funciones con vinculaci�n a los acuerdos del Patronato, Consejo de Gobierno y resoluciones del Rector.''
\par
Tambi�n podemos destacar los art�culos 32 y 33, donde se establece la composici�n y funciones de las Juntas de Facultad, Centro o Escuela.
\par
Art�culo 32
Composici�n de las Juntas
La Junta de Facultad, Escuela o Centro estar� compuesta por miembros natos y electos.
Son miembros natos: El Decano o Director, que presidir� sus reuniones; los Vicedecanos o Subdirectores, el Secretario acad�mico, que levantar� acta de sus sesiones y los Directores de los Departamentos integrados en la Facultad o Escuela.
Son miembros electos: Quienes resulten elegidos en representaci�n del profesorado y de los alumnos de acuerdo con la normativa que reglamentariamente se establezca.
\par
Art�culo 33
Funciones de las Juntas
Las competencias de la Junta de Facultad, Escuela o Centro son:
a) Colaborar con el Decano o Director en la gesti�n de la Facultad, Escuela o Centro.
b) Promover el perfeccionamiento de los planes de estudio y de la metodolog�a docente, as� como el establecimiento de nuevos t�tulos tanto propios como oficiales.
c) Participar en la programaci�n de las actividades de extensi�n universitaria.
d) Velar por la adecuada dotaci�n de los servicios necesarios para su correcto funcionamiento.
e) Cualquier otra competencia que le pueda ser atribuida en el desarrollo de estas Normas de Organizaci�n y Funcionamiento.
\par
\par
\par
\par
\par
\par
\par
\par
\par
\par
\par
\par
\par
Tipos de Voto Electr�nico
\par
\begin{itemize}
	\item Presenciales
	\begin{itemize}
		\item Urna electr�nica (Sistema DRE - Direct-Recording Electronic - sistema de registro electr�nico directo): facilita el voto a  trav�s de una pantalla t�ctil, teclado u otro dispositivo. La m�quina DRE permite la captura, almacenamiento y escrutinio de los votos.
		\item Sistema reconocedor de marca �ptica:  El votante marca su voto en una papeleta mediante un bol�grafo, por ejemplo, y la inserta en un lector o esc�ner, a trav�s del cual la m�quina autom�ticamente registra el voto para su posterior contabilizaci�n.
	\end{itemize}
	\item Remotos
	\begin{itemize}
		\item Sistema de votaci�n telem�tica a trav�s de Internet: el elector vota mediante una aplicaci�n cliente (normalmente un navegador web) que env�a el voto a trav�s de Internet al servidor donde queda almacenado.
		\item Sistema de votaci�n telem�tica a trav�s de dispositivos m�viles: el elector vota mediante una aplicaci�n cliente que env�a el voto a trav�s de una red m�vil e Internet, dependiendo del caso, al servidor donde queda almacenado.
	\end{itemize}
\end{itemize}

\par
\par
\par
\par
\par
\par
\par
\par
\par
\par
REQUISITOS DESEABLES EN LOS SISTEMAS DE VOTO ELECTR�NICO
\begin{description}
	\item[Autenticidad]: S�lo los votantes autorizados pueden votar.
	\item[Anonimato]: El voto es secreto.
	\item[Verificabilidad]: El votante puede asegurarse de que su voto se ha contado adecuadamente.
	\item[Imposibilidad de coacci�n]: El voto emitido no puede ser mostrado.
	\item[Posibilidad de emitir un voto nulo].
	\item[Fiabilidad]: el sistema debe asegurar que no se producen alteraciones de los resultados.
	\item[Auditabilidad]: se debe poder comprobar que el funcionamiento de los elementos que intervienen en el proceso es correcto.
	\item[Usabilidad]: cualquier votante debe ser capaz de emitir un voto en un tiempo razonable.
\end{description}


\par
\par
\par
\par
\par
\par
Aplicaciones
\begin{description}
	\item[Gesti�n del censo electoral]: altas, bajas, informes, solicitud, tramitaci�n del voto por correo...
	\item[Gesti�n de candidatos]: solicitud, aprobaciones, difusi�n del perfil y propaganda...
	\item[Gesti�n del proceso electoral]: apertura de urnas, cierre de urnas, descifrado, introducci�n de votos por correo, escrutinio, recuento, presentaci�n de actas electorales...
	\item[Aplicaciones de voto]: todas aquellas que mecanizan la ejecuci�n del voto, el almacenamiento y su posterior recuento.
	\item[Presentaci�n general de actas y resultados electorales].
\end{description}
\chapter{TempEleccionJuntaEscuela}\label{tempEleccionJuntaEscuela}
\lhead{Cap�tulo \ref{tempEleccionJuntaEscuela}}
\rhead{TempEleccionJuntaEscuela}
%*******************************************************************************
\par
Empezamoss

Cada categor�a de profesores (colaboradores, adjuntos, agregados y
catedr�ticos) elegir� a dos representantes de entre los profesores con
contrato a media jornada o jornada completa.
Respecto a los alumnos:
Cada titulaci�n elegir� a dos representantes de entre todos los
delegados de la titulaci�n (dos de Teleco, dos de Inform�tica, dos de
Arquitectura y dos de Ingenier�a de la Edificaci�n)

 Los profes votamos por categor�as (los agregados a los suyos, etc.).
   La diferencia es que en el censo est�n todos (jornada completa, media
   jornada y tiempo parcial) pero s�lo son elegibles de media jornada
   para arriba.

 - Las categor�as de profes son disjuntas; s�lo votas en la tuya.

 - En cuanto a los alumnos, cada grupo tiene dos delegados (delegado y
   subdelegado). Recuerda que en Arq. hay varios grupos en cada curso,
   eso hace un censo m�s amplio.


%%%%%%%%%%%%%%%%%%%%%%%%%%%%%%%%%%%%%%%%%%%%%%%%%%%%%%%%%%%%%


%%%%%%%%%%%%%%%%%%%%%%%%%%%%%%%%%%%%%%%%%%%%%%%%%%%%%%%%%%%%%
%% BIBLIOGRAPHY AND OTHER LISTS
%%%%%%%%%%%%%%%%%%%%%%%%%%%%%%%%%%%%%%%%%%%%%%%%%%%%%%%%%%%%%
%% A small distance to the other stuff in the table of contents (toc)
\addtocontents{toc}{\protect\vspace*{\baselineskip}}




%% The Bibliography
%% ==> You need a file 'literature.bib' for this.
%% ==> You need to run BibTeX for this (Project | Properties... | Uses BibTeX)
\addcontentsline{toc}{chapter}{Bibliograf�a} %'Bibliography' into toc
\nocite* %Even non-cited BibTeX-Entries will be shown.
\bibliographystyle{unsrt} %Style of Bibliography: plain / apalike / amsalpha / ...
\bibliography{biblio} %You need a file 'literature.bib' for this.
\rhead{Bibliograf�a}




%%%%%%%%%%%%%%%%%%%%%%%%%%%%%%%%%%%%%%%%%%%%%%%%%%%%%%%%%%%%%
%% APPENDICES
%%%%%%%%%%%%%%%%%%%%%%%%%%%%%%%%%%%%%%%%%%%%%%%%%%%%%%%%%%%%%

\appendix
%% ==> Write your text here or include other files.
%%\chapter{Licencia Open Source}\label{osi}
\lhead{Anexo \ref{osi}}
\rhead{Licencia Open Source}

\par
Este tipo de licencia se usa para programas inform�ticos con copyright, en casos donde: el software de dominio p�blico (esto significa sin licencia), cumple todos estos criterios siempre y cuando todo el c�digo fuente est� disponible, y est� reconocido por la Open Source Initiative (OSI) y se le permita usar la marca de la misma.  
\par
Una licencia es considerada Open Source cuando ha sido aprobada por la OSI, mediante los criterios explicados a continuaci�n.
\par
\section{Definici�n de una licencia Open Source}
\par
La Open Source Initiative utiliza los siguientes criterios para determinar si una licencia de software puede o no considerarse software abierto. Esta definici�n se basa en las Directrices de software libre de Debian, y est� escrita y adaptada primeramente por Bruce Perens. Es similar pero no igual a la definici�n de licencia de software libre.
\par
\begin{enumerate}
	\item Libre redistribuci�n: el software debe poder ser regalado o vendido libremente.
	\item C�digo fuente: el c�digo fuente debe estar incluido u obtenerse libremente.
	\item Trabajos derivados: la redistribuci�n de modificaciones debe estar permitida.
	\item Integridad del c�digo fuente del autor: las licencias pueden requerir que las modificaciones sean redistribuidas solo 	parches.
	\item Sin discriminaci�n de personas o grupos: nadie puede dejarse fuera.
	\item Sin discriminaci�n de �reas de iniciativa: los usuarios comerciales no pueden ser excluidos.
	\item Distribuci�n de la licencia: deben aplicarse los mismos derechos a todo el que reciba el programa.
	\item La licencia no debe ser espec�fica de un producto: el programa no puede licenciarse solo como parte de una distribuci�n mayor.
	\item La licencia no debe restringir otro software: la licencia no puede obligar a que alg�n otro software que sea distribuido con el software abierto deba tambi�n ser de c�digo abierto.
	\item La licencia debe ser tecnol�gicamente neutral: no debe requerirse la aceptaci�n de la licencia por medio de un acceso por clic de rat�n o de otra forma espec�fica del medio de soporte del software.
 \end{enumerate}
\par
\section{La licencia}
\par
Puede obtenerse m�s informaci�n, as� como la versi�n completa de la licencia ver \cite{osiweb}
	

%%\chapter{Licencia P�blica General (GPL)}\label{gpl}
\lhead{Anexo \ref{gpl}}
\rhead{Licencia P�blica General (GPL)}

\par
Es una licencia creada por la FSF (Free Software Foundation) orientada principalmente a proteger la libre distribuci�n, modificaci�n y uso de software. Esta licencia declara que el cualquier software cubierto por ella es software libre, con el objetivo de protegerlo de intentos de apropiaci�n que restrinjan esas libertades a los usuarios.
\par
La definici�n de esta licencia viene marcada por el respeto a una serie de derechos que la FSF considera fundamentales para el usuario de software libre. 
\par

\section{Derechos del usuario de software libre}
\par
A continuaci�n se muestran, los cuatro derechos fundamentales que definen esta licencia.
\par
\subsubsection{El derecho a utilizar}
\par
El primer derecho o libertad, el que trata sobre el derecho a utilizar software, aunque pueda sorprender, tiene mucho sentido. . Cuando una persona ''compra'' un programa de ordenador que no es software libre por lo general no dispone del derecho de utilizaci�n ilimitada: el usuario est� limitado a utilizar el programa para determinados objetivos (prohibido usar este programa de forma comercial) o en determinados sitios (prohibido usar este programa en el pa�s X y el pa�s Y) o en un n�mero determinado de m�quinas (prohibido usar este programa en m�s de una m�quina al mismo tiempo). Estas restricciones son muy habituales cuando hablamos de software privativo, y el software libre debe respetar el derecho a que �stas no existan en la utilizaci�n.
\par
\subsubsection{El derecho a entender}
\par
La segunda libertad para el usuario: el derecho a entender c�mo funcionan los programas que nos distribuyen, y a adaptarlo a nuestras necesidades. Este derecho no existe cuando hablamos de software privativo: por lo general, el software privativo se distribuye en forma de ejecutables (equivalentes a los ficheros ''.exe'' en entornos windows) sin que le acompa�e el c�digo fuente correspondiente. El c�digo fuente de un programa es su forma entendible y modificable por un programador. 
\par
\subsubsection{El derecho a distribuir}
\par
El derecho a distribuir programas de ordenador de forma gratuita o, alternativamente, cobrando algo a cambio de hacerlo. Es natural, ya que la industria del software privativo hace cont�nuos esfuerzos para intentar convencer a la sociedad de que copiar programas de ordenador es algo que no debe hacerse. La FSF entiende que el poder copiar sin necesidad de grandes recursos (con una unidad de grabaci�n basta) y la caracter�stica peculiar de que la copia no pierde calidad respecto al original no es algo malo: por el contrario, es casi lo mejor que tiene el software. Copiar programas de ordenador y distribuirlas es algo que beneficia a la sociedad. Es de sentido com�n. Realizar copias de programas privativos es algo ilegal en la mayor�a de los pa�ses. Por eso proporcionamos software libre: es perfectamente legal copiarlo. De esta forma tanto el usuario como la sociedad se benefician, y nadie sale perdiendo (la copia original no funciona peor por haber hecho una o millones de copias).
\par
Es importante un detalle: el software libre no tiene por qu� ser gratis. Es perfectamente posible distribuir software libre a cambio de dinero. As� es como pueden ganarse la vida los programadores y distribuidores. Ahora bien, eso no justifica el hecho de vulnerar los derechos de la gente que paga por obtener una copia del programa: el usuario puede distribuir sus propias copias, cobrando por ello si lo desea.
\par
\subsubsection{El derecho a mejorar}
\par
El derecho a mejorar el software y distribuir las mejoras, es tal vez el que m�s controversia genera. El usuario de software privativo no puede mejorar los programas que utiliza: aunque quisiera y supiera hacerlo, por lo general no tiene acceso al c�digo fuente. Y aunque lo tuviera (puede distribuirse el c�digo fuente y no obstante no ser software libre) ser�a ilegal modificar ese c�digo fuente.
\par
Sin embargo, el software libre siempre se distribuye con su c�digo fuente, y adem�s es totalmente legal modificarlo. Y otra cosa importante: el usuario tambi�n tiene derecho a no distribuir sus mejoras si no quiere. Una persona puede descargar o comprar software libre, introducirle mejoras, y no redistribuir ni hacer p�blicas dichas mejoras. 
\par
\section{La licencia}
\par
Puede obtenerse m�s informaci�n, y la versi�n completa de la licencia en \cite{fsfweb}, y una versi�n de la misma en castellano en \cite{gnuweb}.

%%\chapter{Contenido del CD}\label{cd}
\lhead{Anexo \ref{cd}}
\rhead{Contenido del CD}
\par
Junto con este documento, se adjunta un CD que contiene los siguientes directorios:
\par
\textbf{ACABAR DE REDACTAR ESTE APARTADO}
\par
\begin{itemize}
	\item ROUNDUP. Version Original. Bla bla bla...
	\item ROUNDUP. Version Modificada. Bla bla bla...
	\item Utilidades. los ficheros .sh 
	\item Pruebas. Esta carpeta contiene el c�digo de las pruebas autom�ticas, en formato propio de Selenium, en Java, en HTML, y en Python.
	\item Memoria. Esta carpeta incluye copia en formato digital de este documento.
\end{itemize}


%% MARCADEAGUA
%\ClearShipoutPicture
%% FIN MARCADEAGUA

\end{document}


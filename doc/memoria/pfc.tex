%% Para arreglar el error que da el ifpdf
%% Sacado de internet:
%% Yes, \ifpdf is defined both in the class file and in the package ifpdf that is loaded by graphics.cfg. Workaround:
%%  \RequirePackage{ifpdf}
%%  \documentclass{JHEP3}
%% Then \ifpdf will be overwritten by the class, but with a similar meaning (the class is wrong for negative numbers of \pdfoutput). But the warning of ifpdf is not triggered and the package will not be loaded again later, because the package is already loaded.
\RequirePackage{ifpdf}

%%%%%%%%%%%%%%%%%%%%%%%%%%%%%%%%%%%%%%%%%%%%%%%%%%%%%%%%%%%%%
%% HEADER
%%%%%%%%%%%%%%%%%%%%%%%%%%%%%%%%%%%%%%%%%%%%%%%%%%%%%%%%%%%%%
%\documentclass[a4paper,oneside,12pt,draft]{report}
\documentclass[a4paper,oneside,12pt]{report}
% Alternative Options:
%	Paper Size: a4paper / a5paper / b5paper / letterpaper / legalpaper / executivepaper
% Duplex: oneside / twoside
% Base Font Size: 10pt / 11pt / 12pt
\usepackage{helvet}
\renewcommand{\familydefault}{\sfdefault}

%% Normal LaTeX or pdfLaTeX? %%%%%%%%%%%%%%%%%%%%%%%%%%%%%%%%
%% ==> The new if-Command "\ifpdf" will be used at some
%% ==> places to ensure the compatibility between
%% ==> LaTeX and pdfLaTeX.
\newif\ifpdf
\ifx\pdfoutput\undefined
	\pdffalse              %%normal LaTeX is executed
\else
	\pdfoutput=1           
	\pdftrue               %%pdfLaTeX is executed
\fi


%% Fonts for pdfLaTeX %%%%%%%%%%%%%%%%%%%%%%%%%%%%%%%%%%%%%%%
%% ==> Only needed, if cm-super-fonts are not installed
%\ifpdf
	%\usepackage{ae}       %%Use only just one of these packages:
	%\usepackage{zefonts}  %%depends on your installation.
%\else
	%%Normal LaTeX - no special packages for fonts required
%\fi


%% Language %%%%%%%%%%%%%%%%%%%%%%%%%%%%%%%%%%%%%%%%%%%%%%%%%
\usepackage[spanish]{babel}
\usepackage[T1]{fontenc}
\usepackage[latin1]{inputenc}
\usepackage{soul}	% para tachar texto


%% Packages for Graphics & Figures %%%%%%%%%%%%%%%%%%%%%%%%%%
\ifpdf %%Inclusion of graphics via \includegraphics{file}
	\usepackage[pdftex]{graphicx} %%graphics in pdfLaTeX
\else
	\usepackage[dvips]{graphicx} %%graphics and normal LaTeX
\fi
%\usepackage[hang,tight,raggedright]{subfigure} %%Subfigures inside a figure
%\usepackage{pst-all} %%PSTricks - not useable with pdfLaTeX



% will apply to all captions
\usepackage[labelfont={footnotesize},textfont={it,footnotesize}]{caption}
% will apply to all subcaptions
\usepackage[labelfont={footnotesize},textfont={it,footnotesize},singlelinecheck=off,justification=raggedright]{subcaption}
%Command             10pt    11pt    12pt
%\tiny               5       6       6
%\scriptsize         7       8       8
%\footnotesize       8       9       10
%\small              9       10      10.95
%\normalsize         10      10.95   12
%\large              12      12      14.4
%\Large              14.4    14.4    17.28
%\LARGE              17.28   17.28   20.74
%\huge               20.74   20.74   24.88
%\Huge               24.88   24.88   24.88



%% Math Packages %%%%%%%%%%%%%%%%%%%%%%%%%%%%%%%%%%%%%%%%%%%%
\usepackage{amsmath}
\usepackage{amsthm}
\usepackage{amsfonts}

%% Lists packages %%%%%%%%%%%%%%%%%%%%%%%%%%%%%%%%%%%%%%%%%%%
%\usepackage{enumerate}
\usepackage[shortlabels]{enumitem}


%% Line Spacing %%%%%%%%%%%%%%%%%%%%%%%%%%%%%%%%%%%%%%%%%%%%%
\usepackage{setspace}
%\singlespacing        %% 1-spacing (default)
\onehalfspacing       %% 1,5-spacing
%\doublespacing        %% 2-spacing


%% Other Packages %%%%%%%%%%%%%%%%%%%%%%%%%%%%%%%%%%%%%%%%%%%
%\usepackage{a4wide} %%Smaller margins = more text per page.
\usepackage{fancyhdr} %%Fancy headings
\usepackage{longtable} %%For tables, that exceed one page
\usepackage{rotating} %%permite rotar tablas
\usepackage{multirow} %% permite tener multilineas en una tabla
\usepackage[left=3cm, right=3cm, top=3cm, bottom=3cm]{geometry}

\usepackage{array}

\usepackage{wrapfig}

\showboxdepth=5
\showboxbreadth=5

\setlength{\headheight}{16pt}

%%\usepackage{tex4ht}
%%%%%%%%%%%%%%%%%%%%%%%%%%%%%%%%%%%%%%%%%%%%%%%%%%%%%%%%%%%%%
%% Remarks
%%%%%%%%%%%%%%%%%%%%%%%%%%%%%%%%%%%%%%%%%%%%%%%%%%%%%%%%%%%%%
%
% TODO:
% 1. Edit the used packages and their options (see above).
% 2. If you want, add a BibTeX-File to the project
%    (e.g., 'literature.bib').
% 3. Happy TeXing!
%
%%%%%%%%%%%%%%%%%%%%%%%%%%%%%%%%%%%%%%%%%%%%%%%%%%%%%%%%%%%%%

%%%%%%%%%%%%%%%%%%%%%%%%%%%%%%%%%%%%%%%%%%%%%%%%%%%%%%%%%%%%%
%% Options / Modifications
%%%%%%%%%%%%%%%%%%%%%%%%%%%%%%%%%%%%%%%%%%%%%%%%%%%%%%%%%%%%%

%\input{options} %You need a file 'options.tex' for this
%% ==> TeXnicCenter supplies some possible option files
%% ==> with its templates (File | New from Template...).

%% MARCADEAGUA
%%%%%%%%%%%%%%%%%%%%%%%%%%%%%%%%%%%%%%%%%%%%%%%%%%%%%%%%%%%%%
%% Marca de agua (mientras se escribe el documento)
%%%%%%%%%%%%%%%%%%%%%%%%%%%%%%%%%%%%%%%%%%%%%%%%%%%%%%%%%%%%%
\usepackage{wallpaper}
 


%% SVN Version %%%%%%%%%%%%%%%%%%%%%%%%%%%%%%%%%%%%%%%%%%%%%%
%%%%%%%%%%%%%%%%%%%%%%%%%%%%%%%%%%%%%%%%%%%%%%%%%%%%%%%%%%%%%
%\usepackage{svn-multi}
%\svnidlong
%{$HeadURL$}
%{$LastChangedDate$}
%{$LastChangedRevision$}
%{$LastChangedBy$}
%\svnid{$Id$}
%% Based on a TeXnicCenter-Template by Tino Weinkauf.
%%%%%%%%%%%%%%%%%%%%%%%%%%%%%%%%%%%%%%%%%%%%%%%%%%%%%%%%%%%%%

%% GIT Version %%%%%%%%%%%%%%%%%%%%%%%%%%%%%%%%%%%%%%%%%%%%%%
%%%%%%%%%%%%%%%%%%%%%%%%%%%%%%%%%%%%%%%%%%%%%%%%%%%%%%%%%%%%%
\usepackage[missing=Help!,notags={No tags?},dirty=Eww!]{gitinfo2}
%%%%%%%%%%%%%%%%%%%%%%%%%%%%%%%%%%%%%%%%%%%%%%%%%%%%%%%%%%%%%


 
%% Links en la tabla de contenidos
\usepackage[linktoc=all]{hyperref}
%% Contadores para los cap�tulos(0), secciones(1), subsecciones(2), subsubsecciones(3), paragraphs(4), subparagraphs(5)...
\setcounter{secnumdepth}{5}
\setcounter{tocdepth}{5}
%\usepackage{titlesec}
\hypersetup{
    colorlinks,
    citecolor=black,
    filecolor=black,
    linkcolor=black,
    urlcolor=blue
}

\usepackage{cite}

%% Bibliograf�a en espa�ol
%% %%% Este paquete recupera el idioma indicado al llamar a babel y lo utiliza para las citas bibliogr�ficas
\usepackage{babelbib}



% Para formatear la fecha
\usepackage[nodayofweek]{datetime}


% Para el glosario
\usepackage[acronym]{glossaries}



%% Para poner notas
\usepackage{xargs}
\usepackage{xcolor}
\usepackage[colorinlistoftodos,prependcaption,textsize=tiny]{todonotes}
%\usepackage[colorinlistoftodos,prependcaption,textsize=tiny,spanish]{todonotes}
%\usepackage[colorinlistoftodos,prependcaption,textsize=tiny,spanish,obeyDraft]{todonotes}
\newcommandx{\notasDuda}[2][1=]{\todo[linecolor=red,backgroundcolor=red!25,bordercolor=red,#1]{#2}}
\newcommandx{\notasCambio}[2][1=]{\todo[linecolor=blue,backgroundcolor=blue!25,bordercolor=blue,#1]{#2}}
\newcommandx{\notasInfo}[2][1=]{\todo[linecolor=green,backgroundcolor=green!25,bordercolor=green,#1]{#2}}
\newcommandx{\notasMejorar}[2][1=]{\todo[linecolor=purple,backgroundcolor=purple!25,bordercolor=purple,#1]{#2}}
\newcommandx{\notasOculto}[2][1=]{\todo[disable,#1]{#2}}



\usepackage{tabularx}
\newcolumntype{Y}{>{\centering\arraybackslash}X}
\newcolumntype{Z}{>{\raggedright\arraybackslash}X}

%%%%%%%%%%%%%%%%%%%%%%%%%%%%%%%%%%%%%%%%%%%%%%%%%%%%%%%%%%%%%
%% DOCUMENT
%%%%%%%%%%%%%%%%%%%%%%%%%%%%%%%%%%%%%%%%%%%%%%%%%%%%%%%%%%%%%
\begin{document}

%% Por defecto, las vi�etas van con circulitos
\renewcommand\labelitemi{$\bullet$}

%% MARCADEAGUA
%\ULCornerWallPaper{1}{imgs/marcadeagua_endesarrollo.png}
%% FIN MARCADEAGUA

%% File Extensions of Graphics %%%%%%%%%%%%%%%%%%%%%%%%%%%%%%
%% ==> This enables you to omit the file extension of a graphic.
%% ==> "\includegraphics{title.eps}" becomes "\includegraphics{title}".
%% ==> If you create 2 graphics with same content (but different file types)
%% ==> "title.eps" and "title.pdf", only the file processable by
%% ==> your compiler will be used.
%% ==> pdfLaTeX uses "title.pdf". LaTeX uses "title.eps".
\ifpdf
	\DeclareGraphicsExtensions{.pdf,.jpg,.png}
\else
	\DeclareGraphicsExtensions{.eps}
\fi

\pagestyle{fancy} %No headings for the first pages
%\fancyfoot[CO, CE]{\thepage}
\fancyfoot[CO]{\thepage}


%% Title Page %%%%%%%%%%%%%%%%%%%%%%%%%%%%%%%%%%%%%%%%%%%%%%%
%% ==> Write your text here or include other files.

%% The simple version:
%\begin{figure}
	%\centering
		%\includegraphics[width=0.50\textwidth]{imgs/logo_usp_12star.jpg}
%\end{figure}
%
%
%\title{Aqu� va el nombre del proyecto}
%\author{Jos� Carlos Jim�nez G�mez}
%%\date{7 de febrero de 2012} %%If commented, the current date is used.
%
%
%\maketitle


%% The nice version:
%% Based on a TeXnicCenter-Template by Tino Weinkauf.
%%%%%%%%%%%%%%%%%%%%%%%%%%%%%%%%%%%%%%%%%%%%%%%%%%%%%%%%%%%%%

%%%%%%%%%%%%%%%%%%%%%%%%%%%%%%%%%%%%%%%%%%%%%%%%%%%%%%%%%%%%%
%% Deckblatt
%%%%%%%%%%%%%%%%%%%%%%%%%%%%%%%%%%%%%%%%%%%%%%%%%%%%%%%%%%%%%
%%
%% ATTENTION: You need a main file to use this one here.
%%            Use the command "\input{filename}" in your
%%            main file to include this file.
%%

%% Definici�n de autor, t�tulo y fecha para reutilizar
\author{Jos� Carlos Jim�nez G�mez}
\title{<T�TULO DEL\\PROYECTO FINAL DE CARRERA>}

\newcommand{\director}{Ra�l Garc�a Garc�a}
\newcommand{\miUniversidad}{Universidad San Pablo - CEU}
\newcommand{\miFacultad}{Escuela Polit�cnica Superior}
\newcommand{\miCarrera}{Ingenier�a en Inform�tica}

\date{\today}
\newdateformat{mydate}{\monthname[\THEMONTH] de \THEYEAR} 
%http://www.howtotex.com/packages/customize-the-date-format-in-your-latex-documents/#sthash.rmRWrHms.dpuf


\makeatletter	%% Para poder usar las definiciones anteriores. Hay que cerrar con \makeatother

\begin{titlepage}
	\begin{center}

		\Large
		\vspace{1cm}
		%\textsf{UNIVERSIDAD SAN PABLO - CEU\\}
		\textsf{\textsc{\miUniversidad}}
		\\
		%\vspace{1cm}
		\large
		%\textsf{ESCUELA POLIT�CNICA SUPERIOR\\}
		\textsf{\textsc{\miFacultad}}
		\\
		\vspace{2cm}
		%\textsf{INGENIER�A INFORM�TICA\\}
		\textsf{\textsc{\miCarrera}}
		\\
		\vspace{2cm}

		\begin{figure}[htbp]
			\centering
				\includegraphics[width=0.30\textwidth]{imgs/logoceu.jpg}
				%\includegraphics[width=0.50\textwidth]{imgs/logo_usp_12star.jpg}
			\label{fig:logoceu}
		\end{figure}
		%\begin{figure}
			%\centering
				%\includegraphics[width=0.50\textwidth]{imgs/logo_usp_12star.jpg}
		%\end{figure}

		\large
		\vspace*{2cm}
		\textsf{\textsc{Proyecto Final de Carrera}}
		\\

		\LARGE
		\vspace*{1cm}
		\textbf{\textsf{\@title}}
		\vspace{1cm}

		\large
		\textsf{Autor: \textbf{\@author}\\
		%Director: Ra�l Garc�a Garc�a}
		Director: \textbf{\director}}

		\vspace{1cm}


		%\textsf{Enero 2008}\\ %%Date - better you write it yourself.

		%\textsf{\date{\today}}\\ %%Date - better you write it yourself.

		\mydate
		\date{\today}

		\textsf{\@date}


	\end{center}

\end{titlepage}

\makeatother	%% Necesario este ''cierre'' para el \makeatletter para poder usar las referencias a title, author y date
 %%You need a file 'titlepage.tex' for this.
%% ==> TeXnicCenter supplies a possible titlepage file
%% ==> with its templates (File | New from Template...).

\renewcommand{\thepage}{\roman{page}}
\setcounter{page}{2}

%% P�gina temporal que muestra la revisi�n del SVN de la documentaci�n
\svnidlong
{$HeadURL$}
{$LastChangedDate$}
{$LastChangedRevision$}
{$LastChangedBy$}
\svnid{$Id$}
%% Based on a TeXnicCenter-Template by Tino Weinkauf.
%%%%%%%%%%%%%%%%%%%%%%%%%%%%%%%%%%%%%%%%%%%%%%%%%%%%%%%%%%%%%

\chapter*{Versi�n SVN}

%%%%%%%% SVN Version
\par
Version control information :\\
Last changed by : \svnFullAuthor {\svnauthor }\\
Last changed date : \svndate \\
Last changed revision : \svnrev \\
Document URL : \svnmainurl \\
Document filename \svnmainfilename \\

%%%%%%%% Fin SV Version

%% P�gina reservada para la calificaci�n del proyecto
\chapter*{Calificaci�n}
\addcontentsline{toc}{chapter}{Calificaci�n}
\begin{center}
Pagina reservada para la calificaci�n del proyecto
\end{center}


%\renewcommand{\thepage}{\roman{page}}
%\setcounter{page}{1}


%% Resumen, abstract y agradecimientos %%%%%%%%%%%%%%%%%%%%%%%%
\chapter*{Resumen}
\addcontentsline{toc}{chapter}{Resumen}

\vspace{-30pt}
El presente proyecto desarrolla una soluci�n tecnol�gica de voto por Internet con el que realizar las Elecciones a la Junta de Escuela de la Escuela Polit�cnica Superior de la Universidad San Pablo CEU, sita en Madrid, Espa�a. \\

Trata de ofrecer una prueba de concepto de un sistema seguro de voto por Internet que hace uso del nuevo \gls{DNIe} 3.0 como herramienta de identificaci�n remota del votante. \\

En el momento de la redacci�n de esta memoria, existen soluciones implementadas para este tipo de problemas, pero ning�n sistema actual permite utilizar el nuevo documento de identidad espa�ol para identificar de forma remota al votante. \\

Por tanto, podemos considerar que con este desarrollo se establece el primer sistema de voto por Internet que utiliza el \gls{DNIe} 3.0 para identificar al votante con tecnolog�a \gls{NFC}. \\

Para desarrollar el sistema se ha decidido adaptar una soluci�n ya existente, Helios Voting. Este proyecto, creado por Ben Adida y nacido en el \gls{MIT}, es considerado un est�ndar de facto en votaci�n electr�nica basada en protocolos de Verificaci�n Punto-a-Punto y en un esquema criptogr�fico homom�rfico. Es el proyecto libre m�s completo para aquellos procesos electorales con riesgo bajo de coacci�n. \\

No obstante, para poder cumplir con los objetivos del \gls{PFC}, que contiene el uso del \gls{DNIe} 3.0 como documento digital de identificaci�n de usuario, ha sido necesario realizar una integraci�n de sistemas. El proyecto Helios Voting, pese a proporcionar un gran n�mero de opciones de login, no soporta por defecto identificaci�n con certificados digitales, lo cual es b�sico para poder utilizar los que contiene el \gls{DNIe}. Por ello, ha sido necesario dise�ar un m�dulo de identificaci�n alternativo, basado en protocolo oAuth 2.0 con un servidor web configurado para aceptar estos certificados. \\

Para facilitar el uso de este documento, se ha integrado tambi�n una app de Android desarrollada por la Polic�a. Esta app requiere una adaptaci�n para las necesidades del proyecto, pero permite a los votantes usar sus dispositivos m�viles para votar utilizando el sensor \gls{NFC} de los mismos y su propio \gls{DNIe} 3.0, sin necesidad de requerir de hardware externo como los lectores de chips con contacto. Esta aproximaci�n posibilita que realmente se pueda votar desde cualquier lugar con conexi�n a Internet. \\

\chapter*{Abstract}
\addcontentsline{toc}{chapter}{Abstract}
\begin{otherlanguage}{british}
	This document shows the development of an Internet voting solution to be used to hold the Elections for the Junta de Escuela in the Escuela Polit�cnica Superior of the Universidad San Pablo CEU, placed in Madrid, Spain. \\
	
	It tries to offer a proof of concept of a secure Internet voting system which uses the new \gls{DNIe} 3.0 as a tool for the remote voter identification. \\
	
	At the moment of writing this document, there are several solutions implemented to solve this kind of problems, but there is none system which allow the use of the new Spanish ID card to identify the voter remotely. \\
	
	The goal of this project, along with the implementation of the solution, is that it will serve as a start point for future developments with voting and  the new \gls{DNIe} with \gls{NFC} chip. \\
	
	So then, we can consider that this project establishes the first Internet voting system that uses the \gls{DNIe} 3.0 to identify the voter using its \gls{NFC} chip. \\
	
	To accomplish the solution, it has been decided to adapt an existing solution, Helios Voting. This project, created by Ben Adida in the \gls{MIT}, is considered as a de-facto standard in electronic voting based in End-to-end Verifiability protocols and a homomorphic cryptography. It is the more complete free software project for low coercion risk elections. \\
	
	Neverthless, to fulfill the goals of this \gls{PFC}, which includes the use of the\gls{DNIe} as digital user identification, it has been necessary a system integration. Helios Voting, though it offers several login options, it does not support digital certificates login by default, which is basic for using the ones included in the \gls{DNIe} card. This causes the need to design an alternative identification module, based in oAuth 2.0 protocol with a web server configured to accept these digital certificates. \\
	
	To ease the use of this document, an Android app developed by Spanish Police Department has been integrated. This app has needed to be adapted in order to fulfill the requirements of the project, but it allows the voters to use their mobile devices to cast a vote using their \gls{NFC} sensor and their own \gls{DNIe} 3.0 cards, with no needs of external hardware as contact chip readers. This approach makes possible to cast votes from anywhere with an Internet connection. \\
\end{otherlanguage}
\chapter*{Agradecimientos}
\par
No quiero dejar pasar esta oportunidad sin agradecer a todas aquellas personas que en mayor o menor medida han ayudado a que este proyecto pudiera realizarse. No puedo nombrar a todos, pero si quiero reconocer espec�ficamente el valor a algunos de ellos:
\begin{itemize}
	\item A Ruth, porque sin su apoyo jam�s habr�a comenzado esta andadura universitaria, y sin su inestimable ayuda, personal y did�ctica, no habr�a podido concluirla.
	\item A mis padres, a mi hermano... Por estar ah�. Por el apoyo moral y econ�mico durante la carrera. Por lo que les ha tocado aguantarme todo este tiempo: ex�menes, trabajos y malos humores. Y al resto de mi familia, los que tambi�n han compartido estos a�os conmigo, y a los que nos han dejado pero siguen presentes.
	\item A mi director de proyecto, Gianluca Cornetta, y a mi profesor Ra�l Garc�a, por todo el tiempo, apoyo e interes que ambos han dedicado en este proyecto, y en general por mi carrera.
	\item A la Universidad San Pablo CEU, y a las Fuerzas Armadas Espa�olas, por darme la oportunidad de participar en este proyecto.
	\item A mis profesores, por lo que he aprendido de ellos, tanto en lo t�cnico y lo profesional como en lo humano. Por las lecciones magistrales que he recibido de ellos, tanto did�cticas como de la propia vida.
	\item A esos amigos que me han sabido dar una palmada en la espalda cuando me ha hecho falta. A los que me han sabido escuchar. A los que se han tomado una ca�a conmigo cuando ha hecho falta. A los ''Jedis'', a la gente del ''Sanagus'', a los asiduos del ''Boomerang'', a los del ''CIB'', a quienes tengo desperdigados por Espa�a, y en general a todos los que me han hecho pasar tan buenos momentos estos a�os.
 	\item A los compa�eros de trabajo. De Telef�nica I+D y AXPE. A los que esos �ltimos meses de proyecto me han acompa�ado y dado �nimos para acabar. Pero por supuesto a Ra�l, ya que sin �l no les habr�a conocido (ni hubiera comenzado con buen pie mis andaduras profesionales).
	\item A mis compa�eros de estudios. A los que s� que nunca perder� el contacto con ellos y a  los que hoy considero mis amigos. Al ''Gang of five''. Por todo ese tiempo que hemos compartido juntos, y la ayuda que siempre he recibido de ellos.
	\item Al Dr. Emilio Gonz�lez. Por ser un modelo de superaci�n en quien me he podido fijar. Por poder tener su tesis en mente cuando me ha costado avanzar con mi proyecto. Por hacerme con ello y otras muchas cosas recordar que, luchando, a todo se llega.
	\item A Carmen, por ese tiempo que ha invertido siempre en hacer que yo tenga m�s tiempo. Por ayudarme a que los momentos dif�ciles no lo fueran tanto.
	\item A mi buen amigo Carlos. Mi querido ''Betism�n''. Que sabe tan bien como yo que, si no hubiera tirado de m� en los malos momentos, me habr�a quedado en la cuneta y nunca habr�a podido finalizar mis estudios.
	\item Pero muy especialmente a quien me ha apoyado desde el principio hasta el final, que me ha levantado cada vez que me he ca�do y ha compartido cada momento bueno y malo a lo largo de toda la carrera. A quien ha sufrido cada d�a que me ha visto quedarme sin dormir, y pacientemente ha aguantado todos los cambios de humor. A quien sin duda debo por encima de todo el poder presentar hoy este proyecto. A mi madre.
\end{itemize}
\par
A todos ellos, y a todos los que no he podido nombrar, gracias.


%% Indice %%%%%%%%%%%%%%%%%%%%%%%%%%%%%%%%%%%%%%%
\lhead{}
\rhead{�ndice General}
\tableofcontents %Table of contents
\addcontentsline{toc}{chapter}{�ndice General}

%% The List of Figures %%%%%%%%%%%%%%%%%%%%%%%%%%
\clearpage
\listoffigures
\addcontentsline{toc}{chapter}{�ndice de Figuras}

%% The List of Tables %%%%%%%%%%%%%%%%%%%%%%%%%%%
\clearpage
\renewcommand{\tablename}{Tabla}
\renewcommand{\listtablename}{�ndice de tablas}
\listoftables
\addcontentsline{toc}{chapter}{�ndice de Tablas}

\pagestyle{fancy}
%\fancyfoot[CO, CE]{\thepage}
\fancyfoot[CO]{\thepage}

%\pagestyle{plain} %Now display headings: headings / fancy / ...

\newpage

%% Chapters %%%%%%%%%%%%%%%%%%%%%%%%%%%%%%%%%%%%%%%%%%%%%%%%%
%% ==> Write your text here or include other files.
\renewcommand{\thepage}{\arabic{page}}
%\setcounter{page}{1}

\chapter*{Introducci�n}
\rhead{Introducci�n}
\par
Este proyecto trata de entrar en la problem�tica del voto electr�nico remoto y presencial, de las reticencias sociales y tecnol�gicas que influyen en su reducida implantaci�n en procesos electorales de gran importancia y alto n�mero de electores. Para ello, vamos a reproducir la situaci�n a escala reducida. Plantearemos una posible soluci�n al proceso necesario para llevar a cabo las Elecciones a la Junta de Escuela de la Escuela Polit�cnica Superior de la Universidad San Pablo - CEU.
\par
Con este planteamiento es obvio que no vamos a solucionar las trabas t�cnicas y sociales del voto por internet a nivel de unas elecciones legislativas en, por ejemplo, Espa�a. Es un tema que se escapa del objetivo de este PFC, pero s� que vamos a tratar de identificar algunos de los agentes influyentes y buscar una posible soluci�n para la elecci�n a la Junta de Escuela.
\par
As�, conseguiremos dos objetivos. Por un lado, estudiar la dificultad existente para la implantaci�n del voto electr�nico en las elecciones nacionales. Por otro, un soporte electr�nico al proceso completo de las Elecciones a la Junta de Escuela, con el cual obtendremos una mejora significativa en el mismo respecto a procesos anteriores.
\par
La forma de llegar a la soluci�n buscada debe comenzar identificando los factores que afectan a un proceso electoral general y, a continuaci�n, personalizar los que se encuentran en el que vamos a estudiar.
Una vez identificados estos agentes, definiremos las fases que comportan unas elecciones y estudiaremos c�mo podr�an ser apoyadas tecnol�gicamente, evaluando c�mo llegar al punto �ptimo de integraci�n con el sistema tradicional para mejorar el proceso.
\par
La primera fase se concentrar� en desarrollar los sistemas asociados a la fase preelectoral. En ella, se recoge el censo electoral y se identifican tanto los candidatos como los diferentes cargos que se votan.
%%OJO: �hay que hablar de la l�gica de la elecci�n? De c�mo se vota y qui�n para elegir el qu� y c�mo??
\par
La segunda fase, la electoral, la identificamos con los procesos que se requieren durante el periodo que dura la elecci�n (ya sea un d�a o varios). Esta consistir� en desarrollar los sistemas de identificaci�n y validaci�n de votantes, el sistema de votaci�n, ss
\chapter{Estado de la cuesti�n}\label{estadoCuestion}
\lhead{Cap�tulo \ref{estadoCuestion}}
\rhead{Estado de la cuesti�n}


\setcounter{epihc}{\value{footnote}}\stepcounter{epihc}
\epigraphhead[70]{\epigraph{Las elecciones no resuelven por s� mismas los problemas, aunque son el paso previo y necesario para su soluci�n.}{\textit{Adolfo Su�rez\footnotemark[\value{epihc}]}}}
\footnotetext[\value{epihc}]{{Presidente del Gobierno de Espa�a (1976-1981) \url{https://es.wikipedia.org/wiki/Adolfo_Su\%C3\%A1rez} Cita del discurso de cierre de campa�a [14/06/1977]}}
\setcounter{footnote}{\value{epihc}}

	% *************************************************************************************************************** %
	% 			ESTADO DE LA CUESTI�N
	% *************************************************************************************************************** %
	
	
	
		% *************************************************************************************************************** %
		% 			VOTO ELECTR�NICO
		% *************************************************************************************************************** %
		\section{Voto electr�nico}\label{votoElectronico}
		Cuando se habla de voto electr�nico, una primera acepci�n del t�rmino se refiere a los procesos electorales cuyas fases pueden llevarse a cabo haciendo uso de tecnolog�as de la informaci�n. Dentro de estas fases susceptibles de ser implementadas con protocolos inform�ticos se incluyen el registro de votantes, dise�o de mapas de distritos o circunscripciones electorales, y la gesti�n, administraci�n y log�stica electoral; as� como el escrutinio provisional o definitivo, transmisi�n de resultados y difusi�n de los mismos, o el sufragio del voto en s� mismo.
	
			\begin{figure}[ht]
				\centering
					\includegraphics[width=0.80\textwidth]{imgs/categorizationVotingSystems.png}
				\caption{Categorizaci�n de los sistemas de voto}
				\label{fig:categorizationVotingSystems}
			\end{figure}

		No obstante, una definici�n m�s simple del voto electr�nico se refiere �nicamente a este �ltimo acto de votar, ya sea a trav�s Internet o simplemente utilizando sistemas que no est�n intercomunicados a trav�s de Internet, aunque s� con un servidor receptor del voto.
		
		\begin{figure}[ht]
			\centering
				\includegraphics[width=0.95\textwidth]{imgs/categorizacionEVotingTesisCodina.png}
			\caption{Tipos de e-Voting \cite{Codina:tesis,USEAC:2011}}
			\label{fig:categorizacionEVotingTesisCodina}
		\end{figure}
		
		
		\subsection{Niveles}\label{votoElectronicoNiveles}
	
			Podemos estudiar el voto electr�nico separ�ndolo en varios niveles, dependiendo de su implantaci�n en el proceso.
			
			\begin{itemize}
				\item[$\circ$] Nivel 0
				\item[$\circ$] Voto electr�nico sustitutivo
				\item[$\circ$] Voto electr�nico remoto
				\begin{itemize}
					\item[$\cdot$] Voto telem�tico en local de votaci�n
					\item[$\cdot$] Voto por Internet
				\end{itemize}
			\end{itemize}
			
			\begin{itemize}
				\item Nivel 0
					
					Es el sistema de voto tradicional, sin hacer uso de elementos electr�nicos para llevar a cabo ninguna fase del proceso. Es el sistema que se ha venido utilizando desde las primeras votaciones hasta bien entrado el siglo XX y todav�a en uso en muchos territorios del planeta.
					
				\item Voto electr�nico sustitutivo
					
					En este nivel, se sustituyen algunos procedimientos manuales o elementos utilizados en el voto tradicional por sistemas electr�nicos determinados. Lo que se intenta es que el proceso de votaci�n sea lo m�s parecido al que se ha venido llevando a cabo, pero pudiendo utilizar avances t�cnicos que mejoren el procedimiento en algunos de los puntos del mismo. As�, dependiendo de la legislaci�n, el nivel democr�tico y social y la aceptaci�n de la innovaci�n tecnol�gica, se han adoptado procesos electorales en los que se hace uso de algunos elementos tales como tarjetas magn�ticas o documento de identidad electr�nico (para identificar al votante o incluso para emitir el voto), urnas de votaci�n electr�nica que recuentan los votos de forma autom�tica (RFID, lector c�digo de barras, etc.), pantallas de votaci�n para selecci�n de candidaturas (en EEUU es una de las formas en las que se elige la opci�n a votar), sistemas de totalizaci�n y consolidaci�n de resultados (para evitar el escrutinio manual), e incluso sistemas para guiar el recuento definitivo pasados unos d�as de la jornada electoral. As� podemos encontrar muchos m�s ejemplos.
					
					Como se puede observar, todos los sistemas que se tienen en cuenta en este nivel est�n orientados a sustituir un elemento del proceso tradicional de votaci�n. Todos est�n pensados para tener una funci�n en el local electoral, ya sea para la identificaci�n del votante, emisi�n del voto, escrutinio o (en otro tipo de local electoral) recuento definitivo. Aqu� podemos observar, de paso, diferentes fases del proceso electoral, que son f�cilmente reconocibles.
					
				\item Voto electr�nico remoto
					
					En este nivel, el concepto del voto traspasa el local electoral com�n. Se trata de que el voto se transmita desde un punto de votaci�n a una ''urna remota''. Dependiendo del punto de origen, podemos dividir este grupo en dos subgrupos, uno en el que los diferentes colegios electorales est�n interconectados entre si y otro en el que el voto se emite desde cualquier punto con conexi�n a Internet.
					
					\begin{itemize}
						\item Voto telem�tico en local de votaci�n
							
							En esta primera aproximaci�n al voto telem�tico sigue pens�ndose en el sistema de voto tradicional en cuanto a que el votante ha de acudir a un local de votaci�n acondicionado para ejercer su derecho al voto. En este local, encontrar�a una serie de sistemas de identificaci�n (tanto personal frente a los miembros de mesa - como en el sistema tradicional - como telem�tico frente a una autoridad certificadora remota a trav�s de una identificaci�n digital) para superar el primer paso del proceso. Una vez cerrada la votaci�n, se conectar�an los diferentes colegios electorales para comunicar cada uno sus escrutinios y pasar los resultados para la fase de totalizaci�n.
							
						\item Voto por Internet
							
							La aproximaci�n del voto por Internet es la m�s ambiciosa en t�rminos tecnol�gicos y de seguridad. En esta, el votante puede ejercer su derecho al voto desde cualquier punto conectado a Internet, como puede ser su propia casa o el lugar en el que se encuentre de viaje. La identificaci�n del votante debe ser digital y remota. El voto emitido tiene que ser transmitido a la urna electr�nica remota que corresponda. No obstante, desde un punto de vista sociol�gico, este sistema tiene todav�a una serie de retos que debe cumplir, como es el acceso universal al proceso de votaci�n, ya que es complicado asegurar que la totalidad de la poblaci�n podr�a hacer uso de un sistema inform�tico de este tipo. Adem�s, encontramos dificultades en cuanto a fraude electoral, ataques al sistema, tolerabilidad al fallo, etc.
							
							
					\end{itemize}
				\end{itemize}
				
				El t�rmino voto electr�nico se utiliza tambi�n para referirse a los sistemas electorales que tratan de automatizar algunas fases del proceso, como son la autenticaci�n de votantes, la votaci�n y  el escrutinio de los votos y/o difusi�n de los resultados. En el nivel 3 de la clasificaci�n anterior se pueden encuadrar todos estos sistemas que, adem�s de hacer uso de tecnolog�as de la informaci�n para la automatizaci�n de estos procesos, se basen en una comunicaci�n de redes telem�ticas para interconectar votantes con mesas electorales - urnas digitales - y estas con los centros de procesamiento de resultados.
				
				Son estos sistemas los que est�n en auge para los investigadores de protocolos electr�nicos electorales. Con el aumento de la participaci�n ciudadana en Internet, en la sociedad digital, los usuarios realizan todo tipo de procesos cotidianos a trav�s de la red de forma remota, ya sea interactuar con las entidades estatales o municipales con tr�mites burocr�ticos, multas o pagando impuestos, gestionando los recursos familiares o de la empresa con el banco desde casa o el despacho, o incluso compras por Internet o consumici�n de ocio digital. Con este panorama es cuesti�n de tiempo que cierto sector de la poblaci�n demande una actualizaci�n de los procesos de votaci�n, igual�ndolos a las posibilidades de ubicuidad de los que disfrutan el resto de servicios comentados.
				He aqu� donde el voto electr�nico remoto tiene que estudiarse si es el candidato ideal para cubrir este nicho o si, por el contrario, los riesgos de seguridad y procedimiento que sus detractores le achacan realmente imposibilitar�n este cambio en un corto per�odo de tiempo.
				
				La introducci�n de este paradigma en el mundo de los procesos electorales plantea para los expertos, por tanto, un conjunto de retos muy importante, tanto desde el punto de vista tecnol�gico - sobre todo a nivel de seguridad del sistema y privacidad del votante - como a nivel social, ya que estos sistemas electr�nicos deben garantizar al votante al menos la misma confianza que la que le proporciona el sistema de voto tradicional, para que se pueda plantear aceptar el cambio.
				
				
				\subsection{Verificabilidad vs. Secreto}\label{estadocuestion.votoelectronico.verifvssecreto}
				El voto electr�nico es uno de los retos m�s importantes y complejos del mundo tecnol�gico hoy en d�a. Aunque pueda parecer que en otras �reas m�s modernas hay desarrollos mucho m�s complejos, el problema del voto electr�nico sigue teniendo factores de dif�cil resoluci�n.
				
				Uno de los retos m�s complicado de resolver para los desarrolladores de estos protocolos y sistemas es el de la \textit{dualidad Verificabilidad-Secreto}.
				
				Aqu� se enfrentan dos requisitos b�sicos de un sistema de voto electr�nico: la posibilidad de un elector de \textbf{verificar} que el voto que ha emitido ha sido correctamente incluido en el escrutinio y el derecho fundamental del \textbf{secreto} de voto del propio elector.
				
				Con el fin de visualizar el problema, planteo un ejemplo t�pico.
				

\begin{center}
\begin{tcolorbox}[breakable, enhanced, colback=white, opacityframe=.1, width=0.9\linewidth]
							Para comenzar con el ejemplo, introducimos una serie de actores:
							\begin{description}
								\item[Marta y Bego�a:] dos votantes.
								\item[Ram�n:] quiere influenciar el voto de Marta.
								\item[Alphas y Betas:] las dos opciones entre las que los votantes han de elegir en la Elecci�n.
							\end{description}
							
							Marta tiene intenci�n de votar por los Alphas, mientras que Bego�a desea votar a los Betas. Por su parte, Ram�n tiene un gran inter�s en que Marta vote por los Betas.
							
							Como hemos avanzado, en cualquier proceso electoral hay un conflicto entre la verificabilidad y el secreto. Cualquier votante querr�a verificar que el proceso de su voto ocurre correctamente, desde su inclusi�n en la urna hasta el preciso conteo en el escrutinio. De forma particular, Marta quiere verificar que su voto es apropiadamente escrutado como Alphas. No obstante, si Marta consigue suficiente informaci�n como para poder demostrar a Ram�n de que vot� por los Betas, aparece la amenaza de la compraventa de votos. Sabiendo que el voto emitido es demostrable a un tercero, antes de que Marta vote, Ram�n podr�a ofrecerle dinero u otros recursos a cambio de que su voto sea para los Betas en vez de para los Alphas. 
							
							Naturalmente, junto con la amenaza de la compraventa de votos, aparece tambi�n la de la coerci�n. Como Marta puede mostrar el contenido de su voto a Ram�n, puede ser coaccionada por �ste para que vote por Betas como �l quiere, en lugar de los Alphas, como ella pretend�a, por miedo a represalias.
							
							Lo que se debe buscar con el voto electr�nico es que Marta consiga la informaci�n suficiente para verificar personalmente que su voto a Carn�voros ha sido efectivamente emitido y escrutado como Alphas, pero que no consiga la informaci�n necesaria como para probar a Juan el contenido de su voto.
							
							Si Marta vota a Alphas y Bego�a a Betas, ambas deber�an tener la seguridad de que sus votos han sido correctamente incluidos en el escrutinio seg�n sus opciones elegidas. Las dos pueden decirle a Ram�n que han votado a Betas, con lo que Marta estar� mintiendo y Bego�a dir� la verdad, pero Ram�n no notar� la diferencia. Quiz�, al ver Ram�n que no tiene seguridad para incentivar o coartar a Marta, desista de tratar de comprar su voto (o de coartarlo) y se acabe as� con la amenaza.

					\end{tcolorbox}
			\end{center}

			Resumiendo, es muy complicado conseguir que los Sistemas de Voto sean confiables ya que presentan requisitos de seguridad que chocan entre ellos:
			\begin{itemize}
				\item El sistema debe asegurar la integridad de la elecci�n para que todos los votantes est�n convencidos de que los votos se cuentan correctamente.
				\item El sistema debe asegurar la confidencialidad de los votos para proteger la privacidad del votante, prevenir la venta de votos y para defender a los votantes de ser coartados.
			\end{itemize}
			
			La integridad del sistema es f�cil de obtener por medio de un muestreo p�blico de los votos emitidos. Pero esto destruir�a autom�ticamente la confidencialidad del voto.
			
			La confidencialidad se puede obtener a trav�s del voto secreto, pero con ello es muy dif�cil asegurar la integridad del sistema.
			
			Debido a la naturaleza de unas elecciones, la violaci�n de cualquiera de estos requisitos del voto electr�nico puede derivar en consecuencias dram�ticas.
			Por tanto, el reto del dise�o de un sistema electoral digital reside en encontrar el punto medio en el que asegurar la integridad del sistema no comprometa el derecho del votante al secreto de su voto.
			
				%************************************************************************************************************** %
				% 			Requisitos de voto electr�nico
				%************************************************************************************************************* %
				\subsection{Requisitos del voto electr�nico}\label{introduccionRequisitosVotoElectronico}
				Un dogma que deben cumplir los sistemas de voto electr�nico es la consigna de aportar al proceso al menos las mismas garant�as de seguridad que el sistema tradicional al que est� sustituyendo / complementando.
				
				El voto presencial tradicional permite un recuento de la votaci�n una vez acabado el proceso y abierta la urna que contiene los votos f�sicos. 
				
				En el siguiente nivel, lo mismo le ocurre a la ciertos sistemas de voto electr�nico que hacen uso de urnas digitales pero generan un recibo o papeleta f�sica, que almacenan tambi�n como una urna f�sica.
				
				En cuanto al �ltimo nivel, el correspondiente al voto electr�nico remoto, esto no est� tan claro, pues la mayor�a de estos sistemas no generan un resguardo f�sico de los votos electr�nicos emitidos, por lo que es complicado pensar en un recuento en caso de fallo o de duda de la autoridad electoral o del propio electorado.

					\subsubsection{Fujioka, Okamoto y Ohta}\label{introduccion.requisitosvotoelectronico.fujioka}
				\par
				Seg�n publican \textit{Fujioka, Okamoto y Ohta} \cite{Fujioka93}, un sistema de voto secreto es \textit{seguro} si cumple con los siguientes requisitos:
				\begin{description}[font=$\bullet$\ \ ]
					\item[Completitud (Completeness):] Todos los votos v�lidos son contados correctamente.
					\item[Solidez (Soundness):] Un votante deshonesto no puede interrumpir la votaci�n.
					\item[Privacidad (Privacy):] Todos los votos deben ser secretos.
					\item[Unicidad (Unreusability):] Ning�n votante puede votar dos veces.
					\item[Elegibilidad (Elegibility):] Nadie que no tenga permitido el voto puede votar.
					\item[Fiabilidad (Fairness):] Nada debe afectar la votaci�n.
					\item[Verificabilidad (Verifiability):] Nadie puede falsificar el resultado de la votaci�n.
				\end{description}
				A estos requisitos b�sicos, el equipo de Fujioka a�ade otros seis que considera importantes para la correcta implementaci�n de un sistema de voto electr�nico:
				\begin{description}[font=$\bullet$\ \ ]
					\item[Robustez (Robustness):] El sistema debe ser capaz de tolerar una cierta cantidad de condiciones de fallas, a la vez que debe ser capaz de manejar y responder a estas situaciones.
					\item[Verificabilidad Universal (Universal Verifiability):] Cualquier actor debe poder verificar el resultado de las votaciones.
					\item[Sin recibo (Receipt Freeness):] El votante no necesita una prueba del voto realizado, debe ser incapaz de probar a un tercero el contenido de su voto
					\item[Incoercebilidad (Incoercibility):] El votante no puede ser coartado por un tercero para que vote por una opci�n en concreto. Se debe asegurar la libertad del voto. 
					\item[Sin duplicados (Non-Duplication):] Nadie puede duplicar el voto de otra persona.
					\item[Participaci�n P�blica (Public Participation):] La lista de qui�nes votaron o qui�nes no lo hicieron ha de ser p�blica.
					\item[Correcci�n Privada de Errores (Private Error Correction):] El votante tiene la capacidad de probar que su voto no fue contado correctamente sin tener que revelar qu� opci�n vot�.
				\end{description}
				
				\subsubsection{Universidad de Extremadura}\label{introduccion.requisitosvotoelectronico.unex}
				A partir de esta primera definici�n de los requisitos del voto electr�nico, muchos equipos de desarrolladores o te�ricos de infraestructuras para el voto electr�nico han redactado sus propias interpretaciones, aunque suelen ser an�logas a las ofrecidas por Fujioka.
				Por ejemplo, estas son las propiedades que debe tener un sistema de voto electr�nico a trav�s de Internet, seg�n publican desde la Universidad de Extremadura \cite{estudioInfraestructutaVotoElectronicoUNEX} son las siguientes:
				\begin{description}[font=$\bullet$\ \ ]
					\item[Universal]
					Todas las personas que tienen derecho al voto deben poder hacerlo usando el sistema telem�tico.
					
					Por ello, debe cumplir con los requisitos para acceso al proceso a trav�s de Internet si es el caso.
					
					\item[Libre]
					Las personas con derecho a voto tienen la libertad para escoger si emitir su voto a trav�s del sistema por Internet o de alguna otra forma que se haya implementado para llevar a cabo el proceso. No se les puede imponer un sistema por encima de otro.
					
					Adem�s, los votantes tienen la libertad de elecci�n en la forma y contenido del voto, incluso en si desean abstenerse.
					
					Los votantes han de ser agn�sticos en cuanto a la tecnolog�a del sistema, por lo que este debe permitirles el voto sin importar el sistema operativo, navegador o dispositivo m�vil que utilicen. De la misma forma, debe superar las barreras de accesibilidad f�sica o social.
					
					\item[Directo]
					Los votantes son los encargados de realizar su voto, sin posibilidad de delegarlo en otra persona.
					
					Este requisito plantea retos en cuanto a la identificaci�n y autenticaci�n del votante, requiriendo la aplicaci�n de protocolos criptogr�ficos en el proceso.

					\item[Igual]
					Todos los votantes han de ser iguales frente al sistema de votaci�n.
					
					Este requisito admite algunas salvedades, como pueden ser votantes con necesidades especiales.
					
					\item[Secreto]
					Cada votante es la �nica persona que puede conocer el contenido de su voto. Adem�s, ning�n voto debe poder ser asociado al votante que lo ha emitido.
					
				\end{description}
				
Hasta aqu� est�n las caracter�sticas inherentes a un sistema de votaci�n tradicional, el realizado hasta ahora en cualquier proceso electoral que haya habido en Espa�a, por ejemplo. A continuaci�n, a�aden una serie de propiedades ligadas al voto electr�nico:
				\begin{description}[font=$\bullet$\ \ ]
					\item[Autentificaci�n]
					S�lo se permite el voto a los votantes que hayan sido correctamente autorizados.
					
					En caso de procesos con votaci�n mixta, por Internet y de forma presencial, el censo electr�nico ha de ir actualiz�ndose para contener los votantes de ambas posibilidades con el fin de evitar que un votante vote en m�s de una ocasi�n, como podr�a ser presencialmente y por Internet.
					
					\item[Unicidad]
					Un votante con derecho al voto s�lo debe poder votar una �nica vez.
					
					Para esto es esencial tener un censo electr�nico actualizado con los votantes que emiten su voto tanto presencialmente como por Internet, o cualquier otro canal que se proporcione.
					
					\item[Integridad]
					Todo voto, una vez se ha registrado en el sistema, no puede ser modificado o eliminado de �ste.
					
					El sistema debe aportar herramientas de seguridad de los datos, como \textit{backups}, as� como para realizar y almacenar auditor�as que corroboren la invariabilidad de los votos emitidos.
					
					\item[Confidencialidad]
					El contenido de un voto registrado en el sistema s�lo lo conoce el votante que lo emiti�. Ninguna entidad puede llegar a averiguar el contenido de alg�n voto.
					
					\item[Fiabilidad]
					Una vez un voto es almacenado en el sistema, �ste debe asegurar que no puede perderlo. Por ello el sistema debe ser tolerante a todo tipo de fallos en dispositivos y conexiones.
					
					\item[Flexibilidad]
					El sistema debe permitir que los votantes puedan emitir su voto a trav�s de Internet con independencia tecnol�gica respecto a sistema operativo, dispositivo m�vil, navegador web, etc.
					
					\item[Comodidad]
					El sistema debe proporcionar a los votantes una buena experiencia de usuario a la hora de votar, con independencia de las habilidades o conocimientos t�cnicos que posean.

					\item[Ergonom�a]
					El sistema debe proporcionar ayudas t�cnicas a los votantes para facilitarles el proceso de voto. Esto incluye ayuda interactiva y gesti�n de errores.
					
				\end{description}
				
				
				\subsubsection{Bokslag y Vries}\label{articuloEvaluacionVotoElectronicoHolandes}
				Otro ejemplo que indica que, pese a las variaciones, hay requisitos fundamentales reconocidos por la mayor�a de los autores, nos lleva a una visi�n muy actual de estos requisitos inherentes al voto electr�nico. En el art�culo \cite{DBLP:journals/corr/BokslagV16}, Bokslag y Vries, dos autores holandeses, profundizan acerca del voto electr�nico y por Internet en este pa�s. Este art�culo enumera una serie de principios fundamentales para cualquier soluci�n de e-voting bas�ndose en los acordados por el Consejo de Europa\cite{ministers2005legal}.Seg�n estos principios, se debe asegurar que el \textbf{sufragio} sea:
				\begin{description}[font=$\bullet$\ \ ]
					\item[Universal:] Cualquier persona tiene derecho a votar y a postularse para la Elecci�n (entendiendo la existencia de l�mites basados en condiciones como la edad o la nacionalidad).
					\item[Igualitario:] Cada votante puede emitir el mismo n�mero de votos.
					\item[Libre:] Todo votante tiene el derecho de formarse y expresar su propia opini�n libremente, sin ser coaccionado por ninguna influencia exterior.
					\item[Secreto:] Todo votante tiene el derecho de poder votar de forma individual y secreta.
					\item[Directo:] Los votos emitidos por los votantes deben ser los que directamente determinen la/s persona/s elegida/s en la votaci�n.
				\end{description}
				
				Seg�n el mismo art�culo, estos principios fundamentales se traducen en los siguientes requisitos:
				\begin{description}[font=$\bullet$\ \ ]
					\item[Transparencia/Integridad:] Asegurar que el p�blico general y los asociados al proceso electoral tienen confianza en la soluci�n implementada. 
					\item[Voto secreto/privacidad:] Proteger el secreto del voto en todas las etapas del proceso de votaci�n.
					\item[Unicidad:] Asegurar que cada voto emitido es contado y que cada uno es contado tan s�lo una vez.
					\item[Derecho al voto:] Asegurar que s�lo aquellos votantes con derecho de voto pueden votar en el proceso.
					\item[Verificabilidad/auditor�a:] El votante deber�a poder comprobar que su voto ha sido correctamente contado en el escrutinio. Si no lo puede comprobar, al menos auditores independientes deber�an poder comprobar la integridad de los resultados de la elecci�n.
					\item[Accesibilidad:] Garantizar la accesibilidad al proceso del mayor n�mero de personas, especialmente aquellas con alg�n tipo de discapacidad.
					\item[Libertad del votante/resistencia a la coacci�n:] Mantener el derecho del votante a expresar su opini�n mediante el voto sin coacci�n o una influencia externa.
					\item[Disponibilidad:] Asegurar la disponibilidad del sistema durante el tiempo que tiene lugar la votaci�n.
				\end{description}			
															
				En el art�culo mencionado se realiza una valoraci�n muy interesante de varios sistemas de votaci�n electr�nica y por Internet con resultados basados en esta clasificaci�n.
				
				\subsubsection{Resumen}\label{estadocuestion_requisitoVotoElectronic_resumen}
				Se han presentado tres definiciones de requisitos del voto electr�nico/por Internet.
				
				En primer lugar se han introducido los requisitos ''cl�sicos'', definidos por el importante equipo de Fujioka, Okamoto y Ohta en el a�o 1993. Pr�cticamente establecieron el punto de partida para las reglas que deb�an cumplir los sistema de voto electr�nico. A partir de esta definici�n, me ha parecido interesante aportar otras dos, de entre los incontables ejemplos que se pueden encontrar en la bibliograf�a especializada, que mostrasen las variaciones a partir de estas originales.
				Para ello, se han recogido las de una Universidad espa�ola puntera en proyectos de software libre como es la Universidad de Extremadura, para poner en esta memoria un ejemplo de nuestro pa�s, adem�s de que, si se consulta el documento original, se observa que se enunciaron teniendo en mente un sistema de voto electr�nico por Internet. Y se ha incluido un art�culo holand�s por ser interesante entre los m�s recientes publicados, intentando dar una definici�n de vanguardia.
				
				La conclusi�n, revisando las tres definiciones, es que los requisitos son interpretables y cada autor propone los que cree convenientes bas�ndose en su propia idea o experiencia. Pero es claramente visible que pese a esta interpretaci�n libre, existe un conjunto de requisitos que parece que todos estos autores consideran inherentes al voto electr�nico, pues, de una forma u otra, los recogen la mayor�a de las recopilaciones estudiadas.
														
			As� entre los requisitos b�sicos vemos que los autores aqu� recogidos siempre hablan de \textit{integridad}, \textit{privacidad}, \textit{verificabilidad}, \textit{unicidad}, \textit{fiabilidad}, \textit{disponibilidad}, \textit{derecho de voto} ... Aspectos que debe cumplir todo sistema de votaci�n electr�nica para que el proceso pueda ser llevado a cabo con un nivel aceptable de seguridad y confianza por parte de los intervinientes y observadores de la Elecci�n.
				
		% *************************************************************************************************************** %
		% 			EXPERIENCIAS DE VOTO ELECTR�NICO
		% *************************************************************************************************************** %
		\section{Experiencias de Voto por Internet}\label{estadoExperienciasVotoPorInternet}
		En lo referente al voto electr�nico hay numerosos proyectos llevados a cabo, tanto desde el mundo empresarial como estatal o universitario.
		
		Muchos de ellos se utilizan hoy en d�a. Se pueden destacar a niveles estatales todas las elecciones en la que se usan urnas electr�nicas o m�quinas cuenta-votos, como ocurre en pa�ses como Venezuela, Estados Unidos, India y varios m�s. Hay otros estados cuya prioridad es el estudio de la implantaci�n de este tipo de herramientas para sus procesos electorales, como es el caso actual de Argentina. Lo mismo ocurre con muchos proyectos surgidos desde �mbitos acad�micos o empresariales, donde se desarrollan sistemas que permiten el uso de tecnolog�a para la fase de votaci�n. Incluso algunos permiten el volcado de informaci�n de los votos de la urna en el sistema de recuento de voto, aunque una vez la urna ha sido cerrada, no en el momento en el que el votante introduce su voto en el Sistema.
		
		No obstante, la naturaleza de este proyecto implica que nos centremos en aquellos procesos que utilizan la variante del voto electr�nico consistente en el voto por Internet, siendo �ste transmitido al sistema de recuento en el momento en el que se introduce el voto.
		
		En este sentido, en cuanto al estado de la cuesti�n del voto por Internet, como hemos destacado, la experiencia m�s ambiciosa es, sin duda, las elecciones que se llevan a cabo en Estonia (\ref{ivotingEstonia}) donde, desde el a�o 2005, se utiliza un sistema de voto por Internet accesible por la totalidad del censo que desee hacer uso de �l.
		
		En clave nacional podemos destacar el desempe�o de empresas como Scytl\footnote{\url{https://www.scytl.com/es/clientes/}}, que ha implementado sistemas de voto por Internet para voto desde el extranjero para algunos condados de Estados Unidos, ciertos cantones de Suiza y varias provincias de India, la mayor democracia del mundo (en n�mero de votantes). Otra empresa espa�ola, Indra, tambi�n tiene soluciones tecnol�gicas de voto por Internet utilizadas para elegir las c�pulas directivas de organismos como la Guardia Civil, universidades como la \gls{UAH} o la \gls{UNED} e incluso de partidos pol�ticos, como es el caso de la direcci�n de \gls{UPyD}.
		
		
% *************************************************************************************************************** %
% 			VOTO POR INTERNET
% *************************************************************************************************************** %
		%\section{Voto por Internet (i-voting)}\label{Voto por Internet (i-voting)}
			Dentro de las soluciones de voto electr�nico telem�tico, es importante el desarrollo que se ha hecho en el voto por Internet. 
			
			Seg�n un estudio\footnote{\url{https://www.jbisa.nl/download/?id=17700076&download=1}} de un ente holand�s, once pa�ses han desarrollado pruebas piloto para elecciones a trav�s de voto electr�nico por Internet a nivel nacional, aunque cuatro de ellos ya hab�an abandonado sus proyectos. Los pa�ses que continuaban probando de distinta forma soluciones de voto por Internet y que recoge dicho estudio ser�an Australia, Canad�, Estonia, Francia, India, Noruega y Suiza. Las motivaciones de los pa�ses en desarrollar herramientas de votaci�n remota difieren en cada uno de ellos, dependiendo del tipo de votantes al que va dirigido este tipo de votaci�n. Indica como ejemplo el de Francia, motivado por la necesidad de incrementar la participaci�n electoral de los expatriados o el de Estonia, pa�s que ha apostado por un desarrollo tecnol�gico en la mayor�a de entornos gubernamentales, buscando incrementar la participaci�n tanto de los votantes ocasionales como de acercar a los que se suelen abstener. 
			
			Seg�n otro estudio presentado por la municipalidad de Guelph, en Ontario, con vistas a sus propias Elecciones Municipales con voto por Internet, se destacan una serie de razones que diferentes territorios han tenido en cuenta para invertir en el desarrollo e implantaci�n del iVoting. Estas razones se resumen en la tabla \ref{tab:principalesRazonesVotoInternet}.
			
			Observando esta tabla, se descubre que hay un motivo en el que coinciden por unanimidad los territorios citados, que es la Accesibilidad de los votantes al proceso electoral, permitiendo que votantes con discapacidades o que se encuentren fuera de su circunscripci�n electoral, por ejemplo, puedan ejercer su voto. Los siguientes motivos tenidos en cuenta para la inversi�n en Voto por Internet son la b�squeda de un aumento de la participaci�n y el posicionamiento de liderazgo en cuanto a Gobierno Electr�nico.
			
			Es interesante destacar de esta tabla que aunque la mayor�a apuesta por el Voto por Internet como una herramienta para incrementar la participaci�n electoral, tan s�lo dos territorios se centran en la participaci�n de los j�venes. De hecho, Noruega, que es uno de esos dos territorios, ni siquiera esgrime el motivo de la participaci�n electoral como uno que le lleve a invertir en iVoting por encima de otros.


			% Para forzar nueva l�nea dentro de una celda
		\newcommand{\specialcell}[2][c]{%
			\begin{tabular}[#1]{@{}c@{}}#2\end{tabular}}
		%Foo bar & \specialcell{Foo\\bar} & Foo bar \\    % vertically centered
		%Foo bar & \specialcell[t]{Foo\\bar} & Foo bar \\ % aligned with top rule
		%Foo bar & \specialcell[b]{Foo\\bar} & Foo bar \\ % aligned with bottom rule
		\begin{table}[htbp]
			\centering
				\begin{threeparttable}
					\renewcommand{\familydefault}{\ttdefault}\normalfont
					\footnotesize
					\centering
						%\begin{tabular}{0.75\textwidth}@{\extracolsep{\fill}}|l|l|m{3cm}| m{3cm}|}
						%\begin{tabular*}{0.95\textwidth}{@{\extracolsep{\fill}}|l|l|rm{2cm}|rm{1cm}|}
						\begin{tabularx}{\textwidth}{|m{.15\textwidth}|Y|Y|Y|Y|Y|Y|Y|}
							\hline
								\multirow{2}{*}{\textbf{Territorio}} & \multicolumn{7}{c|}{\textbf{Razones para adopci�n del Voto por Internet}} \\
							\cline{2-8}
								 & \fontsize{5}{6}\textbf{Participaci�n electoral} 
								 & \fontsize{5}{6}\textbf{Liderazgo en Gobierno Electr�nico} 
								 & \fontsize{5}{6}\textbf{Accesibilidad} 
								 & \fontsize{5}{6}\textbf{Conveniencia} 
								 & \fontsize{5}{6}\textbf{Foco en servicio centrado en el ciudadano} 
								 & \fontsize{5}{6}\textbf{Incrementar la participaci�n de los j�venes} 
								 & \fontsize{5}{6}\textbf{Eficiencia en el escrutinio} \\
							\hline
								\raisebox{-1px}{\setlength{\fboxsep}{0pt}\setlength{\fboxrule}{.2px}\fbox{\includegraphics[width=5mm,height=3mm]{imgs/Flag_of_Estonia.png}}} Estonia	& $\bullet$ 	& $\bullet$ & $\bullet$ & & & & 	\\
							\hline
								\raisebox{-1px}{\setlength{\fboxsep}{0pt}\setlength{\fboxrule}{.2px}\fbox{\includegraphics[width=5mm,height=3mm]{imgs/Flag_of_Switzerland.png}}} Suiza	& $\bullet$ 	& $\bullet$ & $\bullet$ & $\bullet$ &  &  &  	\\
							\hline
								\raisebox{-1px}{\setlength{\fboxsep}{0pt}\setlength{\fboxrule}{.2px}\fbox{\includegraphics[width=5mm,height=3mm]{imgs/Flag_of_Norway.png}}} Noruega	& & & $\bullet$ & & & $\bullet$ & $\bullet$ 	\\
							\hline
								\raisebox{-1px}{\setlength{\fboxsep}{0pt}\setlength{\fboxrule}{.2px}\fbox{\includegraphics[width=5mm,height=3mm]{imgs/Flag_of_Canada.png}}} Edmonton	& & $\bullet$ & $\bullet$ & $\bullet$ & $\bullet$ & & \\
							\hline
								\raisebox{-1px}{\setlength{\fboxsep}{0pt}\setlength{\fboxrule}{.2px}\fbox{\includegraphics[width=5mm,height=3mm]{imgs/Flag_of_Canada.png}}} Markham	& $\bullet$ 	& $\bullet$ & $\bullet$ & $\bullet$ & $\bullet$ & & 	\\
							\hline
								\raisebox{-1px}{\setlength{\fboxsep}{0pt}\setlength{\fboxrule}{.2px}\fbox{\includegraphics[width=5mm,height=3mm]{imgs/Flag_of_Canada.png}}} Halifax	& $\bullet$ 	& & $\bullet$ & & & & \\
							\hline
								\raisebox{-1px}{\setlength{\fboxsep}{0pt}\setlength{\fboxrule}{.2px}\fbox{\includegraphics[width=5mm,height=3mm]{imgs/Flag_of_Canada.png}}} \tiny{Cape Breton}	& $\bullet$ 	& $\bullet$ & $\bullet$ & & & & \\
							\hline
								\raisebox{-1px}{\setlength{\fboxsep}{0pt}\setlength{\fboxrule}{.2px}\fbox{\includegraphics[width=5mm,height=3mm]{imgs/Flag_of_England.png}}} Truro	& $\bullet$ 	& $\bullet$ & $\bullet$ & $\bullet$ & & $\bullet$ &  	\\
							\hline
								Resumen & \scriptsize\specialcell{6/8\\75\%} & \scriptsize\specialcell{6/8\\75\%} & \scriptsize\specialcell{8/8\\100\%} & \scriptsize\specialcell{4/8\\50\%} & \scriptsize\specialcell{2/8\\25\%} & \scriptsize\specialcell{2/8\\25\%} & \scriptsize\specialcell{1/8\\12,5\%} \\
							\hline
						\end{tabularx}
						\begin{tablenotes}
							\scriptsize
							\item[a] Fuente: Presentaci�n del Voto Online para las Elecciones Municipales de Guelph (Ontario, Canad�) en 2014\tnote{b}
							\item[b] \url{https://www.youtube.com/watch?v=FJ2rHI8NNBk}
						\end{tablenotes}
						\caption{Principales razones para considerar la adopci�n del Voto por Internet}\label{tab:principalesRazonesVotoInternet}
					\end{threeparttable}
			\end{table}
			
			\subsection{Estonia}\label{ivotingEstonia}
			Estonia es quiz� el ejemplo m�s destacado en cuanto a la utilizaci�n del voto por Internet en elecciones a nivel estatal. Desde el a�o 2005 lleva usando una soluci�n de voto electr�nico remoto no presencial complementando al voto tradicional.
			
			El impacto del voto electr�nico sobre el electorado estonio ha ido evolucionando en cada comicio. En el 2005, el primer a�o en que se comenz� a utilizar, no lleg� al 2\% de los votantes los que se decantaron por hacerlo por Internet, mientras que en el 2014 y 2015, este porcentaje super� el 30\% de los sufragistas (ver tabla \ref{tab:EvolucionIVotingEstonia}).
			
			
			%\begin{table}[htbp]
				%\centering
					%%\begin{tabular}{0.75\textwidth}@{\extracolsep{\fill}}|l|l|m{3cm}| m{3cm}|}
					%%\begin{tabular*}{0.95\textwidth}{@{\extracolsep{\fill}}|l|l|rm{2cm}|rm{1cm}|}
					%\begin{tabular*}{0.95\textwidth}{@{\extracolsep{\fill}}|l|l|r|r|}
						%\hline
%%							\textbf{Elecci�n} & \textbf{Tipo} & \textbf{Votantes por internet} & \textbf{\% sobre total votantes} \\
							%\textbf{Elecci�n} & \textbf{Tipo} & \textbf{IV} & \textbf{\% IV-TV} \\
						%\hline
							%2005 & Elecciones Locales 						& 9.317 	& 1,90\% 	\\
						%\hline
							%2007 & Elecciones Parlamentarias 			& 30.275 	& 5,50\% 	\\
						%\hline
							%2009 & Elecciones Parlamento Europeo	& 58.669 	& 14,70\% \\
						%\hline
							%2009 & Elecciones Locales 						& 104.413 & 15,80\% \\
						%\hline
							%2011 & Elecciones Parlamentarias 			& 140.846 & 24,30\% \\
						%\hline
							%2013 & Elecciones Locales 						& 133.808 & 21,20\% \\
						%\hline
							%2014 & Elecciones Parlamento Europeo 	& 103.151	& 31,30\%	\\
						%\hline
							%2015 & Elecciones Parlamentarias	 		& 176.491	& 30,05\%	\\
						%\hline
					%\end{tabular*}
				%\caption{Evoluci�n del voto por Internet en Estonia}
				%\begin{flushleft}
					%\caption*{\textbf{IV}: Votantes que votaron a trav�s de Internet.}
					%\caption*{\textbf{\%IV-TV}: \% de votantes que votaron a trav�s de Internet sobre el total de votantes.}
					%\caption*{Fuente: Comisi�n Nacional Electoral de Estonia / Vabariigi Valimiskomisjon \footnote{\url{http://www.vvk.ee/voting-methods-in-estonia/engindex/statistics}}}
				%\end{flushleft}
				%\begin{tablenotes}
					%\small
					%\item Fuente: Comisi�n Nacional Electoral de Estonia / Vabariigi Valimiskomisjon \footnote{fff \url{http://www.vvk.ee/voting-methods-in-estonia/engindex/statistics}}
				%\end{tablenotes}
				%\label{tab:EvolucionIVotingEstonia}
			%\end{table}
			
			
			\begin{table}[htbp]
				\centering
				\begin{threeparttable}
					\centering
						%\begin{tabular}{0.75\textwidth}@{\extracolsep{\fill}}|l|l|m{3cm}| m{3cm}|}
						%\begin{tabular*}{0.95\textwidth}{@{\extracolsep{\fill}}|l|l|rm{2cm}|rm{1cm}|}
						\begin{tabular*}{0.95\textwidth}{@{\extracolsep{\fill}}|l|l|r|r|}
							\hline
	%							\textbf{Elecci�n} & \textbf{Tipo} & \textbf{Votantes por internet} & \textbf{\% sobre total votantes} \\
								\textbf{Elecci�n} & \textbf{Tipo} & \textbf{IV}\tnote{a} & \textbf{\% IV-TV}\tnote{b} \\
							\hline
								2005 & Elecciones Locales 						& 9.317 	& 1,90\% 	\\
							\hline
								2007 & Elecciones Parlamentarias 			& 30.275 	& 5,50\% 	\\
							\hline
								2009 & Elecciones Parlamento Europeo	& 58.669 	& 14,70\% \\
							\hline
								2009 & Elecciones Locales 						& 104.413 & 15,80\% \\
							\hline
								2011 & Elecciones Parlamentarias 			& 140.846 & 24,30\% \\
							\hline
								2013 & Elecciones Locales 						& 133.808 & 21,20\% \\
							\hline
								2014 & Elecciones Parlamento Europeo 	& 103.151	& 31,30\%	\\
							\hline
								2015 & Elecciones Parlamentarias	 		& 176.491	& 30,05\%	\\
							\hline
						\end{tabular*}
						\begin{tablenotes}
							\scriptsize
							\item[a] \textbf{IV}: Votantes que votaron a trav�s de Internet.
							\item[b] \textbf{\%IV-TV}: \% de votantes que votaron a trav�s de Internet sobre el total de votantes.
							\item[c] Fuente: Comisi�n Nacional Electoral de Estonia / Vabariigi Valimiskomisjon\tnote{d}
							\item[d] \url{http://www.vvk.ee/voting-methods-in-estonia/engindex/statistics}
						\end{tablenotes}
						\caption{Evoluci�n del voto por Internet en Estonia\tnote{c}}
					\end{threeparttable}
					\label{tab:EvolucionIVotingEstonia}
			\end{table}
			
			
			\begin{figure}[htbp]
				\centering
					\includegraphics[width=0.95\textwidth]{recursos/GraficoParticipacionEstonia.png}
				\caption{Participaci�n hist�rica de elecciones con i-voting en Estonia.}
				\label{fig:GraficoParticipacionEstonia}
			\end{figure}
			
			\begin{figure}[htbp]
				\centering
					\includegraphics[width=0.95\textwidth]{recursos/GraficoPorcentajeParticipacionEstonia.png}
				\caption{Hist�rico: Porcentaje de voto presencial comparado con el i-Voting en Estonia.}
				\label{fig:GraficoPorcentajeParticipacionEstonia}
			\end{figure}
			
			\begin{figure}[htbp]
				\centering
					\includegraphics[width=0.95\textwidth]{recursos/GraficoPorcentajeInternetEstonia.png}
				\caption{Hist�rico: Porcentaje de voto por Internet frente al total de votos en Estonia.}
				\label{fig:GraficoPorcentajeInternetEstonia}
			\end{figure}
			
			Estonia es el primer estado que utiliza, oficialmente, el voto electr�nico remoto por Internet de forma vinculante. Este sistema puesto en pr�ctica en el a�o 2005 es una parte de un plan de modernizaci�n del pa�s b�ltico. De hecho, previamente a la puesta en producci�n del sistema electoral, se comenz� a desarrollar en el a�o 2000 un despliegue t�cnico importante para la implantaci�n del documento de identidad electr�nico, junto con mecanismos de comunicaci�n con la Administraci�n para facilitar los tr�mites con la misma por parte de los ciudadanos de forma electr�nica y remota.
			
			La ley electoral estonia permite a los votantes ejercer su derecho al voto de tres formas:
			\begin{enumerate}[a)]
				\item \textbf{Voto tradicional} Los votantes pueden acudir a los colegios electorales e introducir su voto en la urna previa identificaci�n del votante por parte de los miembros de la mesa.
				\item \textbf{Voto postal} Los votantes estonios tienen la posibilidad de acudir en unas fechas determinadas anteriores al d�a electoral a unas Estaciones de Votaci�n, que funcionan de forma an�loga a Correos en Espa�a, donde pueden entregar el voto en papel y una acreditaci�n que le identifique. Esta Estaci�n se encarga de hacer llegar el voto y la identificaci�n a la mesa o Distrito Electoral donde el votante est� censado.
				\item \textbf{Voto por Internet} Durante un per�odo de tiempo anterior al d�a electoral, los votantes tienen la posibilidad de entregar el voto por medio de Internet.
			\end{enumerate}
			
			Aunque el votante haya emitido su voto de forma electr�nica, la Ley Electoral estonia permite al mismo ejercer su voto de cualquiera de las otras dos formas invalidando su voto electr�nico. Es decir, que si una vez votado por Internet, el votante decide votar por correo, �ste voto anular� el emitido por Internet. Lo mismo pasar�a si decidiese votar presencialmente el d�a electoral, que su voto emitido por Internet quedar�a anulado y fuera del escrutinio. Este hecho es una medida de la Autoridad Electoral para proteger a los votantes frente a la coerci�n, proveyendo de un mecanismo por el cual un votante que haya elegido una formaci�n determinada por presiones de terceros podr�a libremente cambiar la direcci�n de su voto una vez emitido el primero.\\\hspace*{\fill}
			
			Son requisitos fundamentales de este sistema de voto electr�nico remoto la seguridad, confiabilidad y la precisi�n, as� como proveer de mecanismos eficaces contra la coerci�n. Otra necesidad importante del sistema es su acceso, que debe ser pr�cticamente universal, lo cual implica que ha de ser accesible y sencillo de entender para los usuarios, adem�s de que debe funcionar en la mayor�a de las plataformas tecnol�gicas.
			
			Hay una serie de puntos, recogidos en \cite{Morshed2010}, que consiguen que el sistema satisfaga tales requisitos:
			\begin{enumerate}[1.]
				\item Uso de ID-cards o Mobile ID para la identificaci�n de los votantes.
				\item Un votante puede emitir cualquier n�mero de votos durante el periodo habilitado para la votaci�n electr�nica. El �ltimo voto enviado ser� el �nico que cuente en el escrutinio. No obstante, si el votante se encuentra bajo alg�n tipo de coerci�n, siempre podr� volver a votar m�s adelante (cuando no ejerzan presi�n sobre su decisi�n) y este �ltimo ser� el que cuente. As� se intenta minimizar el riesgo de la coacci�n.
				\item Prioridad del voto tradicional. Si el votante ejerce su derecho al voto de forma presencial, cualquier voto que hubiese emitido de forma electr�nica ser� cancelado y no se contar� en el escrutinio.
				\item Todos los servidores en el sistema de voto son seguros y siempre estar�n bajo monitorizaci�n durante el periodo de la votaci�n.
				\item El servidor de almacenamiento de voto est� detr�s de un firewall. Nadie puede acceder a este servidor desde Internet.
				\item El servidor de conteo de votos est� offline, sin conexi�n a Internet y asegurado por medio de clave privada compartida.
				\item Todas las comunicaciones a trav�s de Internet usan cifrado SSL.
				\item El cifrado y la firma digital usan un mecanismo de cifrado RSA.
			\end{enumerate}

			En las conclusiones de la tesis \cite{tesisEPerezBelleboni} se destaca de este proceso:
			\begin{itemize}
				\item Interesante por ser una elecci�n a nivel nacional y vinculante.
				\item N�mero de votantes en tendencia creciente a nivel nacional.
				\item Debilidades:
					\begin{itemize}
						\item No se hace uso de mecanismos seguros que garanticen la protecci�n de la privacidad del voto.
						\item El voto no est� protegido por primitivas de firma ciega, anonimizadores (el voto se almacena entre 4 y 10 d�as junto a la identificaci�n del votante) ni mecanismos equivalentes, sino que se traslada a este sistema de voto por Internet las debilidades ya existentes en el voto tradicional.
					\end{itemize}
			\end{itemize}
			
			La importancia que tiene el sistema de voto utilizado en Estonia radica en el hecho de que provee un mecanismo de voto por Internet a un potencial de votantes consistente en el 100\% de la poblaci�n con derecho de voto de un estado democr�tico. Hasta el momento de su implantaci�n, esto no ocurr�a. Se daban casos en los que se proporcionaba un sistema electr�nico remoto a diversos espectros de la poblaci�n, como pod�an ser los residentes en el extranjero, los militares en misiones activas o los focos de posibles proyectos pilotos.
			
			Es destacable que, con el objetivo de alcanzar a la totalidad de la poblaci�n, incluso, el propio estado estonio aumentara el desarrollo de infraestructura tecnol�gica en el pa�s para intentar reducir la brecha digital de sus habitantes, tratando de proveer el acceso a Internet a la mayor parte del pa�s. Igualmente importante fueron los esfuerzos por la certificaci�n digital, teniendo como punto esencial la implantaci�n del documento nacional de identidad electr�nico para la totalidad de la poblaci�n.
			
			La mayor�a de los estados no pueden implantar unas elecciones como las llevadas a cabo en Estonia por diversos motivos, v�ase el miedo a la falta de transparencia, fraude, log�stica o, en muchos casos, ilegalidad con respecto a las leyes electorales actuales.
			
			En diversos pa�ses se ha logrado implementar sistemas reales de votaci�n deslocalizada por Internet en varios de sus territorios, alternativamente y al mismo nivel que el sistema tradicional presencial. Un ejemplo veremos que es Suiza con los proyectos de varios de sus cantones. No es comparable a la experiencia estonia, pues cada cant�n se rige de forma diferente y son diferentes empresas las que realizan los desarrollos del sistema independientemente del resto, adem�s de que no todos los cantones han implementado estos sistemas remotos.
			
%			\notasInfo[inline]{
%			INTERESANTE -- http://elecciones.smartmatic.com/estonia-y-el-voto-por-internet/
			%\\
			%Hist�ricamente, las comisiones electorales del mundo han tenido en el voto remoto una alternativa para hacer el sufragio m�s accesible y conveniente. Por ello, muchos pa�ses permiten a sus conciudadanos ejercer el voto postal, sea desde el territorio nacional o el extranjero. \\
%
			%Sin embargo, junto a los beneficios que ofrece, el voto remoto trae consigo ciertos retos que deben considerarse seriamente: �C�mo se puede verificar que alguien votando desde su casa es realmente la persona que dice ser? �C�mo saber que la persona que emiti� el voto lo hizo libremente, sin ser coaccionado? �C�mo permitir al votante verificar que su voto es tomado en cuenta y c�mo auditarlo?
%
%La revoluci�n digital de las �ltimas d�cadas abri� la posibilidad de lo que ser�a una forma moderna del voto postal, el voto por Internet. En consecuencia, y desde hace por lo menos un par de d�cadas, diferentes comisiones electorales han tratado de dar con la clave de un voto remoto en l�nea que sea seguro, confiable y verificable. No obstante, solo Estonia ha tenido �xito llevando a cabo elecciones nacionales vinculantes por esta v�a. La clave del sistema estonio es que basa su funcionamiento sobre las siguientes premisas:
%
%Voto opcional: El modelo electoral estonio se basa en un sistema de papel. Emitir un voto electr�nico a distancia es opcional para todos los ciudadanos estonios. El voto en l�nea es s�lo uno de los varios m�todos para facultar a los ciudadanos y est� concebido como un sustituto al voto postal.
%
%Voto temprano: La ventana para votar en l�nea tiene una duraci�n de 7 d�as consecutivos y termina tres d�as antes de la jornada electoral. Los electores tambi�n pueden acudir a cualquier centro de votaci�n y votar en persona y con papel y l�piz durante la votaci�n temprana.
%
%Votaci�n m�ltiple: Los votantes pueden emitir varios votos, pero s�lo el �ltimo de ellos es el que cuenta. Y de haber votado en l�nea -una vez o muchas veces-, el ciudadano siempre tendr� el derecho a votar en persona y en papel el d�a de las elecciones, cancelando as� todos los votos anteriores. Si el votante fue coaccionado en su casa, puede votar nuevamente en cualquier momento.
%
%Verificable: El sistema est� dise�ado para que los votantes puedan verificar que su voto fue registrado correctamente, y los observadores pueden auditar el proceso en su totalidad.
%
%Predominio del papel: Un voto emitido en un centro de votaci�n siempre prevalecer� sobre cualquier voto electr�nico. Por lo tanto, cualquier voto digital hecho bajo coerci�n, puede ser f�cilmente invalidado con otro voto digital, o durante la jornada electoral emitiendo un voto manual.
%
%Seguridad: Un atributo clave de la soluci�n de Estonia es el hecho de que ofrece un equilibrio �ptimo de seguridad criptogr�fica, t�cnica y procedimental impidiendo el fraude electoral toda vez que protege la privacidad y el anonimato de los votos.
%
%En conjunto, estas medidas garantizan la implementaci�n exitosa del voto por internet y, por ende, resguardan la legitimidad de la elecci�n. El voto en l�nea estonio trabaja como sustituto al voto por correo y no sustituye al voto tradicional, m�s bien se apalanca en �ste.
%
%Desde el a�o 2005, Estonia se ha apoyado con �xito en Internet para realizar siete elecciones gubernamentales. Como evidencia del nivel de confianza p�blica en el sistema, en las elecciones al Parlamento Europeo ocurridas en mayo de este a�o el 33\% de todo el electorado estonio decidi� emitir su voto en l�nea.
%}
			
			
			
			\subsection{Noruega}\label{ivotingNoruega}
				La implantaci�n del voto electr�nico en Noruega empieza en 1993 con una experiencia piloto utilizando m�quinas de lectura �ptica en la capital, Oslo.
				
				Ya en 2011, en las elecciones municipales se realiza una primera prueba de voto por Internet en la participaron diez municipios (de un total de 429). Los votantes ten�an la posibilidad de votar por Internet durante un per�odo de voto anticipado. No obstante, pod�an votar mediante el tradicional voto en papel el d�a de las elecciones, prevaleciendo este voto f�sico sobre el emitido previamente por Internet.
				
				Dos a�os m�s tarde, en las elecciones parlamentarias de 2013 se realiza una segunda prueba de voto por Internet en 12 municipios. M�s de 250.000 votantes ten�an la posibilidad de utilizar este canal de votaci�n.
				
				En 2014, no obstante, el Gobierno noruego decide dar por concluidas las pruebas de voto por Internet debido a la controversia pol�tica existente y a que los ensayos realizados no dieron muestras de impulsar la participaci�n entre los ciudadanos.
				

			
			%\begin{table}[htbp]
				%\centering
					%\begin{tabularx}{\textwidth}{|m{.6\textwidth}|Y|Y|}
						%\cline{2-3}
							%\multicolumn{1}{l|}{} & \multicolumn{2}{c|}{\textbf{SISTEMA DE VOTACI�N}} \\
						%\cline{2-3}
							%\multicolumn{1}{l|}{} & \textbf{ESTONIA} & \textbf{NORUEGA} \\
						%\hline
							%Identificaci�n digital de votantes	& $\bullet$$\bullet$$\bullet$ 	& $\bullet$$\bullet$	\\
						%\hline
							%Protecci�n frente a la suplantaci�n de votantes	& $\bullet$$\bullet$ 	& $\bullet$$\bullet$	\\
						%\hline
							%Usabilidad de la interfaz de votante	& $\bullet$ 	& $\bullet$	\\
						%\hline
							%Seguridad criptogr�fica de la informaci�n intercambiada	& $\bullet$$\bullet$$\bullet$ 	& $\bullet$$\bullet$$\bullet$	\\
						%\hline
							%Protecci�n frente a ruptura del secreto del voto por una sola entidad	& $\bullet$$\bullet$$\bullet$ 	& $\bullet$$\bullet$	\\
						%\hline
							%Protecci�n frente a ruptura del secreto del voto por colusi�n entre entidades	& $\bullet$ 	& $\bullet$	\\
						%\hline
							%Protecci�n frente a contabilizaci�n indebida de votos	& $\bullet$$\bullet$ 	& $\bullet$$\bullet$	\\
						%\hline
							%Protecci�n frente a denegaci�n arbitraria de derecho a voto	& $\bullet$$\bullet$$\bullet$ 	& $\bullet$$\bullet$$\bullet$	\\
						%\hline
					%\end{tabularx}
				%\caption{Comparativa de la evaluaci�n de los sistemas de Votos por Internet de Estonia y Noruega conforme al criterio de Seguridad de la Informaci�n}
				%\label{tab:ResultadoDeLaEvaluaci�nConformeAlCriterioDeSeguridadDeLaInformaci�n}
			%\end{table}
			
			
			\begin{table}[htbp]
				\centering
				\begin{threeparttable}
					\centering
						%\begin{tabular}{0.75\textwidth}@{\extracolsep{\fill}}|l|l|m{3cm}| m{3cm}|}
						%\begin{tabular*}{0.95\textwidth}{@{\extracolsep{\fill}}|l|l|rm{2cm}|rm{1cm}|}
						\begin{tabularx}{\textwidth}{|m{.6\textwidth}|Y|Y|}
							\cline{2-3}
								\multicolumn{1}{l|}{} & \multicolumn{2}{c|}{\textbf{\small SISTEMA DE VOTACI�N}} \\
							\cline{2-3}
								\multicolumn{1}{l|}{} & \textbf{\small ESTONIA} & \textbf{\small NORUEGA} \\
							\hline
								Identificaci�n digital de votantes	& $\bullet$$\bullet$$\bullet$ 	& $\bullet$$\bullet$	\\
							\hline
								Protecci�n frente a la suplantaci�n de votantes	& $\bullet$$\bullet$ 	& $\bullet$$\bullet$	\\
							\hline
								Usabilidad de la interfaz de votante	& $\bullet$ 	& $\bullet$	\\
							\hline
								Seguridad criptogr�fica de la informaci�n intercambiada	& $\bullet$$\bullet$$\bullet$ 	& $\bullet$$\bullet$$\bullet$	\\
							\hline
								Protecci�n frente a ruptura del secreto del voto por una sola entidad	& $\bullet$$\bullet$$\bullet$ 	& $\bullet$$\bullet$	\\
							\hline
								Protecci�n frente a ruptura del secreto del voto por colusi�n entre entidades	& $\bullet$ 	& $\bullet$	\\
							\hline
								Protecci�n frente a contabilizaci�n indebida de votos	& $\bullet$$\bullet$ 	& $\bullet$$\bullet$	\\
							\hline
								Protecci�n frente a denegaci�n arbitraria de derecho a voto	& $\bullet$$\bullet$$\bullet$ 	& $\bullet$$\bullet$$\bullet$	\\
							\hline
						\end{tabularx}
						\begin{tablenotes}
							\scriptsize
							\item[a] Fuente: Ponencia \textit{Posibilidades del voto telem�tico en la democracia digital}, por Justo Carracedo\tnote{b}
							\item[b] \url{http://www.criptored.upm.es/descarga/ConferenciaJustoCarracedoTASSI2014.pdf}
						\end{tablenotes}
						\label{tab:ResultadoDeLaEvaluaci�nConformeAlCriterioDeSeguridadDeLaInformaci�n}
						\caption{Comparativa de la evaluaci�n de los sistemas de Votos por Internet de Estonia y Noruega conforme al criterio de Seguridad de la Informaci�n\tnote{a}}
					\end{threeparttable}
			\end{table}
			
			%
			%\begin{table}[htbp]
				%\centering
					%\begin{tabularx}{\textwidth}{|m{.6\textwidth}|Y|Y|}
						%\cline{2-3}
							%\multicolumn{1}{l|}{} & \multicolumn{2}{c|}{\textbf{SISTEMA DE VOTACI�N}} \\
						%\cline{2-3}
							%\multicolumn{1}{l|}{} & \textbf{ESTONIA} & \textbf{NORUEGA} \\
						%\hline
							%Identificaci�n robusta de gestores del sistema	& $\bullet$$\bullet$$\bullet$ 	& $\bullet$$\bullet$$\bullet$	\\
						%\hline
							%Verificaci�n individual de voto	& $\bullet$ 	& $\bullet$	\\
						%\hline
							%Verificabilidad de resultados	& $\bullet$ 	& $\bullet$$\bullet$	\\
						%\hline
							%Protecci�n frente a la coacci�n	& $\bullet$ 	& $\bullet$	\\
						%\hline
							%Protecci�n del sistema frente a falsas acusaciones	& $\bullet$$\bullet$ 	& $\bullet$$\bullet$	\\
						%\hline
							%Capacidad de supervisi�n de los interventores	& $\bullet$$\bullet$ 	& $\bullet$$\bullet$	\\
						%\hline
							%Uso de software p�blico	& $\bullet$ 	& $\bullet$$\bullet$$\bullet$	\\
						%\hline
							%Auditabilidad del sistema	& $\bullet$$\bullet$ 	& $\bullet$$\bullet$	\\
						%\hline
					%\end{tabularx}
				%\caption{Comparativa de la evaluaci�n de los sistemas de Votos por Internet de Estonia y Noruega conforme a los criterios de verificaci�n, auditor�a y procedimiento}
				%\label{tab:ResultadoDeLaEvaluaci�nConformeALosCriteriosDeVerificacionAuditor�aYProcedimiento}
			%\end{table}
			%
			\begin{table}[htbp]
				\centering
				\begin{threeparttable}
					\centering
						%\begin{tabular}{0.75\textwidth}@{\extracolsep{\fill}}|l|l|m{3cm}| m{3cm}|}
						%\begin{tabular*}{0.95\textwidth}{@{\extracolsep{\fill}}|l|l|rm{2cm}|rm{1cm}|}
						\begin{tabularx}{\textwidth}{|m{.6\textwidth}|Y|Y|}
							\cline{2-3}
								\multicolumn{1}{l|}{} & \multicolumn{2}{c|}{\textbf{\small SISTEMA DE VOTACI�N}} \\
							\cline{2-3}
								\multicolumn{1}{l|}{} & \textbf{\small ESTONIA} & \textbf{\small NORUEGA} \\
							\hline
								Identificaci�n robusta de gestores del sistema	& $\bullet$$\bullet$$\bullet$ 	& $\bullet$$\bullet$$\bullet$	\\
							\hline
								Verificaci�n individual de voto	& $\bullet$ 	& $\bullet$	\\
							\hline
								Verificabilidad de resultados	& $\bullet$ 	& $\bullet$$\bullet$	\\
							\hline
								Protecci�n frente a la coacci�n	& $\bullet$ 	& $\bullet$	\\
							\hline
								Protecci�n del sistema frente a falsas acusaciones	& $\bullet$$\bullet$ 	& $\bullet$$\bullet$	\\
							\hline
								Capacidad de supervisi�n de los interventores	& $\bullet$$\bullet$ 	& $\bullet$$\bullet$	\\
							\hline
								Uso de software p�blico	& $\bullet$ 	& $\bullet$$\bullet$$\bullet$	\\
							\hline
								Auditabilidad del sistema	& $\bullet$$\bullet$ 	& $\bullet$$\bullet$	\\
							\hline
						\end{tabularx}
						\begin{tablenotes}
							\scriptsize
							\item[a] Fuente: Ponencia \textit{Posibilidades del voto telem�tico en la democracia digital}, por Justo Carracedo\tnote{b}
							\item[b] \url{http://www.criptored.upm.es/descarga/ConferenciaJustoCarracedoTASSI2014.pdf}
						\end{tablenotes}
						\caption{Comparativa de la evaluaci�n de los sistemas de Votos por Internet de Estonia y Noruega conforme a los criterios de verificaci�n, auditor�a y procedimiento\tnote{a}}
				\label{tab:ResultadoDeLaEvaluaci�nConformeALosCriteriosDeVerificacionAuditor�aYProcedimiento}
					\end{threeparttable}
			\end{table}
			
			
			\subsection{Suiza}\label{ivotingSuiza}
			El estado suizo se divide administrativamente en cantones. Estos cantones son los responsables de la celebraci�n de procesos electorales en sus territorios. Con esto, varios de ellos, impulsados por el Estado, han dedicado mucho esfuerzo al estudio y desarrollo del voto por Internet para poder implementarlo de forma general.
			
			Suiza es un pa�s especial, pues tiene una gran costumbre en la realizaci�n de referendos. Entre 1789 y 2012 se contabilizaron 577 comicios en el pa�s, lo que da una idea de la necesidad de acometer soluciones para combatir el ''cansancio'' democr�tico que puede suponer votar unas seis veces al a�o.
			
			Con el objetivo de la implantaci�n del voto electr�nico, el estado suizo marc� tres fases de actuaci�n como l�nea a seguir para la resoluci�n de estudios y pruebas pilotos previas a una futura utilizaci�n de este sistema con garant�as de viabilidad y seguridad.\cite{suiza1}
			
			En una primera fase, de 2000 a 2002, se realizaron una serie de estudios e investigaciones que derivaron en la creaci�n de programas piloto de voto electr�nico.
			
			De 2002 a 2006, tres cantones - Neuchatel, Ginebra y Zurich, comenzaron a realizar pruebas piloto que mostraron que era posible implantar el voto electr�nico remoto en Suiza, lo cual acentu� el apoyo del Gobierno en el proyecto.
			
			A partir de 2006,  las pruebas piloto se expandieron a otros cantones, utilizando estos los sistemas desarrollados por el cant�n de Zurich o el de Ginebra.
			
			En 2010 ya eran 12 los cantones que realizaron pruebas pilotos en los comicios del 28 de noviembre. El n�mero de votantes que pod�an votar de forma electr�nica ascend�a a 193.236 personas aunque, sin embargo, tan s�lo 28.192, no lleg� al 15\%, lo hicieron de esta forma.
			
			En 2017, Swiss Post, ente organizador de comicios electorales en Suiza public� una \textit{demo}\footnote{\url{https://www.evoting.ch/index_en.php}} de votaci�n por Internet para que los ciudadanos puedan tener contacto con este tipo de sistemas y aclarar as� dudas de funcionamiento y tecnolog�a. 
			
			La tecnolog�a utilizada para este sistema se basa en un protocolo \gls{E2E} verificable. Se puede utilizar en cualquier navegador web. Los votos se cifran en el servidor y se almacenan de forma segura desacopl�ndolos de la informaci�n del votante.
			
			\subsection{Universidades}\label{iVotingUniversidades}
				A nivel de Juntas de Gobierno o de Elecciones de Rector a nivel universitario, merece la pena introducir que existen experiencias de voto por Internet en varias de ellas, destacando las de la \gls{UNED} o las de la \gls{UPV/EHU}.
				
					\begin{wrapfigure}{r}{.10\textwidth}
						\vspace{-20pt}
						\begin{center}
							\includegraphics[width=.1\textwidth]{imgs/unedLogo.jpeg}
						\end{center}
						\vspace{-20pt}
					\end{wrapfigure}
					La \gls{UNED} fue pionera en el voto electr�nico para elegir su claustro dentro del mundo universitario espa�ol. En el a�o 2010, se ali� con la multinacional espa�ola Indra para adaptar su plataforma Net-vote\footnote{\url{http://www.indracompany.com/sites/default/files/indra_pe_netvote_baja.pdf}} al proceso electoral y generar un caso de �xito. 
				
					Para llevar a cabo estos comicios fue necesario adaptar su Reglamento Electoral General.
				
					El proceso consist�a en cuatro fases:
					\begin{enumerate}
						\item \textbf{Preparaci�n}. Preparar las necesidades a cubrir del resto de fases.
						\item \textbf{Fase Preelectoral}. En esta fase, se precarga el sistema con el censo de la elecci�n, los par�metros que la definen y se generan las claves de cifrado de los votos.
						\item \textbf{Votaci�n}. Fase en la que el votante introduce su voto en el sistema y �ste lo guarda cifrado.
						\item \textbf{Recuento}. Se desencriptan los votos para ser contados. A continuaci�n, se publican los resultados.
					\end{enumerate}
				
					Es interesante destacar el apartado de la Identidad Digital de esta primera elecci�n. Se permitieron tres tokens digitales para poder identificar al votante contra el sistema:
					\begin{itemize}
						\item Certficado electr�nico incluido en el \textbf{\gls{DNIe}}
						\item Certificado \textbf{\gls{CERES-FNMT}}, incluido en la tarjeta universitaria inteligente de la \gls{UNED}.
						\item \textbf{Clave concertada} (usuario y contrase�a) de uso exclusivo en este proceso electoral.
					\end{itemize}
				
					Ya en esta elecci�n se apostaba por la identificaci�n con los certificados del \gls{DNIe} 2.0, el que posee chips de contacto. Dentro de las conclusiones de esta experiencia, se observa que el tipo de incidencia m�s com�n en el proceso fue la de \textit{Instalaci�n de la m�quina virtual de JAVA}, necesaria para poder ejecutar el applet de comunicaci�n entre el navegador y los drivers de lectura de los certificados del documento. Se empieza a vislumbrar que la Web no est� realmente preparada para este mecanismo de identificaci�n y resulta ser un cuello de botella muy importante para la identificaci�n remota en Internet.
				
				
					Para consultar de forma r�pida datos sobre esta primera elecci�n, un documento muy resumido y recomendables es \cite{UNED:2010:elecciones:claustro}.
					
					A partir de 2013, los diferentes comicios de esta universidad fueron adjudicados a la empresa espa�ola Scytl. En los de 2013, correspondientes a la elecci�n de Rector, pusieron a disposici�n del proceso su plataforma de voto por Internet Pnyx. Para conocer m�s acerca de la adaptaci�n de esta plataforma a estas elecciones y las diferencias con la anterior, adjunto en la bibliograf�a un documento divulgativo de f�cil lectura. \cite{UNED:2013:elecciones:rector}
				
					\begin{wrapfigure}{l}{.20\textwidth}
						\vspace{-20pt}
						\begin{center}
							\includegraphics[width=.2\textwidth]{imgs/upvehuLogo}
						\end{center}
						\vspace{-20pt}
					\end{wrapfigure}
					Este mismo sistema Pnyx es el que lleva varios a�os implementando Scytl como plataforma de voto para las diferentes elecciones de la Universidad del Pa�s Vasco, adem�s de otras Universidades y organizaciones p�blicas y privadas.			
				
				
							
		% *************************************************************************************************************** %
		% 			VOTO POR INTERNET EN LA EPS
		% *************************************************************************************************************** %
		\subsection{Escuela Polit�cnica Superior - Universidad San Pablo CEU}\label{ivoting_epsceu}
			\begin{wrapfigure}{r}{.50\textwidth}
				\vspace{-20pt}
				\begin{center}
					\includegraphics[width=.5\textwidth]{imgs/uspceuLogo}
				\end{center}
				\vspace{-20pt}
			\end{wrapfigure}
			En la Escuela Polit�cnica Superior de la Universidad San Pablo-CEU ya se realiz� una elecci�n por medio de voto electr�nico. Sucedi� en 2005, cuando en una colaboraci�n entre la Universidad y la multinacional espa�ola Indra se celebr� la primera elecci�n de delegados de clase a trav�s de voto electr�nico con motivo del D�a de Internet, celebrado el 25 de octubre del mismo a�o.\\\hspace*{\fill}
			
			En esta experiencia, m�s de 600 alumnos de los �ltimos cursos de la Escuela Polit�cnica eligieron a sus delegados de clase a trav�s de este sistema.\\\hspace*{\fill}
			
			En la fecha de la elecci�n, cada alumno emiti� su voto a trav�s de un nombre de usuario y una clave personal. Por motivos divulgativos, los organizadores de la elecci�n determinaron que una parte del alumnado censado realizara la votaci�n desde un aula de votaci�n concreta, perteneciente al centro y adecuada para ello; mientras que el resto del alumnado deb�a elegir sus representantes desde alg�n equipo personal fuera del dominio de la Universidad.\\\hspace*{\fill}
			
			Para que estas elecciones a trav�s de Internet pudiesen llevarse a cabo la Universidad San Pablo-CEU tuvo que adaptar su normativa de r�gimen interno, pues la que ten�a originalmente establec�a �nicamente la posibilidad de un sistema de voto presencial.
\iffalse
			\subsection{Universidad de Lovaina}\label{ivotingUniversidadLovaina}
			
			\todo[inline]{Esto ser�a interesante incluirlo}
\fi
		\section{Proyectos varios de voto por Internet }\label{estado_desarrollos}
		
			 En esta secci�n introducimos algunos proyectos de voto por Internet que se han estudiado para la elaboraci�n de este PFC y que han resultado ser interesantes y con propuestas a tener en cuenta para llevar a cabo proyectos de �ndole electoral.
			
			No obstante, una vez que se decidi� que el sistema debe cumplir con la caracter�stica de ser un sistema de votaci�n auditable punto a punto (\gls{E2E} Auditable Voting System), estos proyectos fueron descartados como c�digos base de la adaptaci�n al sistema a implementar para la EPS.
			
			\subsection{Votescript}\label{ivotingVotescript}
			Votescript es un proyecto de espa�ol, gestado en la Universidad Polit�cnica de Madrid, de voto electr�nico remoto basado en locales de votaci�n. Pese a que no responde completamente al objetivo de este \gls{PFC}, pues requiere de un entorno controlado como los locales de votaci�n, su desarrollo telem�tico s� que se basa en una comunicaci�n de los votos a trav�s de Internet. Se podr�a corresponder con el escenario en el que se deba dar la posibilidad de voto presencial de forma complementaria al voto remoto (es uno de los requisitos que se definir�n en el cap�tulo \ref{requisitosFuncionales}).
			
			El esquema de votaci�n telem�tica Votescript tiene su origen en el proyecto de investigaci�n \textit{Votaci�n Electr�nica Segura basada en criptograf�a avanzada} \cite{votescript1}, denominaci�n de la cual adquiere el nombre, Votescript. Este proyecto es una colaboraci�n entre el grupo de la Universidad Polit�cnica y la F�brica Nacional de Moneda y Timbre - Real Casa de la Moneda (una de las principales entidades emisoras de certificados digitales de Espa�a).
			
			A partir de este proyecto de investigaci�n, los autores publican diversos art�culos sobre el funcionamiento y alcance de los resultados obtenidos. En este apartado, nos basamos en la versi�n m�s actual del proyecto, desarrollado en su tesis doctoral \cite{tesisEPerezBelleboni} por una de las autoras del original, la Dra. Emilia P�rez Belleboni.
			
			En esta tesis, adem�s de analizar el estado del voto telem�tico, teniendo en consideraci�n, esquemas, problemas y riesgos, realiza un estudio de varias implementaciones reales a nivel nacional, como por ejemplo un extenso an�lisis del procedimiento electoral electr�nico de Estonia (\ref{ivotingEstonia}). No obstante, a partir de estos an�lisis, desarrolla el esquema que proponen, con base en el Votescript original, evolucion�ndolo para solucionar las debilidades del resto de sistemas y para su aplicaci�n en la elecci�n de representantes para el Parlamento Europeo.
			
			En contraposici�n a los sistemas que estudia en la tesis, el sistema Votescript centra sus esfuerzos en la superaci�n de debilidades identificadas en los anteriores, en especial en la fase de identificaci�n del votante. En elecciones como las del Parlamento Europeo, una entidad supranacional, es muy importante que la identificaci�n de los votantes se pueda realizar electr�nicamente de una forma altamente confiable, pues deben ser v�lidas no s�lo en el pa�s del propio votante, sino en el resto de pa�ses europeos.
			
			El esquema que propone Votescript define la necesidad de unos puntos espec�ficos de votaci�n, centros donde han de acudir los votantes a votar telem�ticamente. En estos centros se implantar�an los medios y equipamientos tecnol�gicos para que el votante emita su voto en un entorno controlado.
			
			Seg�n su documentaci�n, las bases del sistema se pueden adecuar sin problemas a un \textit{sistema abierto} (voto por Internet), pero el precio que implica la comodidad de los votantes de poder votar sin necesidad de trasladarse a locales oficiales incurre en un incremento de los riesgos de coacci�n. 
			
			
			\subsection{SEVI}\label{sevi}
				El Sistema Electr�nico de Votaci�n por Internet (SEVI) \cite{tesisMLLGarcia} es una propuesta de sistema software de voto electr�nico para reemplazar el canal que constituye el correo postal certificado en el proceso electoral de M�xico.
				
				La idea del sistema es que los ciudadanos con derecho a voto que no puedan hacer uso del mismo el d�a del proceso tengan un canal de votaci�n disponible a trav�s de Internet. Este canal sustituir�a al proporcionado por el correo postal, por lo que debe asegurar, como m�nimo, los mismos servicios que ya proporcionaba �ste en procesos electorales anteriores.
				\begin{figure}[htbp]
				\centering
					\includegraphics[width=0.80\textwidth]{imgs/sevi01.png}
				\caption{Primera aproximaci�n de funcionalidad de SEVI}
				\label{fig:primeraAproximacionSEVI}
			\end{figure}
			
			El esquema de votaci�n utilizado en SEVI divide el proceso electoral en cuatro fases:
			\begin{itemize}
				\item Registro
				\item Votaci�n
				\item Resultados
				\item Auditor�a
			\end{itemize}
			
			Al igual que en la tesis del proyecto SELES \cite{tesisCPGarciaZ}, el protocolo de seguridad en el que se basa SEVI es una variante del de Lin-Hwang-Chang, el cual se compone de tres fases que cubren la seguridad del esquema en los m�dulos de votaci�n y generaci�n de resultados. Tambi�n con este protocolo, el acuse de recibo dado a los votantes sirve para cumplir con los requisitos de auditor�a. Otra de las necesidades resueltas es la de la democracia del proceso, pues el hecho de que se puedan identificar los votantes no honestos al emitir el voto aumenta la confianza en el proceso.
			
			No obstante, el protocolo de seguridad que respalda la mayor parte de la interacci�n con el sistema no cubre la fase de registro. Precisamente esta es la fase con la que comienza el sistema y donde se basa parte de la suposici�n de honestidad esperada por el protocolo para el resto del proceso. Para resolver el problema, en SEVI optan por establecer un canal seguro entre la m�quina cliente y la m�quina servidor a trav�s de un protocolo de transferencia segura \gls{SSL}.



% *************************************************************************************************************** %
% 			E2E AUDITABLE VOTING SYSTEMS
% *************************************************************************************************************** %				
		\section{Proyectos de Sistemas de Votaci�n Auditables Punto a Punto}\label{estadocuestion_e2everifiability}
			
				Una propiedad bastante �til para el desarrollo de esquemas para votaci�n por Internet es la llamada \textbf{\textit{verificabilidad punto a punto}} (o punta a cabo seg�n otros autores hispanohablantes) (en la literatura inglesa: end-to-end verifiability; E2E-verifiability). Con ella se puede solucionar el problema de la necesidad de confiar en el proceso que recoge, almacena y cuenta los votos.
				
				Se considera a Josh Benaloh\footnote{\url{http://research.microsoft.com/en-us/um/people/benaloh/}} como el precursor del concepto de verificabilidad \gls{E2E}. Seg�n su propia definici�n del t�rmino, se considera que los requisitos de un sistema completamente verificable \gls{E2E} son \cite{Benaloh:2006:SVE:1251003.1251008,cryptoeprint:2015:233,dhillon:2015}:
			
			\begin{enumerate}
				\item \textbf{Verificabilidad individual}: Los votantes pueden comprobar que sus votos se han registrado con la opci�n que han elegido.
				\item \textbf{Verificabilidad universal}: Cualquiera puede comprobar que todos los votos han sido escrutados con precisi�n.
			\end{enumerate}
			
			Con estas premisas, en \cite{dhillon:2015} se recoge un resumen sobre sistemas \gls{E2E} del cual se puede extraer que los sistemas utilizados para las votaciones en Estonia (\ref{ivotingEstonia}) y en Noruega (\ref{ivotingNoruega}) son casi completamente \gls{E2E} verificables.
			
			Para cumplir el requisito de la verificabilidad individual, el sistema estonio (\ref{ivotingEstonia}) se apoya en una app para smartphones que gestiona la verificaci�n del voto, mientras que en la soluci�n noruega (\ref{ivotingNoruega}) apuestan por un sistema de retorno de c�digo por SMS. A trav�s de ambos protocolos, se facilita al votante una herramienta con la cual asegurarse de que su voto ha sido correctamente incluido en la votaci�n.
			
			Los problemas para satisfacer los requisitos de Benaloh aparecen cuando se estudian las herramientas para cumplir con la verificabilidad universal.
			
			El m�todo tradicional para garantizar este requisito es utilizar las pruebas de conocimiento zero (\ref{primitivaPruebasConocimientoNulo}). Con estas pruebas se trata de convencer a un verificador de que el proceso se ha llevado a cabo correctamente con una alta probabilidad, sin necesidad de que el verificador tenga conocimiento del contenido de los votos incluidos en la elecci�n. En los sistemas implementados en Estonia y Noruega, aunque parecen tener herramientas que incluyen estas pruebas de conocimiento nulo en todas las etapas del proceso, seg�n algunos informes postelectorales\cite{carter2013}, esto no ocurr�a en la totalidad del mismo.
			
			Seg�n estos mismos informes\cite{carter2013}, un ejemplo de sistema electoral por Internet \gls{E2E} verificable es Helios Voting (\ref{estadocuestion_helios}), del que dicen que es un reconocido ``\textit{est�ndar en verificabilidad de voto por Internet}''. Las bonanzas de este sistema es que proporciona al proceso todas las garant�as que aporta la verificabilidad \gls{E2E}, como la capacidad de observar pruebas de conocimiento cero no interactivas que verifican que cada voto fue incluido correctamente y que el escrutinio completo fue computado con precisi�n. No obstante, hay que remarcar que, como su propio desarrollador indica, Helios es un sistema pensado en elecciones con bajo riesgo de coacci�n, lo cual no es v�lido para elecciones nacionales como el modelo estonio o noruego.
			
			Las garant�as que ofrecen los esquemas de verificabilidad \gls{E2E} eliminan la necesidad de los votantes de confiar tanto en los propios clientes (dispositivos, navegadores, sistemas operativos) que utilizan para votar, como los servidores y los trabajadores oficiales que administran los sistemas asociados a la recepci�n, descifrado y conteo de votos, que son los pilares cr�ticos en los que se basa un sistema de voto por Internet. Esto es as� porque los esquemas basados en verificabilidad \gls{E2E} aseguran la inviolabilidad del voto emitido hasta ser contado, eliminando la amenaza de interceptaci�n del mismo en cliente, canal de comunicaci�n o servidor de escrutinio.
			
			Con esta perspectiva de relajada confianza en los dispositivos clientes, se torna como un buen protocolo para el voto por Internet sin locales habilitados. Se debe a que en este marco altamente distribuido, la votaci�n se realiza desde entornos no controlados por el sistema, con lo que hay un alto riesgo de amenazas, pero la verificiabilidad \gls{E2E} se encarga de demostrar la invariabiliad del voto emitido frente al escrutado, por lo que minimiza varios de los riesgos inherentes a los clientes no controlados. (Obvia comentar que sigue habiendo bastantes amenazas que este medio no va a tratar y tendr�n que ser enfrentadas con otras herramientas.)
			
			
			En esta secci�n, vamos a recoger e introducir brevemente, tres ejemplos de proyectos desarrollados con el objetivo de cumplir con las premisas de la verificabilidad punto a punto  (\gls{E2E}-verifiability).
			
			De los tres sistemas, ADDER (\ref{estadocuestion_adder}) es un desarrollo acad�mico, �gora Voting(\ref{estadocuestion_agoravoting}) es un sistema desarrollado en Espa�a con un amplio historial de procesos electorales llevados a cabo, al igual que Helios Voting(\ref{estadocuestion_helios}), considerado un est�ndar de facto de votaci�n basada en verificabilidad \gls{E2E}.
			
				\subsection{ADDER}\label{estadocuestion_adder}
					\textsc{Adder} \cite{conf_acsac_KiayiasKW06} se define como un sistema de voto electr�nico basado en Internet, libre y de c�digo abierto. Desarrollado por la Universidad de Connecticut (EEUU) en 2006, supone una plataforma de eVoting completamente funcional con una serie de caracter�sticas de seguridad como robustez, privacidad del voto, auditabilidad y verificabilidad.
					
					Los desarrolladores de \textsc{ADDER} dividen el voto por Internet en 3 escenarios:
					\begin{description}
						\item[Remoto] En el escenario del voto por Internet remoto, un actor diferente a la autoridad electoral, ya sea el votante o un tercero, es el que tiene el control sobre el cliente de voto y el entorno operativo.
						\item[Kiosko] En el escenario del voto por Internet en modo kiosko, el cliente de voto puede ser instalado por las autoridades de la elecci�n, pero entorno de la votaci�n est� fuera de su control.
						\item[Cabina de voto] En este escenario, las autoridades electorales tienen control tanto del cliente de voto como del entorno en el que se lleva a cabo.
					\end{description}
					
					\textsc{ADDER} fue dise�ado para el primero de los escenarios, el voto telem�tico remoto, pero dependiendo de los requisitos de seguridad, es adaptable a los otros dos. Adem�s, es un sistema capaz de llevar a cabo elecciones tanto a peque�a como a gran escala.
					
					\begin{figure}[h]
						\centering
							\includegraphics[width=\textwidth]{imgs/ADDER_diagrama_secuencia.png}
						\caption{Diagrama de secuencia del procedimiento para una elecci�n con ADDER \cite{conf_acsac_KiayiasKW06}.}
						\label{fig:ADDER_diagrama_secuencia}
					\end{figure}
					
					Procedimiento de la elecci�n:
					
					\begin{itemize}
						\item El primer paso en el procedimiento del proceso electoral con este sistema es que el administrados ha de alimentar al sistema con la lista de usuarios (votantes y autoridades) y candidatos.
					
						\item Las autoridades acceden al sistema para pariticipar en la generaci�n de claves criptogr�ficas. Se genera una clave p�blica para el sistema y privadas para cada una de las autoridades.
					
						\item En el per�odo de votaci�n, cada votante ingresa en el sistema y descarga la clave p�blica de �ste. Con esta clave cifra su voto. El voto cifrado se guarda en un �rea reservada al votante.
					
						\item Al finalizar el per�odo de votaci�n, el servidor cuenta los votos y publica el resultado cifrado.
					
						\item Las autoridades proporcionan informaci�n basada en el resultado cifrado y sus claves privadas para descifrar el resultado. El sistema combina estas claves individuales y los descifrados parciales del resultado para componer el resultado electoral, que se publica.
					
						\item El sistema, por tanto, se implementa alrededor de un tabl�n de anuncios, un servidor de autenticaci�n (\textit{gatekeeper}) y un cliente. Haciendo uso de secreto compartido a la hora de repartir trozos de la clave privada del sistema y del resultado cifrado entre las diferentes autoridades con responsabilidad en el sistema.

					\end{itemize}					
					
					\begin{figure}[h]
						\centering
							\includegraphics[width=\textwidth]{imgs/ADDER_diagrama.png}
						\caption{Arquitectura de ADDER.}
						\label{fig:ADDER_diagrama}
					\end{figure}
					
					Los objetivos que persigue el sistema \textsc{ADDER} son:
					\begin{description}
						\item[Transparencia] Toda la informaci�n del tabl�n de anuncios es p�blica y puede ser consultada por cualquier observador. Aqu� se incluyen votos cifrados, claves p�blicas y escrutinios.
						\item[Verificabilidad universal] Cualquier resultado electoral obtenido por el sistema deber�a ser verificado por cualquier observador. A trav�s de los logs y trazas del sistema se puede realizar una auditor�a de cualquiera de los procesos.
						\item[Privacidad] Todos los votantes pueden confiar en que sus votos se mantienen secretos. S�lo el recuento es accesible al p�blico.
						\item[Confianza distribuida] Cada procedimiento del proceso electoral est� supervisado por varias autoridades. El recuento no puede llevarse a cabo sin la cooperaci�n de un determinado n�mero de autoridades.
					\end{description}
					
					
					
					Soluciones propuestas por \textsc{ADDER}:
					\begin{enumerate}
						\item \textit{Autenticaci�n de usuarios}. Para la autenticaci�n de usuarios, \textsc{ADDER} emplea un sistema an�logo a Kerberos que denominan \textit{gatekeeper}.
						\item \textit{Privacidad del voto}. Para contrarrestar el conflicto entre privacidad del voto y verificabilidad universal y la necesidad de acceso al contenido del voto para asegurar el correcto conteo, el sistema utiliza t�cnicas de cifrado homom�rfico.
						\item \textit{Verificabilidad universal}. Para cumplir con este requisito, el sistema se apoya en un tabl�n de anuncios, en el que se publica informaci�n relevante al proceso. Junto al sistema, se han implementado una serie de herramientas libres y de c�digo abierto que permiten hacer uso de los datos publicados en el tabl�n y realizar una serie de tareas:
						\begin{enumerate}
							\item \textit{Recuento de los votos encriptados}. Gracias a las propiedades del cifrado homom�rfico, no necesita usar claves privadas para contar los votos, pues no es necesario que los descifre para realizar el recuento. El programa puede repetir el proceso ejecutado en el servidor.
							\item \textit{Verificaci�n de todas las pruebas}. Cada votante puede comprobar la pruebas de validaci�n de su voto.
							\item \textit{Descifrado del recuento final}. Una vez que todas las autoridades han terminado sus descifrados parciales, la suite de verificaci�n recalcula los coeficientes de Lagrange y desencripta la suma final.
							\item \textit{Verificaci�n del hash}.
						\end{enumerate}
						\item Verificabilidad del votante.
						
					\end{enumerate}
					
					
					
					
				
				\subsection{�gora Ciudadana - �gora Voting}\label{estadocuestion_agoravoting}
					
					\begin{wrapfigure}{l}{.2\textwidth}
						\vspace{-30pt}
						\begin{center}
							\includegraphics[width=.2\textwidth]{imgs/agoraVotingLogo}
						\end{center}
						\vspace{-20pt}
					\end{wrapfigure}
					�gora Voting se inici� como un proyecto de software libre cuyo objetivo era el de proporcionar una plataforma de democracia l�quida, incluyendo un protocolo de voto criptogr�ficamente seguro.
					
					

					
					Este sistema se compone de compone de varios elementos:
					\begin{description}
						\item [Registro] \hfill \\
							Plataforma web que sirve para la verificaci�n de votantes.
							\begin{itemize}
								\item \gls{API} web y AngularJS \textit{single web application}.
								\item Base de datos con acceso al padr�n.
								\item Servidor con certificado \gls{TLS}.
								\item Fail2ban\footnote{\url{https://www.fail2ban.org}} y Cloudfare\footnote{\url{http://www.cloudfare.com}} para protecci�n contra \gls{DDoS}.
								\item Redundancia de Hardware.
							\end{itemize}
							
						\item [Cabina de voto] \hfill \\
							Plataforma web a trav�s de la cual los votantes emiten sus votos.
							\begin{itemize}
								\item Servidor con validaci�n \gls{TLS}.
								\item Cabina de voto Javascript para votar o auditar.
								\item Librer�as de cifrado Javascript (Helios, SJCL).
								\item Base de datos PostgreSQL replicada para la urna electr�nica.
								\item Fail2ban para protecci�n ante DOS y ataques de fuerza bruta.
								\item Cloudfare para protecci�n ante ataques DOS.
								\item Autenticaci�n de cliente para votantes registrados y validados.
							\end{itemize}
							
						\item [Servicios de la Autoridad de la Elecci�n] \hfill  \\
							Servicios web http que sirven para generar claves p�blicas, mezclar, descifrar y totalizar.
							\begin{itemize}
								\item Cola HTTP para orquestraci�n as�ncrona.
								\item Validaci�n \gls{TLS} en cliente y servidor.
								\item Librer�a Verificatum\footnote{\url{http://www.verificatum.com}} para mixnets.
								\item Librer�a OpenSTV\footnote{\url{http://www.opavote.com/openstv}} para totalizaci�n.
							\end{itemize}
							
						\item [Verificador de la Elecci�n] \hfill \\ 
							Una aplicaci�n Python/Java \textit{stand alone} que verifica los datos publicados de la elecci�n.
					\end{description}
					
					
					Las primitivas criptogr�ficas que se emplean en este sistema son:
					\begin{itemize}
						\item Esquema de cifrado homom�rfico ElGamal
						\item Esquema de cifrado de umbral Pedersen.
						\item Mixnet verificable universal con pruebas de conocimiento cero.
						\item Heur�stica Fiat-Shamir para transformar pruebas de  conocimiento cero de verificabilidad en prueba verificables no interactivas.
					\end{itemize}
					
				
				Comentar que �gora Voting dej� de ser un proyecto activo y pas� de software libre a ser una plataforma comercial de voto por Internet llamada nVote. Est� implementada en Python y tiene bastantes funcionalidades comunes con Helios Voting.
				
				
				
				
				
				
				En �gora Voting se observa que hay una gran influencia de Helios Voting, especialmente en el inicio de ambos proyectos.
				
				La diferencia fundamental en la evoluci�n de sendos proyectos se encuentra en puntos como que �gora continu� utilizando mixnets para desacoplar votos de los votantes mientras que Helios dej� de utilizar este protocolo, virando hacia el homomorfismo en la totalizaci�n.
				
				Como se ha adelantado, �gora Voting dej� de ser un proyecto de software libre. Todav�a se puede encontrar en Github\footnote{\url{https://github.com/agoravoting}}, pero sus creadores han decidido desarrollar un producto comercial de nombre nVotes\footnote{\url{https://nvotes.com}}.
				
				Este proyecto ha sido utilizado en varios procesos electorales reales. Varios de ellos han sido votaciones interna de partidos pol�ticos, como Podemos o sus partidos asociados.
				
				La �ltima gran elecci�n llevada a cabo por nVotes ha sido la de Decide Madrid\footnote{\url{https://decide.madrid.es}} en 2017.
				\hfill \\
				
				
				\subsection{Helios}\label{estadocuestion_helios}
					\begin{wrapfigure}{l}{.2\textwidth}
						\vspace{-20pt}
						\begin{center}
							\includegraphics[width=.2\textwidth]{imgs/heliosLogo}
						\end{center}
						\vspace{-20pt}
					\end{wrapfigure}
					Helios Voting\footnote{\url{https://github.com/benadida/helios}} es un sistema de voto por Internet cuyos desarrolladores consideran que es el primer sistema de voto con auditor�a abierta (open-audit) basado en web. Actualmente es un proyecto en desarrollo, funcional y p�blicamente accesible. Cualquier organismo interesado puede descargar el c�digo fuente, configurar un proceso electoral y llevar a t�rmino la elecci�n, junto con que cualquier observador puede auditar todo el proceso.

					El creador del proyecto es Ben Adida\footnote{\url{http://ben.adida.net/}}, doctor en Ingenier�a Inform�tica en Criptograf�a y Seguridad de la Informaci�n por el \gls{MIT}\footnote{\url{http://mit.edu/}}. Para su doctorado, base de este proyecto, fue asesorado\footnote{\url{https://vote.heliosvoting.org/about}} por otro Doctor como Ron Rivest\footnote{\url{http://people.csail.mit.edu/rivest/}}, experto en Criptograf�a y Voto Electr�nico. 
					
					Para ver el rigor acad�mico del proyecto desde sus inicios y durante su desarrollo, Adida tambi�n ha sido asesorado por profesionales de reputada experiencia en la criptograf�a y el Voto Electr�nico como son Lawrence Lessig\footnote{\url{https://es.wikipedia.org/wiki/Lawrence_Lessig}}, de la Universidad de Stanford\footnote{\url{https://www.stanford.edu/}} Y Josh Benaloh\footnote{\url{https://www.microsoft.com/en-us/research/people/benaloh/}}, doctor por el \gls{MIT} y Yale\footnote{\url{https://www.yale.edu/}}, considerado el precursor de los sistemas de verificabilidad punto a punto (\ref{estadocuestion_e2everifiability}).				
					
					
				
				Este proyecto es apropiado para realizar procesos electorales para organismos que necesiten que estos sean confiables y con voto secreto, aunque eso s�, siempre que los comicios se celebren en un ambiente en el que la coacci�n del voto no sea una amenaza. Este detalle es importante, ya que muestra una de las \textit{debilidades} de esta implementaci�n con respecto a un proceso electoral de gran escala.
				
				Helios trata de ser un sistema de voto por Internet simple comparado con otros protocolos criptogr�ficos, centr�ndose en la \textit{auditabilidad p�blica} como elemento diferencial. Con esta propiedad, cualquier organismo puede apoyarse en Helios para llevar a cabo la elecci�n y, aunque resultase que Helios estuviera corrupto, la integridad de la elecci�n puede ser verificada por los observadores.
				
				El proyecto Helios ha sufrido una evoluci�n desde sus primeras versiones a la actual, cambiando protocolos y esquemas criptogr�ficos y superando vulnerabilidades que se le iban encontrando. Un buen documento para consultar esta evoluci�n desde la primera versi�n de Helios hasta la �ltima de 2016 es \cite{cryptoeprint:2015:233}.
				
				En este documento \cite{cryptoeprint:2015:233} se indica que las versiones de Helios 2.0 y 3.1.4,  sufr�an vulnerabilidades que permit�an violar el secreto de voto y la verificabilidad. La versi�n de 2012 redujo esta vulnerabilidad, aunque sigui� siendo insuficiente para asegurar la verificabilidad. La versi�n 2016 implementa un sistema de autenticaci�n externa que permite satisfacer el peligro de violaci�n de la verificabilidad.
				
				
				Las primitivas criptogr�ficas que encontramos en este sistema de votaci�n por Internet son \cite{Maene:2014}:  
					\begin{itemize}
						\item Cifrado homom�rfico multiplicativo: ElGamal (\ref{primitivaCifradoHomomorfico_elgamal})
						\item Cifrado homom�rfico aditivo: ElGamal Exponencial (\ref{primitivaCifradoHomomorfico_elgamal_exponencial})
						\item Cifrado de umbral (+ Secreto Compartido) (\ref{primitivaSecretoCompartido})
					\end{itemize}
				 
				A lo largo de esta memoria se desgranar�n las particularidades de este sistema (\ref{planteamiento_HeliosVoting}, \ref{arquitectura_servidorVotacion}), sus ventajas e inconvenientes y el por qu� ha sido el elegido como base del Sistema (\ref{analisis_integracion_sistemaVotacion}).
	
				
\iffalse				
\notasInfo[inline]{la siguiente info para analizar est� sacada de \url{https://eprint.iacr.org/2015/942.pdf}}
\fi
%TEMPORAL ------------------------------------------------------------------------
				%6 Case Study: Helios
%Helios is an open-source, web-based electronic voting system,18 which has been
%deployed in the real-world: the International Association of Cryptologic Research
%(IACR) has used Helios annually since 2010 to elect board members
%[BVQ10,HBH10],19 the Catholic University of Louvain used Helios to elect their
%university president in 2009 [AMPQ09], and Princeton University has used Helios
%since 2009 to elect student governments [Adi09].20
%Informally, Helios can be modelled as an election scheme (Setup, Vote,Tally)
%such that:
%Setup generates a key pair for an asymmetric homomorphic encryption scheme,
%proves correct key generation in zero-knowledge, and outputs the public
%key coupled with the proof.
%Vote enciphers the vote to a ciphertext, proves correct ciphertext construction
%in zero-knowledge, and outputs the ciphertext coupled with the proof.
%Tally proceeds as follows. First, any ballots on the bulletin board for which
%proofs do not hold are discarded. Secondly, the ciphertexts in the remaining
%ballots are homomorphically combined,21 the homomorphic combination
%is decrypted to reveal the election outcome, and correctness of
%decryption is proved in zero-knowledge. Finally, the election outcome and
%proof of correct decryption are output.
%The original Helios scheme [AMPQ09] is vulnerable to attacks against ballot secrecy
%[CS13, CS11, SC11]. The current version of Helios is intended to mitigate
%against these attacks.22 In particular, it incorporates Smyth's recommendation
%to reject ballots containing zero-knowledge proofs that have been previously
%observed [Smy12, �4]. For clarity, we write Helios 3.1.4 for the current version
%of Helios. Bernhard [Ber14, �6.11] and Bernhard et al. [BCG+15a, �D.3]
%show that variants of Helios 3.1.4 using the strong Fiat-Shamir heuristic satisfy
%notions of ballot secrecy.23 These notions assume ballots are recorded-as-cast,
%i.e., cast ballots are preserved with integrity through the ballot collection process
%[AN06, �2]. Unfortunately, ballot secrecy is not satisfied without this assumption,
%because Helios 3.1.4 uses malleable ballots (as do the variants studied
%by Bernhard [Ber14] and Bernhard et al. [BCG+15a]), which are incompatible
%with ballot secrecy (�5).
%Theorem 9. Helios 3.1.4 does not satisfy ballot secrecy.
%Proof sketch. Suppose an adversary calls the left-right oracle to derive a ballot,
%exploits malleability to derive a related ballot, and outputs a bulletin board
%containing the related ballot.24 The board is balanced, because it does not
%contain the ballot output by the left-right oracle. And the election outcome will
%allow the adversary to win the game.
%We omit a formal proof of Theorem 9.
%The proof sketch of Theorem 9 does not immediately give way to a real-world
%attack against Helios. Nevertheless, we can derive an attack (as the following
%example demonstrates) by extrapolating from the proof sketch and Cortier \&
%Smyth's permutation attack, which asserts: given a ballot b for vote v, we can
%exploit malleability to derive a ballot b'
%for vote v'
%[CS13, �3.2.2]. Suppose
%Alice, Bob and Charlie are voters, and Mallory is an adversary that wants to
%convince herself that Alice did not vote for a candidate v. Further suppose
%Alice casts a ballot b1 for vote v1, Bob casts a ballot b2, and Charlie casts a
%ballot b3. Moreover, suppose that either Bob or Charlie voted for v. (Thus,
%we exclude election outcomes without any votes for candidate v, which would
%permit Mallory to trivially convince herself that Alice did not vote for candidate
%v.) Let us assume that votes for v
%0 are not expected. Mallory proceeds as follows:
%she intercepts ballot b1, exploits malleability to derive a ballot b such that v = v1
%implies b is a vote for v
%0
%, and casts ballot b. It follows that the tallier will compute
%the election outcome from bulletin board {b, b2, b3}. If the outcome does not
%contain any votes for v
%0
%, then Mallory is convinced that Alice did not vote for v.
%This attack also works against the variants of Helios 3.1.4 studied by Bernhard
%and Bernhard et al., however, neither Bernhard [Ber14, �6.11] nor Bernhard et
%al. [BCG+15a, �D.3] were able to detect the attack,25 because interception is
%not possible when ballots are recorded-as-cast.26
%Recommendation: adopt non-malleable ballots. We have seen that nonmalleable
%ballots are necessary for ballot secrecy (�5), hence, future Helios releases
%should adopt non-malleable ballots. The specification for the next Helios
%release [Adi14] makes some progress in this direction. Moreover, a liberal
%interpretation of that specification by Smyth, Frink \& Clarkson [SFC15]
%leads to a variant of Helios, named Helios 4.0, which defines non-malleable ballots
%[SHM15]. Proving whether Helios 4.0 satisfies ballot secrecy is a direction
%for future work. And a successful proof would provide strong motivation for
%future Helios releases being based upon Helios 4.0.
			
		
			
			\subsection{Comparaci�n �gora - Helios}\label{comparativaAgoraVoting}
				En la tesis de Codina Lligo�a \cite{Codina:tesis}, el autor realiza un estudio de diversos sistemas de votaci�n, entre ellos �gora Voting (en su versi�n agora-ciudadana 2.0) y Helios 4.
				
				Entre las m�ltiples caracter�sticas que valora, recoge informaci�n acerca de la interfaz, licencia, requisitos de voto electr�nico, antecedentes y comunidad de soporte del proyecto.
				
				Los resultados de la comparativa los resume en la tabla \ref{tab:tablaComparativaAgoraHelios}, en la cual los valores van desde el 0 (valoraci�n negativa) al 2 (valoraci�n positiva).
			
				
				\begin{table}[ht]
				\centering
					%\begin{tabular}{0.75\textwidth}@{\extracolsep{\fill}}|l|l|m{3cm}| m{3cm}|}
					%\begin{tabular*}{0.95\textwidth}{@{\extracolsep{\fill}}|l|l|rm{2cm}|rm{1cm}|}
					\begin{tabular*}{0.95\textwidth}{@{\extracolsep{\fill}}|l|r|r|}
						\hline
%							\textbf{Elecci�n} & \textbf{Tipo} & \textbf{Votantes por internet} & \textbf{\% sobre total votantes} \\
							\textbf{Criterio} & \textbf{agora-ciudadana 2.0} & \textbf{Helios 4} \\
						\hline
							Especificaciones IMI 			& 2			& 2 	\\
						\hline		
							Interfaces 								& 2			& 2 	\\
						\hline		
							Licencia 									& 2			& 2 	\\
						\hline		
							Redes Sociales 						& 2			& 2 	\\
						\hline
							e-Voting (Seguridad) 			& 1,04	& 1,46 		\\
						\hline
							Elegibility 							& 1,75	& 1,5 	 \\
						\hline
							Democracy 								& 0,33 	& 0,33  \\
						\hline	
							Privacy 									& 0,85 	& 1,92  \\
						\hline	
							Verifiability 						& 0,54 	& 1,54  \\
						\hline
							Fairness 									& 1,33	& 1,4		\\
						\hline
							Robustness 								& 0,67 	& 1,67		\\
						\hline
							Coercion-Resistance 			& 1 		& 1		\\
						\hline		
							Receipt-Freeness 					& 2 		& 2		\\
						\hline
							Correctness 							& 0,88 	& 1,75		\\
						\hline
							M�todos de votaci�n 				& 2 		& 2		\\
						\hline
							Desarrollo 								& 1,75 	& 1,38		\\
						\hline	
							Antecedentes del proyecto & 1,75 	& 1,75		\\
						\hline
							Actividad del proyecto 		& 2 		& 1,75		\\
						\hline
							Gobierno del proyecto 		& 1,5 	& 1,25		\\
						\hline
							Nivel de industrializaci�n del proyecto & 1,75 & 0,75		\\
						\hline
					\end{tabular*}
				\caption{Comparativa �gora - Helios, por Codina Lligo�a \cite{Codina:tesis}}
				\label{tab:tablaComparativaAgoraHelios}
			\end{table}
			
			
			\begin{figure}[!ht]
	    		\centering
	    		\begin{minipage}{0.45\textwidth}
	    		    	\centering
	
				\includegraphics[height=6cm]{imgs/comparativaAgoraHelios01.png}
	
				\caption{Comparativa �gora - Helios en cuanto a caracter�sticas de Proyecto, por Codina Lligo�a \cite{Codina:tesis}}
	
				\label{fig:estadocuestion.comparativa.agora.helios.proyecto}
	    		\end{minipage}\hfill
	    		\begin{minipage}{0.45\textwidth}
	        		\centering
	
					\includegraphics[height=6cm]{imgs/comparativaAgoraHelios02.png}
	
					\caption{Comparativa �gora - Helios en cuanto a requisitos del Voto por Internet de Proyecto, por Codina Lligo�a \cite{Codina:tesis}}
				\label{fig:estadocuestion.comparativa.agora.helios.requisitos}
	    		\end{minipage}
			\end{figure}

			
			En el estudio, el autor divide las valoraciones en dos, las correspondientes al nivel de desarrollo de los proyectos y a las caracter�sticas de seguridad.
			
			En las primeras, da una mejor valoraci�n al proyecto �gora Voting. Sin embargo, en el segundo grupo, el que se corresponder�a con la mayor�a de los requisitos inherentes al voto electr�nico enunciadas por diferentes autores, la mejor nota se la lleva el proyecto Helios Voting, m�s correcto desde el punto de vista acad�mico y de la seguridad en el voto remoto.
			
			Seg�n observa, Helios supera a �gora Voting en estas caracter�sticas porque fue directamente dise�ado teni�ndolas en cuenta, por lo que desde el origen toda la informaci�n se almacena cifrada y existen procedimientos de auditor�a, cumpliendo con los requisitos de Privacidad y Verificabilidad del voto.
			
			Destaca que en ambos sistemas es posible emitir m�s de un voto por votante, aunque s�lo uno de ellos ser� contado.
			
			% *************************************************************************************************************** %
			% 			PRIMITIVAS DE CRIPTOGRAF�A
			% *************************************************************************************************************** %
			
			\section{Mecanismos criptogr�ficos}\label{mecanismosCriptograficos}
			Los sistemas de votaci�n electr�nica son sistemas cr�ticos en cuanto a que la informaci�n que tratan es muy sensible. Han de satisfacer requisitos de anonimato, secreto, verificabilidad, completitud, etc. los cuales requieren que la protecci�n de votante, voto y sistema sea absoluta.
			
			Para conseguir cumplir estos requerimientos b�sicos de seguridad, este tipo de sistemas han de apoyarse en las matem�ticas, concretamente en la criptograf�a. Por ello, existen una serie de mecanismos criptogr�ficos que aportan las herramientas, dise�os y soluciones necesarias para tratar este problea y minimizarlo.
			
			Una primera clasificaci�n de mecanismos criptogr�ficos distingue entre primitivas, esquemas y protocolos criptogr�ficos:
			%% https://www.google.es/url?sa=t&rct=j&q=&esrc=s&source=web&cd=8&ved=0CEsQFjAH&url=http%3A%2F%2Fthoth.cc.e.ipl.pt%2Fclasses%2FSI%2F1314i%2FLI51N-MI1N%2Fresources%2F2225&ei=l4qFVLrYBIavU8zugNAP&usg=AFQjCNEqu0pixZSTmH8THWs9eI9dOQpY9Q&sig2=RgWbVJjYwHyWYOrqoehkEA&bvm=bv.80642063,d.d24&cad=rja
				\begin{description}[font=$\bullet$\ \ ]
					\item[Primitivas] Operaciones matem�ticas, usadas como bloques constructores en la realizaci�n de esquemas. Su caracterizaci�n depende de los problemas matem�ticos que sustentan su uso criptogr�fico. Ej: DES, RSA.
					\item[Esquemas] Combinaci�n de primitivas y m�todos adicionales para la realizaci�n de tareas criptogr�ficas como la firma y el cifrado digital. Ej: DES-CBC-PKCS5Padding; RSA-OAQEP-MGF1-SHA1
					\item[Protocolos] Secuencias de operaciones, a realizar por dos o m�s entidades, que contienen esquemas y primitivas con el prop�sito de dotar a una aplicaci�n de caracter�sticas de seguridad. Ej: TLS %TLS_RSA_WITH_DES_CBC_SHA
%\notasDuda{TLS_RSA_WITH_DES_CBC_SHA}
				\end{description}
			
			\subsection{Primitivas criptogr�ficas}\label{primitivasCriptografia}
				Dentro de los retos tecnol�gicos que propone el voto electr�nico, uno de los m�s importantes es la seguridad. Para poder implementar un sistema seguro que pueda soportar toda la infraestructura necesaria para poder poner en marcha un sistema de voto electr�nico confiable hay que hacer uso de herramientas que sean capaces de asegurar las comunicaciones y el secreto de estas. Es en este escenario donde la criptograf�a es el n�cleo de la soluci�n.
				
				Los requerimientos que se tratan de satisfacer con el uso de la criptograf�a son \cite{tesisSeguridadVMMorales}:
				\begin{itemize}
					\item Privacidad del voto
					\item Autenticaci�n del votante
					\item Integridad de la elecci�n
				\end{itemize}
				
				En la bibliograf�a sobre criptograf�a se establece que existen tres aproximaciones generales para dise�ar sistemas de voto electr�nico bas�ndose en primitivas criptogr�ficas robustas:
				\begin{description}
					\item[Basadas en mixnets] introducidas por David Chaum \cite{Chaum:1981:UEM:358549.358563}
					\item[Basadas en encriptado homom�rfico] introducido por Josh Benaloh \cite{Benaloh:thesis:1987}
					\item[Basadas en firma ciega] introducida por Fujioka et al. \cite{Fujioka93}
				\end{description}
				
				Antes de entrar en los diferentes esquemas de voto electr�nico (\ref{esquemasVotoElectronico}), introducimos una serie de primitivas criptogr�ficas que se utilizan en ellos.				
				
				
					% *************************************************************************************************************** %
					% 			FIRMA CIEGA
					% *************************************************************************************************************** %
					\subsubsection{Firma ciega}\label{primitivaFirmaCiega}
						Los protocolos criptogr�ficos de firma ciega se dan lugar entre dos agentes, un usuario U y un firmante F de forma que F firma digitalmente una serie de datos comunicados por U sin conocer el contenido de estos. \\
						El objetivo de este tipo de protocolos es proporcionar una serie de datos firmados cuyo contenido solamente sea conocido por el actor que env�a, siendo completamente desconocidos para el actor que los firma.\\
						Los protocolos de firma ciega se basan en dos componentes \cite{firmaCiega2007}:
								 \begin{enumerate}
										\item Un protocolo de firma digital. Desarrollado por el actor F, quien es el prestador del servicio de firma.  De tal forma, $S(m)$ es la notaci�n de la firma digital del mensaje $m$.
										\item Dos funciones $f$ y $g$, conocidas �nicamente por el usuario U, de forma que
							 $$g(S(f(m))) = S(m)$$
											
							 La funci�n $f$ se denomina \textit{funci�n de ocultaci�n o de opacidad}. La funci�n $g$ es la \textit{funci�n de recuperaci�n}.
								\end{enumerate}
							
							% Una opci�n, propuesta en el trabajo \cite{firmaCiega2007}, es utilizar el protocolo de firma ciega desarrollado por Chaum \cite{chaumFirmaCiega} basado en el criptosistema RSA. Dicho protolo se podr�a definir as�.
							% Sea $n = p * q$ el producto de dos primos aleatorios suficientemente grandes. El firmante F usar�a el esquema de firma digital RSA con clave p�blica $(n, e)$ y clave privada $d$. Sea k un entero aleatorio tal que $mcd(n, k) = 1$. Las funciones que utiliza el usuario U son:
							% $$f: Zn \longrightarrow Zn$$
							% $$m \longmapsto f(m) = m \cdot k^e \mod{n}$$
							% $$g: Zn \longrightarrow Zn$$
							% $$m \longmapsto g(m) = k^{-1} \cdot \mod{n}$$
							% Sustituyendo (2) y (3) en (1):
							% $$g(S(f(m))) = g(S(m \cdot k^e \mod{n}))
								% = g(m^d \cdot k \mod{n}) = m^d \mod{n} = S(m)$$
						% \end{enumerate}
						
						% \begin{description}x
							% \item[1. Fase de inicailizaci�n]. Sea $0 \leq m \leq n - 1$ el mensaje originado por U y que debe ser firmado por F, y sea $k$ un entero aleatorio elegido por F tal que $0 \leq k \leq n - 1$ y $mcd(k, n) = 1$.
							% \item[2. Fase de 	ocultaci�n]. U calcula $m^* = f(m) = m \dot k^e \mod{n}$ y se lo env�a a F.
							% \item[3. Fase de firma]. F calcula $s^* = S(m^*) = (m^*)^d \mod{n}$ y se lo env�a a U.
							% \item[4. Fase de recuperaci�n]. U calcula $s = S(m) = g(S(m^*)) = k^{-1} \dot s^* \mod{n}$ que es la firma digital del mensaje m por F.
						% \end{description}
					
					%% ************** Merece la pena seguir contando el protocolo propuesto por este art�culo?
					
					% *************************************************************************************************************** %
					% 			SECRETO COMPARTIDO
					% *************************************************************************************************************** %
					\subsubsection{Secreto compartido}\label{primitivaSecretoCompartido}
						Los protocolos criptogr�ficos de secreto compartido dividen un mensaje (secreto) determinado en diferentes fragmentos que se reparten entre los participantes de la comunicaci�n.
						El reparto de informaci�n consiste en los siguientes preceptos:
						\begin{enumerate}
							\item El mensaje (secreto) original �nicamente puede ser reconstruido por un cierto grupo de participantes autorizados.
							\item Los participantes no autorizados no pueden obtener informaci�n sobre el contenido del mensaje original.
						\end{enumerate}
						
						%%% http://upcommons.upc.edu/revistes/bitstream/2099/9843/1/Article018.pdf
						
					
					
					% *************************************************************************************************************** %
					% 			PRUEBAS DE CONOCIMIENTO NULO
					% *************************************************************************************************************** %
					\subsubsection{Pruebas de conocimiento nulo}\label{primitivaPruebasConocimientoNulo}
						Los protocolos basados en pruebas de conocimiento cero o nulo son protocolos criptogr�ficos que se basan en la necesidad de una de las partes en poder demostrar a otra que un enunciado es cierto sin revelar nada m�s que la veracidad del mismo.
						
						Goldwasser, Micali y Rackoff\cite{Goldwasser:1985:KCI:22145.22178} propusieron tres propiedades que deben satisfacer todos los protocolos basados en pruebas de conocimiento cero.					
						\begin{description}
							\item[Completeness]: Si el emisor dice la verdad, en alg�n momento, convencer� al receptor de ello.*
							
							\item[Soundness]: El emisor s�lo puede convencer al receptor si realmente est� diciendo la verdad.
							
							\item[Zero-knowledgeness]: El receptor no conoce nada acerca de la soluci�n real del emisor.
						\end{description}
					
						* A base de repetici�n de una soluci�n, alguien con reputaci�n es capaz de acabar convenciendo al receptor, con hechos, de que lo que demuestra es cierto.
						
						Un sencillo ejemplo para entender el concepto de este tipo de protocolos lo encontramos en una publicaci�n el blog personal de Pablo Della Paolera\cite{dellaPaolera2014}, astr�nomo de la Universidad Nacional de La Plata (Argentina). En este texto, el cient�fico explica que se puede demostrar \textit{algo} correctamente sin necesidad de demostrar \textit{por qu�}.
						
%			\begin{center}
%				\small
%				\fbox{
%					\begin{minipage}{0.9\textwidth}
%						Bas�ndonos en el ejemplo publicado, supongamos dos actores $Marta$ y $Carmen$. 
%						
%						$Carmen$ quiere demostrar que \textit{$Marta$ tiene (o no) la misma cantidad de monedas en su bolsillo izquierdo que en el derecho}. 
%						
%						Para ello, la forma m�s simple de hacerlo ser�a que $Carmen$ le pidiese a $Marta$ las monedas de cada bolsillo y las contase. Por tanto, para demostrar la afirmaci�n de que \textit{$Marta$ realmente tiene (o no) las mismas monedas en cada bolsillo} basta con que las ense�e.
%						
%						En este caso, el problema es que $Carmen$ est� violando la privacidad de $Marta$ pues no deber�a ser necesario que $Carmen$ conozca cu�ntas monedas tiene $Marta$ y, mucho menos, tener que ense�arlas a la audiencia para demostrar la veracidad de sus investigaciones.
%						
%						Esta violaci�n de la privacidad se puede superar aplicando de forma simple una soluci�n basada en prueba de conocimiento cero.
%						
%						Supongamos que $Marta$ tiene $X$ monedas en el bolsillo derecho e $Y$ monedas en el izquierdo. Para no conocer el n�mero de monedas que posee $Marta$, $Carmen$ le pide que piense en un n�mero $Z$ entero y mayor que $X$ e $Y$. A continuaci�n le pide que le diga la diferencia entre $Z$ y $X$ y entre $Z$ e $Y$. Si ambas diferencias son iguales, $Marta$ tiene el mismo n�mero de monedas en ambos bolsillos, del mismo modo que si las diferencias no coinciden, se puede afirmar lo contrario, incluso sabiendo en qu� bolsillo hay un mayor n�mero de monedas. Y todo esto \textbf{sin que $Carmen$ llegue a conocer en ning�n momento cu�ntas monedas posee $Marta$}.
%						
%						Matem�ticamente, se tratar�a de un sistema de 2 ecuaciones con 3 inc�gnitas, por lo que no se puede resolver y no se pueden obtener los valores de $X$ e $Y$:
%						$$Z - X = C$$
%						$$Z - Y = D$$
%						
%						Pese a que no es resoluble, sabiendo los valores de $C$ y $D$ se puede demostrar en qu� bolsillo tiene m�s monedas ($C > D$ � $D > C$) o si se tienen las mismas ($C = D$), sin necesidad de conocer los valores reales.
%						\end{minipage}
%					}
%				\end{center}
%					



\begin{center}
\begin{tcolorbox}[breakable, enhanced, colback=white, opacityframe=.1, width=0.9\linewidth]
%				\begin{center}
%				    \begin{varwidth}{0.8\linewidth}
%	        {\color{blue}
Bas�ndonos en el ejemplo publicado, supongamos dos actores $Marta$ y $Carmen$. 
						
						$Carmen$ quiere demostrar que \textit{$Marta$ tiene (o no) la misma cantidad de monedas en su bolsillo izquierdo que en el derecho}. \\
						
						Para ello, la forma m�s simple de hacerlo ser�a que $Carmen$ le pidiese a $Marta$ las monedas de cada bolsillo y las contase. Por tanto, para demostrar la afirmaci�n de que \textit{$Marta$ realmente tiene (o no) las mismas monedas en cada bolsillo} basta con que las ense�e. \\
						
						En este caso, el problema es que $Carmen$ est� violando la privacidad de $Marta$ pues no deber�a ser necesario que $Carmen$ conozca cu�ntas monedas tiene $Marta$ y, mucho menos, tener que ense�arlas a la audiencia para demostrar la veracidad de sus investigaciones. \\
						
						Esta violaci�n de la privacidad se puede superar aplicando de forma simple una soluci�n basada en prueba de conocimiento cero. \\
						
						Supongamos que $Marta$ tiene $X$ monedas en el bolsillo derecho e $Y$ monedas en el izquierdo. Para no conocer el n�mero de monedas que posee $Marta$, $Carmen$ le pide que piense en un n�mero $Z$ entero y mayor que $X$ e $Y$. A continuaci�n le pide que le diga la diferencia entre $Z$ y $X$ y entre $Z$ e $Y$. Si ambas diferencias son iguales, $Marta$ tiene el mismo n�mero de monedas en ambos bolsillos, del mismo modo que si las diferencias no coinciden, se puede afirmar lo contrario, incluso sabiendo en qu� bolsillo hay un mayor n�mero de monedas. Y todo esto \textbf{sin que $Carmen$ llegue a conocer en ning�n momento cu�ntas monedas posee $Marta$}. \\
						
						Matem�ticamente, se tratar�a de un sistema de 2 ecuaciones con 3 inc�gnitas, por lo que no se puede resolver y no se pueden obtener los valores de $X$ e $Y$:
						$$Z - X = C$$
						$$Z - Y = D$$
						
						Pese a que no es resoluble, sabiendo los valores de $C$ y $D$ se puede demostrar en qu� bolsillo tiene m�s monedas ($C > D$ � $D > C$) o si se tienen las mismas ($C = D$), sin necesidad de conocer los valores reales.
	
%        	}
%    			\end{varwidth}
%				\end{center}
\end{tcolorbox}
\end{center}
						
					La importancia de este esquema criptogr�fico es tal que se ha implementado de forma satisfactoria en campos tan esenciales como la verificaci�n de armas nucleares, permitiendo a los observadores internacionales medir la veracidad de una naci�n al cuantificar su fuerza nuclear sin necesidad de conocer la tecnolog�a o cabezas u otros detalles clasificados que no quieren que sean sacados a la luz.
					
					El ejemplo expuesto, pese a ser extremadamente simple, da una idea de lo que luego, matem�ticamente se encarga de demostrar la criptograf�a.
					
					Recomiendo estos otros ejemplos sobre pruebas de conocimiento cero. En este art�culo\footnote{\url{http://jorgegarciaherrero.com/auditar-algoritmo/}}, se recomiendan tres ejemplos de distinto nivel, muy interesantes: Uno sobre un ganador de loter�a\footnote{\url{http://elprofedefisica.naukas.com/2014/11/12/desafio-del-millon-de-james-randi-2/}}, otro para convencer a un ni�o de que no le est�s haciendo trampa\footnote{\url{http://www.wisdom.weizmann.ac.il/\~naor/PAPERS/waldo.pdf}} y otro sobre Google y sombreros\footnote{\url{https://blog.cryptographyengineering.com/2014/11/27/zero-knowledge-proofs-illustrated-primer/}}.
					
					
					
					%% En wikipedia hay m�s informaci�n sobre los PCC http://es.wikipedia.org/wiki/Prueba_de_conocimiento_cero
					
					
					% *************************************************************************************************************** %
					% 			REDES MIXTAS - MIXNETS
					% *************************************************************************************************************** %
					\subsubsection{Mixnets}\label{primitivaMixnets}
						Los sistema de mixnets, o redes mixtas, se basan en una t�cnica enunciada por David Chaum en 1981 en \cite{Chaum:1981:UEM:358549.358563} que permite establecer un canal an�nimo entre un emisor y un receptor.
						
						El funcionamiento b�sico consiste en que cada servidor, cuando recibe una serie de mensajes, los mezcla y entrega al siguiente servidor. Y as� en todos los servidores hasta llegar al definitivo.
						
						La relaci�n entre una entrada y su correspondiente salida s�lo es conocida por el servidor que realiza el mezclado. De este modo, al final del proceso de mezclado no se puede determinar qu� entrada corresponde con qu� salida. A no ser que haya una conspiraci�n coordinada entre todos los servidores del proceso.
					
						Esta idea de mezclado es aplicable al voto electr�nico en busca de poder desacoplar el voto de su votante, protegiendo as� la privacidad de �ste.
						
						Para una mixnet se utiliza un esquema de cifrado de clave p�blica $(G, E, D)$ donde $(pubK, secK) \leftarrow G()$ genera un par de claves p�blica y secreta.
						$c \leftarrow E_{pubK}(m)$ cifra el mensaje $v$ utilizando la clave p�blica $pubK$
						$D_{secK}(c)$ descifra el mensaje cifrado $c$ utilizando la clave secreta $secK$
						
						El esquema ha de ser correcto y seguro. Para todo mensaje $v$, $D_{secK}(E_{pubK}(m)) = m$ y debe ser imposible conocer la informaci�n de m a partir de $c$.
						
						La mixnet consiste en $n$ servidores, cada uno de los cuales genera un par de claves p�blicas y privadas $(pubK_1, secK_1), (pubK_2, secK_2), ..., (pubK_n, secK_n)$. Supongamos $i$ usuarios que quieren enviar mensajes $m_1, m_2, ..., m_i$ a trav�s de la red. Cada usuario prepara un mensaje cifrado de forma que $c_i = E_{pubK_1}(E_{pubK_2}(...(E_{pubK_n}(m_i))...))$ y lo publica en un tabl�n p�blico.
						
						El primer servidor descifra la primera capa de cada mensaje cifrado mediante $D_{secK_1}(c_i)$ para obtener $c'_i = E_{pubK_2}(...(E_{pubK_1}(m_i))...)$. As�, reordena $c'_i$ y escribe el resultado en un tabl�n p�blico. En este punto, s�lo el primer servidor, que conoce la clave privada $secK_1$, es quien conoce qu� $c_i$ se corresponde con $c'_i$.
						
						Cada uno de los servidores realiza el mismo proceso, de forma que al final el �ltimo servidor es el que obtiene los mensajes originales $m_1, m_2, ..., m_i$ y los publica.
						
						\begin{figure}[htbp]
							\centering
							\includegraphics[width=0.7\textwidth]{imgs/mixnet01}
							\caption{Diagrama mezclado mixnet}
							\label{fig:mixnet.01}
						\end{figure}
						
						Este es el modelo enunciado por Chaum, en el que cada servidor descifra un mensaje cifrado. Existen otras 
						
						De todos modos, como se indica en \cite{artcl:Cabarcas:2015}, no basta con realizar un mezclado de votos para obtener un sistema de votaci�n electr�nica seguro, sino que se necesita un esquema con mecanismos para que los votantes certifiquen su identidad, as� como para que puedan firmar digitalmente los votos de forma an�nima, asegurando la integridad del voto y la privacidad del votante.
						
						\begin{figure}[htbp]
							\centering
							\includegraphics[width=0.7\textwidth]{imgs/mixnet02}
							\caption{Diagrama de mixnet con descifrado de mensajes}
							\label{fig:mixnet.02}
						\end{figure}

					% *************************************************************************************************************** %
					% 			CIFRADO HOMOM�RFICO
					% *************************************************************************************************************** %
					\subsubsection{Cifrado homom�rfico}\label{primitivaCifradoHomomorfico}
						
						El cifrado homom�rfico es un tipo de cifrado basado en algoritmos criptogr�ficos que poseen la propiedad de que si se aplica una funci�n ($f$) a un mensaje cifrado, el resultado de descifrarlo es igual a la aplicaci�n de otra funci�n ($g$) al mensaje sin cifrar.
						
						Definamos $D$ como la funci�n de descifrado y $C$ como la funci�n de cifrado.
						
					    $D(C(m, k), k) = m$
					    
					    $D(f(C(m, k)), k) = g(m)$
					    
					    Hay algoritmos en los que las funciones que se aplican ($f$ y $g$) son id�nticas, de forma que ($f$ = $g$), lo cual resulta muy interesante ya que aplicando la misma funci�n a un mensaje cifrado, al descifrar se obtiene el mismo resultado que resultar�a de aplicar la funci�n al mensaje en claro.
					    
					    $D(f(C(m, k)), k) = f(m)$
						
						En la figura \ref{fig:homomorfismo} se muestra un diagrama con el siguiente ejemplo de cifrado homom�rfico basado en la aplicaci�n de la operaci�n multiplicaci�n:
						
						($m$ es el mensaje a cifrar, $m'$ es el mensaje cifrado)
						
						$ m = [11, 7] \Rightarrow m[0] = 11, m[1] = 7 $
						
						$ f(m) = m[0]  \cdot  m[1] \Rightarrow $
						
						$ f(m) = 11  \cdot  7 = 77 $
						
						$ C(m, k) = m' $
						
						$ C(11|7, k) = [15, 12] = m' $
						
						\hfill
						
						$ f(m') = m'[0]  \cdot  m'[1] \Rightarrow $
						
						$ f(m') = 15  \cdot  12 = 180 $
						
						$ D(180, k) = 77 $
						
						$ D(f(m')) = f(m) $
						
						$ D(f(C(m, k)), k) = f(m) $
						
						\hfill 
						
						Entre los algoritmos parcialmente homom�rficos, podemos destacar los siguientes:
						
						\begin{description}
							\item [RSA] $C(m_1)  \cdot  C(m_2) = C(m_1  \cdot  m_2)$
							\item [ElGamal] $C(m_1)  \cdot  C(m_2) = C(m_1  \cdot  m_2)$
							\item [Benaloh] $C(m_1)  \cdot  C(m_2) = C(m_1 + m_2 \pmod{c})$
							\item [Goldwasser-Micali] $C(x_1)  \cdot  C(x_2) = C(x_1 \oplus x_2)$
							\item [Paillier] $C(m_1)  \cdot  C(m_2) = C((m_1 + m_2) \pmod{c})$
						\end{description}
						
						\begin{figure}[htbp]
							\centering
							\includegraphics[width=0.80\textwidth]{imgs/homomorfismo.png}
							\caption{Ejemplo de homomorfismo basado en la operaci�n multiplicaci�n}
							\label{fig:homomorfismo}
						\end{figure}

					
						\paragraph{ElGamal}\label{primitivaCifradoHomomorfico_elgamal}

							\hfill \\
							
							Uno de los algoritmos m�s utilizados para realizar un cifrado homom�rfico es el de ElGamal \cite{ElGamal:1985:PKC:19478.19480}.
							
							ElGamal es un algoritmo de criptograf�a asim�trica basado en Diffie-Hellman de cifrado de clave p�blica que se usa tanto para cifrado como para firma digital. Se apoya en el problema de logaritmos discretos.
							
							El creador de este algoritmo fue Taher Elgamal\footnote{\url{https://es.wikipedia.org/wiki/Taher_Elgamal}}, en 1984. Es un algoritmo de uso libre.
							
							
							\textbf{Clave}: \hfill \\
							El algoritmo de ElGamal necesita un par de claves p�blica - privada para poder ser utilizado. \cite{Geraldo:2014,Meissen} Para ello, el cifrador del mensaje ha de escoger tres n�meros:
							\begin{itemize}
								\item N�mero primo $p$ cualquiera, tal que el logaritmo discreto no soluble en un tiempo asumible en $Z^*_p$ (grupo multiplicativo m�dulo un primo $p$). Esto quiere decir que $p - 1$ ha de tener un factor primo grande, lo cual provoca que la resoluci�n del problema de logaritmo discreto sea dif�cil).
								\item N�mero aleatorio $g$, que ser� el generador del grupo c�clico $Z^*_p$.
								\item N�mero aleatorio $a$ tal que $a \in {0, ..., p - 1}$, que ser� la clave privada.
							\end{itemize}
							La clave p�blica ser� $(g, p, K)$ donde $K = g^a \mod{p}$. De este modo, la clave privada $a$ se mantendr� en secreto.
							
							Para el \textbf{cifrado} de un mensaje: \hfill \\
							Supongamos que queremos cifrar el mensaje $m$ tal que $1 < m < p-1$ (es decir que $m \in Z_p$).
							
							El cifrador ha de escoger un n�mero aleatorio $b$ tal que $a \in {2, ..., p - 1}$ que mantendr� en secreto.
							
							El mensaje cifrado se corresponde con la tupla $C_b(m, b) = (y_1, y_2)$, donde:
							\begin{itemize}
								\item[] $y_1 = g^b \pmod{p}$
								\item[] $y_2 = K^b  \cdot  m \pmod{p}$
							\end{itemize}
							
							
							
							
							Para \textbf{descifrar} el mensaje $m$ se utiliza el peque�o teorema de Fermat\footnote{\url{https://es.wikipedia.org/wiki/Peque�o_teorema_de_Fermat}}, por el cual podemos inferir que:
							
							$y_1^{-a}  \cdot  y_2 \pmod{p} = (g^b)^{-a}  \cdot  K^b  \cdot  m \pmod{p} = g^{-ab}  \cdot  (g^a)^b  \cdot m \pmod{p} = (g^a)^{-b}  \cdot  (g^a)^b  \cdot m \pmod{p} = (g^a)^{b-b}  \cdot m \pmod{p} = (g^a)^0  \cdot m \pmod{p} = 1 \cdot m \pmod{p} = m \pmod{p}$
							
							Por tanto:\hfill \\
							$m = D(C_b(m, b)) = D((y_1, y_2)) = y_1^{p-1-a}  \cdot  y_2 \pmod{p}$
							
							Un ejemplo de este algoritmo es el siguiente:
							
							Bego�a escoge los siguiente valores:
							\begin{itemize}
								\item $p = 101$ (primo aleatorio. Supongamos, aunque no es cierto en este ejemplo, que $p - 1 = 100$ tiene un factor primo grande.)
								\item $g = 5$
								\item $a = 16$ (clave privada)
							\end{itemize}							
							
							Con estos valores, se calcula la clave p�blica.
							
							$K = g^a \pmod{p} = 5^16 \pmod{101} = 31. => K = 31$
							
							$pubK = (g, p, K) = (5, 101, 31)$
							
							Marta quiere cifrar el mensaje $m = 6$ (tal que $1 < m < p-1 = 1 < 6 < 100$, para lo que escoge un n�mero $b$ aleatorio $b = 7$ de forma que $2 < b < p - 1 = 2 < 7 < 100$.
							
							Para cifrar el texto, ha de utilizar la clave p�blica generada por Bego�a y calcular:
							\begin{itemize}
								\item[] $y_1 = g^b \pmod{p} = 5^7 \pmod{101} = 52$
								\item[] $y_2 = K^b  \cdot  m \pmod{p} = 31^7  \cdot  6 \pmod{101} = 47$
							\end{itemize}

							
							As� el mensaje $m = 9$ cifrado ser�:
							
							$C_b(m, b) = C_7(9, 7) = (y_1, y_2) = (52, 47) => C(9) = (52, 47)$
							
							Con este mensaje cifrado, Bego�a debe utilizar su clave privada para poder obtener el texto en claro:
							
							$D(C_b(m, b)) = D(C_7(9, 7)) = D((y_1, y_2)) = D((52, 47)) = y_1^{p-1-a}  \cdot  y_2 \pmod{p} = 52^{101-1-16}  \cdot  y_2 \pmod{101} = 52^84  \cdot  47 \pmod{101} = 6$							

							
							Este algoritmo es un sistema de cifrado homom�rfico respecto de la operaci�n multiplicaci�n. Con esto, se obtiene un proceso en el que el producto de de dos mensajes cifrados equivale al cifrado del producto de ambos mensajes.
							
							\[C (x_{1}) \cdot C (x_{2}) = C (x_{1} \cdot x_{2})\]
							
							
							Definamos:
							\begin{itemize}
								\item[] $a$: clave secreta
								\item[] $(g, p, K)$: clave p�blica
								\item[] $g$: generador
								\item[] $m$: mensaje
								\item[] $b$: aleatoriedad
							\end{itemize}
								
							
							$C(m) = (y_1, y_2) = (g^b \pmod{p}, K^b  \cdot  m \pmod{p})$
							
							$C(m_1)  \cdot  C(m_2) = (g^{b_1}, m_1 \cdot K^{b_1})(g^{b_2}, m_2 \cdot K^{b_2}) = (g^{r_1+r_2}, (m_1 \cdot m_2) \cdot K^{r_1+r_2}) = (g^t, (m_1 \cdot m_2) \cdot h^t) = C(m_1 \cdot m_2)$
							
							Un ejemplo pr�ctico de este homomorfismo:
							
							$C(6) = (5^7 \pmod{101}, 31^7  \cdot  6 \pmod{101}) = (52, 47)$

							$C(8) = (5^7 \pmod{101}, 31^7  \cdot  8 \pmod{101}) = (52, 29)$

							Para descifrar ambos mensajes, utilizar�amos el siguiente proceso:
							
							$m_1 = y_1^{p-1-a}  \cdot  y_2 \pmod{p} = 52^{101-1-16}  \cdot  47 \pmod{101} = 6$
							
							$m_2 = y_1^{p-1-a}  \cdot  y_2 \pmod{p} = 52^{101-1-16}  \cdot  29 \pmod{101} = 8$
							
							Si multiplicamos los dos mensajes cifrados:
							
							$C(6)  \cdot  C(8) = (52, 47)  \cdot  (52, 29) = (52^2, 47 \cdot 29)$
							
							$D(C(6)  \cdot  C(8)) = D((52^2, 47 \cdot 29)) = [y_1^{p-1-a}  \cdot  y_2 \pmod{p}] = (52^2)^{101-1-16}  \cdot  (47 \cdot 29) \pmod{101} = 48 = 6  \cdot  8$
							
							As�, se observa que si el resultado de descifrar el producto de dos mensajes cifrados es igual al producto de los mensajes sin cifrar.
							
						\paragraph{ElGamal Exponencial}\label{primitivaCifradoHomomorfico_elgamal_exponencial}		
						\hfill \\
							
							La propiedad de ElGamal en cuanto a homomorfismo es muy interesante, pero de cara al voto por Internet, presenta ciertos problemas. El principal es que la operaci�n en la que se basa es el producto. Para un sistema orientado a procesos electorales, ser�a mucho m�s �til que el homomorfismo fuera sobre la suma, ya que el escrutinio no es otra cosa que la suma (totalizaci�n) de votos. \cite{Geraldo:2014,Meissen}							
							As�, existe una variante de ElGamal que se denomina ElGamal Exponencial. En esta variante, en lugar de encriptar el mensaje $m$ como en ElGamal tradicional, se cifra $g^{m}$, donde $g$ es un generador (normalmente se suele reutilizar el mismo que se utiliza para generar la clave p�blica), transformando el sistema en un homomorfismo aditivo, es decir, sobre la operaci�n suma.
							
							\[C (g^{x_{1}}) \cdot C (g^{x_{2}}) = C (g^{x_{1} + x_{2}})\]
							
							Para esta variante de ElGamal, definamos:
							\begin{itemize}
								\item[] $a$: clave secreta
								\item[] $(g, p, K)$: clave p�blica
								\item[] $g$: generador
								\item[] $m$: mensaje
								\item[] $b$: aleatoriedad
							\end{itemize}
							
							y modifiquemos \cite{Martins:2014} la funci�n del ElGamal multiplicativo de modo que
							
							$C(m) = (y_1, y_2) = (g^b \pmod{p}, g^m  \cdot  K^b \pmod{p})$
							
							Desarrollando:
							
							$C(m_1)  \cdot  C(m_2) = (g^{b_1}, g^{m_1}  \cdot  K^{b_1})(g^{b_2}, g^{m_2}  \cdot  K^{b_2}) = (g^{b_1}  \cdot  g^{b_2}, g^{m_1}  \cdot  K^{b_1}  \cdot  g^{m_2}  \cdot  K^{b_2}) = (g^{b_1+b_2}, g^{m_1+m_2}  \cdot  K^{b_1+b_2}) = (g^t, g^{m_1 + m_2}  \cdot  K^{b_1+b_2}) = C(g^{m_1 + m_2})$ 
							
							
							El descifrado en este caso se obtiene de este modo:
							
							$y_1^{-a} \cdot y_2 \pmod{p} = (g^b)^{-a} \cdot g^m \cdot K^b \pmod{p} = g^{-ab} \cdot g^m \cdot g^{ab} \pmod{p} = g^m \pmod{p}$
							
							$D(C(y_1, y_2)) = y_1^{-a} \cdot y_2 \mod{p} = g^m \mod{p}$
							
							$m = \log_b{\dfrac{y_2}{y_1^a}}$
							
							Un ejemplo pr�ctico de este homomorfismo aditivo:
							
							$C(6) = (5^7 \pmod{101}, 31^7  \cdot  5^6 \pmod{101}) = (52, 68)$

							$C(8) = (5^7 \pmod{101}, 31^7  \cdot  5^8 \pmod{101}) = (52, 84)$

							Para descifrar ambos mensajes, utilizar�amos el siguiente proceso:
							
							$D(C(6)) => y_1^{p-1-a}  \cdot  y_2 \pmod{p} = 52^{101-1-16}  \cdot  68 \pmod{101} = 71 => 71 = g^m \pmod{101}$
							
							Para obtener el descifrado, estoy haciendo uso de una herramienta online que resuelve el logaritmo discreto\footnote{\url{https://www.alpertron.com.ar/DILOG.HTM}} de forma que obtiene el exponente de la ecuaci�n $Base^{Exponent} = Power \pmod{Modulus}$
							
							Para ello, se pasan 3 par�metros (Base, Power y Modulus) y la aplicaci�n calcula el exponente.
							
							As�, Base = 5, Power = 71, Modulus = 101 $ => m = 6$
							
							$D(C(8)) => y_1^{p-1-a}  \cdot  y_2 \pmod{p} = 52^{101-1-16}  \cdot  84 \pmod{101} = 58 => 58 = g^m \pmod{101}$
							
							Base = 5, Power = 58, Modulus = 101 $ => m = 8$
							
							Si multiplicamos los dos mensajes cifrados:
							
							$C(6)  \cdot  C(8) = (52, 68)  \cdot  (52, 84) = (52^2, 68 \cdot 84) = (2704, 5712)$
							
							$D(C(6)  \cdot  C(8)) = D((2704, 5712)) => [y_1^{p-1-a}  \cdot  y_2 \pmod{p}] => (2704)^{101-1-16}  \cdot  5712 \pmod{101} = 78$
							
							Base = 5, Power = 78, Modulus = 101 $=> m = 14 = 8 + 6$
							
							As�, se observa que, con este cambio en el algoritmo, si el resultado de descifrar el producto de dos mensajes cifrados es igual a la suma de los mensajes sin cifrar.
			
			
			
			% *************************************************************************************************************** %
			% 			ESQUEMAS DE VOTO ELECTR�NICO
			% *************************************************************************************************************** %
			\subsection{Esquemas de Voto Electr�nico}\label{esquemasVotoElectronico}

				Los sistemas de voto electr�nico est�n formados por un dise�o conceptual y el llamado esquema o paradigma de voto electr�nico (E-Voting Schemes - EVS). El esquema es el n�cleo del sistema, lo que asegura que los requisitos se cumplan.
				
				La pr�ctica totalidad de estos esquemas usan mecanismos y principios criptogr�ficos.
				
				Los esquemas de voto electr�nico se basan en una primitiva criptogr�fica o en un conjunto de ellas. Por eso, hay una serie de esquemas publicados apoyados sobre alguna de las primitivas introducidas en el apartado anterior.
				
				Podemos clasificar varios tipos de esquemas de voto electr�nico entre los m�s usados seg�n las publicaciones de una serie de expertos en el campo del voto electr�nico criptogr�fico:
				
				\begin{itemize}
					\item Esquema de Voto Electr�nico basado en \textbf{Cifrado Homom�rfico}
					
						El votante emite su voto codificado y el recuento se realiza sin descodificar los votos. De esta forma se consigue que no se vulnere el secreto del voto. Para poder realizar esta descodificaci�n, el elector debe instalar alg�n software desarrollado por la autoridad electoral para realizar las operaciones criptogr�ficas.
					
					\item Esquema de Voto Electr�nico basado en \textbf{Canales An�nimos}
					
						Se trata de un esquema bastante seguro, aunque complejo al mismo tiempo. Se trata el anonimato del votante ocultando el origen de los votos que recibe el sistema.
					
					\item Esquema de Voto Electr�nico basado en \textbf{Mixnets}
					
						El esquema basado en mixnets (redes mixtas) define la existencia de una serie de servidores enlazados. Cada uno de estos servidores recibe un grupo de mensajes encriptados, los reordena, los vuelve a encriptar de forma aleatoria y los env�a al siguiente servidor. Con este proceso se consigue que no sea posible asociar la informaci�n de los mensajes de entrada con los de salida, rompiendo la relaci�n votante-voto del sistema.
						
						La desencriptaci�n de los votos se puede realizar tanto en cada servidor (por medio de su propia clave) como al finalizar el proceso utilizando una clave distribuida entre varios de los servidores.
						
						%******************* LITERAL ******** La principal cr�tica a este esquema es que las pruebas de correctitud son costosas. Existen algunas implementaciones comerciales de sistemas de elecci�n electr�nica basadas en este esquema. \notasCambio{LITERAL}
						
					\item Esquema de Voto Electr�nico basado en \textbf{Secreto Compartido}
					
						En el esquema de voto electr�nico el votante comparte su voto entre varias autoridades electorales. Una vez finalizado el proceso de votaci�n, cada autoridad computa los votos que ha recibido y los pone en com�n con el resto de autoridades electorales que toman parte en la elecci�n. As� se obtiene el resultado total del proceso.
											
					\item Esquema de Voto Electr�nico basado en \textbf{Pruebas de Conocimiento Nulo}
					
					\item Esquema de Voto Electr�nico basado en \textbf{Firma Ciega}
					
						En un Esquema de Firma Ciega, el firmante no conoce el contenido del mensaje que firma, ya que el emisor del mismo realiza un proceso previo para ocultar su contenido, lo que se conoce por \textit{cegar} el mensaje.
						
						Se caracteriza porque la entidad firmante no adquiere ning�n conocimiento sobre el contenido del mensaje que est� firmando, aunque, con posterioridad, la firma obtenida puede ser verificada como v�lida tanto por esta entidad firmante como cualquier otra entidad que disponga de la informaci�n necesaria.
						
						Se caracteriza porque la entidad firmante no adquiere ning�n conocimiento sobre el contenido del mensaje que est� firmando, aunque, con posterioridad, la firma obtenida puede ser verificada como v�lida tanto por esta entidad firmante como cualquier otra entidad que disponga de la informaci�n necesaria.
						
						Los esquemas que se basan en protocolos con firma ciega suelen usar canales an�nimos para enviar tanto la firma como el voto cifrado a la autoridad electoral, con lo que protege el anonimato del votante.
						
						Podemos encontrar este esquema en soluciones como la propuesta en 1992 por Fujioka en \cite{Fujioka93}, la cual sirvi� de base a Cranor\footnote{\url{http://lorrie.cranor.org/}} para la implementaci�n de un prototipo (Sensus\footnote{\url{http://lorrie.cranor.org/voting/sensus/}}\footnote{\url{http://lorrie.cranor.org/pubs/hicss/hicss.html}}).
						
						El esquema desarrollado en Sensus divide el proceso en cuatro etapas: \textit{inicializaci�n}, \textit{registro}, \textit{votaci�n} y \textit{recuento}.				
					\item Esquema de Voto Electr�nico basado en \textbf{papeletas precifradas}
					
						Este esquema de voto electr�nico aparece en la tesis de Morales Rocha \cite{tesisSeguridadVMMorales}.
				
				\end{itemize}

				Aprovechando que en el �ltimo esquema se cita la tesis de Morales Rocha \cite{tesisSeguridadVMMorales}, vamos a introducir una alternativa en cuanto al tipo de esquemas de voto electr�nico existentes. En esta tesis, el autor define cuatro grupos de esquemas de voto electr�nico remoto. Estos se diferencian en la forma en la que usan los elementos criptogr�ficos para tratar de resolver los requisitos de seguridad de un sistema electoral:
				\begin{itemize}
					\item Esquemas basados en \textbf{firma ciega}
					\item Esquemas basados en \textbf{mixnets}
					\item Esquemas basados en \textbf{cifrado homom�rfico}
					\item Esquemas basados en \textbf{papeletas precifradas}
				\end{itemize}
				
				El propio autor de la tesis citada \cite{tesisSeguridadVMMorales}, incorpora (en la p�gina 109) un resumen con las ventajas y desventajas que ofrece cada uno de estos esquemas (tabla \ref{tab:ResumenVentajasDesventajasEsquemasVotoElectronico}).				
				\begin{table}[htbp]
					\centering
					\fontsize{10}{9}\selectfont
					\begin{tabularx}{\textwidth}{|Z|Y|Y|}
						\hline
							\normalsize\centering\textbf{Clasificaci�n} & \normalsize\textbf{Ventajas} & \normalsize\textbf{Desventajas} \\
						\hline
								%\begin{itemize}[leftmargin=1em]
									%\renewcommand\labelitemi{}
									%\item 
									Esquemas basados en firma ciega
								%\end{itemize}
								& 
								\parbox[c]{\hsize}{
									\begin{itemize}[leftmargin=1em,font=$\bullet$]
										\renewcommand\labelitemi{}
										\item Protegen la privacidad del votante al separar los procesos de autenticaci�n y voto. 
									\end{itemize} 	
								}
							& 
								\parbox[c]{\hsize}{
									\begin{itemize}[leftmargin=1em,font=$\bullet$]
										\renewcommand\labelitemi{}
										\item La protecci�n del anonimato puede verse afectada si un atacante monitorea el canal de comunicaci�n.
										\item Con el conocimiento de la clave privada de la autoridad de autenticaci�n se pueden a�adir votos no leg�timos. 
									\end{itemize} 	
								}
							\\
						\hline
							Esquemas basados en mixnets &
							\parbox[c]{\hsize}{
								\begin{itemize}[leftmargin=1em,font=$\bullet$]
									\renewcommand\labelitemi{}
									\item Protegen la privacidad del votante a trav�s de las permutaciones llevadas a cabo. 
								\end{itemize} 	
							}
							& 
							\parbox[c]{\hsize}{
								\begin{itemize}[leftmargin=1em,font=$\bullet$]
									\renewcommand\labelitemi{}
									\item Dif�cil verificaci�n de que los servidores mix han actuado correctamente.
									\item En el caso de mixnets de descifrado, el terminal de votaci�n requiere de alta capacidad de c�mputo. 
								\end{itemize} 	
							}
							\\
						\hline
							Esquemas basados en cifrado homom�rfico & 
							\parbox[c]{\hsize}{
								\begin{itemize}[leftmargin=1em,font=$\bullet$]
									\renewcommand\labelitemi{}
									\item Protegen la privacidad del votante al no tener que descifrar los votos individualmente para llevar a cabo el escrutinio. 
								\end{itemize} 	
							}
							& 
							\parbox[c]{\hsize}{
								\begin{itemize}[leftmargin=1em,font=$\bullet$]
									\renewcommand\labelitemi{}
									\item No soportan todo tipo de elecciones.
									\item Son susceptibles a ataques en donde votantes deshonestos pueden enviar un mensaje que represente m�s de un voto para un candidato. 
								\end{itemize} 	
							}
							\\
						\hline
							Esquemas basados en papeletas precifradas & 
							\parbox[c]{\hsize}{
								\begin{itemize}[leftmargin=1em,font=$\bullet$]
									\renewcommand\labelitemi{}
									\item Protegen la privacidad del votante ya que este env�a como voto un c�digo cuya relaci�n con el candidato es desconocida para el servidor de votaci�n. 
									\item Evitan ataques de c�digo malicioso que trate de alterar o conocer el contenido del voto. 
									\item El voto puede ser enviado desde un dispositivo con baja capacidad de c�mputo. 
									\item Permiten al votante verificar que su voto se ha recibido correctamente en el servidor de votaci�n. 
								\end{itemize} 	
							}
							& 
							\parbox[c]{\hsize}{
								\begin{itemize}[leftmargin=1em,font=$\bullet$]
									\renewcommand\labelitemi{}
									\item Posibles alteraciones en las papeletas precifradas sin detecci�n, lo cual ocasionar�a que el votante env�e un voto diferente al deseado.
									\item Se pueden presentar problemas de log�stica en la distribuci�n de las papeletas a los votantes. 
									\item Votantes no pueden verificar que su voto fue incluido en el escrutinio sin arriesgar un ataque de coerci�n masiva. 
									\item Poca usabilidad al tener que teclear c�digos de votaci�n. 
								\end{itemize} 	
							}
							\\
						\hline
					\end{tabularx}
				\caption{Resumen de ventajas y desventajas de los esquemas de voto electr�nico seg�n Morales Rocha \cite{tesisSeguridadVMMorales} (p. 109)}
				\label{tab:ResumenVentajasDesventajasEsquemasVotoElectronico}
			\end{table}
				
				
				
				Est� fuera del alcance de este proyecto el estudio de estos esquemas y sus evoluciones, pero nos basamos en esta informaci�n para el desarrollo del sistema que se implementa. Para ahondar en ellos, recomiendo la lectura de la citada tesis de Morales Rocha \cite{tesisSeguridadVMMorales}, as� como el cap�tulo 4 de la tesis de la Dra. Emilia P�rez Belleboni \cite{tesisEPerezBelleboni}, en la cual se expone una recopilaci�n de informaci�n muy concisa sobre multitud de esquemas y sistemas que los implementan, seg�n las necesidades que se necesiten cubrir.
			 


\iffalse
		\subsection{Prueba de conocimiento cero}\label{zeroKnowledgeProof}
		La t�cnica de Prueba de Conocimiento Cero (Zero Knowledge Proof - ZKP) en criptograf�a permite a un actor probar un mensaje a otro actor verificador sin revelar el contenido del mismo.
\fi				
				
				
	% *************************************************************************************************************** %
	% 			ESTADO ACTUAL DE LAS TECNOLOG�AS
	% *************************************************************************************************************** %
	\section{Estado actual de las tecnolog�as}
		% *************************************************************************************************************** %
		% 			CERTIFICADOS DIGITALES
		% *************************************************************************************************************** %
		\subsection{Certificados Digitales}\label{estadoCertificadosDigitales}
		La web de la F�brica Nacional de Moneda y Timbre \cite{webCertFNMT} indica que \textit{un certificado digital es un documento electr�nico que asocia una clave p�blica con la identidad de su propietario}.
		
		Complementa la definici�n a�adiendo que ''\textit{adicionalmente, adem�s de la clave p�blica y la identidad de su propietario, un certificado digital puede contener otros atributos para, por ejemplo, concretar el �mbito de utilizaci�n de la clave p�blica, las fechas de inicio y fin de la validez del certificado, etc. El usuario que haga uso del certificado podr�, gracias a los distintos atributos que posee, conocer m�s detalles sobre las caracter�sticas del mismo}''.
		
		La utilidad de los certificados digitales, simplificando el contexto, se resume en \textit{asegurar que una determinada clave p�blica pertenece a un usuario en concreto}.
		
		Con las tecnolog�as actuales, la econom�a ha virado su desarrollo hacia el comercio electr�nico y las relaciones remotas. Muchas transacciones que antes se realizaban en persona han evolucionado al mundo digital, por lo que, para la mayor�a de ellas es indispensable poseer mecanismos que puedan demostrar que los sujetos intervinientes en la comunicaci�n est�n un�vocamente identificados y con la seguridad de que no se produce suplantaci�n.
		
		Una herramienta fundamental para cumplir con este prop�sito ha sido el desarrollo de las certificaciones digitales.
		
		Los certificados digitales permiten cifrar las comunicaciones, permitiendo tan s�lo al destinatario de estas acceder al contenido.
		
		Los certificados electr�nicos est�n expedidos por una Autoridad de Certificaci�n e identifica a una persona con un par de claves criptogr�ficas, una p�blica y otra privada, generadas mediante un algoritmo matem�tico. Ambas son complementarias, de forma que lo que cifra una s�lo lo puede descifrar la otra, y viceversa. La diferencia entre ellas es que la clave privada est� pensada para que nunca salga del certificado y permanezca siempre bajo control del firmante. La clave p�blica se puede enviar a otros usuarios.
		
		Tienen como objetivo validar y certificar que una firma electr�nica se corresponde con una persona concreta. Por esta raz�n, para dar fe de que el certificado se corresponde con una persona concreta, es por lo que los certificados est�n firmados, a su vez por la Autoridad de Certificaci�n.
		
		% *************************************************************************************************************** %
		% 			DNI ELECTR�NICO
		% *************************************************************************************************************** %
		\newpage
		\subsection{DNIe}\label{estadoDNIe}
		
			\begin{wrapfigure}{r}{.3\textwidth}
				\vspace{-10pt}
				\begin{center}
					\includegraphics[width=.3\textwidth]{imgs/dnie03.jpg}
				\end{center}
				\vspace{-20pt}
				\caption{Anverso de un especimen del DNIe 3.0}\label{fig:anverso.dnie.3.0}
			\end{wrapfigure}
			El DNIe es un documento con una antig�edad de m�s de setenta a�os. Tras los primeros, ideados en 1944 y estrenados en 1961, entre 2006 y 2015 se a�adi� al documento un chip con certificados v�lidos para la identificaci�n y firma digital. Era el llamado \gls{DNIe} 2.0. A partir de 2015 y sobre todo ya en 2016, el documento fue evolucionado a una tercera versi�n, la cual, como elemento m�s novedoso, a�ad�a al chip de contacto otro chip sin contacto con tecnolog�a \gls{RFID}, m�s concretamente, \gls{NFC}.
			
			Para utilizar el nuevo chip sin contacto, el usuario necesita un dispositivo (smartphone o tablet) con tecnolog�a NFC y una app que d� el servicio que requiera de identificaci�n conect�ndose al documento.
			
			\begin{wrapfigure}{l}{.3\textwidth}
				\vspace{-20pt}
				\begin{center}
					\includegraphics[width=.3\textwidth]{imgs/dnie02.jpg}
				\end{center}
				\vspace{-20pt}
				\caption{Reverso de un especimen del DNIe 3.0}\label{fig:anverso.dnie.3.0}
			\end{wrapfigure}		
			Hay que tener en cuenta que no basta con acercar el \gls{DNIe} al dispositivo m�vil con NFC para que �ste lea la informaci�n contenida en aqu�l. El usuario deber� introducir dos c�digos, uno es el \gls{PACE}, que encontrar� en el anverso del documento f�sico, y el otro es su \gls{PIN} personal que establece en las m�quinas habilitadas para ello en comisar�as.
			
			
			El chip electr�nico almacena los datos personales y la fotograf�a del titular del documento, su firma manuscrita digitalizada y el patr�n de su huella dactilar. Junto a estos datos personales, encontramos los certificados digitales de autenticaci�n y de firma electr�nica.
			
			Este conjunto de informaci�n y certificados permite al titular del documento identificarse tanto de forma f�sica como remota, as� como para realizar tr�mites electr�nicos en los que se requiera la identificaci�n un�voca de la persona.
			
			\begin{figure}[htbp]
				\centering
				\includegraphics[width=0.7\textwidth]{imgs/DNI_PERSPECTIVA_3.jpg}
				\caption{Especificaciones m�s relevantes del DNIe 3.0}
				\label{fig:dnie.3.0.especificaciones}
			\end{figure}
						
			La firma electr�nica cuenta con una clave p�blica y otra privada. Al realizar una operaci�n electr�nica, el receptor utiliza la clave p�blica para asegurarse de la identificaci�n del emisor, mientras que �ste utiliza su clave privada para firmar la operaci�n.
			
			El certificado de autenticaci�n asegura que la persona que est� utilizando el documento realmente es quien dice ser.
			
			
			Una buena fuente de informaci�n para consultar las funcionalidades del \gls{DNIe} 3.0 es la ofrecida en el propio portal\footnote{\url{https://www.dnielectronico.es/PortalDNIe/}} del \gls{DNIe} por la Polic�a Nacional de Espa�a en \cite{policia:DNIe:NFC:2015}.
					
			De esta fuente se obtienen los datos comparativos que se muestran en la tabla \ref{tbl:comparacion.dnie.2.3} acerca de las caracter�sticas de los chips de las versiones 2.0 y 3.0 del \gls{DNIe}.


			\newcolumntype{L}[1]{>{\raggedright\let\newline\\\arraybackslash\hspace{0pt}}m{#1}}
			\newcolumntype{C}[1]{>{\centering\let\newline\\\arraybackslash\hspace{0pt}}m{#1}}
			\newcolumntype{R}[1]{>{\raggedleft\let\newline\\\arraybackslash\hspace{0pt}}m{#1}}
			\begin{table}[htbp]
				\centering
				%\fontsize{5}{4}\selectfont
				\small
				\def\arraystretch{1.25}%  1 is the default, change whatever you need
				\setstretch{1}
				\begin{threeparttable}
					\centering
					\begin{tabular*}{\textwidth}{| L{0.2\textwidth} | L{0.356\textwidth} | L{0.356\textwidth} |}
						\hline
						\begin{center}
							\textbf{Caracter�stica}
						\end{center}
						& 
						\begin{center}
							\textbf{DNIe}
						\end{center}
						& 
						\begin{center}
							\textbf{DNIe 3.0}
						\end{center}
						\\
						\hline
			\textbf{Chip} & ST19WL34 & SLE78CLFX408AP de Infineon Technologies \\
			\hline
			\textbf{Sistema operativo} & DNIe v1.1 & DNIe v4.0 \\
			\hline
			\textbf{Capacidad} & 32Kb & 400KB memoria Flash, 8KB memoria RAM \\
			\hline
			\textbf{Interfaz} & Single (con contacto) & Dual (con contacto y sin contacto) \\
			\hline
			\textbf{Criptolibrer�a} & & RSA \\
			\hline
			\textbf{Certificaci�n} & & CC EAL5+ \\
			\hline
			\textbf{Zona p�blica:} Accesible en lectura sin restricciones
				&
				\multicolumn{2}{l|}{\parbox{0.6\textwidth}{
				\begin{itemize}
					\item Certificado CA intermedio emisora
					\item Claves Diffie-Hellman
					\item Certificado x509 de componente
				\end{itemize}}
				}
				\\
				\hline
			\textbf{Zona privada:} Accesible en lectura por el ciudadano, mediante el uso de su \gls{PIN} 
				&
				\multicolumn{2}{l|}{\parbox{0.6\textwidth}{\begin{itemize}
					\item Certifiacdo de Firma (no repudio)
					\item Certificado de Autenticaci�n
				\end{itemize}}
				}
				\\
				\hline
			\textbf{Zona de seguridad:} Accesible en lectura por el ciudadano, en los Puntos de Actualizaci�n del DNI.
				&
				\multicolumn{2}{l|}{\parbox{0.6\textwidth}{
				\begin{itemize}
					\item Datos de filiaci�n del ciudadano (los mismos que est�n impresos en el soporte f�sico del DNI)					\item Imagen de la fotograf�a
					\item Imagen de la firma manuscrita
				\end{itemize}}
				}
				\\
				\hline
			\textbf{Datos criptogr�ficos:} Claves de ciudadano
				&
				\multicolumn{2}{l|}{\parbox{0.72\textwidth}{
				\begin{itemize}
					\item Clave RSA p�blica de autenticaci�n (Digital Signature)
					\item Clave RSA p�blica de no repudio(ContentCommitment)
					\item Clave RSA privada de autenticaci�n (Digital Signature)
					\item Clave RSA privada de firma (ContentCommitment)
					\item Patr�n de impresi�n dactilar
					\item Clave P�blica de root CA para certificados card-verificables
					\item Claves Diffie-Hellman
				\end{itemize}}
				}
				\\
			\hline
					\end{tabular*}
					\begin{tablenotes}
						\scriptsize
						\item[] Fuente: Portal del DNI electr�nico. Ministerio del Interior. Direcci�n General de la Polic�a. Cuerpo Nacional de Polic�a. \url{https://www.dnielectronico.es/PortalDNIe/PRF1_Cons02.action?pag=REF_1078&id_menu=[26_\%2030]}.
					\end{tablenotes}
					\caption{Comparativa DNIe - DNIe 3.0.}
					\label{tbl:comparacion.dnie.2.3}
				\end{threeparttable}
			\end{table}


			El proceso a seguir cuando se est� en posesi�n todos los certificados necesarios para operar con el \gls{DNIe} es (para cada uno de los certificados)\cite{inteco:DNIe:2007}:
			\begin{itemize}
				\item Verificar que el certificado fue firmado usando la clave privada que
corresponde a la clave p�blica de su emisor. Este paso no es necesario
para el certificado de la CA ra�z.
				\item Verificar la validez del certificado, es decir, no ha caducado.
				\item Realizar la validaci�n \gls{OCSP} contra el servidor de la \gls{FNMT}.
			\end{itemize}
		
		
			Durante esta memoria se desglosa m�s informaci�n acerca del \gls{DNIe} 3.0, ya que es uno de los elementos claves del \gls{PFC}. Por ello, para mayor informaci�n, recomiendo visitar aquellos puntos en los que se explican ventajas y desventajas, motivos de adopci�n tecnol�gica, implementaci�n, integraci�n. (\ref{fase_electoral_electoral_identificacion}, \ref{analisis:sistema:autenticacion:identificacion}, \ref{solucion:aplicacion:autenticacion})

		
		% *************************************************************************************************************** %
		% 			NFC
		% *************************************************************************************************************** %
		\subsection{NFC}\label{estadoNFC}
			\gls{NFC} (\textbf{N}ear \textbf{F}ield \textbf{C}ommunication - Comunicaci�n de Campo Cercano) se trata de una tecnolog�a de comunicaci�n inal�mbrica de corto alcance.
			
			El origen de esta tecnolog�a est� en el RFID, tecnolog�a ya bastante utilizada en m�ltiples campos de nuestra vida cotidiana. Un buen ejemplo son las etiquetas en prendas en tiendas de ropa o en productos de supermercado. Si tratas de llevarte algo sin pagar, al pasar por unos arcos a la salida preparados con cierto campo magn�tico, las etiquetas RFID activan una alarma, la cual se desactiva una vez se ha abonado el producto.
			
			\begin{wrapfigure}{r}{.3\textwidth}
				\vspace{-20pt}
				\begin{center}
					\includegraphics[width=.3\textwidth]{imgs/rfid.jpg}
				\end{center}
				\vspace{-20pt}
				\caption{Logo de RFID}
				\label{fig:logoRFID}
				\vspace{-10pt}
			\end{wrapfigure}
			
			\gls{RFID} es una tecnolog�a de comunicaci�n inal�mbrica centrada en la transmisi�n de un identificador mediante ondas de radio. As�, a cualquier producto se le puede asociar un identificador un�voco a trav�s de una etiqueta RFID y as� poder ser detectado y le�do por un lector apropiado de forma inal�mbrica.
			
			En el caso de \gls{NFC}, se trata de un concepto similar, pero buscando la transmisi�n de datos m�s complejos que un simple identificador.
			
			\gls{NFC}, se basa en la norma ISO 14443 (\gls{RFID}), un est�ndar internacional relacionado con tarjetas de identificaci�n electr�nica, especialmente tarjetas de proximidad. Como establece esta norma est�ndar, \gls{NFC} se comunica mediante inducci�n en un campo magn�tico. Por tanto cada tarjeta \gls{NFC} cuenta con una antena en espiral, con lo que cuando se ambas (origen-destino) se colocan dentro de un campo electromagn�tico cercano, se puede realizar la transmisi�n de datos. Cumpliendo igualmente con la norma, esta tecnolog�a trabaja en una banda de frecuencia de 13,56 MHz.
			
			\begin{figure}[htbp]
				\centering
					\includegraphics[width=0.80\textwidth]{imgs/nfc_usim_img.png}
				\caption{Arquitectura de un chip \gls{NFC}}
				\label{fig:arquitecturaNFC}
			\end{figure}
			
			
			Este campo electromagn�tico debe ser producido por uno de los dispositivos o por ambos al mismo tiempo. Esto provoca que los dispositivos que implementan el est�ndar NFCIP-1 tienen la capacidad de funcionar de dos modos:
			
			\begin{description}
				\item[Modo Activo:] En el modo activo, ambos dispositivos generan su propio campo electromagn�tico.
				\item[Modo Pasivo:] En el modo pasivo, uno de los dispositivos genera el campo electromagn�tico y el otro se aprovecha de la modulaci�n de la carga para poder transferir de datos. Es el dispositivo que inicia la comunicaci�n el que genera el campo.
			\end{description}
			
			Este protocolo de comunicaciones puede trabajar a diferentes velocidades: 106, 212, 424 � 848 Kbits/s.
			La velocidad de la transmisi�n se negocia al comienzo de la comunicaci�n entre los dispositivos, aunque puede regularse seg�n sea necesario variando ciertos par�metros.
			
			Seg�n \gls{NFC} Forum, organizaci�n fundada en 2004 por Philips, Nokia y Sony y que hoy re�ne alrededor de 170 miembros, la conexi�n entre dos dispositivos se realiza de forma autom�tica cuando estos se encuentran cerca el uno del otro, digamos, a unos 5cm de distancia. De todos modos, esta distancia es variable en cuanto a que se puede ver afectada por ciertos factores como pueden ser el tipo de emisor/receptor, temperatura, aislantes, etc. Este foro fija la m�xima distancia para operar con este canal de transmisi�n en 20cm, con el prop�sito de cubrir la seguridad de la informaci�n, que esta no pueda ser accedida de forma remota por un atacante a distancia.
			
			Dentro de los modos de comunicaci�n activo y pasivo tambi�n se definen 3 modos de comunicaci�n:
			
			\begin{description}
				\item[Modo Lectura / Escritura] Con este modo, las aplicaciones pueden transferir datos en un formato definido por el NFC Forum, aunque es un modo no considerado seguro.
				\item[Modo emulaci�n \gls{NFC} Card] Este modo permite al dispositivo actuar como una Smartcard est�ndar, es decir, la transmisi�n de datos se encuentra securizada.
				\item[Modo Peer to Peer] Este modo permite la comunicaci�n directa entre dos dispositivos a nivel de enlace.
			\end{description}
			
			
			Hoy en d�a, el uso del \gls{NFC} est� bastante extendido en cuanto a que muchos dispositivos entre smartphone y tablets lo implementa de serie. Adem�s, est� desarroll�ndose el pago con tarjetas de d�bito/cr�dito contactless, por lo que muchos comercios ya poseen \gls{TPV}s para tarjetas sin contacto, o incluso con pago a trav�s del m�vil por \gls{NFC}.
			
			Igualmente, desde hace un tiempo, existen tarjetas \gls{NFC} con las que se pueden realizar tareas al poner en contacto con el dispositivo m�vil. Un ejemplo de funcionamiento ser�a colocar una etiqueta NFC junto a la puerta y que al pasar el m�vil por ella se encienda la luz de la habitaci�n por medio de una bombilla inteligente. O colocar una tarjeta junto a la puerta de casa que indique al m�vil que desconecte el WiFi al salir de casa para disminuir el consumo de bater�a.
			
			Tambi�n se utiliza el \gls{NFC} en marketing. Hay anuncios que incorporan etiquetas NFC en ellos. Concretamente, en junio de 2010, una empresa de Motril dise�� el primer anuncio en Espa�a que incorporaba una etiqueta NFC situada bajo pegatinas colocadas en motos que realizaban una ruta deportiva.
			
			Ya en 2015, con la salida del \gls{DNIe} 3.0, la tecnolog�a \gls{NFC} lleg� a la identificaci�n de los ciudadanos espa�oles. Esta nueva versi�n del carnet de identidad complementa el chip con contacto que inclu�a la primera versi�n del \gls{DNIe} con un chip sin contacto que realiza las mismas funciones sin necesidad de un lector de chips del primer tipo.
			
	
	\iffalse
		\subsection{Django}\label{estado_django}
		Django es un framework para aplicaciones web escrito en Python. Adem�s, tiene la caracter�stica de ser gratuito y un proyecto open source.
		\fi
\iffalse
\url{http://upcommons.upc.edu/bitstream/handle/2099.1/24909/memoria.pdf}
Django �s l�eina principal amb la que est� programada l�aplicaci� ISMe. �s
un framework de desenvolupament web de codi obert, desenvolupat en python,
que respecta el paradigma conegut com Model Template View (MTV), que t�
l�objectiu de facilitar la creaci� de webs complexes. Django posa �mfasis en el
re-�s, la connectivitat i extensibilitat de components, el desenvolupament r�pid i
el principi de no-repetici� (DRY, de l�angl�s Don't Repeat Yourself). Python �s 
18 Desenvolupament de nous serveis de signatura electr�nica sobre ISMe
utilitzat en totes les parts del framework, incl�s en configuracions, arxius, i en els
models de dades.
Podr�em classificar a Django com part de la tercera generaci� del
desenvolupament web, tal com es pot apreciar en la figura 3.4:
Fig. 3.4 Visi� general Django
Django �s un framework poder�s, que ens permet disminuir el temps de
desenvolupament, utilitzant una metodologia que es coneix amb el nom de Model
Template View. Per comen�ar a entendre l�arquitectura MTV hem de fixar-nos
en l�analogia amb MVC (Model-Vista-Controlador). El model en Django segueix
sent Model, la vista en Django s�anomena Plantilla (template), el controlador en
Django es diu Vista.
B�sicament els 3 components principals del framework tenen com a funci� crear
una comunicaci� entre ells per extraure informaci� vital de base de dades i
presentar-les al navegador. La figura 3.5 ajuda a entendre millor aquesta relaci�:
Fig. 3.5 Funcionament Django
1- El navegador envia una sol�licitud.
2- El UrlConf interpreta la sol�licitud i ubica la vista apropiada.
3- La vista interactua amb el model per obtenir les dades.
4- La vista crida la plantilla corresponent.
5- La plantilla renderitza la resposta de la sol�licitud del navegador.
El model
El model defineix les dades guardades i es troba en forma de classes. Cada tipus
de dada que ha de ser emmagatzemada es troba en una variable amb certs 
Entorn de treball i desenvolupament 19
par�metres i m�todes. Tot aix� permet indicar i controlar el comportament
d�aquestes dades.
La vista
La vista es presenta en forma de funcions. El seu prop�sit �s determinar quines
dades seran visualitzades i fer determinades operacions amb aquestes. El ORM
(Mapejador Objecte-relacional) de Django permet escriure codi python en
comptes de sent�ncies SQL per fer les consultes que necessita la vista.
La plantilla
La plantilla �s b�sicament una p�gina HTML amb algunes etiquetes extres
pr�pies de Django que en si no nom�s crea contingut en HTML (tamb� XML,
CSS, Javascript, CSV, etc). Parlant de les funcions b�siques amb HTML, la
plantilla rep les dades de la vista i despr�s les organitza per la presentaci� al
navegador web. Les etiquetes que Django utilitza per les plantilles permeten que
sigui flexible per als dissenyadors del front-end.
La configuraci� de les rutes
Django posseeix un mapeig de URL�s que permet controlar el desplegament de
les vistes. Aquesta configuraci� �s coneguda com UrlConf. El treball del UrlConf
�s llegir la URL que l�usuari sol�licita, trobar la vista apropiada per la sol�licitud i
passar qualsevol variable que la vista necessiti per completar el seu treball.
Permet que les rutes que manegui Django siguin agradables i entenedores per
l�usuari.
\fi		
		
		

	
		
				
			
\chapter{Planteamiento}\label{planteamiento}
\lhead{Cap�tulo \ref{planteamiento}}
\rhead{Planteamiento}
%*******************************************************************************
\section{Objetivos finales del proyecto}\label{objetivos}
%*
%*
%*
%*
%*
%*

	\subsection{Descripci�n del sistema real}\label{sistemareal}

\section{Alcance del proyecto}\label{alcance}

\section{Especificaci�n de requisitos}\label{requisitos}
\begin{enumerate}
	\item Votaci�n por internet
	\item Disponibilidad 24/7
	\item Todos los requisitos del voto electr�nico
\end{enumerate}
	
\chapter{Riesgos}\label{riesgos}
\lhead{Cap�tulo \ref{riesgos}}
\rhead{Riesgos}
%*******************************************************************************
\section{Identificaci�n y gesti�n de riesgos}\label{idGestRiesgos}
%*
%*
%*
%*
%*
%*

	\subsection{Identificaci�n de riesgos}\label{identificacionRiesgos}
\par
El objetivo del an�lisis es determinar qu� solucion se va a tomar, independientemente de la forma en que se har�, o de las decisiones que se tomen para la construcci�n, cuestiones que se abordar�n m�s adelante en el dise�o (apartado \ref{diseno}). Por tanto, en el presente apartado, nuestro objetivo es mostrar la elecci�n de la l�gica de aplicaci�n que deber� seguir el sistema que estamos construyendo.
\par
\subsection{Elecci�n de la logica de la aplicaci�n}\label{logica}
\par
La elecci�n de la l�gica a utilizar debe encajar con todos los requisitos planteados en el apartado \ref{requisitos}, que se adapte al sistema real descrito en el apartado \ref{sistemareal} y a los objetivos indicados en \ref{objconcr}.
\par
Existen diferentes tipos de software donde podr�amos intentar encajar estas necesidades. De las diferentes opciones que se barajan, se tiene que descartar cualquier opci�n que no cumpla alg�n requisito o no pueda corresponderse correctamente con el sistema real y los objetivos descritos.
\par
\subsubsection{Opciones estudiadas}
\par
Para poder evaluar las distintas opciones, y siguiendo las instrucciones que acabamos de indicar, se han tomado principalmente cuatro criterios para su evaluaci�n (tal y como adelant�bamos en el punto \ref{plan_analisis}:
\par
\begin{enumerate}
	\item \textbf{Modelo.} La opci�n tiene que adaptarse al modelo real, descrito en el apartado \ref{sistemareal} y al modelo funcional descrito en \ref {mod_func_corregido}.
	\item \textbf{Disponibilidad.} La informaci�n relativa a los eventos tienen que estar disponibles para cualquier usuario que necesite acceder a ella.
	\item \textbf{Administraci�n.} Los eventos deben poder ser administrados, gestionados, asignados, etc, con facilidad.
	\item \textbf{Evoluci�n.} Los usuarios podr�n estudiar con total facilidad la evoluci�n de los eventos, acceder y administrar hist�ricos de eventos con facilidad.
\end{enumerate}
\par
A continuaci�n se muestran las opciones opciones posibles, de las cuales se han descartado las no totalmente v�lidas para obtener la soluci�n �ptima.
\par
\begin{itemize}
	\item \textbf{Correo Electr�nico.} El correo electr�nico, o E-mail, es probablemente el sistema de env�o de mensajes m�s popular. Encaja perfectamente con el modelo funcional descrito en la figura \ref{fig:aproxfunc}, donde un emisor enviaba un mensaje a un receptor (o a m�s de uno). En cambio no se corresponde con el modelo ampliado donde todos los usuarios enviaban la informaci�n a un contenedor y pod�an disponer de ella cuando fuera necesario, como se apreciaba en la figura \ref{fig:modampli}.
	\par 
Si un usuario comenzara a participar de una discusi�n, no tendr�a disponible la informaci�n que pudiera haber sido enviada anteriormente. Adem�s el control de ese paso de mensajes no es algo sencillo de administrar, asignar, reasignar o cambiar de estado, pues no es su objetivo. La revisi�n de la evoluci�n de los asuntos no es tampoco sencilla pues �nicamente podr�a llevarse a cabo mediante reenv�os de mensajes, que podr�an reenviarse de un modo incorrecto, a usuarios incorrectos. 
	\par
Resumiendo, el correo electr�nico no es v�lido pues su objetivo es el env�o simple de informaci�n, pero no el control de la evoluci�n de la misma, requisito indispensable para nuestro sistema.
	\item \textbf{Foros, grupos de correo y de discusi�n.} Quiz�s se asimilen m�s al modelo funcional de informaci�n en contenedor de la figura \ref{fig:modampli}. Su objetivo precisamente es ese, permitir a los usuarios que publiquen una informaci�n, que quede a disposici�n del resto de usuarios, e incluso nos permitir�an clasificar esa informaci�n por asuntos, que podr�an simular los eventos.
	\par
Ser�a la soluci�n m�s b�sica que podr�a acercarse al modelo que buscamos, pero funcionalmente dejan bastante que desear. Ni nos permiten realizar una revisi�n muy buena de la evoluci�n de los eventos, ni mucho menos nos permite administrar, asignar, reasignar o cambiar el estado de los eventos.	
	\item \textbf{Wiki.} Una wiki es un sitio web colaborativo que puede ser editado por varios usuarios, creando, modificando, o borrando el contenido de forma interactiva, f�cil y r�pida. La facilidad de participaci�n en una wiki podr�a mostrar que se trata de una opci�n interesante. Pero estos sistemas �nicamente son interesantes para publicar informaci�n y enriquecerla de un modo colectivo. No son v�lidos para gestionar eventos, y funcionalmente tienen todos los inconvenientes de las dos opciones anteriores, a�adiendo adem�s la imposibilidad de revisi�n de la evoluci�n de un asunto.
	\item \textbf{Mensajer�a instantanea.} Si bien los chats y otros sistemas de mensajer�a instantanea son una herramienta potent�sima para la comunicaci�n en tiempo real, evidentemente es la peor opci�n si queremos revisar la evoluci�n de cualquier asunto. No hay que olvidar que el objetivo de este sistema es la asignaci�n de eventos entre actores, y la revisi�n de la misma por otros usuarios, y nada tiene que ver con lo que estos progamas ofrecen.
	\item \textbf{Issue Tracker.} Estos sistemas precisamente est�n ideados para gestionar listas de reportes y eventos, mediante la asignaci�n y cambios de estados en los mismos. Corresponden con el modelo, la informaci�n puede estar en todo momento disponible para cualquier usuario que participe, el seguimiento del hist�rico de cada evento es sencillo, pero sobre todo es que est�n dise�ados para hacer ese tipo de administraci�n de los eventos que requerimos para obtener la soluci�n. A continuaci�n se describir�n en profundidad estos sistemas, y se justificar� la elecci�n de los mismos.
\end{itemize}
\par
En la tabla \ref{tab:ComparativaLogica}, se puede ver un peque�o resumen de las opciones estudiadas, donde en un vistazo r�pido se pueden apreciar los inconvenientes y ventajas de cada uno, en base a los criterios seguidos para evaluarlos.
\par
\begin{table}[htbp]
	\centering
		\begin{tabular}{|l|c|c|c|c|}
		\hline
			&\textbf{Modelo}&\textbf{Disponibilidad}&\textbf{Admin.}&\textbf{Seguimiento}\\
		\hline
		\hline
			\textbf{E-Mail}&No&Mal&No&Mal\\
		\hline
			\textbf{Foros, grupos}&Si&Si&Mal&Mal\\
		\hline
			\textbf{Wiki}&Si&Si&Mal&Mal\\
		\hline
			\textbf{Chat}&No&Mal&Mal&Mal\\
		\hline
			\textbf{Issue Tracker}&Si&Bien&Bien&Bien\\
		\hline
			
		\end{tabular}
	\caption{Resumen comparativo de las l�gicas posibles}
	\label{tab:ComparativaLogica}
\end{table}

\subsubsection{Justificaci�n de la opcion escogida: issue tracking}
\par
Los sistemas \textbf{issue tracking system} (tambi�n conocidos como bug tracking system, trouble ticket system o incident ticket system), son sistemas que gestionan y mantienen tickets (issues) que representan los asuntos a controlar. Estos tickets tienen un contenido, relativo al asunto que tratan, m�s un estado y una asignaci�n. Adem�s es caracter�stica de estos sistemas el archivado de un hist�rico de la vida de cada ticket, incluyendo sus cambios de contenido, estado y asignaci�n, de un modo optimo para su consulta, an�lisis, elaboraci�n de  estad�sticas, etc.
\par
Un ejemplo de funcionamiento podr�a venir marcado por el siguiente flujo de trabajo:
\par
\begin{enumerate}
	\item Un usuario crea un ticket sobre un evento ocurrido. Lo cumplimenta con la informaci�n apropiada, le indica un estado \textit{por resolver}, y se lo asigna al usuario apropiado.
	\item El segundo usuario puede cambiar la informaci�n y asignar el ticket a otro usuario.
	\item Llegado el momento en que se ha resuelto el evento, el usuario propietario del ticket en ese momento lo podr� cambiar al estado \textit{resuelto}. 
	\item El ticket no desaparece: aunque el asunto ya est� resuelto y desasignado, y no se vuelva a operar con �l, todo su hist�rico sigue almacenado en el sistema. 
\end{enumerate}
\par
En todo momento, todo el hist�rico de actuaciones sobre el ticket podr� ser consultado con total facilidad. Cualquier cambio realizado por cualquier usuario queda reflejado y puede ser consultado incluso una vez resuelto el asunto.
\par
La arquitectura de estos sistemas suele consistir en una base de datos que funciona como repositorio de la informaci�n relativa a los tickets, que es manejada por la propia l�gica de negocio situada en otra capa distinta (no necesariamente en distintas ubicaciones f�sicas). La presentaci�n al usuario suele aportarse mediante otra capa adicional, mediante una interfaz sencilla (interfaz web, por ejemplo).
\par
Su funcionamiento los hace muy populares en servicios de atenci�n telef�nica o call centers, donde se gestionan incidencias: en estos casos lo que se hace es crear un ticket para cada incidencia, y son ideales para llevar un control y un seguimiento de las mismas (cada nueva anotaci�n sobre la incidencia se introduce como un evento nuevo), gestionandose adem�s la asignaci�n de la incidencia en cada momento.
\par
El sistema de gesti�n de eventos que queremos desarrollar, tiene un funcionamiento muy similar al ejemplo mostrado:
\par
\begin{enumerate}
	\item Un usuario de rol 1 da de alta un evento, con estado \textit{por resolver} aporta la informaci�n necesaria y lo asigna a un usuario de rol 2.
	\item Entre ambos usuarios pueden aportar nueva informaci�n, pueden asignarse el evento entre ellos, o incluso a otros elementos intervinientes (medios de evacuaci�n, usuarios de rol 3, etc). 
	\item Una vez que la eventualidad se ha solucionado, se cambiar� a estado \textit{resuelto}.
	\item Aunque el evento se dar� como resuelto no se eliminar� porque se debe mantener el hist�rico de actuaciones.
\end{enumerate}
\par
Cada nueva aportaci�n, asignaci�n, cambio de estado o de urgencia debe quedar registrada para que el usuario de nivel 3 de quien dependan pueda obtener estad�sticas e informaci�n, o incluso dar instrucciones a los niveles inferiores. Y esa informaci�n quedar� disponible durante toda la vida del ticket, incluido cuando el evento ya ha sido resuelto.
\par
La similitud funcional entre lo ofrecido por este tipo de sistemas, y la funcionalidad requerida en este proyecto, y el cumplimiento de los criterios de estudio antes indicados, justifican que esta opci�n es idonea como soluci�n al problema.
\par
\chapter{Soluci�n}\label{solucion}
\lhead{Cap�tulo \ref{solucion}}
\rhead{Descripci�n detallada de la soluci�n}
%*******************************************************************************
% *************************************************************************************************************** %
% 			SOLUCI�N
% *************************************************************************************************************** %
%%%%%%%\section{Identificaci�n y gesti�n de riesgos}\label{idGestRiesgos}
%*
%*
%*
%*
%*
%*

	%%%%%%%%\subsection{Identificaci�n de riesgos}\label{identificacionRiesgos}
	\todo[inline]{He eliminado los diagramas que ten�a porque ya no se corresponden con el nuevo sistema.}
	%\begin{figure}[htbp]
		%\centering
			%\includegraphics[width=1\textwidth]{imgs/sistema.png}
		%\caption{Diagrama de flujo del Sistema}
		%\label{fig:sistema}
	%\end{figure}
	%
	%\begin{figure}[htbp]
		%\centering
			%\includegraphics[width=0.60\textwidth]{imgs/flujoUsuario.jpg}
		%\caption{Esquema del flujo que sigue el votante}
		%\label{fig:flujoUsuario}
	%\end{figure}
	%\begin{figure}[htbp]
		%\centering
			%\includegraphics[width=0.60\textwidth]{imgs/flujoSistema.jpg}
		%\caption{Esquema del flujo del Sistema}
		%\label{fig:flujoSistema}
	%\end{figure}
	
		
		\section{Dise�o}\label{dise�o}
		
			Tal como se estudia en el an�lisis (\ref{analisis}), la soluci�n que se propone para este Proyecto consiste en adaptar el sistema Helios Voting a las necesidades de la EPS.
			
			Para ello, habr� que tener en cuenta qui�nes son los actores que han de interactuar, as� como los cambios que hay que realizar para llegar a implementar el sistema en este nuevo escenario.
			
			
		
			\subsection{Dise�o del esquema de votaci�n}\label{disenhoEsquemaVoto}
				\subsubsection{Registro}
					La fase de registro de votantes en el sistema no ser� interactivo en cuanto a que no es el propio votante el que debe inscribirse para poder votar en las elecciones, sino que es la Autoridad Electoral la que lo registra en el censo. En esta fase, pues, se trata de establecer el censo de votantes que tienen autoridad para votar en el proceso electoral.
					
					Como se advierte en el an�lisis se ha tomado en consideraci�n que sean los administradores del sistema quienes tengan responsabilidad sobre el tratamiento del censo, por lo que se ha de cargar en el sistema y �ste es el que lo va a tratar.
					
					El censo ha de cargarse en dos servicios del sistema, tanto en el servidor de autenticaci�n como en el sistema de votaci�n.
					
					El censo del subsistema de autenticaci�n se utilizar� para llevar el control de los votantes que tienen derecho de acceso al sistema, por lo que a los votantes se les pueden a�adir otros usuarios necesarios para llevar a t�rmino la votaci�n, como pueden ser administradores o auditores, aunque estos no tengan derecho de voto. As� se conformar�a la base de usuarios activos del sistema.
					
					En el subsistema de voto tambi�n se vuelca el censo para cada uno de los diferentes procesos de voto que conformen la elecci�n.
					
					El procedimiento a seguir consistir� en que la Autoridad Organizadora del Proceso Electoral, la Universidad, proveer� una lista del censo a los administradores del sistema. El administrador utilizar� la funci�n de carga de votantes con la lista proporcionada para realizar la carga inicial de votantes para cada una de las subelecciones que se configuren.
					
					La lista proporcionada por la Universidad debe contener la siguiente informaci�n de cada uno de los votantes:
						\begin{itemize}
							\item Nombre y apellidos
							\item DNI
							\item Clase / grupo de votantes
							\item E-mail
						\end{itemize}
						
					La carga de los votantes a trav�s de su aplicaci�n se realiza subiendo un fichero csv con la informaci�n requerida. Este fichero se pone a disposici�n de la cola de procesos, la cual, llegado el momento volcar� cada uno de los registros en la base de datos del sistema de votaci�n.
					
					\notasDuda{Falta definir el proceso de carga de votantes para el servicio de autenticaci�n.}
					
				\subsubsection{Identificaci�n / Autenticaci�n}
					El servicio de identificaci�n es un subsistema clave en el proceso electoral. En �l recae parte de la responsabilidad de la robustez del sistema, en cuanto a que debe asegurar varios de los requisitos b�sicos que definen el voto electr�nico en concreto:
					
		\notasInfo[inline]{Esto es as� seg�n los requisitos de Fujioka, si se utilizan los de la UNEX, se puede modificar.}
		\begin{description}
			\item[Solidez:] Debe asegurar que un votante deshonesto no tenga capacidad de acceder al sistema e interrumpir la votaci�n. Es decir, que s�lo debe dar acceso a los votantes que realmente deben ingresar al sistema de votaci�n.
			
			\item[Elegibilidad:] Este requisito implica que el sistema debe controlar que ning�n votante que no tenga permitido el voto pueda votar. Aunque es el proceso de votaci�n el que debe controlar esta circunstancia cuando un usuario trata de emitir un voto, el sistema de votaci�n, de forma an�loga al requisito anterior, tambi�n debe proteger el sistema evitando el acceso a aquellos que, directamente, no tengan permisos para votar.
			
			\item[Sin duplicados:] El sistema debe evitar que un votante duplique o reemplace el voto de otro. Igualmente, aunque es el sistema de votaci�n el que debe tener mecanismos que controlen esta situaci�n, la primera barrera debe ser la servicio de autenticaci�n del votante.
			
		\end{description}

					
			 El sistema de identificaci�n del votante se apoya en el protocolo oAuth2.
			\notasInfo[inline]{Aqu� desarrollamos la implementaci�n oAuth que hemos desarrollado para este sistema}
					
					
					\begin{figure}[!ht]
						\centering
							\includegraphics[width=.95\textwidth]{imgs/flujoOAuth01.jpg}
						\caption{Flujo oAuth para servicio de autenticaci�n}
						\label{fig:flujo.oauth}
					\end{figure}
					
					
					
					
					
					
					
					
				\subsubsection{Elecci�n de candidatura}
				\subsubsection{Votaci�n}
				\subsubsection{Escrutinio}
				\subsubsection{Difusi�n de resultados}
				
			\subsection{Dise�o de la arquitectura}
				\todo[inline]{Aqu� metemos la explicaci�n de los dos servidores, el de auth y el de Helios, con sus conexiones por Internet. Podemos adelantar la implementaci�n pensando en dos raspberrys? Aqu� se habla de los dos Apaches en uno de los servidores (el de DNIe y el otro)??}
				
				\begin{figure}[!ht]
						\centering
							\includegraphics[width=.5\textwidth]{imgs/arquitectura01.jpg}
						\caption{Arquitectura}
						\label{fig:arquitecura01}
					\end{figure}
					
					\begin{figure}[!ht]
						\centering
							\includegraphics[width=.5\textwidth]{imgs/arquitectura02.jpg}
						\caption{Arquitectura nivel 2}
						\label{fig:arquitectura02}
					\end{figure}
				
				
			\subsection{Dise�o de la capa de datos}
				\todo[inline]{Al hablar de Helios ya se puso un diagrama de datos. Repetir destacando los cambios. A�adir el diagrama E-R del servidor de oAuth}
			
			\subsection{Dise�o de la red}
				\todo[inline]{Para esto s� que necesito ayuda, por el tema de Firewalls y esas cosas.}
				
			\subsection{Dise�o de la app de identificaci�n}
							
			\subsection{Dise�o de la interfaz de usuario}\label{dise�o_interfaz_usuario}
				Para la interfaz de usuario hay varias necesidades que se han debido de satisfacer.
				
				Por un lado, la imagen corporativa. Al tratarse de un proceso electoral dise�ado para una entidad, la plataforma en la que se basa deber�a mostrar inequ�vocamente la imagen de la entidad que lo organiza.
				
				Otro aspecto a tener en cuenta ser� la adaptaci�n a dispositivos m�viles, pues lo que se busca es el voto seguro desde este tipo de dispositivos, algo que no cubre con suficiencia la versi�n actual de Helios Voting.
				
				\subsubsection{Estructura de la p�gina web}
					\todo[inline]{Cambios en la interfaz original de Helios. Responsive....}
					
				\subsubsection{Estructura de la aplicaci�n m�vil}
					\todo[inline]{Interfaz de usuario de la app.}
					
				\subsubsection{Accesibilidad}
					\todo[inline]{Destacar accesibilidad}
					
				\subsubsection{Imagen corporativa}
					\todo[inline]{Como imagen de marca, bla bla bla, al no haber un logo del proceso, bla bla bla, se puede introducir uno o utilizar el de la Universidad, bla bla}
					La interfaz de Helios Voting se corresponde con un sistema neutro para realizar elecciones en la plataforma web de ejemplo que tienen publicada junto con el c�digo fuente del proyecto.
					
					Esta interfaz est� bien para el prop�sito que tienen de, por un lado, mostrar un ejemplo de funcionamiento del proyecto y, por otro, dar una herramienta funcional para la realizaci�n de peque�os procesos electorales seguros por Internet para peque�os grupos que no necesiten una implementaci�n propia.
					
					A la hora de utilizar esta herramienta para montar un sistema electoral propio, en infraestructuras de un cliente, esta interfaz no se corresponde con la que deber�a tener un proyecto serio.
					Cualquier proceso electoral suele y debe tener una imagen corporativa propia que pueda identificarse claramente con el propio proceso y con las autoridades que lo organizan, con la finalidad de que el votante identifique claramente el origen del proceso y no d� lugar a equ�vocos, adem�s de que, en cierto modo, aumentan la seguridad del mismo en el proceso.
					
					La Universidad, al no haber estar habituada a procesos electorales por Internet, no suele llevar a cabo el desarrollo de una imagen corporativa para este tipo de eventos. No obstante, s� que lo hace con otros que suele organizar, ya sean acad�micos o sociales, adem�s de que, como la mayor�a de entidades, tiene construida una imagen de marca propia.
					
					Bas�ndonos en esta imagen de marca propia de la Universidad San Pablo CEU, se ha dise�ado una gama de colores y una serie de logos e im�genes para dar una imagen corporativa al proceso electoral. Aunque se ha utilizado la interfaz base del proyecto Helios, ha habido que modificarla para hacer que se relacione la web con la Autoridad que organiza la Elecci�n.
					
					\notasInfo[inline]{Dar una muestra de colores y de logos de la elecci�n}
					
					\begin{figure}[!ht]
						\centering
							\includegraphics[width=.5\textwidth]{imgs/logo_usp_ceu.jpg}
						\caption{Logo de la Universidad San Pablo CEU en su web}
						\label{fig:logo.universidad_san_pablo_ceu}
					\end{figure}
				
				
					\notasInfo[inline]{Colores...accesibilidad!}				
				
				
				
				
				
				
				
				\todo[inline]{Desde aqu� hasta el comienzo de \ref{solucion_protocolo} hay que moverlo a An�lisis.}
				En varios de los sistemas estudiados que se han desarrollado para intentar implantar el voto electr�nico a un nivel medio, como pueden ser los mexicanos SELES \ref{seles} y SEVI \ref{sevi} o los espa�oles de V�ctor Moreno \cite{moreno07} \notasDuda{???????}o Votescript \ref{ivotingVotescript}\notasDuda{???????} se observa que se realiza una divisi�n del proceso electoral en cuatro fases (Registro, Votaci�n, Consolidaci�n de resultados y Auditor�a). En el desarrollo de este sistema vamos a identificar las mismas fases, pero con matices.
				
				As�, en una primera visi�n global del sistema, en este se definen cuatro fases:
				\begin{itemize}
					\item Preelectoral
					\item Votaci�n
					\item Consolidaci�n de resultados
					\item Postelectoral
				\end{itemize}
				
				Realmente, la mayor diferencia con las fases definidas en los esquemas anteriores se corresponden con el alcance de la primera y la �ltima fase.
				La fase Preelectoral, denominada com�nmente en los ejemplos estudiados en la Introducci�n como fase de Registro, en este sistema tiene un alcance mayor. En este proceso electoral no se requiere que el votante se registre para poder votar. El censo lo proporciona la Autoridad Electoral y se carga en el sistema. Igualmente, en los d�as previos a la jornada electoral el sistema permitir� a los votantes comprobar si est�n en el censo y qu� informaci�n contiene �ste, tanto personal - para asegurarse de que podr�n identificarse - como de permisos de cara a realizar la votaci�n.
				
				
				
				Las fase de Votaci�n tambi�n tiene un alcance diferente. En primer lugar, empieza con la identificaci�n del votante en el sistema electoral. Una vez el votante ha sido correctamente identificado por el sistema (tal como lo har�a contra los miembros de la mesa en el voto tradicional), debe recibir una boleta electr�nica que le ofrezca las opciones entre las que, por su circunscripci�n, deba elegir la que desea votar. Una vez seleccionado, es el momento en el que realmente el votante realiza la votaci�n, traspasando el voto de forma digital al sistema, a la \textit{urna digital} donde se anonimizar�n y almacenar�n hasta la fase de consolidaci�n.
				
				En la fase de consolidaci�n de resultados, el sistema se encargar� del conteo de los votos que han sido emitidos
				\todo[inline]{continuar...}
				
				La fase Postelectoral, que denominan Auditor�a, prefiero dejarla con este nombre, ya que considero que la auditor�a del sistema es una operativa que se realiza durante toda la jornada electoral, no s�lo al finalizar �sta. No obstante, es cierto que al final se llevar�n a cabo auditor�as de los resultados y el funcionamiento. Adem�s de las auditor�as llevadas a cabo por los auditores \textit{oficiales}, se va a implementar un mecanismo que permita a los propios votante auditar que su voto ha sido correctamente incluido y contado en el proceso. Esta fase postelectoral tambi�n tiene m�s pasos ... \todo[inline]{continuar con fase postelectoral}
				
			\todo[inline]{Antes de este punto hay que hacer un resumen de los diferentes esquemas de votaci�n, teniendo estos como Firma ciega, mixnets, etc...}


\iffalse
			\subsection{Protocolo}\label{solucion_protocolo}
			\textst{
				Como se ha comentado en cap�tulos anteriores, hay una multitud de soluciones propuestas para el voto telem�tico.
				
				Teniendo en cuenta el objetivo de este Proyecto Fin de Carrera, de los sistemas implementados a gran escala, a nivel nacional o regional, podemos destacar Estonia, Noruega y los cantones suizos como las tres experiencias m�s exitosas y aquellas de las que se pueden estudiar las soluciones, esquemas y protocolos utilizados. No obstante, el alcance de las mismas supera sobremanera el de este proyecto. Igualmente, muchas decisiones las toman en base a satisfacer requisitos que resultan muy importantes en su an�lisis, pero que en este trabajo no se ha considerado que tengan igual trascendencia, y viceversa, por lo que se han de tomar diferentes consideraciones frente a los mismos problemas dependiendo del impacto que suponen en cada proyecto.
				}
				Tambi�n se han presentado casos de proyectos de voto telem�tico pensados a menor escala. Entre ellos, hay muchas soluciones que, en parte, podr�an satisfacer los requisitos de este proyecto. No obstante en ninguno de ellos encontramos un protocolo que se adapte completamente a los requerimientos planteados, ya que, en alg�n momento, se analiza un elemento que los hace diferir. Por ejemplo, un proyecto ya maduro como Votescript (\ref{ivotingVotescript}) realiza un estudio acad�mico y t�cnico muy profundo acerca del voto telem�tico pero, por su propia definici�n, el modelo de identificaci�n y emisi�n del voto lo sit�an f�sicamente en centros de votaci�n. Este elemento es diferencial para este proyecto, pensado en el voto telem�tico remoto, aunque puede integrarse cuando se estudian alternativas para que aquellos votantes que, por alg�n motivo, no pueden o quieren votar por Internet de forma remota tengan la oportunidad de ejercer su derecho de sufragio desde un lugar habilitado para ello por la propia Escuela.
				\textst{
				A partir de los esquemas criptogr�ficos estudiados y con ayuda de algunos protocolos ya publicados en otros proyectos, el siguiente paso es dise�ar el protocolo de votaci�n que se adapte a las necesidades del Proyecto, cumpliendo con los requisitos y asegurando los niveles de seguridad planteados.
				
				En muchas de las soluciones estudiadas se observa que no recibe la importancia necesaria la fase de identificaci�n del votante. Los mecanismos de identificaci�n y autenticaci�n del mismo resultan laxos desde el punto de vista de la seguridad ante el fraude electoral. Por ello han sido descartadas las soluciones basadas en identificaci�n por medio de bases de datos con el t�pico protocolo de usuario/contrase�a o incluso con elementos de seguridad de una generaci�n algo posterior, como pin, patrones, captchas, operaciones aritm�ticas o m�todos similares con mayor o menor complejidad. Igualmente, se han descartado aquellos m�todos de identificaci�n que requieran la presencia f�sica del votante frente a los responsables de la mesa de votaci�n, ya que se busca el dise�o de un sistema remoto. As� descartamos protocolos de identificaci�n como los publicados por Votescript, en el que el votante acude a un centro o local de votaci�n, se identifica ante la mesa electoral y recibe un token criptogr�fico personalizado con el que se le permite ejercer el sufragio.
				
				La mayor�a de las soluciones estudiadas previamente a la realizaci�n de esta memoria centran sus esfuerzos en la fase de votaci�n. Buscan la elaboraci�n de un protocolo robusto, basado en esquemas criptogr�ficos, que permita la mayor seguridad posible al cumplimiento de los requisitos fundamentales del voto electr�nico, dotando al sistema de privacidad del votante, 
			}
				\todo[inline]{Hay que modificarlo. Es anterior a la decisi�n de utilizar Helios}
\fi				
				
				
				
				
				
				
				
				
				
				
				
				
				
				%\subsubsection{Descripci�n del sistema}\label{solucion_descripcion}
				%El sistema contar� de cinco fases, determinadas por el flujo temporal de la votaci�n.
				%Preelectoral, Identificaci�n, Votaci�n, Escrutinio y publicaci�n de resultados.
				%Adicionalmente, se tendr� en cuenta un sistema de auditor�a, de car�cter transversal a este flujo, ya que debe estar disponible durante todo el proceso de votaci�n.
				%
				%\todo[inline]{No he podido conseguir reglamentaci�n oficial de la elecci�n, as� que, b�sicamente, propongo yo las fases y la problem�tica ... esto, con palabras aqu� escrito y bien puesto }
				%
				
\iffalse
				\section{ESTO ES EL PFC}
				\todo[inline]{ESTO NO VA EN EL PFC, ES UNA EXPLICACI�N PARA TENER PRESENTE QU� ES EL PFC YU PODER DESARROLLAR LA MEMORIA EN TORNO A LA IDEA QUE TENEMOS.}
				El sistema que se propone en este PFC es un sistema integral. Busca sostener el proceso electoral desde el comienzo hasta el final del mismo. Por ello empieza en el momento mismo de definici�n del censo y no termina hasta que la publicaci�n de resultados y su auditor�a son oficializadas por el �rgano rector de la Elecci�n.
				
				La primera fase, preelectoral, es aquella previa al d�a electoral, en la cual se definen las bases en las que se rige el proceso electoral.
				
				As�, es imprescindible cumplimentar varias acciones por parte de los desarrolladores, administradores y �rgano electoral.
				
				En primer lugar, es fundamental la elaboraci�n de un censo electoral. En �ste se recogen los potenciales votantes, aquellos con derecho a voto, identificando, adem�s, la circunscripci�n \notasDuda{Cir-cuns-crip-ci�n??? No hay una forma mejor de expresarlo??} a la que pertenece. En unas elecciones legislativas, una circunscripci�n electoral se puede definir como el conjunto de electores a partir del cual se procede la distribuci�n de los esca�os asignados, en funci�n de la distribuci�n del los votos sufragados. En las elecciones legislativas espa�olas, las circunscripciones se corresponden con las provincias espa�olas (excepto en el caso de Aturias, que est� subdividida en 3 distritos electorales, y la Regi�n de Murcia, que lo hace en 5). Esto significa que del total de diputados que se eligen en este proceso para la totalidad de Espa�a, en vez de repartirlos con el recuento total de los votos, se reparten los cargos por cada circunscripci�n, dependiendo del n�mero de electores de cada una, con lo que los votantes censados en una circunscripci�n, digamos por ejemplo la provincia de M�laga, elegir�n a un n�mero determinado de diputados que ser�n quienes les representen en el Congreso junto a los elegidos en el resto de territorios espa�oles. En las Elecciones al Parlamento Europeo, sin embargo, Espa�a act�a como una �nica circunscripci�n, por lo que los diputados que representar�n al pa�s en la c�mara supranacional se obtendr�n a base de repartir los esca�os con respecto al total de votos recogidos en todo el territorio espa�ol.
				
				Algo parecido es lo que se va a definir en el censo electoral. Adem�s de recoger de forma un�voca a los electores con derecho al voto, se tendr�n que sumar las ************\notasCambio{*********} necesarias para su correcta identificaci�n, as� como la ``circunscripci�n'' a la que pertenece, es decir, el grupo sobre el que debe escoger a sus representantes, con el fin de que la opci�n de voto que el sistema le presente y la que introduzca en el sistema sea correcta.
				
				Se vislumbran aqu� dos requisitos del voto electr�nico que necesitan ser satisfechos para la integridad del proceso electoral.
				
				En primer lugar, es b�sico que el censo defina claramente los votantes con derecho al voto y provea de la informaci�n necesaria para que se pueda comprobar la identidad del votante en el momento en el que se disponga a votar. En las elecciones con voto tradicional esto se consegu�a a�adiendo datos personales tales como el n�mero del DNI, del Pasaporte o, en caso de estas elecciones, el n�mero de identificaci�n del alumno. As�, al acudir a la mesa electoral todos los votantes ten�an estos datos con los que se pod�an identificar frente a los miembros de la misma, los cuales tienen la potestad de permitirles votar o no.
				
				Integridad del voto. El hecho de relacionar cada votante con una ``circunscripci�n'' es esencial a la hora de mantener la integridad de la votaci�n, pues hay que tener en cuenta los candidatos a los que cada votante puede votar, ya que no son los mismos para todos. Igual que en unas legislativas espa�olas un votante de M�laga no elige entre los mismos candidatos que lo hace un votante de Lugo, en estas elecciones, un alumno elige sus representantes entre los delegados de curso, mientras que los profesores, por su parte, lo hacen entre otros colegas profesores. Es indispensable, pues, gestionar correctamente estas relaciones ya que no se deben recoger votos de votantes a candidatos a los que no tiene derecho a elegir.
				
				En el caso de esta elecci�n, es la propia Escuela Polit�cnica Superior la que debe proveer el censo oficial a los administradores del sistema, los cuales proceder�n a cargarlo en el mismo a trav�s de los mecanismos implementados para ello.
				\notasInfo[inline]{(Aqu� encontramos un primer punto de auditor�a importante).}
				\notasInfo[inline]{(En algunos pa�ses, en vez de elaborarse un censo oficial, son los propios votantes los que han de registrarse)}
				
				
				Es requisito de la Instituci�n que convoca el proceso electoral el definir las ``reglas del juego''. En este caso, el �rgano de la EPS encargado de la celebraci�n de las elecciones ha de definir los mecanismos de votaci�n para que el sistema se pueda adaptar y mantener .......
				\todo[inline]{continuar...}
				
				Candidatos. Es necesario que los candidatos puedan presentar su candidatura e incorporarse al sistema para que �ste pueda gestionarlos para presentarlos como opciones a los votantes determinados, adem�s de en el momento de consolidaci�n de los votos y posterior publicaci�n de resultados.
				En muchos procesos se realizan desarrollos que permiten a los partidos pol�ticos registrar sus listas electorales y/o candidatos de forma remota durante el plazo determinado que la Ley Electoral les indica. As�, los partidos inscriben a sus representantes en el proceso electoral. En el caso de esta elecci�n, debido a su car�cter tan localizado no vemos necesidad de ello y corresponde a la Escuela Polit�cnica Superior proporcionar el listado de candidatos elegible y las circunscripciones a las que se presentan.
				\todo[inline]{Para futuros desarrollos, pensando en la escalabilidad del sistema, se podr�a desarrollar este punto para que este proceso sea independiente de los �rganos electorales de la EPS}.
				
				En las elecciones tradicionales, es tambi�n necesaria la formaci�n de las mesas electorales, con la definici�n del n�mero de ellas que son necesarias y la designaci�n de los miembros que van a formar parte de ella. En una elecci�n electr�nica y remota, como la que hemos dise�ado, el concepto de mesa se puede mantener, sobre todo para poder gestionar las circunscripciones y para continuar con las estad�sticas de participaci�n tradicionales, basadas en agrupaciones y disgregaciones de mesas. Sin embargo, al transformarse en un concepto l�gico, se pierde el sentido de la designaci�n de los miembros de mesa, por lo que no ser� un punto a tener en cuenta en el proceso.
				
				
				\todo[inline]{Pasamos a la siguiente fase: Identificaci�n}
				Una vez acometidas todas las gestiones de la fase preelectoral, pasamos a la fase correspondiente al llamado D�a Electoral (aunque realmente la elecci�n en vez de en un d�a, se pueda alargar a lo largo de un per�odo de tiempo mayor).
				Tratando de emular a las elecciones tradicionales, esta fase comienza con la apertura de los colegios electorales y las mesas que los componen. En el caso digital, ser�n los miembros designados por la Junta Electoral los que, previa identificaci�n y requerimiento de sus credenciales digitales, pongan en marcha el sistema en su fase electoral. Ser� una apertura de los colegios de forma virtual, permitiendo que los votantes puedan acceder al sistema y proceder a votar.
				La fase de identificaci�n del votante es una fase realmente importante. En las elecciones de voto tradicional, el proceso normal consiste en que el votante acude a la mesa electoral y muestra a los miembros de mesa alguna identificaci�n de curso legal, respaldada por alguna instituci�n estatal reconocida y capacitada. Los miembros de la mesa electoral contrastan la identificaci�n presentada con la informaci�n recogida en el censo electoral de dicha mesa y deciden si es suficiente o no para permitir al votante introducir su voto en la urna. En el caso de las elecciones legislativas espa�olas los documentos que se pueden mostrar son DNI, pasaporte o permiso de conducir. Todos estos documentos son v�lidos para votar incluso estando caducados. Han de mostrar la fotograf�a del votante para permitir la identificaci�n por parte de los miembros de mesa, por lo que, aunque sea v�lido que est�n caducados, no se permite utilizar el resguardo de DNI en tr�mite.
				\todo[inline]{En el caso de las elecciones de la EPS, los documentos v�lidos son .-......}
				Es requisito el sustituir este sistema de identificaci�n del elector por otro en el que no sea necesaria la presencia f�sica de �ste ni de los miembros de mesa para permitir el voto, aunque manteniendo el mismo nivel de seguridad en el proceso. Aqu� se hace indispensable estudiar las opciones de identificaci�n digital que se pueden implementar para .............
				\todo[inline]{continuar}
				
				Lo ideal es disponer de documentos que contengan tokens criptogr�ficos propios que puedan ser utilizados en los diferentes procesos de identificaci�n y voto. Por ello, vamos a utilizar documentos que los disponen.
				
				As�, los documentos v�lidos para ejercer el derecho al voto ser�n el DNIe (tanto la primera versi�n como la denominada 3.0, presentada en enero de 2015) y la TUI \notasMejorar{***InfoTUI***} de la Universidad San Pablo-CEU. En sendos documentos encontramos elementos criptogr�ficos que identifican un�vocamente a su due�o. Adem�s encontramos en ellos certificados para la firma digital, que ser�n necesarios para la fase de votaci�n.
				
				El votante se identifica con su documento digital de forma remota. Es necesario que disponga de un lector de chip electr�nico conectado al dispositivo desde el que va a realizar el voto, aunque utilizando DNIe con lector de chip sin contacto, no har�a falta si se hace uso de un dispositivo con sensor de radiofrecuencia, con capacidad para leer informaci�n a trav�s de NFC.
				
				A trav�s de la app Android (o la app web), el votante accede al servicio de votaci�n por Internet. El primer paso es la identificaci�n del votante. Es la primera vez que har� uso de los certificados del DNIe. En este caso, la app leer� (con NFC o chip con contacto) el certificado de Autenticaci�n del DNIe, por el cual se asegura la identidad del votante. Con la identidad del votante verificada (por la DGP), se contrasta con el censo, para comprobar:
				\begin{itemize}
					\item Si el votante existe en el censo.
					\item Si el votante ha votado previamente.
					\item Los datos censales del votante, para comprobar circunscripci�n, mesa electoral y, por ende, ser capaz de obtener los candidatos entre los que puede escoger.
				\end{itemize}
				
				Una vez verificado el votante y comprobados sus datos censales, se procede a construir la boleta con los candidatos que entre los que le corresponde elegir bas�ndose en su circunscripci�n electoral. El sistema ha de presentar la boleta al votante y permitir que �ste marque la o las opciones que permita el sistema electoral para constituir el voto a emitir.
				
				Una vez constituido el voto (papeleta), hay que proceder a la votaci�n digital. Para ello nos basamos en cifrado y firma ciega. As�, el primer paso es que la app utiliza la clave p�blica de la Entidad Electoral para cifrar el voto. Con el voto cifrado, el votante ha de firmarlo. La firma se realiza con el certificado de Firma que posee el DNIe. As�, el votante firma un conjunto de [voto cifrado + votante], que es el paquete que se pasar� al subsistema de gesti�n del voto.
				
				Una vez el votante ha emitido el voto, el sistema le devuelve un resguardo (c�digo QR como en Estonia, un c�digo alfanum�rico, no s� todav�a) con el cual puede verificar que el voto ha sido correctamente incluido en el sistema. Adem�s, podr� verificar que el voto ha sido correctamente incluido en el escrutinio. \notasCambio[inline]{(No s� si con este resguardo debe poder llegar a la opci�n de voto elegida, todo depende de c�mo tomemos el requisito de coerci�n y qu� es lo que menos le afecta)}
				
				El votante puede votar tantas veces como desee cambiar su voto \notasInfo{As� disminuimos el riesgo de coerci�n}. Para ello, hay un protocolo por el cual cuando un votante emite su voto, todos los anteriores son anulados. \todo[inline]{Hay que definir el protocolo para la anulaci�n de votos por 'revoto'}
				
				El sistema de gesti�n del voto es el encargado de los votos sufragados durante la jornada electoral. El sistema almacena los votos firmados (voto cifrado + votante) en una \textit{urna} digital durante el tiempo que dura la jornada electoral. En caso de recibir un voto de un votante que ya previamente hab�a emitido su voto, debe ser capaz de anular los votos anteriores que �ste hubiese sufragado\notasInfo{Como digo antes, hay que definir c�mo se hace esto de anular votos emitidos}.
				Una vez que el Administrador del Proceso Electoral da por terminada la Jornada Electoral, los Miembros de la Junta Electoral utilizan sus claves para formar la clave maestra que permite dar por terminada la fase de votaci�n y comenzar con el Escrutinio.
				La primera fase del Escrutinio es que los votos firmados deben ser \textit{anonimizados}. Esto lo vamos a realizar en dos pasos. Primero, comprobamos la validez de la firma del voto firmado. Si la firma se corresponde con u voto a descartar, se elimina. Si la firma es v�lida, se extrae (abrimos el sobre donde va la info del votante y el sobre con su voto secreto) del contenido del voto firmado tanto el voto cifrado como la informaci�n asociada del votante. Por un lado, la informaci�n del votante se almacena para sacar un listado de votantes (que podr� compararse con el resultado de votantes del censo). Por otra parte, los votos cifrados pasan a otro almac�n ya sin asociaci�n con su votante. Para terminar de separar los votos de sus votantes, pasamos por un proceso anonimizador que ... \todo[inline]{Aqu� entra en juego ElGamal y sus amigos o las mixnets}.
				Una vez tenemos los votos separados de sus votantes, procedemos a la siguiente fase del escrutinio, que es la de (abrir el sobre del voto secreto) descifrar el voto. El sistema necesita la clave privada de la Entidad Electoral para descifrar los votos que, recordemos, est�n cifrados con la clave p�blica. Con esta clave privada, extraemos el contenido del voto cifrado y obtenemos cada uno de los votos en plano de las urnas digitales.
				Una vez obtenidos el conjunto de los votos en plano de cada urna digital, podemos proceder a la consolidaci�n de los votos. Se realiza el conteo de cada urna y, con los resultados obtenidos, se puede realizar la totalizaci�n para llegar al resultado final de la Elecci�n.
				
				El �ltimo paso del sistema ser� el de la Difusi�n de los Resultados. El sistema de Escrutinio (o Totalizaci�n) informa de los resultados al m�dulo de Difusi�n, el cual les aplicar� el formato necesario para cumplir con las necesidades de publicaci�n de los mismos. En el caso del Proceso Electoral asociado a este proyecto, una web y diversos listados PDFs para poder ser cotejados.

				Ser�a muy interesante que, como en Estonia, los votantes tuvieran una herramienta para poder verificar que su voto ha sido correctamente incluido y escrutado en el Proceso.
				
				Paralelamente a todo el proceso, cada subsistema ha de generar una serie de registros, ficheros logs, que puedan ser visualizados por un conjunto de auditores, observadores u otro grupo de profesionales que tengan que dar cuenta del correcto funcionamiento del Proceso y de la transparencia del mismo, as� como del �xito t�cnico del Sistema.
\fi		
				
\par
En el dise�o de este proyecto, se han seguido diversas pr�cticas pr�pias de las metodolog�as �giles. Destaca la utilizaci�n del dise�o dirigido por pruebas (TDD). 
\par
Siguiendo esta pr�ctica, se ha procedido al dise�o de las pruebas simultaneamente o incluso con anterioridad al dise�o. Se han automatizado las pruebas, de modo que a cada peque�o incremento que se introduce en el sistema, se pueden repetir todas las pruebas, no dando como bueno cada cambio hasta que absolutamente todas las pruebas sean superadas. Las pruebas de caja blanca \footnote{Pruebas de caja blanca: metodolog�a de prueba que se basa en las estructuras de control del dise�o procedimental para generar los casos de prueba que, garanticen que se recorren por lo menos una vez todos los caminos independientes de cada m�dulo, que se ejecutan todas las decisiones l�gicas en su parte verdadera y en su parte falsa, se recorren todos los bucles.} carecen de sentido cuando se emplea la pr�ctica del dise�o dirigido por pruebas(las pruebas deben ser independientes y anteriores a la implementaci�n de las clases y a los cambios que en ellas se pueden hacer) por lo que no se ha realizado ning�n c�lculo de complejidad ciclom�tica \footnote{Complejidad ciclom�tica:  medida cuantitativa de la complejidad l�gica de un programa que define el n�mero de caminos independientes del conjunto b�sico de un m�dulo y por tanto nos proporciona una cota superior del n�mero m�ximo de pruebas que se han de realizar para garantizar que cada sentencia se ejecuta al menos una vez}. El TDD se entiende mejor con las pruebas de caja negra\footnote{Pruebas de caja negra: pruebas que se llevan a cabo sobre la interfaz del software, para probar su funcionalidad, donde los casos de prueba pretenden demostrar que las funciones del software se verifican, que la entrada se acepta de forma adecuada y que se produce una salida correcta, as� como que la integridad de la informaci�n externa se mantiene.}, pues las pruebas de caja negra examinan los aspectos del modelo fundamental del sistema sin tener mucho en cuenta la estructura l�gica interna del software, al dise�arse a priori, estos aspectos no se conocen a�n. 
\par
Esta pr�ctica no solamente garantiza resultados correctos, sino que, adem�s, permite realizar cualquier tipo de mejora con una cierta tranquilidad, al contar con estas pruebas a modo de 'red de seguridad'.
\par
Y no debemos olvidar en ning�n momento que, las ventajas que esta pr�ctica aportan no son solo para este proyecto, sino que podr�n ser reutilizables para futuras fases del proyecto global, y en futuras lineas de desarrollo.
\par
Para ampliar informaci�n sobre el dise�o dirigido por pruebas ver \cite{tdd}, y para ampliar sobre conceptos generales, o m�s propios de las metodolog�as cl�sicas ver \cite{pressman}.

\subsection{Dise�o de las pruebas}
\par
Las pruebas deber�n cumplir unos determinados criterios para su correcto dise�o. A partir de estos criterios se determinar�n los casos de prueba.
\par
\subsubsection{Criterios para el dise�o de las pruebas}
\par
Para el dise�o de las pruebas se tiene en cuenta el cumplimiento de los siguientes criterios a nivel funcional:
\par
\begin{itemize}
	\item Se debe poder crear usuarios.
	\item Se debe poder crear tickets (issues).
	\item Se debe poder leer tickets.
	\item Se debe poder asignar tickets.
	\item Se debe poder acceder a los adjuntos.
	\item Solo el administrador puede admitir usuarios.
	\item Solo los usuarios registrados pueden crear tickets. 
	\item Solo los usuarios registrados pueden asignar tickets. 
	\item Solo los usuarios registrados pueden acceder a los adjuntos. 
\end{itemize}
\par
Adem�s los casos de prueba deber�n comprobar otros factores como:
\begin{itemize}
	\item Se puede acceder al sistema.
	\item La interfaz es v�lida.
	\item El idioma es v�lido.
	\item Funciona en las plataformas descritas.
\end{itemize}
\par
\subsubsection{Casos de prueba}\label{casos}
\par
Para realizar las pruebas se levantar� el servicio y se lanzar�n contra su interfaz en tiempo de ejecuci�n (como se explicar� m�s adelante), de modo que cada  prueba funcional que hagamos, estar� probando a la vez los otros factores que describ�amos. Por tanto, y por los la propia metodolog�a de prueba usada, los casos de prueba se plantear�n desde un punto de vista funcional. 
\par
Dividiremos los casos de prueba en grupos:
\par
\begin{enumerate}
	\item \textbf{Pruebas de Acceso Administrador.} En primer lugar se determinar�n los casos de prueba que determinene como correctos los accesos en modo administrador. Su importancia hace que se separe de los casos de prueba de accesos para otros usuarios. Los casos de prueba contemplados ser�n los siguientes:
\begin{itemize}
	\item Un administrador puede acceder como administrador con contrase�a correcta
	\item No se puede acceder como administrador con una contrase�a incorrecta
	\item No se puede acceder como administrador con una contrase�a nula
	\item No se puede acceder sin login y password 
\end{itemize}
	\item \textbf{Pruebas de Administraci�n.} En este grupo se comprueba que el usuario administrador puede hacer las labores propias del administrador.
\begin{itemize}
	\item Un administrador puede listar clases
	\item Un administrador puede listar usuarios
	\item Un administrador puede crear un usuario
	\item Se puede crear m�s de un usuario
	\item No se puede crear un usuario ya existente.
	\item Un administrador puede editar un usuario
\end{itemize}
	\item \textbf{Pruebas de Usuario. }Se deber� comprobar que un usuario puede acceder y realizar las acciones propias de un usuario pero no de administrador.
\begin{itemize}
	\item Un usuario puede acceder, y accede como usuario no admiinistrador (no se contempla el acceso con contrase�a incorrecta o sin contrase�a, pues eso queda ya comprobado en el caso del administrador)
	\item Un usuario no puede listar clases
	\item Un usuario s� puede listar usuarios
	\item Un usuario no puede crear usuarios
\end{itemize}
	\item \textbf{Pruebas de creaci�n de Issues}
\begin{itemize}
	\item Un usuario puede crear un issue.
	\item Un usuario puede crear otro issue.
	\item Otro usuario puede ver los issues creados.
\end{itemize}
	\item \textbf{Pruebas de manipulaci�n de Issues}
\begin{itemize}
	\item Un usuario puede reasignar un issue
	\item Un usuario puede cambiar la prioridad de un issue
	\item Un usuario puede cambiar el estado de un issue
	\item Un usuario puede cambiar el contenido de un issue
	\item Un usuario puede examinar todas las acciones realizadas con un issue
\end{itemize}
\end{enumerate}
En base a esos casos, las pruebas se han automatizado en base a lo que se indica en el siguiente punto.
\subsection{Automatizaci�n de las pruebas}
Se ha optado por el empleo de SELENIUM IDE\footnote{SELENIUM IDE: herramienta para automatizar pruebas para aplicaciones web, que funciona directamente sobre el internet browser (navegador de internet). Para m�s informaci�n, ver \cite{selenium}} para a realizaci�n de las pruebas. Esta herramienta comprueba si ante determinadas entradas que hagamos sobre la interfaz web, la salida obtenida es la esperada o no. En los siguientes puntos se detalla c�mo se han dise�ado y programado las correspondientes suites.
\par
\subsection{Implementaci�n y realizaci�n de las pruebas}
\par
\subsubsection{Implementaci�n de las pruebas}
\par
La implementaci�n de las pruebas se ha hecho creando suites de pruebas para SELENIUM IDE. Aunque al hacerse al estilo de pruebas de caja negra se han dise�ado independientes al c�digo, s� que habr� que tener en cuenta que peque�os incrementos pueden requerir de un cambio en la programaci�n de las pruebas. El ejemplo m�s claro qeu podemos encontrarnos aqu�, es que si en un primer paso queremos acceder en local, la suite de pruebas buscar� la direcci�n \verb|http://localhost:8080| mientras que en un paso a remoto, la suite de pruebas deber�a buscar en una ubicaci�n concreta que ya no ser�a esa, evidentemente.
\par
La versi�n de este sistema no ha sido aceptada como v�lida hasta que no ha pasado correctamente todas las pruebas que han sido dise�adas y programadas.
\par
En fases futuras de desarrollo, no se deber�n aceptar como v�lidas nuevas versiones si no se pasan todas las pruebas aqu� descritas, as� como las nuevas pruebas que se determinen como necesarias en ese momento. En las futuras fases, en el sistema de las fichas m�dicas, en la integraci�n con el mismo, ser� probable que sea preciso a�adir nuevas pruebas o modificar estas, pero siempre se deber�n tomar estas como referencia.
\par
\subsubsection{Realizaci�n de las pruebas}
\par
Las pruebas se realizar�n a cada cambio realizado en la aplicaci�n, reverti�ndose dicho cambio a una versi�n anterior si no se pasan las pruebas y no se consigue corregir la versi�n para que las pruebas sean superadas.
\par
El procedimiento para realizar la ejecuci�n de las pruebas es el siguiente:
\begin{enumerate}
	\item Se realiza desde cero la inicializaci�n del tracker (ver apartado \ref{inicializacion})
	\item Se levanta el servicio (ver apartado \ref{levantar})
	\item Se abre el navegador FIREFOX en la m�quina donde reside el tracker. Se realizan las pruebas mediante SELENIUM IDE
	\item Se abre el navegador FIREFOX en la m�quina remota, y se realizan las pruebas mediante SELENIUM IDE
	\item Se valoran los resultados
\end{enumerate}
\par
\subsection{Construcci�n de las pruebas}
\par
Sobre los casos de prueba descritos se han construido las siguientes pruebas. Aqu� �nicamente se describen, el c�digo correspondiente a estas pruebas se adjunta a este trabajo.
\par
A continuaci�n se muestran las pruebas que se har�n para resolver cada caso, uno por uno:
\par
\begin{enumerate}
	\item \textbf{Pruebas de Acceso Administrador.}
\begin{itemize}
	\item Un administrador puede acceder como administrador con contrase�a correcta: se accede con usuario admin, y contrase�a correcta; se comprueba que la pantalla muestre el texto \textit{Administraci�n}.
	\item No se puede acceder como administrador con una contrase�a incorrecta: se introduce el usuario admin y contrase�a incorrecta; se comprueba que la pantalla muestre el texto \textit{nombre de usuario o contrase�a inv�lidos}.
	\item No se puede acceder como administrador con una contrase�a nula: se introduce el usuario admin sin contrase�a; se comprueba que la pantalla muestre el texto \textit{nombre de usuario o contrase�a inv�lidos}.
	\item No se puede acceder sin login y password: no se introduce ningun usuario ni contrase�a; se comprueba que la pantalla muestre el texto \textit{nombre de usuario o contrase�a inv�lidos}.
\end{itemize}
	\item \textbf{Pruebas de Administraci�n.} 
\begin{itemize}
	\item Un administrador puede listar clases: un usuario accede como usuario admin con contrase�a correcta; a continuaci�n sigue el link \textit{lista de clases}; se comprueba que aparece el texto \textit{lista de clases}.
	\item Un administrador puede listar usuarios: un usuario accede como usuario admin con contrase�a correcta; a continuaci�n sigue el link \textit{lista de usuarios}; se comprueba que aparece el texto \textit{lista de usuarios}.
	\item Un administrador puede crear un usuario: un usuario accede como usuario admin con contrase�a correcta; a continuaci�n sigue el link \textit{agregar usuario}; introduce los datos m�nimos para crear un usuario; comprueba que aparece el texto \textit{creado}; sigue el  link \textit{lista de usuarios};comprueba que el usuario existe.
	\item Se puede crear m�s de un usuario: un usuario accede como usuario admin con contrase�a correcta; a continuaci�n sigue el link \textit{agregar usuario}; introduce los datos m�nimos para crear un usuario (distintos al anterior); comprueba que aparece el texto \textit{creado}; sigue el  link \textit{lista de usuarios};comprueba que el usuario existe.
	\item No se puede crear un usuario ya existente: un usuario accede como usuario admin con contrase�a correcta; a continuaci�n sigue el link \textit{agregar usuario}; introduce los datos m�nimos para crear un usuario (igual al anterior); comprueba que aparece el texto \textit{Error: node with key}.
	\item Un administrador puede editar un usuario: un usuario accede como usuario admin con contrase�a correcta; a continuaci�n sigue el link \textit{agregar usuario};introduce los datos m�nimos para crear un usuario (distinto a los anteriores); sale del sistema y vuelve a acceder; lista los usuarios; elige el usuario recien creado; cambia el nombre; se comprueba que aparece el texto \textit{Edici�n exitosa}.
\end{itemize}
	\item \textbf{Pruebas de Usuario. }Previamente a la realizaci�n de estas pruebas, se crean tres usuarios solo para las pruebas: tom, dick y harry.
\begin{itemize}
	\item Un usuario puede acceder, y accede como usuario no administrador: el usuario tom accede con el login y el password correcto; se comprueba que aparece el texto \textit{Hola, tom}; se repite el mismo procedimiento con los tres usuarios de prueba.
	\item Un usuario no puede listar clases:un usuario accede como usuario normal con contrase�a correcta; a continuaci�n sigue el link \textit{lista de usuarios}; se comprueba que no aparece el texto \textit{lista de clases}.
	\item Un usuario s� puede listar usuarios:un usuario accede como usuario normal con contrase�a correcta; se comprueba que no aparece el texto \textit{lista de usuarios}.
	\item Un usuario no puede crear usuarios: un usuario accede como usuario normal con contrase�a correcta; a continuaci�n sigue el link \textit{lista de usuarios}; se comprueba que no aparece el texto \textit{agregar usuario}.
\end{itemize}
	\item \textbf{Pruebas de creaci�n de Issues.}
\begin{itemize}
	\item Un usuario puede crear un issue: el usuario tom accede con password correcto; crea un issue sin asignar.
	\item Un usuario puede crear otro issue: el usuario tom accede con password correcto; crea un issue asignado a dick.
	\item Otro usuario puede ver los issues creados.el usuario dick accede con password correcto; sigue el vinculo \textit{mostrar todos};se comprueba que est�n presentes los textos correspondientes a los nombres de los issues antes creados; sigue el v�nculo \textit{sus issues}; se comprueba que est� presente el issue que le fue asignado.
\end{itemize}
	\item \textbf{Pruebas de manipulaci�n de Issues.}  
\begin{itemize}
	\item Un usuario puede reasignar un issue: el usuario dick accede con password correcto; sigue el v�nculo \textit{sus issues}; escoge el issue que tiene asignado; cambia la asignaci�n al usuario harry; se comprueba que aparezca el texto \textit{Edici�n existosa}
	\item Un usuario puede cambiar la prioridad de un issue: el usuario harry accede con password correcto; sigue el v�nculo \textit{sus issues}; escoge el issue que tiene asignado; cambia la prioridad; se comprueba que aparezca el texto \textit{Edici�n existosa}
	\item Un usuario puede cambiar el estado de un issue: el usuario harry accede con password correcto; sigue el v�nculo \textit{sus issues}; escoge el issue que tiene asignado; cambia el estado a \textit{resuelto}; se comprueba que aparezca el texto \textit{Edici�n existosa}; se sigue el vinculo \textit{mostrar todos}; se comprueba que el issue ya no est�; se sigue el vinculo buscar; se elige estado \textit{resuelto}; se comprueba que ah� s� aparece el issue; se accede al issue; se cambia el estado a \textit{in-progress}.
	\item Un usuario puede cambiar el contenido de un issue: el usuario harry accede con password correcto; sigue el vinculo \textit{sus issues}; escoge el issue que tiene asignado; a�ade un mensaje; se comprueba que aparezca el texto \textit{Edici�n existosa}; se comprueba que est� presente el texto del mensaje.
	\item Un usuario puede examinar todas las acciones realizadas con un issue: el usuario harry accede con password correcto; sigue el vinculo \textit{sus issues}; escoge el issue que tiene asignado; se comprueba que aparezca el texto \textit{crea}; se comprueba que aparezca el texto \textit{asignadoa: dick -> harry}; se comprueba que aparezca el texto \textit{prioridad: urgent -> wish}; se comprueba que aparezca el texto \textit{estado: unread -> resolved}; se comprueba que aparezca el texto \textit{estado: resolved -> in-progress}; se comprueba que aparezca el texto \textit{mensajes: +msg}.
\end{itemize}
\end{enumerate}











\par
\subsubsection{Aclaraci�n sobre la realizaci�n de prueba final}
\par
Para la correcta verificaci�n tanto de la aplicaci�n como de manuales, instalables, etc, cuando se determina por concluido el proyecto, se procede a una desinstalaci�n desde cero, y reinstalaci�n siguiendo paso por paso las instrucciones de los manuales, etc.
\par
Una vez completada esta reinstalaci�n, se repiten los pasos indicados en el apartado anterior, d�ndose por buena y definitiva la versi�n al pasarse las pruebas bajo estas condiciones.

\subsection{Observaci�n sobre las pruebas}
\par
Hay que hacer una peque�a observaci�n sobre las pruebas. �stas han sido realizadas con intenci�n de comprobar si se satisfacen los requisitos y alcance descritos en el apartado \ref{alcance} y el apartado \ref{requisitos}. 
\par
No obstante, las pruebas se han realizado en un entorno hardware concreto (Ordenadores personales) y desde un navegador concreto (FIREFOX). En futuras lineas de desarrollo, se deber� tener en cuenta que las pruebas deber�n adaptarse a los entornos en que se est� trabajando. Por ejemplo, cuando si se dise�a una interfaz concreta para las MDA, deber�n hacerse pruebas en las MDA, etc.

%%\chapter{Manuales}\label{manuales}
\lhead{Cap�tulo \ref{manuales}}
\rhead{Manuales}
%*******************************************************************************
\section{Notas sobre los manuales}
\par
El objetivo de este cap�tulo es facilitar unos manuales m�nimos pero suficiente para poder instalar, configurar, utilizar y administrar el sistema, de una forma correcta pero sencilla. 
\par
Aunque podr�amos entender que los contenidos de este cap�tulo podr�an estar dentro de los apartados relativos a la construcci�n del sistema y a la implantaci�n y despliegue, se ha optado por presentarlo a parte, debido a la importancia que tiene dentro de los requisitos descritos. Dicha importancia viene descrita en el apartado \ref{alcance_final}, sobre el alcance final del proyecto, donde se indicaba la creaci�n de los manuales como el tercero de los cuatro objetivos a resolver (los dos primeros quedaron resueltos en el cap�tulo \ref{solucion}, y el cuarto se resolver� en el cap�tulo \ref{lineas_futuras}).
\par
Si se precisa ayuda, o ampliar informaci�n sobre el sistem Linux, en quien reside la aplicaci�n, ver \cite{autounix}. Puede ser muy util tanto a la hora de la instalaci�n, configuraci�n, uso, administraci�n y mantenimiento del sistema.
\section{Manual de Instalaci�n}
\subsection{Requisitos para la instalacion}\label{requisitosinst}
\subsection{Pasos previos a la instalaci�n}
\par
Para poder realizar la instalaci�n de los programas necesitaremos previamente instalar la distribuci�n correspondiente de Linux, en este caso la distribuci�n SUSE 9.3, incluyendo los siguientes paquetes:
\begin{itemize}
\item
	python: int�rprete de Python (versi�n superior a TODO:qu� versi�n?)
\item
 	python-devel: paquete necesario, pues roundup necesita "distutils". Este paquete requiere de la instalaci�n de python-tk y blt.
\end{itemize}

\subsection{Instalaci�n de Roundup Issue Tracker}
\par
A continuaci�n se muestran las distintas acciones a seguir para una correcta instalaci�n y configuraci�n del programa.
	{Operaciones a realizar en modo superusuario}
\par
A continuaci�n se muestran los pasos a seguir, que deber�n ser realizados desde la consola, habi�ndose identificado como "root":

\begin{itemize}
\item
	Elecci�n de la direcci�n para la instalaci�n (en este caso, elegiremos: 
\begin{center}
\verb|/opt/roundup/bin| 
\end{center}
\item
	Ejecutar, situ�ndonos en el directorio donde hayamos descomprimido el programa:
\begin{center}
\verb|python setup.py install --install-scripts=/opt/roundup/bin| .
\end{center}
\end{itemize}
\par
Para cualquier duda que pueda surgir sobre la instalaci�n de roundup, se recomienda encarecidamente ver \cite{roundupweb}.
\par

\subsection{Instalaci�n PostgreSQL}\label{instpost}
\par
\textbf{DECIR LAS INDICACIONES PROPIAS QUE REQUIERE POSTGRES}
\par
Para cualquier duda que pueda surgir sobre la instalaci�n de roundup, se recomienda encarecidamente ver \cite{postweb} y \cite{postgres}.
\par
\subsection{Instalaci�n Xapian}\label{xapinst}
\textbf{DECIR LAS INDICACIONES PROPIAS QUE REQUIERE ROUNDUP PARA ESTO}
\par
Cualquier aclaraci�n adicional que se necesite sobre c�mo debe hacerse la instalaci�n de Xapian, se puede obtener en \cite{xapianweb}.
\par


\section{Manual de Configuraci�n}

\par
En este apartado trataremos sobre c�mo configurar e inicializar el tracker, as� como los complementos necesarios para el funcionamiento del sistema\footnote{Tenemos que tener en cuenta que, como ya hemos comentado en el apartado \ref{decisiones}, utilizaremos el usuario ''\textbf{pfc}''.}.

\subsection{Configuraci�n de los elementos complementarios}
\subsubsection{Configuraci�n de la base de datos}

\subsection{Configuraci�n del tracker}
Para configurar el tracker, se deber�n seguir los siguientes pasos (todos ellos en modo superusuario):
\begin{enumerate}
	\item Crear directorio para incluir el tracker \footnote{NOTA: en el caso hipot�tico de que futuras verisiones consideraran que deber� configurarse m�s de un tracker, se utilizar� un �nico directorio para todos.}. 
\begin{center}
\verb|mkdir /opt/roundup/trackers|
\end{center}
	\item A�adir al path la direcci�n donde se encuentran los scripts (en nuestro caso en /opt/roundup/bin
	\item Ejecutar el siguiente comando: \verb|roundup-admin install|. Al ejecutarse esta opci�n, la aplicaci�n nos pedir� que seleccionemos la plantilla (template) y el sistema gestor de bases de datos subyacente (backend) a utilizar:
\begin {itemize}
	\item Inserte directorio base (Enter tracker home): en nuestro caso introduciremos:
\begin{center}
\verb|/opt/roundup/trackers/pfc/|.
\end{center}
	\item Seleccione plantilla (Select template):por defecto introduciremos 'classic'.
	\item Seleccione base de datos (Select backend): por defecto introduciremos 'anydbm'.
\end {itemize}
	\item Realizaremos las siguientes modificaciones en el archivo 'config.ini'\footnote{NOTA: para facilitar la configuraci�n, se adjunta en el cd de instalaci�n una versi�n del fichero config.ini que podr� tomarse tal cual para sustituir la que genera por defecto.}, cuyo sentido se explica en el apartado \ref{config}, sobre el dise�o de la aplicaci�n. (para lo cual, comprobaremos los correspondientes permisos y ajustaremos seg�n sea preciso):
\begin {itemize}
	\item \verb|[main]|
\begin {itemize}
	\item \verb|admin_email = pfc|
	\item \verb|dispatcher_email = pfc|
\end {itemize}
	\item \verb|[tracker]|
\begin {itemize}
	\item \verb|web = http://localhost:8080/pfc|\footnote{NOTA: Esta direcci�n deber� ser modificada de acuerdo a la ubicaci�n que tenga el equipo que sirva realmente la aplicaci�n. Indicar Localhost �nicamente valdr� para un escenario de pruebas en que cliente y servidor est�n en el mismo equipo}
	\item \verb|email = pfc|
\end {itemize}
	\item \verb|[mail]|
\begin {itemize}
	\item \verb|domain = localhost|
	\item \verb|host = localhost|
\end {itemize}
	
\end {itemize}
\end {enumerate}
\par
Para cualquier duda que pueda surgir sobre sobre la configuraci�n de roundup, se recomienda encarecidamente ver \cite{roundupweb}.
\par
\subsection{Inicializaci�n del tracker}\label{inicializacion}
\par
Antes que nada, hay que tener en cuenta que se tienen que verificar los permisos antes y despu�s de realizar este paso, pues al crearse nuevas carpetas puede ser que los permisos de �stas nos impidan realizar alg�n paso.
\par
La inicializaci�n del tracker debe hacerse tambi�n como superusuario. Los pasos a seguir ser�n los siguientes
\begin{enumerate}
	\item Se ejecutar� el comando:
\begin{center}
\verb|roundup-admin initialize|
\end{center}
	\item Se nos pedir� introducir el directorio base del tracker:
\begin{center}
\verb|/opt/roundup/trackers/pfc/|
\end{center}
	\item Se nos pide introducir una contrase�a de administraci�n (en este caso hemos utilizado 'pfc')
\par
\end{enumerate}


\section{Manual de Uso}
\par
Una vez configurado el programa, e inicializado el tracker, en este cap�tulo se indicar� c�mo se debe proceder a levantar el servicio. Tambi�n se realiza alguna peque�a aclaraci�n sobre los distintos manuales que se deben utilizar.
\subsection{Levantar el servicio}\label{levantar}
\par
Para arrancar el tracker y que el sistema entre en funcionamiento, se debe levantar el servicio. 
\par
Esta operaci�n se deber� realizar tambi�n en caso de que la m�quina caiga, o tras cualquier operaci�n de mantenimiento del sistema. Es decir, cada vez que ocurra un evento por el cual debamos volver a poner en funcionamiento el sistema.
\par
Para levantar el servicio, deberemos trabajar como usuario normal, y no como superusuario (como hemos hecho en la instalaci�n y en la configuraci�n). Para ello, se deben revisar los permisos en las carpetas involucradas, y comprobar que sean los pertinentes.
\par
Se ejecutar�n en la consola las siguientes instrucciones, de acuerdo a los casos expuestos en la instalaci�n y la configuraci�n:
\begin{center}
\verb|export PATH=$PATH:/opt/roundup/bin|
\end{center}
\begin{center}
\verb|roundup-server pfc=/opt/roundup/trackers/pfc|
\end{center}
\par
Por comodidad en el uso, se recomienda encarecidamente el uso del script ''levantar.sh'' que se adjunta en el CD, y contiene s�mplemente esas dos instrucciones. Para utilizar este script, habr�a que alojarlo dentro del directorio ''home/pfc''. Para levantar la aplicaci�n bastar� unicamente con situarnos en ese directorio desde la consola y ejecutar:
\begin{center}
\verb|./levantar.sh|
\end{center}
\par
\subsection{Manual de usuario}
\par
Para un correcto uso del sistema, se ven involucrados los siguientes documentos:
\begin{itemize}
	\item Documentaci�n de uso de Round-up Issue Tracking (incluida en el CD adjunto, y tambi�n disponible y actualizada en \cite{roundupweb}).
	\item Cualquier manual, normativa o documentaci�n propuesta para la utilizaci�n de este sistema, comunicaci�n en los roles m�s bajos, etc, as� como cualquier otro documento que sea considerado pertinente por las Fuerzas Armadas.
\end{itemize}
\par
Aunque el funcionamiento del sistema es muy intuitivo y sencillo, cualquier duda podr� ser aclarada en la documentaci�n del programa, por lo que esa documentaci�n debe estar disponible. \par
Igualmente, el programa no deber� entrar en producci�n hasta que los usuarios finales tengan pleno conocimiento de las normativas de uso impuestas por las Fuerzas Armadas. 
 \label{uso}
\section{Notas sobre la administraci�n y el mantenimiento}
Hablar de que existe m�s documentaci�n para los backend,selenium etc.

Hacer una aclaraci�n sobre las cosas que debe hacer el administrador. (crear usuarios... ver guia de administraci�n de roundup
 \label{notasadmin}

\chapter{L�neas futuras}\label{lineasFuturas}
\lhead{Cap�tulo \ref{lineasFuturas}}
\rhead{L�neas futuras}

\iffalse
\todo[inline]{Uso de Enigma \url{http://enigma.media.mit.edu/enigma_full.pdf}(8.9, p�gina 13) para el voto por Internet.}
\todo[inline]{{tro: \url{http://www.pabloyglesias.com/cifrado-homomorfico-blockchain/}}}

\todo[inline]{Realizar auditor�a de la criptograf�a y seguridad de los procesos}
\todo[inline]{Mejorar el protocolo oAuth y montar un sistema REST para poder usado como servicio web y llamado desde diferentes plataformas: web estilo Angular, app m�vil con llamadas Http directas, app de escritorio...}

\todo[inline]{Investigar m�todos de identificaci�n - biometr�a, SIM con IDs ... }

\todo[inline]{Carlos, aseg�rate de que comentamos en alg�n sitio que es el primer sistema con el que se puede votar con el nuevo \gls{DNIe} 3.0. Importante por la novedad.}
\fi
	Este proyecto representa una prueba de concepto de c�mo se puede implementar un sistema de voto por Internet que utiliza el \gls{DNIe} 3.0 como herramienta para la identificaci�n digital del votante y que busca facilitar que el acceso seguro al sistema de voto se pueda realizar desde dispositivos m�viles tales como un smartphone o una tablet Android.
	
	No obstante, como cualquier prueba de concepto, tras la elaboraci�n de la memoria del proyecto y la implementaci�n del mismo, se advierten varios �mbitos donde hay carencias tecnol�gicas o se observan oportunidades de mejora en el desarrollo.
	
	\section{Auditor�a de seguridad / criptograf�a}
		El principal punto en el que se deber�a intervenir es en la seguridad criptogr�fica del proyecto. Cierto es que el desarrollo se basa en un sistema como Helios Voting, que est� avalado tecnol�gicamente por instituciones como el \gls{MIT} y por grandes expertos internacionales en criptograf�a (avanzados en \ref{}). Pero en el momento en el que se ha debido desarrollar una soluci�n para admitir el \gls{DNIe} como herramienta de identificaci�n digital, ya se ha impactado en la seguridad del sistema. En caso de querer llevar esta soluci�n a la pr�ctica para un proceso electoral real, es necesario que se realice un estudio de los elementos criptogr�ficos utilizados en el proyecto, as� como de la seguridad de cada uno de los elementos implementados. Especialmente cr�tico es esto teniendo en cuenta que se trata de un sistema accesible por Internet, cuyo core est� escrito hace algunos a�os.
	
		Por ello, en caso de querer avanzar en este sentido, recomiendo desarrollar estas cuestiones incluyendo una auditor�a de seguridad rigurosa del sistema. Por un lado permitir�a incrementar la seguridad y fiabilidad del mismo y, por otro, dota al sistema de capacidades legales que puedan cumplir criterios de los estamentos u organizaciones que puedan requerir de este servicio electoral.
	
	\section{Procurar escalabilidad del sistema}
		El c�digo del core de Helios Voting est� escrito en Python sobre un framework Django.
		
		Una posible l�nea de desarrollo futuro para este proyecto podr�a ser la migraci�n del mismo hacia sistemas modernos escalables. En estos momentos, la posibilidad de escalar el sistema implementado consiste en migrarlo a m�quina con mayor potencia y recursos. Est� muy limitado en caso de aumentar el n�mero de votantes.
		
		Pienso que ser�a una muy buena l�nea de desarrollo el migrar el sistema a una tecnolog�a que permita escalabilidad de forma simple. Para ellos, pienso en tecnolog�as como Nodejs y microservicios. Nodejs permite la implementaci�n de sistemas estables, con una comunidad inmensa sobre la que apoyarse, con una ingente cantidad de m�dulos que reutlizar. Esto proporciona una herramienta importante para favorecer el desarrollo �gil. Junto a esta tecnolog�a, se puede introducir el uso de microservicios. Migrando el sistema al uso de microservicios podemos mejorar el rendimiento del sistema. Por un lado es m�s f�cil probar la funcionalidad del sistema, ya que con pruebas unitarias de cada microservicio, aseguras la funcionalidad de cada uno. Adem�s el uso de microservicios facilita la f�cil escalabilidad del sistema, pues si la carga aumenta, en lugar de una migraci�n a una m�quina m�s potente, se pueden lanzar nuevas instancias de cada uno de los microservicios m�s impactados para que estos trabajen en paralelo.
		
		Considero que este desarrollo es bastante interesante porque permite el crecimiento de las capacidades del proyecto con una inversi�n muy baja.
		
	\section{Pruebas}
		Un sistema de voto, y m�s si es accesible por Internet, necesita un sistema de tests que sea robusto, estable y minucioso. Hay que asegurar en la medida de lo posible que el sistema no falla.
		
		Este proyecto carece de un sistema fiable de test, lo que identifica este punto como la mayor vulnerabilidad del sistema.
		
		Es fundamental que se desarrolle una bater�a de pruebas para el sistema que permita asegurar su fiabilidad.
		
		Para ello, una posibilidad que recomiendo, es la reescritura del sistema siguiendo una metodolog�a de desarrollo basada en \gls{TDD}. 
		
		As�, se desarrollar�a el sistema obligando al mismo a pasar los tests pensados para cada m�dulo y para el sistema de forma global.
		
		Esta metodolog�a permite desarrollar de forma segura y fiable, permitiendo, igualmente, refactorizar c�digo con la misma seguridad, con lo que al final el la velocidad de desarrollo se acelera bastante.
		
		Un sistema tan importante como una votaci�n, que debe minimizar los errores al m�ximo por la criticidad de los mismos, deber�a apoyarse en un sistema automatizado de tests. Deber�an implementarse tests unitarios y punto a punto para todos los subsistemas y funcionalidades del sistema. Con esto, se puede asegurar la integridad del mismo y facilitar los cambios en el software, provocando un software m�s estable.
	
	 
	 
	 Combinando estas l�neas de desarrollo futuro pienso en una metodolog�a de desarrollo TDD, con herramientas de Integraci�n Continua, lenguages como Python o Nodejs y el uso de microservicios. Creo que se puede implementar un sistema estable y con un rendimiento muy elevado, que facilite el desarrollo de funcionalidades de forma segura. Este sistema se basar�a en pruebas automatizadas lo que, adem�s de asegurar la estabilidad del sistema frente a cambios, permitir�a la refactorizaci�n del c�digo de forma segura, incrementando la adaptabilidad del sistema y su rendimiento.
	 
	 
	 \section{Sistema de identificaci�n del votante}
	 	Por limitaciones tecnol�gicas del componente WebView de Android se debi� desarrollar una soluci�n alternativa de identificaci�n del votante contra el sistema basada en oAuth. Este desarrollo entiendo que es el punto m�s vulnerable del sistema, por lo que considero que es una buena l�nea de desarrollo para el futuro.
	 	
	 	Ser�a interesante sustituir este mecanismo de identificaci�n por uno realmente seguro.
	 	
	 	Por un lado, se podr�a estudiar el componente WebView de Android para observar c�mo es su implementaci�n. Se puede comentar con los desarrolladores de Google el problema que tiene en cuanto al uso de certificados de cliente y navegaci�n segura por una web (en contra de un sistema \gls{REST}) para ver si hay opciones de implementaci�n de un nuevo m�dulo que cumpla con las necesidades. Tambi�n cabe la posibilidad de implementar un componente WebView modificado para cumplir con los requisitos de este sistema. 
	 	
	 	Otra opci�n, no obstante es cambiar el sistema para que pueda funcionar con el componente WebView de Android. Para ello ser�a necesario convertirlo en un sistema completamente \gls{REST}. Con ello, se puede pensar en la implementaci�n de una app Android que no s�lo sirva para la autenticaci�n, sino para la votaci�n completa. esto implica reescribir el sistema pasando de web a Android. El backend del sistema ser�a el mismo, s�lo que servir�a HTML a las conexiones realizadas con navegadores clientes de escritorio, pero desde apps Android se basar�a en peticiones HTTPS directas contra servicios \gls{REST} incluyendo el certificado en cada una de ellas.
	 	
		No obstante, la limitaci�n del sistema ha sido provocada por los requisitos del mismo. En concreto, el uso del \gls{DNIe} 3.0 con su chip \gls{NFC}. Otra l�nea de desarrollo puede ser la b�squeda de nuevos servicios de identificaci�n digitales y la implementaci�n para este sistema. Para ello pienso que es interesante la investigaci�n en identificaci�n digital basada en biometr�a (huellas digitales, lectura ocular, facial, etc.), pasaporte digital, mobId (usado en Estonia\ref{ivotingEstonia}, \cite{Morshed2010}), bitcoins, multifactor, etc. Considero que es un mundo con unas posibilidades enormes que merece la pena ser investigado para encontrar soluciones que adaptar e incorporar al sistema desarrollado en este proyecto.

\chapter{Conclusiones}\label{conclusiones}
\lhead{Cap�tulo \ref{conclusiones}}
\rhead{Conclusiones}
%*******************************************************************************
\par
En este trabajo nos hemos centrado en encontrar una soluci�n para los roles m�s bajos de la jerarqu�a empleada en el sistema de telemedicina de las Fuerzas Armadas Espa�olas. Las soluciones aqu� propuestas se han centrado en solucionar toda la problem�tica existente en estos niveles de la jerarqu�a, y en crear un sistema que satisfaga todas las necesidades requeridas.
\par
Dentro de estos niveles se han descrito dos tipos de abstracci�n, representados por dos sistemas distintos: un consistir�a en un sistema que contenga la informaci�n de las fichas m�dicas que prev�n los est�ndares OTAN, y otro, que es el que en este proyecto estamos estudiando, que gestionar�a los eventos sucedidos en el tr�mite de las urgencias m�dicas.
\par
Puesto que los temas derivados de niveles superiores de la jerarqu�a ya se encuentran resueltos, este proyecto no ha tocado en ning�n momento ning�n aspecto relativo a los sistemas que resuelven esos problemas.
\par
Para la creaci�n del sistema planteado en este proyecto, y el buen cumplimiento de los requisitos planteados se han cumplido los siguientes objetivos: se ha dise�ado un modelo que representaba de un modo apropiado la situaci�n real que se quer�a resolver; se ha encontrado una soluci�n software capaz de resolver el problema; se han creado los manuales precisos para su correcta instalaci�, configuraci�n y uso; se ha facilitado el desarrollo para futuras lineas de desarrollo.
\par
Para obtener esos resultados no solo se han elegido muy cuidadosamente todas las opciones posibles (desde la aplicaci�n a utilizar, el sistema operativo o la base de datos, a los lenguajes de programaci�n, los modelos para el despliegue, etc), sino que adem�s se ha puesto una importancia muy especial en las pruebas, que han sido automatizadas, y se han utilizado algunas pr�cticas de metodolog�as �giles que han enriquecido mucho este trabajo. Todos estos factores han hecho que esos resultados sean muy satisfactorios.
\par
Los unicos inconvenientes que se han encontrado son los derivados a la insuficiencia de este sistema por si solo (en este caso carece de sentido, si no es correctamente integrado en las otras fases de desarrollo descritas), y a la posible dificultad que podr�a tener este sistema llegado el momento de ingresarse con los sistemas ya existentes, en futuras fases de desarrollo.
\par
Pero independientemente de estos inconvenientes, se ha encontrado una soluci�n que satisface todos los requisitos y objetivos propuestos. Adem�s, en todo momento se ha facilitado el trabajo para las futuras fases de desarrollo, como se puede ver a lo largo del documento en todas las decisiones tomadas, y de forma m�s concreta en el cap�tulo dedicado a las futuras lineas de desarrollo.
%%\chapter{Glosario}\label{glosario}
%\lhead{Cap�tulo \ref{glosario}}
%\rhead{Glosario}

%from documentation
%\newacronym[?key-val list?]{?label ?}{?abbrv ?}{?long?}
%above is short version of this
% \newglossaryentry{?label ?}{type=\acronymtype,
% name={?abbrv ?},
% description={?long?},
% text={?abbrv ?},
% first={?long? (?abbrv ?)},
% plural={?abbrv ?\glspluralsuffix},
% firstplural={?long?\glspluralsuffix\space (?abbrv ?\glspluralsuffix)},
% ?key-val list?}



%\newglossaryentry{DNIe}
%{name={DNIe}, 
%description={Documento Nacional de Indentidad Electr�nico}
%}

\newacronym{DNIe}{DNIe}{Documento Nacional de Identidad Electr�nico}
\newacronym{CAN}{CAN}{Card Access Number}
\newacronym{PACE}{PACE}{Password Authentication Connection Establishment}
\newacronym{PIN}{PIN}{Personal Identification Number}
\newacronym{NFC}{NFC}{Near Field Communication}
\newacronym{CRL}{CRL}{Certificate Revocation List (Lista de Revocaci�n de Certificados)}
\newacronym{OCSP}{OCSP}{Online Certificate Status Protocol}
\newacronym{E2E}{E2E}{End-to-End (Punto a punto)}
\newacronym{UNED}{UNED}{Universidad Nacional de Educaci�n a Distancia}
\newacronym{UPV/EHU}{UPV/EHU}{Universidad del Pa�s Vasco / Euskal Herriko Unibertsitatea}
\newacronym{CERES-FNMT}{CERES-FNMT}{CERtificaci�n ESpa�ola - F�brica Nacional de Moneda y Timbre}
\newacronym{PFC}{PFC}{Proyecto Final de Carrera}
\newacronym{W3C}{W3C}{World Wide Web Consortium}
\newacronym{PC}{PC}{Personal Computer}





%\makeglossaries
%%\chapter{Temp}\label{temp}
\lhead{Cap�tulo \ref{temp}}
\rhead{Temp}
%*******************************************************************************
\par
Seg�n el documento \textbf{NORMAS DE ORGANIZACI�N Y FUNCIONAMIENTO DE LA UNIVERSIDAD SAN PABLO-CEU}, en su Art�culo 9, ''Las Facultades, Escuelas y Centros integrados o adscritos son las instancias responsables de la organizaci�n de la ense�anza e investigaci�n, de acuerdo con las directrices emanadas de los �rganos superiores de la Universidad, y de los procesos acad�micos, administrativos y de gesti�n conducentes a la obtenci�n de t�tulos de car�cter oficial y validez en todo el territorio nacional, as� como de aquellas otras funciones que determinen las presentes Normas de Organizaci�n y Funcionamiento y los restantes reglamentos universitarios.''
\par
A partir de esta definici�n, en el Cap�tulo II De los �rganos acad�micos, encontramos el Art�culo 22 Tipos de �rganos, donde se establece que (1c) que las Juntas de Facultad, Escuela o Centro son �rganos colegiados. Y encontramos su definici�n en el Art�culo 31 Las Juntas de Centros, donde podemos leer que ''La Junta de Facultad, Escuela o Centro es el �rgano colegiado de gobierno del mismo, que ejerce sus funciones con vinculaci�n a los acuerdos del Patronato, Consejo de Gobierno y resoluciones del Rector.''
\par
Tambi�n podemos destacar los art�culos 32 y 33, donde se establece la composici�n y funciones de las Juntas de Facultad, Centro o Escuela.
\par
Art�culo 32
Composici�n de las Juntas
La Junta de Facultad, Escuela o Centro estar� compuesta por miembros natos y electos.
Son miembros natos: El Decano o Director, que presidir� sus reuniones; los Vicedecanos o Subdirectores, el Secretario acad�mico, que levantar� acta de sus sesiones y los Directores de los Departamentos integrados en la Facultad o Escuela.
Son miembros electos: Quienes resulten elegidos en representaci�n del profesorado y de los alumnos de acuerdo con la normativa que reglamentariamente se establezca.
\par
Art�culo 33
Funciones de las Juntas
Las competencias de la Junta de Facultad, Escuela o Centro son:
a) Colaborar con el Decano o Director en la gesti�n de la Facultad, Escuela o Centro.
b) Promover el perfeccionamiento de los planes de estudio y de la metodolog�a docente, as� como el establecimiento de nuevos t�tulos tanto propios como oficiales.
c) Participar en la programaci�n de las actividades de extensi�n universitaria.
d) Velar por la adecuada dotaci�n de los servicios necesarios para su correcto funcionamiento.
e) Cualquier otra competencia que le pueda ser atribuida en el desarrollo de estas Normas de Organizaci�n y Funcionamiento.
\par
\par
\par
\par
\par
\par
\par
\par
\par
\par
\par
\par
\par
Tipos de Voto Electr�nico
\par
\begin{itemize}
	\item Presenciales
	\begin{itemize}
		\item Urna electr�nica (Sistema DRE - Direct-Recording Electronic - sistema de registro electr�nico directo): facilita el voto a  trav�s de una pantalla t�ctil, teclado u otro dispositivo. La m�quina DRE permite la captura, almacenamiento y escrutinio de los votos.
		\item Sistema reconocedor de marca �ptica:  El votante marca su voto en una papeleta mediante un bol�grafo, por ejemplo, y la inserta en un lector o esc�ner, a trav�s del cual la m�quina autom�ticamente registra el voto para su posterior contabilizaci�n.
	\end{itemize}
	\item Remotos
	\begin{itemize}
		\item Sistema de votaci�n telem�tica a trav�s de Internet: el elector vota mediante una aplicaci�n cliente (normalmente un navegador web) que env�a el voto a trav�s de Internet al servidor donde queda almacenado.
		\item Sistema de votaci�n telem�tica a trav�s de dispositivos m�viles: el elector vota mediante una aplicaci�n cliente que env�a el voto a trav�s de una red m�vil e Internet, dependiendo del caso, al servidor donde queda almacenado.
	\end{itemize}
\end{itemize}

\par
\par
\par
\par
\par
\par
\par
\par
\par
\par
REQUISITOS DESEABLES EN LOS SISTEMAS DE VOTO ELECTR�NICO
\begin{description}
	\item[Autenticidad]: S�lo los votantes autorizados pueden votar.
	\item[Anonimato]: El voto es secreto.
	\item[Verificabilidad]: El votante puede asegurarse de que su voto se ha contado adecuadamente.
	\item[Imposibilidad de coacci�n]: El voto emitido no puede ser mostrado.
	\item[Posibilidad de emitir un voto nulo].
	\item[Fiabilidad]: el sistema debe asegurar que no se producen alteraciones de los resultados.
	\item[Auditabilidad]: se debe poder comprobar que el funcionamiento de los elementos que intervienen en el proceso es correcto.
	\item[Usabilidad]: cualquier votante debe ser capaz de emitir un voto en un tiempo razonable.
\end{description}


\par
\par
\par
\par
\par
\par
Aplicaciones
\begin{description}
	\item[Gesti�n del censo electoral]: altas, bajas, informes, solicitud, tramitaci�n del voto por correo...
	\item[Gesti�n de candidatos]: solicitud, aprobaciones, difusi�n del perfil y propaganda...
	\item[Gesti�n del proceso electoral]: apertura de urnas, cierre de urnas, descifrado, introducci�n de votos por correo, escrutinio, recuento, presentaci�n de actas electorales...
	\item[Aplicaciones de voto]: todas aquellas que mecanizan la ejecuci�n del voto, el almacenamiento y su posterior recuento.
	\item[Presentaci�n general de actas y resultados electorales].
\end{description}
%%\chapter{TempEleccionJuntaEscuela}\label{tempEleccionJuntaEscuela}
\lhead{Cap�tulo \ref{tempEleccionJuntaEscuela}}
\rhead{TempEleccionJuntaEscuela}
%*******************************************************************************
\par
Empezamoss

Cada categor�a de profesores (colaboradores, adjuntos, agregados y
catedr�ticos) elegir� a dos representantes de entre los profesores con
contrato a media jornada o jornada completa.
Respecto a los alumnos:
Cada titulaci�n elegir� a dos representantes de entre todos los
delegados de la titulaci�n (dos de Teleco, dos de Inform�tica, dos de
Arquitectura y dos de Ingenier�a de la Edificaci�n)

 Los profes votamos por categor�as (los agregados a los suyos, etc.).
   La diferencia es que en el censo est�n todos (jornada completa, media
   jornada y tiempo parcial) pero s�lo son elegibles de media jornada
   para arriba.

 - Las categor�as de profes son disjuntas; s�lo votas en la tuya.

 - En cuanto a los alumnos, cada grupo tiene dos delegados (delegado y
   subdelegado). Recuerda que en Arq. hay varios grupos en cada curso,
   eso hace un censo m�s amplio.

%%%%%%%%%%%%%%%%%%%%%%%%%%%%%%%%%%%%%%%%%%%%%%%%%%%%%%%%%%%%%


%%%%%%%%%%%%%%%%%%%%%%%%%%%%%%%%%%%%%%%%%%%%%%%%%%%%%%%%%%%%%
%% BIBLIOGRAPHY AND OTHER LISTS
%%%%%%%%%%%%%%%%%%%%%%%%%%%%%%%%%%%%%%%%%%%%%%%%%%%%%%%%%%%%%
%% A small distance to the other stuff in the table of contents (toc)
\addtocontents{toc}{\protect\vspace*{\baselineskip}}




%% The Bibliography
%% ==> You need a file 'literature.bib' for this.
%% ==> You need to run BibTeX for this (Project | Properties... | Uses BibTeX)
\cleardoublepage
\phantomsection
\addcontentsline{toc}{chapter}{Bibliograf�a} %'Bibliography' into toc
\nocite* %Even non-cited BibTeX-Entries will be shown.
\bibliographystyle{acm} %Style of Bibliography: plain / apalike / amsalpha / ...
%\bibliographystyle{ieeetr} %Style of Bibliography: plain / apalike / amsalpha / ...
\bibliography{biblio} %You need a file 'literature.bib' for this.
\rhead{Bibliograf�a}




%%%%%%%%%%%%%%%%%%%%%%%%%%%%%%%%%%%%%%%%%%%%%%%%%%%%%%%%%%%%%
%% APPENDICES
%%%%%%%%%%%%%%%%%%%%%%%%%%%%%%%%%%%%%%%%%%%%%%%%%%%%%%%%%%%%%

\appendix
%% ==> Write your text here or include other files.
%%\chapter{Licencia Open Source}\label{osi}
\lhead{Anexo \ref{osi}}
\rhead{Licencia Open Source}

\par
Este tipo de licencia se usa para programas inform�ticos con copyright, en casos donde: el software de dominio p�blico (esto significa sin licencia), cumple todos estos criterios siempre y cuando todo el c�digo fuente est� disponible, y est� reconocido por la Open Source Initiative (OSI) y se le permita usar la marca de la misma.  
\par
Una licencia es considerada Open Source cuando ha sido aprobada por la OSI, mediante los criterios explicados a continuaci�n.
\par
\section{Definici�n de una licencia Open Source}
\par
La Open Source Initiative utiliza los siguientes criterios para determinar si una licencia de software puede o no considerarse software abierto. Esta definici�n se basa en las Directrices de software libre de Debian, y est� escrita y adaptada primeramente por Bruce Perens. Es similar pero no igual a la definici�n de licencia de software libre.
\par
\begin{enumerate}
	\item Libre redistribuci�n: el software debe poder ser regalado o vendido libremente.
	\item C�digo fuente: el c�digo fuente debe estar incluido u obtenerse libremente.
	\item Trabajos derivados: la redistribuci�n de modificaciones debe estar permitida.
	\item Integridad del c�digo fuente del autor: las licencias pueden requerir que las modificaciones sean redistribuidas solo 	parches.
	\item Sin discriminaci�n de personas o grupos: nadie puede dejarse fuera.
	\item Sin discriminaci�n de �reas de iniciativa: los usuarios comerciales no pueden ser excluidos.
	\item Distribuci�n de la licencia: deben aplicarse los mismos derechos a todo el que reciba el programa.
	\item La licencia no debe ser espec�fica de un producto: el programa no puede licenciarse solo como parte de una distribuci�n mayor.
	\item La licencia no debe restringir otro software: la licencia no puede obligar a que alg�n otro software que sea distribuido con el software abierto deba tambi�n ser de c�digo abierto.
	\item La licencia debe ser tecnol�gicamente neutral: no debe requerirse la aceptaci�n de la licencia por medio de un acceso por clic de rat�n o de otra forma espec�fica del medio de soporte del software.
 \end{enumerate}
\par
\section{La licencia}
\par
Puede obtenerse m�s informaci�n, as� como la versi�n completa de la licencia ver \cite{osiweb}
	

%%\chapter{Licencia P�blica General (GPL)}\label{gpl}
\lhead{Anexo \ref{gpl}}
\rhead{Licencia P�blica General (GPL)}

\par
Es una licencia creada por la FSF (Free Software Foundation) orientada principalmente a proteger la libre distribuci�n, modificaci�n y uso de software. Esta licencia declara que el cualquier software cubierto por ella es software libre, con el objetivo de protegerlo de intentos de apropiaci�n que restrinjan esas libertades a los usuarios.
\par
La definici�n de esta licencia viene marcada por el respeto a una serie de derechos que la FSF considera fundamentales para el usuario de software libre. 
\par

\section{Derechos del usuario de software libre}
\par
A continuaci�n se muestran, los cuatro derechos fundamentales que definen esta licencia.
\par
\subsubsection{El derecho a utilizar}
\par
El primer derecho o libertad, el que trata sobre el derecho a utilizar software, aunque pueda sorprender, tiene mucho sentido. . Cuando una persona ''compra'' un programa de ordenador que no es software libre por lo general no dispone del derecho de utilizaci�n ilimitada: el usuario est� limitado a utilizar el programa para determinados objetivos (prohibido usar este programa de forma comercial) o en determinados sitios (prohibido usar este programa en el pa�s X y el pa�s Y) o en un n�mero determinado de m�quinas (prohibido usar este programa en m�s de una m�quina al mismo tiempo). Estas restricciones son muy habituales cuando hablamos de software privativo, y el software libre debe respetar el derecho a que �stas no existan en la utilizaci�n.
\par
\subsubsection{El derecho a entender}
\par
La segunda libertad para el usuario: el derecho a entender c�mo funcionan los programas que nos distribuyen, y a adaptarlo a nuestras necesidades. Este derecho no existe cuando hablamos de software privativo: por lo general, el software privativo se distribuye en forma de ejecutables (equivalentes a los ficheros ''.exe'' en entornos windows) sin que le acompa�e el c�digo fuente correspondiente. El c�digo fuente de un programa es su forma entendible y modificable por un programador. 
\par
\subsubsection{El derecho a distribuir}
\par
El derecho a distribuir programas de ordenador de forma gratuita o, alternativamente, cobrando algo a cambio de hacerlo. Es natural, ya que la industria del software privativo hace cont�nuos esfuerzos para intentar convencer a la sociedad de que copiar programas de ordenador es algo que no debe hacerse. La FSF entiende que el poder copiar sin necesidad de grandes recursos (con una unidad de grabaci�n basta) y la caracter�stica peculiar de que la copia no pierde calidad respecto al original no es algo malo: por el contrario, es casi lo mejor que tiene el software. Copiar programas de ordenador y distribuirlas es algo que beneficia a la sociedad. Es de sentido com�n. Realizar copias de programas privativos es algo ilegal en la mayor�a de los pa�ses. Por eso proporcionamos software libre: es perfectamente legal copiarlo. De esta forma tanto el usuario como la sociedad se benefician, y nadie sale perdiendo (la copia original no funciona peor por haber hecho una o millones de copias).
\par
Es importante un detalle: el software libre no tiene por qu� ser gratis. Es perfectamente posible distribuir software libre a cambio de dinero. As� es como pueden ganarse la vida los programadores y distribuidores. Ahora bien, eso no justifica el hecho de vulnerar los derechos de la gente que paga por obtener una copia del programa: el usuario puede distribuir sus propias copias, cobrando por ello si lo desea.
\par
\subsubsection{El derecho a mejorar}
\par
El derecho a mejorar el software y distribuir las mejoras, es tal vez el que m�s controversia genera. El usuario de software privativo no puede mejorar los programas que utiliza: aunque quisiera y supiera hacerlo, por lo general no tiene acceso al c�digo fuente. Y aunque lo tuviera (puede distribuirse el c�digo fuente y no obstante no ser software libre) ser�a ilegal modificar ese c�digo fuente.
\par
Sin embargo, el software libre siempre se distribuye con su c�digo fuente, y adem�s es totalmente legal modificarlo. Y otra cosa importante: el usuario tambi�n tiene derecho a no distribuir sus mejoras si no quiere. Una persona puede descargar o comprar software libre, introducirle mejoras, y no redistribuir ni hacer p�blicas dichas mejoras. 
\par
\section{La licencia}
\par
Puede obtenerse m�s informaci�n, y la versi�n completa de la licencia en \cite{fsfweb}, y una versi�n de la misma en castellano en \cite{gnuweb}.

%%\chapter{Contenido del CD}\label{cd}
\lhead{Anexo \ref{cd}}
\rhead{Contenido del CD}
\par
Junto con este documento, se adjunta un CD que contiene los siguientes directorios:
\par
\textbf{ACABAR DE REDACTAR ESTE APARTADO}
\par
\begin{itemize}
	\item ROUNDUP. Version Original. Bla bla bla...
	\item ROUNDUP. Version Modificada. Bla bla bla...
	\item Utilidades. los ficheros .sh 
	\item Pruebas. Esta carpeta contiene el c�digo de las pruebas autom�ticas, en formato propio de Selenium, en Java, en HTML, y en Python.
	\item Memoria. Esta carpeta incluye copia en formato digital de este documento.
\end{itemize}



%%%%%%%%%%%%%%%%%%%%%%%%%%%%%%%%%%%%%%%%%%%%%%%%%%%%%%%%%%%%%
%% NOTAS (PARA BORRADOR)
%%%%%%%%%%%%%%%%%%%%%%%%%%%%%%%%%%%%%%%%%%%%%%%%%%%%%%%%%%%%%
\addtocontents{toc}{\protect\vspace*{\baselineskip}}
\addcontentsline{toc}{chapter}{Notas (para borrador)} %'Bibliography' into toc
\listoftodos[Notas (para borrador)]
\rhead{Notas}

\end{document}


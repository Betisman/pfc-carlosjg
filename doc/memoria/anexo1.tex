\chapter{Licencia Open Source}\label{osi}
\lhead{Anexo \ref{osi}}
\rhead{Licencia Open Source}

\par
Este tipo de licencia se usa para programas inform�ticos con copyright, en casos donde: el software de dominio p�blico (esto significa sin licencia), cumple todos estos criterios siempre y cuando todo el c�digo fuente est� disponible, y est� reconocido por la Open Source Initiative (OSI) y se le permita usar la marca de la misma.  
\par
Una licencia es considerada Open Source cuando ha sido aprobada por la OSI, mediante los criterios explicados a continuaci�n.
\par
\section{Definici�n de una licencia Open Source}
\par
La Open Source Initiative utiliza los siguientes criterios para determinar si una licencia de software puede o no considerarse software abierto. Esta definici�n se basa en las Directrices de software libre de Debian, y est� escrita y adaptada primeramente por Bruce Perens. Es similar pero no igual a la definici�n de licencia de software libre.
\par
\begin{enumerate}
	\item Libre redistribuci�n: el software debe poder ser regalado o vendido libremente.
	\item C�digo fuente: el c�digo fuente debe estar incluido u obtenerse libremente.
	\item Trabajos derivados: la redistribuci�n de modificaciones debe estar permitida.
	\item Integridad del c�digo fuente del autor: las licencias pueden requerir que las modificaciones sean redistribuidas solo 	parches.
	\item Sin discriminaci�n de personas o grupos: nadie puede dejarse fuera.
	\item Sin discriminaci�n de �reas de iniciativa: los usuarios comerciales no pueden ser excluidos.
	\item Distribuci�n de la licencia: deben aplicarse los mismos derechos a todo el que reciba el programa.
	\item La licencia no debe ser espec�fica de un producto: el programa no puede licenciarse solo como parte de una distribuci�n mayor.
	\item La licencia no debe restringir otro software: la licencia no puede obligar a que alg�n otro software que sea distribuido con el software abierto deba tambi�n ser de c�digo abierto.
	\item La licencia debe ser tecnol�gicamente neutral: no debe requerirse la aceptaci�n de la licencia por medio de un acceso por clic de rat�n o de otra forma espec�fica del medio de soporte del software.
 \end{enumerate}
\par
\section{La licencia}
\par
Puede obtenerse m�s informaci�n, as� como la versi�n completa de la licencia ver \cite{osiweb}
	

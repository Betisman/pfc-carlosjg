\par
Una vez configurado el programa, e inicializado el tracker, en este cap�tulo se indicar� c�mo se debe proceder a levantar el servicio. Tambi�n se realiza alguna peque�a aclaraci�n sobre los distintos manuales que se deben utilizar.
\subsection{Levantar el servicio}\label{levantar}
\par
Para arrancar el tracker y que el sistema entre en funcionamiento, se debe levantar el servicio. 
\par
Esta operaci�n se deber� realizar tambi�n en caso de que la m�quina caiga, o tras cualquier operaci�n de mantenimiento del sistema. Es decir, cada vez que ocurra un evento por el cual debamos volver a poner en funcionamiento el sistema.
\par
Para levantar el servicio, deberemos trabajar como usuario normal, y no como superusuario (como hemos hecho en la instalaci�n y en la configuraci�n). Para ello, se deben revisar los permisos en las carpetas involucradas, y comprobar que sean los pertinentes.
\par
Se ejecutar�n en la consola las siguientes instrucciones, de acuerdo a los casos expuestos en la instalaci�n y la configuraci�n:
\begin{center}
\verb|export PATH=$PATH:/opt/roundup/bin|
\end{center}
\begin{center}
\verb|roundup-server pfc=/opt/roundup/trackers/pfc|
\end{center}
\par
Por comodidad en el uso, se recomienda encarecidamente el uso del script ''levantar.sh'' que se adjunta en el CD, y contiene s�mplemente esas dos instrucciones. Para utilizar este script, habr�a que alojarlo dentro del directorio ''home/pfc''. Para levantar la aplicaci�n bastar� unicamente con situarnos en ese directorio desde la consola y ejecutar:
\begin{center}
\verb|./levantar.sh|
\end{center}
\par
\subsection{Manual de usuario}
\par
Para un correcto uso del sistema, se ven involucrados los siguientes documentos:
\begin{itemize}
	\item Documentaci�n de uso de Round-up Issue Tracking (incluida en el CD adjunto, y tambi�n disponible y actualizada en \cite{roundupweb}).
	\item Cualquier manual, normativa o documentaci�n propuesta para la utilizaci�n de este sistema, comunicaci�n en los roles m�s bajos, etc, as� como cualquier otro documento que sea considerado pertinente por las Fuerzas Armadas.
\end{itemize}
\par
Aunque el funcionamiento del sistema es muy intuitivo y sencillo, cualquier duda podr� ser aclarada en la documentaci�n del programa, por lo que esa documentaci�n debe estar disponible. \par
Igualmente, el programa no deber� entrar en producci�n hasta que los usuarios finales tengan pleno conocimiento de las normativas de uso impuestas por las Fuerzas Armadas. 

\section{Configuraci�n del Servidor Web del Sistema de Autenticaci�n} \label{App:ConfiguracionServidorWebOauth}

El framework web Django provee su propio Servidor Web, pero para poder utilizar un canal seguro \gls{HTTPS} y configurar el acceso con el \gls{DNIe} es preciso utilizar un Servidor Web como Apache en el que apoyar el framework. Aqu� una lista de pasos que se han llevado a cabo para su configuraci�n. 

\begin{lstlisting}[language=bash]
	# Instalamos Apache2 y OpenSSL
	sudo apt-get install apache2
	sudo apt-get install openSSL
	
	cd /etc/apache2/mods-available
	
	# Activamos SSL en Apache
	sudo a2enmod ssl
	
	sudo apt-get install libapache2-mod-wsgi
	sudo a2enmod wsgi
	
	cd /etc/apache2/sites-available
	sudo cp 000-default.conf sitednie.conf
	cd /etc/apache2/sites-available
	
	# Preparamos los certificados para el DNIe
	mkdir ~/Descargas/certificados
	cd certificados/
	
	# Creamos una clave
	openssl genrsa -des3 -out server.key 2048
	
	# Petici�n del certificado, asoci�ndolo con la clave que acabamos de crear
	openssl req -new -key server.key -out server.csr
	
	# Pese a que no somos una entidad certificadora, firmamos nuestro propio certificado,
	# as� obtenemos un certificado autofirmado.
	openssl x509 -req -days 365 -in server.csr -signkey server.key -out server.crt
	
	#Bajamos el certificado de la AC Ra�z del DNIE desde:
	#http://www.dnielectronico.es/PortalDNIe/PRF1_Cons02.action?pag=REF_077
	wget http://www.dnielectronico.es/ZIP/ACRAIZ-SHA2.CAB
	
	# Lo extraemos
	cabextract ACRAIZ-SHA2.CAB
	
	# Creamos el certificado x509
	openssl x509 -in ACRAIZ-SHA2.crt -inform DER -out ACRAIZ-SHA2.crt -outform PEM
	sudo cp ACRAIZ-SHA2.crt acraiz-dnie.cer
	
	cd /etc/apache2/sites-available
	sudo cat default-ssl.conf >> sitednie.conf
	
	# Configuramos el servidor web para que escuche por un puerto seguro, con SSL y 
	# requiriendo los certificados del DNIe
	sudo gedit sitednie.conf &	
\end{lstlisting}

El fichero de configuraci�n, en nuestro caso en la ruta \textit{/etc/apache2/sites-available/sitednie.conf} para Apache es el siguiente:


\begin{lstlisting}[language=XML]
<!-- Configuraci�n de Django en Apache -->
WSGIDaemonProcess / python-path=/home/carlos/pfc/pfc-carlosjg/src/oauth2server/oauth2server:/home/carlos/pfc/pfc-carlosjg/src/oauth2server/venv/lib/python2.7/site-packages

<IfModule mod_ssl.c>
	
	<!-- Servidor HTTPS para certificado del DNIe -->
	<VirtualHost *:443>
		ServerAdmin webmaster@localhost
		ServerName eleccionesuspceu.com

		DocumentRoot /home/carlos/pfc/pfc-carlosjg/src/oauth2server/oauth2server/proj

		LogLevel trace1 ssl:debug
		ErrorLog ${APACHE_LOG_DIR}/error.log
		CustomLog ${APACHE_LOG_DIR}/access.log combined

		<!-- Configuraci�n de certificados SSL -->
		SSLEngine on
		SSLProtocol all -SSLv2	
		SSLCipherSuite ALL:!ADH:!EXPORT56:RC4+RSA:+HIGH:+MEDIUM:+LOW:+SSLv2:+EXP:+eNULL

		SSLCertificateFile	/home/carlos/Descargas/certificados/server.crt
		SSLCertificateKeyFile /home/carlos/Descargas/certificados/server.key

		SSLCACertificateFile /home/carlos/Descargas/certificados/acraiz-dnie.cer 

		SSLVerifyClient require
		SSLVerifyDepth 2 

		<!-- Para activar la validaci�n OCSP en servidor, hay que descomentar las dos siguientes l�neas -->
		#SSLOCSPEnable on
		#SSLOCSPDefaultResponder "http://ocsp.dnie.es/"

		<FilesMatch "\.(cgi|shtml|phtml|php)$">
				SSLOptions +StdEnvVars +ExportCertData
		</FilesMatch>
		<Directory /usr/lib/cgi-bin>
				SSLOptions +StdEnvVars +ExportCertData
		</Directory>
		SSLOptions +StdEnvVars +ExportCertData

		BrowserMatch "MSIE [2-6]" \
				nokeepalive ssl-unclean-shutdown \
				downgrade-1.0 force-response-1.0
		BrowserMatch "MSIE [17-9]" ssl-unclean-shutdown

		<Directory /home/carlos/pfc/pfc-carlosjg/src/oauth2server/oauth2server/proj/>
		    <Files wsgi.py>
				Require all granted
			</Files>
		</Directory>
		WSGIProcessGroup /
		WSGIScriptAlias / /home/carlos/pfc/pfc-carlosjg/src/oauth2server/oauth2server/proj/wsgi.py
	</VirtualHost>

	<!-- Servidor HTTPS para comunicaci�n segura con el Sistema de Votaci�n -->
	<VirtualHost *:442>
		ServerAdmin webmaster@localhost

		DocumentRoot /home/carlos/pfc/pfc-carlosjg/src/oauth2server/oauth2server/proj

		LogLevel debug ssl:debug		
		ErrorLog ${APACHE_LOG_DIR}/error.log
		CustomLog ${APACHE_LOG_DIR}/access.log combined

		SSLEngine on

		SSLCertificateFile	/home/carlos/Descargas/certificados/server.crt
		SSLCertificateKeyFile /home/carlos/Descargas/certificados/server.key

		SSLVerifyClient optional_no_ca
		SSLVerifyDepth  2

		<FilesMatch "\.(cgi|shtml|phtml|php)$">
				SSLOptions +StdEnvVars +ExportCertData
		</FilesMatch>
		<Directory /usr/lib/cgi-bin>
				SSLOptions +StdEnvVars +ExportCertData
		</Directory>
		SSLOptions +StdEnvVars +ExportCertData

		
		BrowserMatch "MSIE [2-6]" \
				nokeepalive ssl-unclean-shutdown \
				downgrade-1.0 force-response-1.0
		BrowserMatch "MSIE [17-9]" ssl-unclean-shutdown

		<Directory /home/carlos/pfc/pfc-carlosjg/src/oauth2server/oauth2server/proj/>
		    <Files wsgi.py>
				Require all granted
			</Files>
		</Directory>

		WSGIProcessGroup /
		WSGIScriptAlias / /home/carlos/pfc/pfc-carlosjg/src/oauth2server/oauth2server/proj/wsgi.py

	</VirtualHost>
</IfModule>
\end{lstlisting}

A continuaci�n hay que editar el fichero \textit{/etc/apache2/ports.conf} si es necesario:

\begin{lstlisting}[language=bash]
	sudo gedit /etc/apache2/ports.conf
\end{lstlisting}
\begin{lstlisting}[language=XML]
	### Modificar si es necesario:
	<IfModule ssl_module>
		Listen 443
		Listen 1443
	</IfModule>
\end{lstlisting}

Cargamos la configuraci�n modificada en Apache y reseteamos el servicio:
\begin{lstlisting}[language=bash]
	sudo a2ensite sitednie.conf
	sudo service apache2 restart
\end{lstlisting}


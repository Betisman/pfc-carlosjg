\chapter*{Resumen}
\addcontentsline{toc}{chapter}{Resumen}

\vspace{-30pt}
El presente proyecto desarrolla una soluci�n tecnol�gica de voto por Internet con el que realizar las Elecciones a la Junta de Escuela de la Escuela Polit�cnica Superior de la Universidad San Pablo CEU, sita en Madrid, Espa�a. \\

Trata de ofrecer una prueba de concepto de un sistema seguro de voto por Internet que hace uso del nuevo \gls{DNIe} 3.0 como herramienta de identificaci�n remota del votante. \\

En el momento de la redacci�n de esta memoria, existen soluciones implementadas para este tipo de problemas, pero ning�n sistema actual permite utilizar el nuevo documento de identidad espa�ol para identificar de forma remota al votante. \\

Por tanto, podemos considerar que con este desarrollo se establece el primer sistema de voto por Internet que utiliza el \gls{DNIe} 3.0 para identificar al votante con tecnolog�a \gls{NFC}. \\

Para desarrollar el sistema se ha decidido adaptar una soluci�n ya existente, Helios Voting. Este proyecto, creado por Ben Adida y nacido en el \gls{MIT}, es considerado un est�ndar de facto en votaci�n electr�nica basada en protocolos de Verificaci�n Punto-a-Punto y en un esquema criptogr�fico homom�rfico. Es el proyecto libre m�s completo para aquellos procesos electorales con riesgo bajo de coacci�n. \\

No obstante, para poder cumplir con los objetivos del \gls{PFC}, que contiene el uso del \gls{DNIe} 3.0 como documento digital de identificaci�n de usuario, ha sido necesario realizar una integraci�n de sistemas. El proyecto Helios Voting, pese a proporcionar un gran n�mero de opciones de login, no soporta por defecto identificaci�n con certificados digitales, lo cual es b�sico para poder utilizar los que contiene el \gls{DNIe}. Por ello, ha sido necesario dise�ar un m�dulo de identificaci�n alternativo, basado en protocolo oAuth 2.0 con un servidor web configurado para aceptar estos certificados. \\

Para facilitar el uso de este documento, se ha integrado tambi�n una app de Android desarrollada por la Polic�a. Esta app requiere una adaptaci�n para las necesidades del proyecto, pero permite a los votantes usar sus dispositivos m�viles para votar utilizando el sensor \gls{NFC} de los mismos y su propio \gls{DNIe} 3.0, sin necesidad de requerir de hardware externo como los lectores de chips con contacto. Esta aproximaci�n posibilita que realmente se pueda votar desde cualquier lugar con conexi�n a Internet. \\

\chapter*{Resumen}
\addcontentsline{toc}{chapter}{Resumen}
%\par
%Con el creciente desarrollo de la tecnolog�a y su implantaci�n en la mayor�a de los campos de la vida cotidiana, es inevitable pensar en soluciones electr�nicas para un elemento tan importante de nuestra sociedad como son los procesos electorales.
%\par 
%Encontramos procesos electorales en multitud de organizaciones, desde estados nacionales a empresas privadas, pasando por juntas de administraciones u organismos p�blicos.
%\par
%En cambio, contrario a lo que puede parecer por el intenso uso de las nuevas tecnolog�as en campos como las transacciones bancarias, telemedicina, comunicaciones o gestiones con la Administraci�n, en el mundo electoral no se est� terminando de introducir el voto telem�tico a gran escala. De hecho, aunque encontramos algunas excepciones como pueden ser Estonia, Venezuela, Brasil y algunos territorios m�s reducidos, no se utiliza a estos niveles en la totalidad del proceso electoral, quedando reducido a algunas fases del proceso o, simplemente, a ninguna.
%\par
%En este PFC, vamos a evaluar la implantaci�n del voto telem�tico a peque�a escala para tratar de escalar los problemas que comportan a nivel nacional. Para ello, vamos a realizar un sistema de voto por Internet que soportar� de forma �ntegra las elecciones a la Junta de Escuela de la Escuela Polit�cnica Superior de la Universidad San Pablo CEU.
%\par
%A partir de este desarrollo, trataremos de hacer frente, a peque�a escala, a los problemas que nos encontramos en estas grandes elecciones, aunque para ello tengamos que establecer requisitos que resulten exagerados para la consecuci�n de la elecci�n que implementamos por su simplicidad frente a un proceso a nivel nacional o auton�mico.
 %
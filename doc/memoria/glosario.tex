\chapter{Glosario de T�rminos}\label{glos}
\lhead{Cap�tulo \ref{glos}}
\rhead{Glosario de t�rminos}

\begin{itemize}
	\item SELENIUM IDE: herramienta para automatizar pruebas para aplicaciones web, que funciona directamente sobre el internet browser (navegador de internet). Para m�s informaci�n, ver \cite{selenium}
	\item MDA: Medical Digital Assistant. Ordenador de mano, similar a una PDA (Personal Digital Assistant), pero orientada para el uso por personal m�dico y de muy util aplicaci�n en telemedicina. 
	\item WI-FI: conjunto de est�ndares para redes inal�mbricas basados en las especificaciones IEEE 802.11
	\item MNFMC: Multinational Field Medical Card. Ficha para gestionar el desarrollo de las lesiones de los soldados, en caso de resultar heridos, descrita en el estandard OTAN STANAG 2132.
	\item Modelo Cliente-Servidor: arquitectura que consiste b�sicamente en que un programa (el cliente) realiza peticiones a otro programa (el servidor) que es quien realiza las operaciones precisas y le da respuesta al primero.
	\item Cliente Ligero: se trata de un cliente, dentro de una arquitectura cliente-servidor, que tiene muy poca o ninguna l�gica del programa y por lo tanto depende principalmente del servidor central para las tareas de procesamiento.
	\item SGBD: Sistema Gestor de Bases de Datos. Software dedicado a servir de interfaz entre la base de datos, el usuario y las aplicaciones que la utilizan.
	\item ACID: Atomicity, Consistency, Isolation and Durability (Atomicidad, Consistencia, Aislamiento y Durabilidad). Se dice de un SGBD que es \textit{'ACID compliant'} cuando es capaz de realizar transacciones seguras, por cumplir esos cuatro principios.
	\item Transacci�n: Interacci�n con una estructura de datos que, a�n siendo compleja y pudiendo estar compuesta por varios procesos que se han de aplicar uno despu�s del otro, queremos que sea equivalente a una interacci�n at�mica. Es decir, que se realice de una sola vez y que la estructura a medio manipular no sea jam�s alcanzable por el resto del sistema.
	\item Trigger: evento que se ejecuta cuando se cumple una condici�n establecida al realizar una operaci�n de inserci�n (INSERT), actualizaci�n (UPDATE) o borrado (DELETE), en una base de datos.
	\item HTML: HyperText Markup Language, (Lenguaje de Etiquetas de Hipertexto). Es un lenguaje de marcado dise�ado para estructurar textos y presentarlos en forma de hipertexto, que es el formato est�ndar de las p�ginas web.
	\item CSS: Cascading Style Sheets (Las hojas de estilo en cascada). Son un lenguaje formal usado para definir la presentaci�n de un documento estructurado escrito en HTML o XML.
	\item SSH: Secure SHell. Protocolo que sirve para acceder a m�quinas remotas a trav�s de una red mediante un int�rprete de comandos.
	\item Complejidad ciclom�tica:  medida cuantitativa de la complejidad l�gica de un programa que define el n�mero de caminos independientes del conjunto b�sico de un m�dulo y por tanto nos proporciona una cota superior del n�mero m�ximo de pruebas que se han de realizar para garantizar que cada sentencia se 
ejecuta al menos una vez
	\item Pruebas de caja blanca: metodolog�a de prueba que se basa en las estructuras de control del dise�o 
procedimental para generar los casos de prueba que, garanticen que se recorren por lo menos una vez todos los caminos independientes de cada m�dulo, que se ejecutan todas las decisiones l�gicas en su parte verdadera y en su parte falsa, se recorren todos los bucles.
	\item Pruebas de caja negra: pruebas que se llevan a cabo sobre la interfaz del software, para probar su funcionalidad, donde los casos de prueba pretenden demostrar que las funciones del software se verifican, que la entrada se acepta de forma adecuada y que se produce una salida correcta, as� como que la integridad de la informaci�n externa se mantiene. 

	
\end{itemize}
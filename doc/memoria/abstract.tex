\chapter*{Abstract}
\par
El sistema actual de telemedicina de las Fuerzas Armadas en Espa�a, separa los distintos integrantes en cuatro roles, desde los de intervenci�n m�s r�pida (tales como veh�culos ambulancia blindados, pr�ximos a los nidos de heridos), a los de intervenci�n menos r�pida pero con mayor capacidad m�dica (hospitales de la infraestructura nacional).
\par 
En la actualidad, est� perfectamente resuelto todo lo relativo a los sistemas de intercambio de informaci�n y comunicaci�n de eventos entre los roles de niveles 3 y 4. La problem�tica de estos roles, ya resuelta, correspond�a con la capacidad de facilitar asistencia desde la infraestructura nacional a los puestos m�dicos fuera del territorio nacional.
\par
En cambio, a�n no se han creado sistemas para solucionar las problem�ticas existentes en los roles 1 y 2, donde las unidades se encuentran en zonas m�s desplazadas, con menos medios, pero con mayor velocidad de actuaci�n.
\par
El objetivo de este proyecto es aportar soluciones para las problem�ticas planteadas en estos roles 1 y 2.